%!TEX root = ../main.tex

\begin{figure}[t!]
\protbox{$(2,3)$-constructions}{

\textbf{Parameters.} Let $\secparam$ be the security parameter and define parameters $n, m, t$ as functions of $\secparam$ such that $m \geq n, m \geq t$.

\textbf{Public values.}
Let $\mat{A} \in \Z_2^{m \times n}$ and $\mat{B} \in \Z_3^{t \times m}$ be fixed public matrices chosen uniformly at random. The matrices can also be chosen to be full-rank circulant matrices.

\begin{construction}[Mod-2/Mod-3 wPRF Candidate~\cite{boneh2018-darkmatter}]
The \ttwPRF candidate is a family of functions $\PRFfunc : \Z_2^{m \times n} \times \Z_2^n \to \Z_3^t$ with key-space $\keyspace_\secparam = \Z_2^{m \times n}$, input space $\inspace_\secparam = \Z_2^n$ and output space $\outspace_\secparam = \Z_3^t$. For a key $\mat{K} \in \keyspace_\secparam$, we define $\sfF_{\mat{K}}(x) = \PRFfunc(\mat{K}, x)$ as follows:

\begin{enumerate}[topsep=0pt]
    \item On input $x \in \Z_2^n$, first compute $w = \BLMap_2(\mat{K}, x) = \mat{K}x$.
    \item Output $y = \LMap^{\mat{B}}_3\left(\Convert_{(2,3)}(w)\right)$. That is, view $w$ as a vector over $\Z_3$ and then output $y = \mat{B}w$. 
\end{enumerate}
\label{construction:23-central-wprf}
\end{construction}

\begin{construction}[Mod-2/Mod-3 OWF Candidate]
The \ttOWF candidate is a function $\OWFfunc : \Z_2^n \to \Z_3^t$ with input space $\inspace_\secparam = \Z_2^n$ and output space $\outspace_\secparam = \Z_3^t$. We define $\sfF(x) = \OWFfunc(x)$ as follows:

\begin{enumerate}[topsep=0pt]
    \item On input $x \in \Z_2^n$, first compute $w = \LMap^{\mat{A}}_2(x) = \mat{A}x$.
    \item Output $y = \LMap^{\mat{B}}_3\left(\Convert_{(2,3)}(w)\right)$. That is, view $w$ as a vector over $\Z_3$ and then output $y = \mat{B}w$.
\end{enumerate}
\label{construction:23-owf}
\end{construction}


\medskip

\begin{minipage}{0.5\textwidth}
\begin{center} %!TEX root = ../../main.tex

\scalebox{0.8}{
\begin{tikzpicture}
    \begin{scope}
        \draw[twostyle] (-0.74,0.65) -- (0.74,1.05) -- (0.74,-1.05) -- (-0.74,-0.65) -- (-0.74,-0.55) -- (0.5,-0.55) -- (0.5,0.55) -- (-0.74,0.55) -- cycle;
         \draw (-0.8,0.65) node[left] {$\mat{K}$};
         \draw (-0.8,-0.65) node[left] {$x$};
    \end{scope}
    \begin{scope}[shift={(0.95,0)}]
    \tikzmath{\sidelen=0.6;\maxnum=4;}
    \foreach \num in {0,...,\maxnum} {
        \tikzmath{\yshift=(\num-(\maxnum)/2)*(\sidelen + 0.3);}
        \node [draw,regular polygon, regular polygon sides=3,scale=0.5,shift={(0,\yshift)},rotate=90] {};
    }
    \end{scope}
    \begin{scope}[shift={(1.85,0)}]
    \node [draw,trapezium,minimum height=1.48cm,trapezium left angle=75,trapezium right angle=75, rotate=-90,threestyle] {\rotatebox{90}{$\mat{B}$}};
    \draw (0.75,0) node[right] {$y$};
    \end{scope}
\end{tikzpicture}
}
 \end{center}
\end{minipage}
\begin{minipage}{0.5\textwidth}
\begin{center} %!TEX root = ../../main.tex

\scalebox{0.8}{
\begin{tikzpicture}
    \begin{scope}
     \node [draw,trapezium,minimum height=1.48cm,trapezium left angle=75,trapezium right angle=75, rotate=90,twostyle] {\rotatebox{-90}{$\mat{A}$}};
    \node[shift={(-1,0)}] {$x$};
    \end{scope}
    \begin{scope}[shift={(0.95,0)}]
    \tikzmath{\sidelen=0.6;\maxnum=4;}
    \foreach \num in {0,...,\maxnum} {
        \tikzmath{\yshift=(\num-(\maxnum)/2)*(\sidelen + 0.3);}
        \node [draw,regular polygon, regular polygon sides=3,scale=0.5,shift={(0,\yshift)},rotate=90] {};
    }
    \end{scope}
    \begin{scope}[shift={(1.85,0)}]
    \node [draw,trapezium,minimum height=1.48cm,trapezium left angle=75,trapezium right angle=75, rotate=-90,threestyle] {\rotatebox{90}{$\mat{B}$}};
    \draw (0.75,0) node[right] {$y$};
    \end{scope}

    \begin{scope}[shift={(0,-1.5)}]
        \draw (0,0) node[right] {\ttOWF};
    \end{scope}

\end{tikzpicture}
}
 \end{center}
\end{minipage}

}
\caption{$(2,3)$-constructions}
\label{fig:primary-constructions}
\end{figure}

