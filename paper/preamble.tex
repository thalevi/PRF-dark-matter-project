%!TEX root = ../main.tex

%--------------------------------------------------------------%
%-----------------------Common Packages------------------------%
%--------------------------------------------------------------%
\usepackage{xcolor}
\usepackage{amsfonts,amssymb,amsmath,amsthm}
\usepackage{enumitem}
\usepackage[style=alphabetic,backend=bibtex,maxnames=4,minnames=3,maxbibnames=99]{biblatex}
\usepackage[pdfauthor={},pdfpagelabels=true,linktocpage=true,hidelinks,linktoc=all,colorlinks]{hyperref}
\usepackage[nameinlink,capitalize]{cleveref}
\hypersetup{linkcolor = {teal}, citecolor = {magenta}, urlcolor = {black}}

\usepackage{stmaryrd}
\usepackage{graphicx}
\usepackage{calc}
\usepackage{multirow,makecell}
%--------------------------------------------------------------%


%--------------------------------------------------------------%
%---------------------------Notation---------------------------%
%--------------------------------------------------------------%

% Standard math notation
\newcommand{\N}{\mathbb{N}}     % Natural numbers
\newcommand{\Z}{\mathbb{Z}}     % Integers
\newcommand{\R}{\mathbb{R}}     % Real numbers
\newcommand{\F}{\mathbb{F}}     % Finite field

\newcommand{\ceil}[1]{\left\lceil {#1} \right\rceil}
\newcommand{\floor}[1]{\left\lfloor {#1} \right\rfloor}
\newcommand{\abs}[1]{\left| {#1} \right|}

\newcommand{\xor}{\oplus}

\newcommand{\figref}[1]{Figure~\protect\ref{#1.fig}}

\newcommand*\BitAnd{\mathbin{\&}}
\newcommand*\BitOr{\mathbin{|}}
\newcommand*\ShiftLeft{\ll}
\newcommand*\ShiftRight{\gg}
\newcommand*\BitNeg{\ensuremath{\mathord{\sim}}}

% Crypto Notation
\newcommand{\secparam}{\lambda}
\newcommand{\bits}{\{0,1\}}
\newcommand{\keyspace}{\mathcal{K}}
\newcommand{\inspace}{\mathcal{X}}
\newcommand{\outspace}{\mathcal{Y}}

%-------General notation---------
\newcommand{\mat}[1]{\MakeUppercase{\mathbf{#1}}}
\newcommand{\thtext}{\textnormal{th}}

% sf font
\newcommand{\sfF}{\mathsf{F}}





%--------------------------------------------------------------%



%--------------------------------------------------------------%
%-----------------------Specific Commands----------------------%
%--------------------------------------------------------------%
\newcommand{\pzero}{\textnormal{P}_0}
\newcommand{\pone}{\textnormal{P}_1}
\newcommand{\ptwo}{\textnormal{P}_2}
\newcommand{\partyi}{\textnormal{P}_i}
\newcommand{\party}{\textnormal{P}}


\newcommand{\share}[1]{\left\llbracket #1 \right\rrbracket}

\newcommand{\PRFfunc}{\mathsf{F}_\secparam}
\newcommand{\OWFfunc}{\mathsf{F}_\secparam}
\newcommand{\map}[1]{\textsf{map23}_{#1}}
\newcommand{\Fconvert}{\mathcal{F}_{23}}
\newcommand{\unmask}{\mathcal{F}_{\textnormal{unmask}}}

\newcommand{\prot}{\boldsymbol{\pi}}
\newcommand{\protunmask}{\prot_\textnormal{unmask}}

\newcommand{\LinMap}{\textsf{LinMap}}
\newcommand{\BiLinear}{\textsf{BiLinear}}
\newcommand{\Convert}{\textsf{Convert}}

%--------------------------------------------------------------%





%--------------------------------------------------------------%
%-----------------------LaTeX Formatting-----------------------%
%--------------------------------------------------------------%
% Comments: Add your name as necessary
\ifcomments
    \newcommand{\mahimna}[1]{\textsf{\color{blue}{[Mahimna: {#1}]}}}
\else
    \newcommand{\mahimna}[1]{}
\fi

% Environments
\newtheorem{theorem}{Theorem}[section]
\newtheorem{corollary}{Corollary}[theorem]
\newtheorem{lemma}[theorem]{Lemma}
\newtheorem*{claim}{Claim}

\theoremstyle{definition}
\newtheorem{definition}[theorem]{Definition}
\newtheorem{example}[theorem]{Example}
\newtheorem{construction}[theorem]{Construction}
\newtheorem{protocol}[theorem]{Protocol}
\newtheorem{functionality}[theorem]{Functionality}
\theoremstyle{remark}
\newtheorem{remark}[theorem]{Remark}



% Boxes
\newcommand{\boxsize}{0.95}
\newcommand{\protbox}[2]{\noindent\fbox{\small\hbox{\quad\begin{minipage}{\boxsize\columnwidth}\begin{center}{\bf #1}\end{center}#2\end{minipage}\quad}}}

% Operators
% \DeclareMathOperator*\bigboxplus{\ensuremath{\boxplus}}

\newcommand{\bigboxplus}{
  \mathop{
    \vphantom{\bigoplus} 
    \mathchoice
      {\vcenter{\hbox{\resizebox{\widthof{$\displaystyle\bigoplus$}}{!}{$\boxplus$}}}}
      {\vcenter{\hbox{\resizebox{\widthof{$\bigoplus$}}{!}{$\boxplus$}}}}
      {\vcenter{\hbox{\resizebox{\widthof{$\scriptstyle\oplus$}}{!}{$\boxplus$}}}}
      {\vcenter{\hbox{\resizebox{\widthof{$\scriptscriptstyle\oplus$}}{!}{$\boxplus$}}}}
  }\displaylimits 
}
