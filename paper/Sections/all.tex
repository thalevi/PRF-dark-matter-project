%!TEX root = ../main.tex

\begin{abstract}
We study new candidates for symmetric cryptographic primitives that leverage alternation between linear functions over $\Z_2$ and $\Z_3$ to support fast protocols for secure multiparty computation (MPC). This continues the study of weak pseudorandom functions of this kind initiated by Boneh et al.\ (TCC 2018) and Cheon et al.\ (PKC 2021).

We make the following contributions. 
\begin{itemize}
\item {\bf Candidates.} We propose new designs of symmetric primitives based on alternating moduli. These include candidate one-way functions, pseudorandom generators, and weak pseudorandom functions.
% in which the key, the  input and the output are all over $\Z_2$. 
We propose concrete parameters based on cryptanalysis. 
\item {\bf Protocols.} We provide a unified approach for securely evaluating modulus-alternating primitives in different MPC models. For the original candidate of Boneh et al., our protocols obtain roughly 2x improvement or more in all performance measures. We report efficiency benchmarks of an optimized implementation.
\item {\bf Applications.} We showcase the usefulness of our candidates for a variety of applications. This includes short ``Picnic-style'' signature schemes, as well as protocols for oblivious pseudorandom functions, hierarchical key derivation, and distributed key generation for function secret sharing.
\end{itemize}



\end{abstract}

%!TEX root = ../main.tex

\section{Introduction}
\label{sec:introduction}
%\greg{Does anyone know why the PDF doesn't have the index on the left hand side with links to all of the sections?}

%-------------------------------------------%
Symmetric-key primitives like one-way functions (OWFs)~\cite{levin1985-owf},
% Not sure about citation for OWF, but can cite Levin's survey: https://arxiv.org/abs/cs/0012023 
 pseudorandom generators (PRGs)~\cite{blum1984-prg,yao1982-prg}, 
and pseudo-random functions (PRFs)~\cite{goldreich1984-prf} are deployed in innumerable settings, and serve as fundamental building blocks of modern cryptography. While traditional use cases primarily considered settings where the function evaluation was done by a single party, many applications (recently also arising in the context of cryptocurrencies)
% \mahimna{cite more here}) 
require evaluation in a distributed fashion to avoid single points of failure. This motivates the study of secure multiparty computation (MPC) protocols for evaluating such symmetric-key primitives in a setting where inputs, outputs, and keys are secret-shared or distributed between two or more parties.

Towards this goal, a long line of work~\cite{damgard2010-aes, pinkas2009-aes, wang2017-mpc} has made substantial progress on concretely efficient MPC protocols for distributing the computation of symmetric primitives, such as AES or SHA-256, which are widely used in practice. Unfortunately, the constructions themselves were not designed with distributed evaluation in mind, and are often clunky to optimize for. More recent work (see~\cite{albrecht2015-lowmc, grassi2016-mpcfriendly, boneh2018-darkmatter,albrecht2019-feistal-mpc, aly2020-design-mpc} and references therein) has therefore proposed to start from scratch by designing \textit{MPC-friendly} primitives from the ground up. 
%So far, the primary focus has been on MPC-friendly PRFs. 
In this work, we continue this line of research by proposing a new suite of \textit{simple} MPC-friendly candidate designs for a number of symmetric primitives.

%: (weak)\footnote{A weak PRF is one whose security only holds when evaluated on {\em random} inputs. In many applications of strong PRF, a weak PRF can be used instead by first applying a hash function (modeled as a random oracle) to the input~\cite{someweakPRFref}.
%} PRFs, OWFs, and PRGs.


\paragraph{Our MPC setting.} We focus on the {\em semi-honest} setting of security for simplicity. This is considered adequate in many cases. In particular, it suffices for the construction of signature
schemes via an ``MPC-in-the-head'' technique~\cite{ishai2007-zkmpc,chase2017-picnic}.
While recent general techniques from the literature~\cite{BBCGI19,BGIN19} can be used to extend some of our protocols to the malicious security model with a low amortized cost, we leave such an extension to future work. We consider protocols for both two parties (2PC) and multiple parties, both with and without an honest majority assumption, and both with and without preprocessing. In the following, we consider by default the setting of (semi-honest) 2PC with preprocessing. However, our contributions apply to the other settings as well. 

\paragraph{Efficiency metrics for MPC.}   Concretely efficient MPC protocols can be divided into two broad categories: protocols based on {\em garbled circuits}~\cite{Yao} and protocols based on {\em linear secret sharing}~\cite{GMW,BGW,CCD}. Protocols based on garbled circuits have low round complexity but their communication cost will be prohibitively high for our purposes. We will therefore focus on protocols based on secret sharing. Roughly speaking, the complexity of evaluating  a given function $f$ using such protocols is determined by the {\em size} and the {\em depth} of a {\em circuit} $C$ that evaluates $f$.  Here we assume that $C$ is comprised of atomic gates of two kinds: {\em linear gates} (computing modular addition or multiplication by a public value) and MPC-friendly {\em nonlinear gates}  that are supported by efficient subprotocols. A typical example for a nonlinear gate is modular multiplication of two secret values. Given such a representation for $f$, the {\em communication} cost of an MPC protocol for $f$ scales linearly with the {\em size} of $C$, namely the number of gates weighted by the ``MPC cost'' of each gate, whereas the {\em round complexity} scales linearly with the {\em depth} of $C$, namely the number of gates on a longest input-output path. Since linear gates do not require any interaction, they do not count towards the size or the depth. We use the term ``nonlinear size'' and ``nonlinear depth'' to refer to the size and the depth when excluding linear gates. 

\paragraph{Our design criteria.}  The above efficiency metrics for MPC are quite crude, since not all kinds of nonlinear gates are the same. However, they still serve as a good intuitive guideline for the design of MPC-friendly primitives. More concretely, we would like to design primitives with the following goals in mind.

\begin{itemize}
    \item \textit{Low nonlinear depth}. Minimizing round complexity calls for minimizing nonlinear depth. Unfortunately, constructions like AES or even MPC-friendly ones such as LowMC~\cite{albrecht2015-lowmc}
    %and Rasta~\cite{cite} 
    have quite a high nonlinear depth, which leads to high-latency protocols when using the secret-sharing approach.
    
\item \textit{Small nonlinear size}.
 For keeping the communication complexity low, we would like to minimize the number of nonlinear gates and make them as ``small'' and ``MPC-friendly'' as possible. 

\item \textit{High algebraic degree}.
Security of block ciphers and (weak) PRFs provably requires high algebraic degree. While there are low-degree implementations of weaker primitives such as OWF and PRG~\cite{MQref,Goldreich,ApplebaumIK04}, these typically come at the price of bigger input size and higher nonlinear size~\cite{couteau2018-goldreich-prg}.
    
    \item \textit{Simplicity}.  A simple design is almost always easier to implement and prone to fewer errors and attacks. This is particularly valuable since a substantial amount of work has previously gone into implementations that resist timing and cache side-channels. Simple constructions are also easier to reason about and cryptanalyze, which builds confidence in their security, and may serve as interesting objects of study from a theory perspective~\cite{Goldreich,MilesV15,AkaviaBGKR14}.
\end{itemize}

\iffalse     
    Recent techniques for securely generating many instances of ``OT'' and ``VOLE'' correlations  with low communication and computation costs~\cite{BCGI18,BCGKIS19,more} motivate primitives that can benefit from these
    
    and can be generated in the \textit{offline} phase. This randomness is usually assumed to be given to the protocol parties by a trusted dealer. Several MPC protocols~\cite{?} have used a varying types correlated randomness to increase efficiency in the input-dependent \textit{online} phase.

    \hspace*{1em} Our final goal is to design candidates that take advantage of two special types of correlations: Obvious transfer (OT) correlations, and (vector) oblivious linear evaluation (OLE/VOLE) correlations. A major selling point for these correlations over other types is their ease of generation even without a trusted dealer. Recent work~\cite{?} for instance, has shown how pseudorandom correlation generators (PCGs) can be used to generate millions of OT and VOLE correlations with little to no communication. Furthermore, it is not necessary to store all of these correlations; they can stored in a compressed form, and expanded only when needed~\cite{?}.


    \hspace*{1em} In this paper, we strive to find the best of both worlds. The first design goal for our constructions therefore, is to achieve high nonlinearity for the overall computation while maintaining a low nonlinear \textit{depth}. Intuitively, this will enable us to achieve low round complexity (using a secret-sharing based approach) without the communication overhead of garbled circuits.

    \hspace*{1em} This means that although our protocols are in the preprocessing model, the correlations we require can be efficiently generated and compressed without a trusted dealer. Furthermore, since we also aim for small nonlinear sized constructions, the upshot of our design is that distributed protocols will be efficient even in an online-only setting. 
\fi

\paragraph{\bf The alternating moduli paradigm.} The above design goals seem inherently at odds with each other. How can ``high algebraic degree'' co-exist with ``small gates'' and ``low nonlinear depth''? Towards settling this apparent conflict, a new design paradigm was recently proposed by Boneh et al.~\cite{boneh2018-darkmatter} and further explored by Cheon et al.~\cite{cheon2020-adventures}. The idea is to break the computation into two or more parts, where each part includes a linear function over a {\em different modulus}. The simplest choice of moduli, which also seems to lead to the best efficiency, is 2 and 3. 

Boneh et al.~\cite{boneh2018-darkmatter} proposed a weak PRF\footnote{A weak PRF is one whose security only holds when evaluated on {\em random} inputs. In many applications of strong PRF, a weak PRF can be used instead by first applying a hash function (modeled as a random oracle) to the input} (wPRF) candidate with the following simple description: the input $x$ is a vector over  $\Z_2$ and the secret key specifies a matrix $\mat{K}$ over  $\Z_2$. The PRF first computes the matrix-vector product $\mat{K}x$ over  $\Z_2$, then interprets the result as a vector over $\Z_3$ in the natural way,  and finally applies a public, compressive linear mapping over $\Z_3$ to obtain an output vector $y$ over $\Z_3$.

The above mapping from $x$ and $\mat K$ to $y$ has two nonlinear steps: The first is the matrix-vector product over $\Z_2$, whose cost can be reduced when the matrix $\mat K$ has a special form. The second is a {\em conversion} of a mod-2 vector to a mod-3 vector, which consists of small  (finite-size) parallel nonlinear gates.  Overall, the nonlinear depth is 2. Why is this a high-degree function? Viewing both the input and the (binary representation of) the output as vectors over $\Z_2$, high degree over $\Z_2$ comes from the final linear mapping over $\Z_3$. Viewing the input as a vector over $\Z_3$, high degree comes from the linear mapping over $\Z_2$ defined by the key.  Despite its simplicity, the design can be conjectured to have a good level of security with small input and key size (say, 256 bits for 128-bit security). It mostly resisted the initial cryptanalysis, where attacks found in~\cite{cheon2020-adventures} require a very big number of samples and are quite easy to circumvent by slightly modifying the design (as suggested in~\cite{cheon2020-adventures}).

\iffalse
The main technique by which we achieve all three of our design goals is to consider alternating linear functions over different (prime) moduli. The only nonlinear part of the constructions will now be to switch between the different moduli, which leads to a very small round complexity. Furthermore, Boneh et al.~\cite{boneh2018-darkmatter} showed that linear functions over one moduli cannot be well approximated by low degree polynomials over a different moduli. Consequently, even a simple design that alternates between different moduli can result in a highly non-linear overall computation. In other words, this approach provides high algebraic degree with low nonlinear depth. 

As the simplest possible instance of this approach, we consider alternating linear functions over $\Z_2$ and $\Z_3$. Boneh et al. found that linear functions over $\Z_2$ are highly nonlinear over $\Z_3$ while linear functions over $\Z_3$ are highly nonlinear over $\Z_2$. Furthermore, using small moduli allows for efficient designs for ``switching'' between the moduli. In fact, we show how conversions between the moduli are “small” nonlinear gates and can be implemented efficiently from OT and VOLE correlations.
\fi

A primary motivation for the alternating moduli paradigm was its MPC-friendliness. Indeed, several MPC protocols were proposed in~\cite{boneh2018-darkmatter}. These protocols demonstrated significant efficiency advantages over earlier MPC-friendly designs, mainly in the setting of 2PC with preprocessing or 3-party computation with an honest majority. 

Another, very different, motivation is the goal of identifying simple function classes that are ``hard to learn.''  Indeed,  the conjectures from~\cite{boneh2018-darkmatter} imply hardness of learning results for low complexity classes such as (depth-2) $\mathsf{ACC}^0$ circuits, sparse $\Z_3$ polynomials, or width-3 branching programs. These conjectures are also of interest outside the field of cryptography~\cite{Chen19,FilmusIKK20,ChenR20,KabanetsKLMO20}, which further motivates cryptanalysis efforts.

\paragraph{Remaining challenges.} The initial works of~\cite{boneh2018-darkmatter,cheon2020-adventures} have only scratched the surface of the kind of questions one may ask. 
\begin{itemize}
\item
What about simpler symmetric primitives such as OWF and PRG? MPC protocols for these primitives are motivated by many applications, including Picnic-style post-quantum digital signatures~\cite{chase2017-picnic,kales2020-picnic} and lightweight distributed key generation for function secret sharing.
\item Are there similar candidates where the input, output, and key are all over $\Z_2$? This too is motivated by natural applications.
%\item Does reversing the order of mod-2 and mod-3 lead to a better design? 
\item Can the MPC protocols given in~\cite{boneh2018-darkmatter} be further improved? Can the preprocessing be realized at a low amortized cost? This motivates an additional design criterion: ``PCG-friendliness,'' leveraging recent advances in pseudorandom correlation generators~\cite{BCGI18,BCGIKRS19,yang2020-ferret}.
\end{itemize}

% All the candidates we propose are based on similar interplays between linear mappings in $\Z_2$ and $\Z_3$.
%\mahimna{write something about complexity theory implications}

\subsection{Our Contribution}


Motivated by the above questions, we make the following contributions.

\iffalse
\paragraph{\bf \textnormal{OT} and \textnormal{VOLE} friendliness}.
    MPC protocols in the preprocessing modelcan take advantage of input-independent correlated randomness to better handle nonlinear gates~\cite{Beaver,BDOZ,SPDZ,PowerOfCorrelatedRandomness,TinyTables,BGI19}. However, some kinds of correlated randomness are better than other. Recent work~\cite{?} for instance, has shown how pseudorandom correlation generators (PCGs) can be used to generate millions of ``OT'' and ``VOLE'' correlations with very low and computation costs. Furthermore, it is not necessary to store all of these correlations; they can stored in a compressed form, and expanded only when needed~\cite{?}. Roughly speaking, an OT correlation is useful for evaluating a ``simple'' nonlinear gates (such as Boolean AND) and a VOLE correlation is useful for scalar-vector products. We would like to maximize the potential benefit from these recent techniques.


\mahimna{still writing}
\fi


\subsubsection{New candidate constructions.}
We introduce several candidate constructions for OWF, PRG, and (weak) PRF, all based on alternation between linear mappings over $\Z_2$ and $\Z_3$.

\begin{itemize}
    \item \textbf{Candidate OWF.} 
    We expand on the general structure of the $(2,3)$-wPRF candidate from~\cite{boneh2018-darkmatter} to construct a candidate OWF. Recall that the wPRF candidate computes $\mat{B}(\mat{K}x)$ where $\mat{K}$ is the secret key (over $\Z_2$) and $\mat{B}$ is a compressive $\Z_3$ linear mapping. For our $(2,3)$-OWF candidate, we replace the secret key matrix with another randomly sampled (expanding) public matrix $\mat{A}$. Specifically, given $\mat{A} \in \Z_2^{m \times n}$ and $\mat{B} \in \Z_3^{t \times m}$ where $m \geq n,t$, our OWF candidate is defined as $\mathsf{F}(x) = \mat{B}(\mat{A}x)$ where $\mat{A}x$ is first reinterpreted as a 0/1 vector over $\Z_3$.

    \item \textbf{Candidate wPRF.}
    The wPRF candidate from~\cite{boneh2018-darkmatter} had inputs over $\Z_2$ but outputs over $\Z_3$. This is not suitable for applications in which the output should be further processed using secret sharing over $\Z_2$. To this end, we propose an ``LPN-style'' wPRF candidate where both the input and output are over $\Z_2$. Specifically, given a secret key matrix $\mat{K} \in \Z_2^{m \times n}$ and a public compressive mapping $\mat{B} \in \Z_2^{t \times m}$, for an input $x \in \Z_2^n$, our LPN-wPRF candidate first computes an intermediate vector
    \[
        w = [(\mat{K}x \bmod 2) + (\mat{K}x \bmod 3) \bmod 2] \bmod 2
    \]
    where for $\mat{K}x \bmod 3$, both $\mat{K}$ and $x$ are first reinterpreted over $\Z_3$. Then, the candidate is defined as $\mathsf{F}_\mat{K}(x) = \mat{B}w$. Intuitively, each intermediate vector bit can be thought of as a deterministic Learning-Parity-with-Noise (LPN) instance with a noise rate of 1/3. The noise is deterministically generated and is dependent on the input $x$ and a specific column of $\mat{A}$. A similar candidate was considered in~\cite{boneh2018-darkmatter} (as their alternate candidate) but it only outputs a single bit (it uses $\mat{K} \in \Z_2^{1 \times n}$ and outputs the intermediate vector directly). Our candidate generalizes this to multiple output bits. But more importantly, it also does not output the intermediate vector directly and instead applies an additional compressive linear mapping (using $\mat{B}$). We show how this allows our candidate to resist standard attacks on LPN.


    \item \textbf{Candidate PRG.}
    We also propose a candidate length-doubling PRG that is similar to our LPN-wPRF. Specifically, we use a public matrix $\mat{A} \in \Z_2^{m \times n}$ instead of the key for the first linear mappings. It follows the same structure as the LPN-wPRF, by first expanding the input to the intermediate vector $w$ and then applying a contracting $\Z_2$ linear mapping $\mat{B}$. Choosing $m$, i.e., the length of the intermediate vector, to be large enough, we can ensure that the final compressive mapping still results in an output of size $t = 2n$.

    % Let $\mat{K}$ be a secret key matrix in $\Z_2^{m \times n}$. For an input $x$, the $(2,3)$-wPRF candidate is defined as $\mathsf{F}_\mat{K}(x) = \mat{B}(\mat{K}x)$ where $\mat{B} \in \Z_3^{t \times m}$ is a public matrix and $w = \mat{K}x$ is first reinterpreted as a 0/1 vector over $\Z_3$. \yuval{Describe LPN-style PRFs here. Maybe make this the last bullet, so that we can start with the new stuff?}
\end{itemize}

\subsubsection{Cryptanalysis and implications on parameter choices.}
Given that the constructions heavily mix linear operations over $\Z_2$ and $\Z_3$,
we will rely on the arguments of Boneh et al.~\cite{boneh2018-darkmatter},
and conjecture that algebraic attacks do not threaten their security.
Instead, we will focus of combinatorial attacks and statistical tests.

Our most interesting attack on the
candidate OWF reduces the inversion problem to a particular type of subset-sum problem,
where addition simultaneously involves operations over $\Z_2$ and $\Z_3$.
Thus, we can invert the OWF by applying a variant of recent advanced algorithms
for solving subset-sum that are based on the
\emph{representation technique}~\cite{Howgrave-GrahamJ10,BeckerCJ11,BonnetainBSS20}.
Compared to a standard meet-in-the-middle approach,
this attack forced us to increase the parameters by about $30\%$.

For the candidate wPRF and PRG constructions,
the security analysis is mainly focused on
statistical distinguishers that exploit a bias in the output.
The strength of such a bias depends on the minimal distance of the
code generated by the second linear operation of the construction.
As this code is generated at random, we use the probabilistic method
(in a similar way it is used to obtain the Gilbert–Varshamov bound for linear codes)
to argue that its minimal distance is sufficiently large, except with negligible probability.


In Table~\ref{table:concrete} we summarize the recommended
concrete parameters for our constructions 
with the goal of obtaining $s$-bit security. For the \ttOWF and \ttwPRF constructions
we give both aggressive and more conservative parameter sets.
Note that the OWF and PRG use the minimal secret input (and output)
sizes, while for wPRFs, we use a larger secret. 
This is a result of different tradeoffs between security and performance.
For example, we could have set $n = s$ for the \ttwPRF, but cryptanalysis would
force setting $m$ to be much larger than $2s$ and result is less efficient protocols.
On the other hand, setting $n = 2s$ for the \ttOWF would also require 
doubling the size of the output,\footnote{
Otherwise, each output would have $2^s$ preimages and there would be no security advantage.}
once again, degrading efficiency.

Our constructions are new and it is reasonable that some of them would be broken
and require updating the parameter sets (even the ``conservative'' ones).
Conversely, if for some of our constructions the more aggressive parameter sets would not be broken, 
we would gain confidence in their security.

One of the main questions we leave open is how to better exploit
the structured matrices used in our constructions in cryptanalysis.
This question is particularly interesting for the wPRF constructions where
the attacker obtains several samples,
and can perhaps utilize the structured matrices to
combine their information in more efficient attacks.


\begin{table}[t]
\small
\begin{centering}
\begin{tabular}{l|l|l}
Construction    & Parameters             & Comment\\
                & $(n, m, t)$            &   \\\hline
\ttOWF          & $(s, 3.13s, s/\log 3)$ & aggressive \\
                & $(s, 3.53s, s/\log 3)$ & conservative \\ \hline
\ttwPRF         & $(2s, 2s, s/\log 3)$   & aggressive  \\
                & $(2s, 2.5s, s/\log 3)$ & conservative \\ \hline
LPN-PRG         & $(s, 3s, 2s)$          &              \\ \hline
LPN-PRF         & $(2s, 2s, s)$          &              \\ \hline
\end{tabular}
\caption{ \label{table:concrete} Concrete parameters for $s$-bit security.}
\end{centering}
\end{table}

\subsubsection{Distributed protocols and optimized implementations.}
 
\yuval{Start with a sharp statement on how we improve the online complexity of the wPRF protocols from~\cite{boneh2018-darkmatter}  (both communication and rounds). Give some concrete data points for fully distributed PRF, OPRF. Then explain our unified approach for achieving the improvement unified approach that applies also to the other primitives we design. This approach is explained below.}
 
\paragraph{Efficient protocols.} 
We provide efficient protocols for our wPRF candidate constructions in a variety of settings: 2PC with preprocessing, 3PC with one passive corruption, and Oblivious 2PC. For the \ttwPRF candidate, our protocols when compared to the ones from~\cite{boneh2018-darkmatter} in exactly the same setting perform roughly 1.5-5x better in both online communication and preprocessing size. Specifically, in the 2PC setting, our protocol requires 2 rounds, 1536 bits of online communication, and 662 bits of preprocessing. In contrast, the protocol from~\cite{boneh2018-darkmatter} for the same setting requires 4 rounds, roughly 2600 bits of online communication and roughly 3500 bits of preprocessing. Similarly, our OPRF protocol requires 2 rounds, 641 bits of online communication while the one from~\cite{boneh2018-darkmatter} requires 4 rounds and roughly 1800 bits of online communication.

% \paragraph{Efficient modulus conversion gates.}
The key ingredient which gives our protocols their efficiency is a subprotocol for modulus conversion gates that switch between shares in $\Z_2$ and $\Z_3$. While~\cite{boneh2018-darkmatter} used OT in their protocols, we use these modulus conversion gates for better efficiency. We note that these conversion gates can also be used to construct efficient distributed protocols for alternate variants of our constructions.

% Since our constructions use shares over both $\Z_2$ and $\Z_3$, it is necessary to be able to switch between the two. We provide the following protocols to do this efficiently. 
% \yuval{This may be slightly too low-level for intro. } \mahimna{fair point, we can just leave this paragraph here and remove the actual conversion gates. Just specify the actual costs}

\iffalse
\begin{itemize}
    \item \textbf{$\Z_2 \to \Z_3$ conversion.} Consider $x \in \Z_2^l$. Given a sharing $\share{x}$ over $\Z_2$, the goal is to convert to a sharing of the same $x$ but over $\Z_3$. For this, the dealer provides the parties with a shares of a random bit-vector $\tilde{x} \in \Z_2^l$ over $\Z_2$, and shares of the same vector $r = \tilde{x}$, but over $\Z_3$. The conversion protocol starts with the parties reconstructing $\hat{x} = x + \tilde{x}$. Now, a $\Z_3$ sharing of $x^* \in \Z_3^l; x^* = x$ can be computed as:
    \[
        \share{x^*} = \share{\hat{x}} + \share{r} + (\hat{x} \odot \share{r}) \bmod 3
    \]
    where $\odot$ denotes the Hadamard (element-wise) product modulo 3.

    \item \textbf{$\Z_3 \to \Z_2$ conversion.}
    Consider $x \in \Z_3^l$. Given a sharing $\share{x}$ over $\Z_3$, the goal is to convert to a sharing over $\Z_2$ of $x^* = x \bmod 2$. For this, the dealer provides the parties with the following correlated randomness. First it samples a random $\Z_3$ vector $\tilde{x}$ as a mask for $x$. It then computes $u = \tilde{x} \bmod 2$ and $v = (\tilde{x} + \textbf{1}^l \bmod 3) \bmod 2$ where $\textbf{1}^l$ is the all-ones vector of length $l$. Finally, the dealer provides the sharings $\share{x}$ (over $\Z_3$), $\share{u}$ (over $\Z_2$) and $\share{v}$ (over $\Z_2$) to the parties. The conversion protocol starts with the parties reconstructing $\hat{x} = x + \tilde{x}$. Now, each party computes its share of $x^*$ as follows: For each position $j \in [l]$, it sets its share $\share{x^*}_j$ as
    \[
    \share{x^*}_j = 
    \begin{cases*}
           1 - \share{u}_j - \share{v}_j  & \quad if $\hat{x}_j = 0$ \\
           \share{v}_j & \quad if $\hat{x}_j = 1$ \\
           \share{u}_j & \quad if $\hat{x}_j = 2$
    \end{cases*}
    \]
    This can also be thought of an optimized version of sharing a truth table for the function. We only require 2 bits (instead of 3) per element since the third bit of the truth table can be computed as a function of the other two.

\end{itemize}
\fi

\yuval{Include some data points comparing our protocols to other protocols, using TCC paper numbers. Point to table for concrete numbers. Explicitly compare to Legendre PRF. }

\paragraph{Distributing the dealer at a low amortized cost.} The 2PC protocols presented in~\cite{boneh2018-darkmatter} rely on trusted preprocessing to generate two kinds of correlated randomness. The first kind, used to securely multiply the input and the key matrix, can be thought of as a standard multiplication triple over a ring. (Using a circulant matrix for the key, this involves a single multiplication in a ring of polynomials over $\Z_2$.)  It was also pointed out that using efficient pseudorandom correlation generators (PCGs) for {\em vector oblivious-linear evaluation} (VOLE) correlations~\cite{BCGI18,BCGIKRS19,SchoppmannGR019}, this kind of correlation can be generated at a low amortized cost when the same key is reused with multiple inputs. In fact, using more recent PCGs for independent OLE correlations~\cite{boyle2020-lpn-pcg} the latter restriction can be removed, albeit at a considerably higher cost.  The second kind of correlated randomness used in~\cite{boneh2018-darkmatter} is a standard oblivious transfer (OT) correlation, which can also be efficiently generated using either classical~\cite{IKNP} or ``silent''~\cite{BCGIKRS19,yang2020-ferret} OT extension. The latter techniques use a PCG for OT to enable fast local generation of many random instances of OT from a pair of short, correlated seeds.  
However, the main source of improvement of our protocols over the ones from~\cite{boneh2018-darkmatter}  is our use of the {\em modulus conversion} correlations described above. We show how to generate both kinds of correlations from a standard OT correlation using only a {\em single} message, where in the $\Z_2 \to \Z_3$ case the (amortized) communication is $<1.38$ bits per instance, and in the (less commonly used) $\Z_3 \to \Z_2$ case it is $6$ bits per instance. This means that the amortized cost of distributing the dealer in our protocols is typically much lower than the cost of the online protocol that consumes the correlated randomness.


\subsubsection{Applications.}
\greg{As I noted in the applications section, will the two applications be the OPRF and signatures?}

\yuval{ I would add Steven's application of wPRF with input, key and output in $\Z_2$, and applications of distributing FSS key generation. I can take care of the latter.}

\paragraph{Digital signatures.} Using the MPC-friendliness of candidates, we can efficiently prove knowledge of
an input (e.g., of a OWF input, wPRF key or PRG seed), using proof protocols
based on the MPC-in-the-head paradigm~\cite{ishai2007-zkmpc}.  This is the approach
taken by many recently designed post-quantum signature
schemes~\cite{chase2017-picnic,katz2018-pqsigs,beullens2020-sigma-mq,beullens2020-legroast,guilhem2019-bbq,banquet},
as it only requires a secure OWF and hash function, and has opened up the range
of hardness assumptions possible for public-key signatures.  We present the
first public-key signature scheme based on alternating moduli cryptography. 

We provide a detailed description of a signature scheme using our OWF candidate,
as a modification to the Picnic
algorithm~\cite{chase2017-picnic,katz2018-pqsigs,kales2020-picnic,picnic-spec}, a
third round candidate in the NIST Post-Quantum Cryptography Standardization
Process.\footnote{See
\url{https://csrc.nist.gov/projects/post-quantum-cryptography/}.} We replace
the OWF in Picnic (an instance of the LowMC block cipher~\cite{albrecht2015-lowmc}),
update the MPC protocol accordingly, and quantify the resulting signature
sizes.  We find that signatures sizes are slightly shorter, with signatures at
the 128-bit security level (64-bit quantum) having size ranging from 10.3--13.3KB (depending
on parameter choice).  This
shows that OWFs based on alternating moduli are competitive with block-cipher
based designs, and we can choose a OWF with an (arguably) simpler mathematical
description, without sacrificing performance. 

\paragraph{Oblivious pseudorandom functions.} 
%\yuval{Literature pointers, compare with DDH-based (better computation, post-quantum), compare with Legendre-based.}
We construct an OPRF protocol that computes our \ttwPRF candidate in an oblivious setting. In the multi-input setting (where the key is used for multiple evaluations), our protocol requires only 2 rounds and 641 bits of online communication.
Compared to a standard DDH-based OPRF~\cite{jarecki2014-ddhoprf,jarecki2016-ddhoprf}, which require 512 bits of communication for $128$-bit security, our protocol requires slightly higher communication but has a much better performance. Our implementation shows that our OPRF protocol is faster than a \textit{single} scalar multiplication over the Curve25519 elliptic curve. Consequently, we expect our protocol to be faster than a number of OPRF protocols~\cite{freedman2005-oprf,jarecki2009-oprf} that are based on algebraic PRFs and require expensive exponentiations. In another direction, recent works~\cite{grassi2016-mpcfriendly,seres2021-legendre} construct an OPRF protocol from the Legendre PRF~\cite{damgard1988-legendre}. For 128-bit security and only a {\em single} output bit, the recent protocol from~\cite{seres2021-legendre} has online communication cost of $13$KB, substantially higher than ours (with 128 output bits), and with a higher computational cost.


% \paragraph{Hierarchical key derivation functions.} 
%\yuval{Motivates wPRF where inputs, outputs, and keys are over $\Z_2$. Explain the problem with using $(2,3)-wPRF$ by distinguishing between applications for which the output is given in the clear and ones where it needs to be kept secret and reused (such as this one).}

\paragraph{Secret output.} Our \ttwPRF candidate is well suited for applications that have privately held secret-shared inputs but require a public output that is delivered in the clear. However, it is insufficient for applications in which the output of the function needs to itself be kept secret and reused as the input to a subsequent PRF invocation.

One such common application arises in the context of deterministic signatures, which consists of generating a nonce by applying a PRF to the private key. In Schnorr and ECDSA, the nonce and a corresponding signature are sufficient to recover the private key. Thus, the nonce must also be distributed using the same secret-shared structure as the key. Distributed generation of deterministic signatures is once application that has both private input (the private key) and output (the nonce). Another example arises in the context of key derivation functions (KDFs), especially in a hierarchical structure, the output of the PRF may need to be used as an input (or even a key) for another evaluation of the PRF. A related application arises in the context of Bitcoin's BIP-32 derivation~\cite{bitcoin_bip0032}. Motivated by such applications, we propose our LPN-wPRF candidate which has both its input and output over $\Z_2$.

\paragraph{Distributed FSS key generation.} Function secret sharing (FSS)~\cite{boyle2015-fss} is a useful tool for many cryptographic applications; see~\cite{BoyleCGGIKR20,boyle2020-lpn-pcg} for recent examples. In many of these applications, two or more parties need to securely generate FSS keys, which in turn reduces to secure evaluation of a length-doubling PRG. Our LPN-style PRG candidate serves as a good basis for such protocols. In contrast to the black-box FSS key generation protocol of Doerner and shelat~\cite{DoernerS17}, its computational cost only scales logarithmically with the domain size. The optimal conjectured seed length of our PRG candidate ensures that FSS the key size is optimal as well.

%\yuval{First considered by Doerner-shelat in the context of 2PC for RAM programs. Then used for PCG seed generation and MPC for mixed-mode and fixed-point operations. Key ingredient: length-doubling PRG.}

\subsubsection{Future directions.} Our work leaves several interesting avenues for further work. One direction is designing MPC protocols with malicious security while minimizing the extra cost. Recent techniques from~\cite{BBCGI19,BGIN19} can be helpful towards this goal. Another direction is designing and analyzing other symmetric primitives based on the alternating moduli paradigm. Relevant examples include hash functions, strong PRFs, and block ciphers. 
%\yuval{Include a pointer to an appendix with Itai's preliminary results on the hash functions front?} 
In fact, a strong PRF candidate was already suggested in~\cite{boneh2018-darkmatter}.  Analyzing its concrete security is left for future work.


%\newpage
%

%!TEX root = ../main.tex

\section{Preliminaries}
\label{sec:preliminaries}

\mypara{Notation.} We start with some basic notation.
For a positive integer $k$, $[k]$ denotes the set $\{1,\dots, k\}$. $\Z_p$ denotes the ring of integers modulo $p$. We use bold uppercase letters (e.g., $\mat{A}, \mat{K}$) to denote matrices. We use $\textbf{0}^l$ and $\textbf{1}^l$ to denote the all zeros and the all ones vector respectively (of length $l$), and drop $l$ when sufficiently clear. For a vector $x$, by $x \bmod p$, we mean that each element in $x$ is taken modulo $p$. We use $x \getsr \mathcal{X}$ to denote sampling uniformly at random a set $\mathcal{X}$. $\Funcs[\inspace, \outspace]$ denotes the set of all functions from $\inspace$ to $\outspace$. $a \parallel b$ denotes concatenating the strings $a$ and $b$.

For distributed protocols with $N$ parties, we use $\parties = \{\party_1, \dots, \party_N\}$ to denote the set of parties. For a value $x$ in group $\mathbb{G}$, we use $\share{x}$ to denote an additive sharing of $x$ (in $\mathbb{G}$) among the protocol parties, and $\share{x}^{(i)}$ to denote the share of the $i^\thtext$ party. When clear from context (e.g., a local protocol for $\party_i$), we will often drop the superscript. When $\mathbb{G}' =  \mathbb{G}^l$ is a product group (e.g., $\Z_p^l$), for $x \in \mathbb{G}'$, we may also say that $\share{x}$ is a sharing \textit{over} $\mathbb{G}$, similar to the standard practice of calling $x$ a vector over $\mathbb{G}$.

For a $v \in \mathbb{G}$, we use $\randval{v}$ to denote a random mask sampled from the same group, and $\masked{v} = v + \randval{v}$ (where + is the group operation for $\mathbb{G}$) to denote $v$ masked by $\randval{v}$. We use the $+$ operator quite liberally and unless specified, it denotes the group operation (e.g., component-wise addition $\bmod~p$ for $\Z_p^l$) for the summands.


\smallskip
\noindent We now briefly recall standard symmetric primitives.

\begin{definition}[Weak Pseudorandom Function (wPRF)]
Let $\keyspace = \{\keyspace_\secparam\}_{\secparam \in \N}$, $\inspace = \{\inspace_\secparam\}_{\secparam \in \N}$, and $\outspace = \{\outspace_\secparam\}_{\secparam \in \N}$ be ensembles of finite sets indexed by a security parameter $\secparam$. Consider an efficiently computable function family $\{\mathsf{F}_\secparam\}_{\secparam \in \N}$ where each function is given by $\mathsf{F}_\secparam: \keyspace_\secparam \times \inspace_\secparam \to \outspace_\secparam$.
We say that $\{\mathsf{F}_\secparam\}_{\secparam \in \N}$ is an $(l, t, \varepsilon)$-weak pseudorandom function if for infinitely many $\secparam \in \N$ and all adversaries $\advA$ running in time at most $t(\secparam)$, the following holds: taking $f_\secparam \getsr \Funcs[\inspace_\secparam, \outspace_\secparam]$, $k \getsr \keyspace_\lambda$, and $x_1, \dots, x_l \getsr \inspace_\secparam$, we have that,
\[
    \left|\Pr\left[ \advA\left(1^\secparam, {\{ x_i, \mathsf{F}_\secparam(k,x_i)\}}_{i \in [l]} \right)\right]
    - \Pr\left[ \advA\left(1^\secparam, {\{ x_i, f_\secparam(x_i)\}}_{i \in [l]}\right)\right] \right| \leq \varepsilon(\secparam).
\]
\end{definition}


\begin{definition}[One-way Function (OWF)]
Let $\inspace = \{\inspace_\secparam\}_{\secparam \in \N}$, and $\outspace = \{\outspace_\secparam\}_{\secparam \in \N}$ be ensembles of finite sets indexed by a security parameter $\secparam$. Consider an efficiently computable function family $\{\mathsf{F}_\secparam\}_{\secparam \in \N}$ where each function is given by $\mathsf{F}_\secparam: \inspace_\secparam \to \outspace_\secparam$.
We say that $\{\mathsf{F}_\secparam\}_{\secparam \in \N}$ is a $(t, \varepsilon)$-one-way function if for infinitely many $\secparam \in \N$ and all adversaries $\advA$ running in time at most $t(\secparam)$, we have that,
\[
    \Pr\left[x \getsr \inspace; y \gets \mathsf{F}_\secparam(x) : \mathsf{F}_\secparam(\advA(1^{\abs{x}}, y)) = y \right] \leq \varepsilon(\secparam)
\]
\end{definition}

\begin{definition}[Pseudorandom Generator (PRG)]
Let $\inspace = \{\inspace_\secparam\}_{\secparam \in \N}$, and $\outspace = \{\outspace_\secparam\}_{\secparam \in \N}$ be ensembles of finite sets indexed by a security parameter $\secparam$. Consider an efficiently computable function family $\{\mathsf{F}_\secparam\}_{\secparam \in \N}$ where each function is given by $\mathsf{F}_\secparam: \inspace_\secparam \to \outspace_\secparam$.
We say that $\{\mathsf{F}_\secparam\}_{\secparam \in \N}$ is an $(l,t, \varepsilon)$-pseudorandom generator if $\mathsf{F}$ is length-expanding (i.e., $\forall \secparam, \forall x \in \inspace_\secparam$, $\len{x} < \len{\mathsf{F}_\secparam(x)}$) and for infinitely many $\secparam \in \N$ and all adversaries $\advA$ running in time at most $t(\secparam)$, the following holds: taking $x_1, \dots, x_l \getsr \inspace_\secparam$  $y_1, \dots, y_l \getsr \outspace_\secparam$, we have that,
\[
    \left|\Pr\left[ \advA\left(1^\secparam, {\{\mathsf{F}_\secparam(x_i)\}}_{i \in [l]} \right)\right]
    - \Pr\left[ \advA\left(1^\secparam, {\{y_i\}}_{i \in [l]}\right)\right] \right| \leq \varepsilon(\secparam).
\]
\end{definition}

%!TEX root = ../main.tex

\section{Candidate Constructions}
\label{sec:candidates}
In this section, we introduce our suite of candidate constructions for a number of cryptographic primitives: weak pseudo-random function families (wPRF), one-way functions (OWF), and pseudo-random generators (PRG). Our constructions are all based on similar interplays between mod-2 and mod-3 linear mappings. 

\subsection{Basic structure}
Given the wide range of candidates we propose, we find it useful to have a clean way to describe the operations that are performed in our candidate constructions. For this, we take inspiration from the basic formalism of the function secret sharing (FSS) approach to MPC with preprocessing, first introduced by Boyle, Gilboa, and Ishai~\cite{boyle2019-fss-preprocess}. Abstractly, the technique starts by representing an MPC functionality as a circuit, where each gate represents an operation to be performed in the distributed protocol. Then the inputs and outputs of each gate are secret shared and the gate operation is ``split'' using an FSS scheme~\cite{boyle2015-fss,boyle2016-fss-extension}. To evaluate the circuit in a distributed fashion, the dealer first shares a random mask for each input wire in the circuit, and possibly additional correlated randomness. Now, to compute a gate, the masked input is first revealed to all parties, who can then locally compute shares of the output wire or shares of the masked output. This technique can also be viewed as a generalization of the TinyTable protocol~\cite{damgard2017-tinytable}.

While we find it useful to use the formalism from~\cite{boyle2019-fss-preprocess} for representing the circuit to be computed, we do not explicitly require the FSS formalism for splitting the functionality of each gate. The individual operations are quite straightforward, and we instead chose to directly provide the distributed protocols that compute them. Further, by doing so, our protocols can make better use of correlated randomness to reduce the overall protocol cost as compared to the general techniques in~\cite{boyle2019-fss-preprocess}.


\paragraph{Circuit gates.}
We make use of just five types of basic operations, or ``gates,'' which we detail below. All our constructions can be succinctly represented using just these gates. In Section~\ref{sec:distributed_protocols}, we will provide distributed protocols to compute them. Local protocols for the first two gates follow directly from the homomorphic properties of (additive) secret sharing but we mention them for completeness, as well as to enable a pictorial representation of constructions that use them. To cleanly describe both our candidate constructions and their distributed protocols, the gates we describe here depart from the formalism in~\cite{boyle2019-fss-preprocess} in that the input values of the circuit itself are not secret shared. The MPC protocols will instead take in secret shared inputs as necessary and evaluate the circuit in a distributed fashion. We denote by $\gateset$, the set comprising of these gates.

\begin{itemize}
    \item \textbf{Mod-$p$ Public Linear Gate.}
    For a prime $p$, given a public matrix $\mat{A} \in \Z_p^{s \times l}$, the gate $\LMap^\mat{A}_p(\cdot)$ takes as input $x \in \Z_p^l$ and outputs $y = \mat{A}x \in \Z_p^s$.

    \item \textbf{Mod-$p$ Addition Gate.}
    For a prime $p$, the gate $\Add_p(\cdot,\cdot)$ takes input $x,x' \in \Z_p^{l}$ and outputs $y = x + x' \bmod p$.

    \item \textbf{Mod-$p$ Bilinear Gate.}
    For a prime $p$, and positive integers $s$ and $l$, the gate $\BLMap_p^{s,l}(\cdot, \cdot)$ takes as input a matrix $\mat{K} \in \Z_p^{s \times l}$ and a vector $x \in \Z_p^l$ and outputs $y = \mat{K}x \in \Z_p^s$. When clear from context, we will drop the superscript and simply write $\BLMap_p(\mat{K}, x)$.
    
    \item \textbf{$\Z_2 \to \Z_3$ conversion.} For a positive integer $l$, the gate $\Convert^l_{(2,3)}(\cdot)$ takes as input a vector $x \in \Z_2^l$ and returns its equivalent representation $x^*$ in $\Z_3^l$. When clear from context, we will drop the superscript and simply write $\Convert_{(2,3)}(x)$.

    \item \textbf{$\Z_3 \to \Z_2$ conversion.} For a positive integer $l$, the gate $\Convert^l_{(3,2)}(\cdot)$ takes as input a vector $x \in \Z_3^l$ and computes its mapping $x^*$ in $\Z_2^l$. For this, each $\Z_3$ element in $x$ is computed modulo 2 to get the corresponding $\Z_2$ element in the output $x^*$. Specifically, each $0$ and $2$ are mapped to $0$ while each $1$ is mapped to $1$. When clear from context, we will drop the superscript and simply write $\Convert_{(3,2)}(x)$.
\end{itemize}


\noindent Note that the difference between the $\LMap$ and the $\BLMap$ gates is seen in the context of their distributed protocols. For $\LMap$, the matrix $\mat{A}$ will be publicly available to all parties, while the input $x$ will be secret shared. On the other hand, for $\BLMap$, both the key $\mat{K}$ and the input $x$ will be secret shared. We call this gate \textit{bilinear} because its output is linear in both of its (secret-shared) inputs. Also note that although the $\Convert_{(2,3)}$ gate is effectively a no-op in a centralized evaluation, in the distributed setting, the gate will be used to convert an additive sharing over $\Z_2$ to an additive sharing over $\Z_3$.

As described previously, for the (bi)linear mappings, we focus only on constructions that use mod-2 and mod-3 mappings. In \fig~\ref{fig:graphical_gates}, we provide a pictorial representation for each circuit gate. We will connect these pieces together to also provide clean visual representations for all our constructions. 

The homomorphic properties of additive secret sharing directly imply that the gates $\LMap$ and $\textsf{Add}$ can be computed locally without any preprocessing or communication. For the other gates, we provide protocols to evaluate them in a distributed setting (i.e., where all inputs/outputs are secret shared) in Section~\ref{sec:distributed_protocols}.

%!TEX root = ../../main.tex

\begin{figure}[t]
\centering
\begin{tikzpicture}[scale=0.4]
    \begin{scope}[shift={(-15,0)}]
        \node [draw,rectangle,minimum width=1.6cm,minimum height=0.55cm,fill=blue!15] {$\mat{A}$};
        \node[shift={(0,-0.6)}] {$\LMap^\mat{A}_2$ gate};
        \node[shift={(-1,0)}] {$x$};
        \node[shift={(1,0)}] {$y$};

        \node [draw,rectangle,minimum width=1.6cm,minimum height=1.065cm,pattern=north west lines, pattern color=orange!70, shift={(0,-2)}] {$\mat{A}$};
        \node[shift={(0,-2.8)}] {$\LMap^\mat{A}_3$ gate};
        \node[shift={(-1,-2)}] {$x$};
        \node[shift={(1,-2)}] {$y$};

    \end{scope}
    \begin{scope}[shift={(0,0)}]
        \draw[fill=blue!15] (-2,0.68) -- (2,0.68) -- (2,-0.68) -- (-2,-0.68) -- (-2,-0.43) -- (1.75,-0.43) -- (1.75,0.43) -- (-2,0.43) -- (-2,0.68);

        \node[shift={(0,-0.6)}] {$\BLMap_2$ gate};
        \node[shift={(-1,0.25)}] {$\mat{A}$};
        \node[shift={(-1,-0.25)}] {$x$};
        \node[shift={(1,0)}] {$y$};


        \draw[pattern=north west lines, pattern color=orange!70, shift={(0,-5)}] (-2,1.3) -- (2,1.3) -- (2,-1.3) -- (-2,-1.3) -- (-2,-0.8) -- (1.5,-0.8) -- (1.5,0.8) -- (-2,0.8) -- (-2,1.3);

        \node[shift={(0,-2.8)}] {$\BLMap_3$ gate};
        \node[shift={(-1,-1.6)}] {$\mat{A}$};
        \node[shift={(-1,-2.4)}] {$x$};
        \node[shift={(1,-2)}] {$y$};

    \end{scope}   
    \begin{scope}[shift={(15,0)}]
        \node [draw,trapezium,trapezium left angle=65,trapezium right angle=65,minimum height=0.55cm,fill=blue!15,rotate=90] {};
        \node[shift={(0,-0.7)}] {$\Convert_{23}$ gate};
        \node[shift={(-0.5,0)}] {$x$};
        \node[shift={(0.6,0.05)}] {$x^*$};

        \node [draw,trapezium,trapezium left angle=65,trapezium right angle=65,minimum height=0.55cm,pattern=north west lines, pattern color=orange!70,rotate=-90, shift={(2,0)}] {};
        \node[shift={(0,-2.8)}] {$\Convert_{32}$ gate};
        \node[shift={(-0.5,-2)}] {$x$};
        \node[shift={(0.6,-1.95)}] {$x^*$};

    \end{scope}          
\end{tikzpicture}
\caption{Pictorial representations of the circuit gates.}
\label{fig:graphical_gates}
\end{figure}


\paragraph{Construction styles.}
The candidate constructions we introduce follow one of two broad styles which we detail below. A wPRF construction for the first style was first proposed by~\cite{boneh2018-darkmatter}. Here, we also propose a suite of symmetric primitives (e.g., OWFs, PRGs) with the same basic structure.
%!TEX root = ../main.tex

\begin{figure}[h]

\protbox{$(2,3)$-constructions}{

\textbf{Parameters.} Let $\secparam$ be the security parameter and define parameters $n, m, t$ as functions of $\secparam$ such that $m \geq n, m \geq t$.

\textbf{Public values.}
Let $\mat{A} \in \Z_2^{m \times n}$ and $\mat{B} \in \Z_3^{t \times m}$ be fixed public matrices chosen uniformly at random.

\begin{construction}[Mod-2/Mod-3 wPRF Candidate~\cite{boneh2018-darkmatter}]
The wPRF candidate is a family of functions $\PRFfunc : \Z_2^{m \times n} \times \Z_2^n \to \Z_3^t$ with key-space $\keyspace_\secparam = \Z_2^{m \times n}$, input space $\inspace_\secparam = \Z_2^n$ and output space $\outspace_\secparam = \Z_3^t$. For a key $\mat{K} \in \keyspace_\secparam$, we define $\sfF_{\mat{K}}(x) = \PRFfunc(\mat{K}, x)$ as follows:

\begin{enumerate}[topsep=0pt]
    \item On input $x \in \Z_2^n$, first compute $w = \BLMap_2(\mat{K}, x) = \mat{K}x$.
    \item Output $y = \LMap^{\mat{B}}_3\left(\Convert_{(2,3)}(w)\right)$. That is, view $w$ as a vector over $\Z_3$ and then output $y = \mat{B}w$. 
\end{enumerate}
\label{construction:23-central-wprf}
\end{construction}

\begin{center} %!TEX root = ../../main.tex

\scalebox{0.8}{
\begin{tikzpicture}
    \begin{scope}
        \draw[twostyle] (-0.74,0.65) -- (0.74,1.05) -- (0.74,-1.05) -- (-0.74,-0.65) -- (-0.74,-0.55) -- (0.5,-0.55) -- (0.5,0.55) -- (-0.74,0.55) -- cycle;
         \draw (-0.8,0.65) node[left] {$\mat{K}$};
         \draw (-0.8,-0.65) node[left] {$x$};
    \end{scope}
    \begin{scope}[shift={(0.95,0)}]
    \tikzmath{\sidelen=0.6;\maxnum=4;}
    \foreach \num in {0,...,\maxnum} {
        \tikzmath{\yshift=(\num-(\maxnum)/2)*(\sidelen + 0.3);}
        \node [draw,regular polygon, regular polygon sides=3,scale=0.5,shift={(0,\yshift)},rotate=90] {};
    }
    \end{scope}
    \begin{scope}[shift={(1.85,0)}]
    \node [draw,trapezium,minimum height=1.48cm,trapezium left angle=75,trapezium right angle=75, rotate=-90,threestyle] {\rotatebox{90}{$\mat{B}$}};
    \draw (0.75,0) node[right] {$y$};
    \end{scope}
\end{tikzpicture}
}
 \end{center}


\begin{construction}[Mod-2/Mod-3 OWF Candidate]
The OWF candidate is a function $\OWFfunc : \Z_2^n \to \Z_3^t$ with input space $\inspace_\secparam = \Z_2^n$ and output space $\outspace_\secparam = \Z_3^t$. We define $\sfF(x) = \OWFfunc(x)$ as follows:

\begin{enumerate}[topsep=0pt]
    \item On input $x \in \Z_2^n$, first compute $w = \LMap^{\mat{A}}_2(x) = \mat{A}x$.
    \item Output $y = \LMap^{\mat{B}}_3\left(\Convert_{(2,3)}(w)\right)$. That is, view $w$ as a vector over $\Z_3$ and then output $y = \mat{B}w$.
\end{enumerate}

\begin{center} %!TEX root = ../../main.tex

\scalebox{0.8}{
\begin{tikzpicture}
    \begin{scope}
     \node [draw,trapezium,minimum height=1.48cm,trapezium left angle=75,trapezium right angle=75, rotate=90,twostyle] {\rotatebox{-90}{$\mat{A}$}};
    \node[shift={(-1,0)}] {$x$};
    \end{scope}
    \begin{scope}[shift={(0.95,0)}]
    \tikzmath{\sidelen=0.6;\maxnum=4;}
    \foreach \num in {0,...,\maxnum} {
        \tikzmath{\yshift=(\num-(\maxnum)/2)*(\sidelen + 0.3);}
        \node [draw,regular polygon, regular polygon sides=3,scale=0.5,shift={(0,\yshift)},rotate=90] {};
    }
    \end{scope}
    \begin{scope}[shift={(1.85,0)}]
    \node [draw,trapezium,minimum height=1.48cm,trapezium left angle=75,trapezium right angle=75, rotate=-90,threestyle] {\rotatebox{90}{$\mat{B}$}};
    \draw (0.75,0) node[right] {$y$};
    \end{scope}

    \begin{scope}[shift={(0,-1.5)}]
        \draw (0,0) node[right] {\ttOWF};
    \end{scope}

\end{tikzpicture}
}
 \end{center}
\label{construction:23-owf}
\end{construction}
}
\caption{$(2,3)$-constructions}
\label{fig:primary-constructions}
\end{figure}
%!TEX root = ../main.tex

\begin{figure}[p]
\protbox{LPN-style-constructions}{

\textbf{Parameters.} $n, m, t$ are functions of the security parameter $\secparam$.

\textbf{Public values.}
Let $\mat{A} \in \Z_2^{m \times n}$ and $\mat{B} \in \Z_2^{t \times m}$ be fixed public matrices chosen uniformly at random. Alternatively, the matrices can also be chosen to be full-rank circulant matrices.

\begin{construction}[LPN-wPRF Candidate]
The \textnormal{LPN-wPRF} candidate is a family of functions $\PRFfunc : \Z_2^{m \times n} \times \Z_2^n \to \Z_2^t$ with key-space $\keyspace_\secparam = \Z_2^{m \times n}$, input space $\inspace_\secparam = \Z_2^n$ and output space $\outspace_\secparam = \Z_2^t$. For a key $\mat{K} \in \keyspace_\secparam$, we define $\sfF_{\mat{K}}(x) = \PRFfunc(\mat{K}, x)$ as follows:

\begin{enumerate}[topsep=0pt]
    \item On input $x \in \Z_2^n$, first compute $u = \BLMap_2(\mat{K}, x) = \mat{K}x$.
    \item Let $\mat{K}^* = \Convert_{(2,3)}(\mat{K})$ and $x^* = \Convert_{(2,3)}(x)$. Compute $v = \Convert_{(3,2)}(\BLMap_3(\mat{K}^*, x^*)) = \mat{K}^* x^* \bmod 2$. That is, compute $v = (\mat{K}x \bmod 3) \bmod 2$ where both $\mat{K}$ and $x$ are first reinterpreted over $\Z_3$.
    \item Compute $w = u \xor v$ and output $y = \LMap_2^{\mat{B}}(w)$.
\end{enumerate}
\label{construction:23-central-wprf}
\end{construction}
% \begin{center} %!TEX root = ../../main.tex

\scalebox{0.7}{
\begin{tikzpicture}
    \begin{scope}
    %  \node [draw,trapezium,minimum height=1.48cm,trapezium left angle=75,trapezium right angle=75, rotate=90,twostyle] {\rotatebox{-90}{$\mat{A}$}};
    % \node[shift={(-1,0)}] {$x$};
    \draw[twostyle] (-0.74,0.65) -- (0.74,1.05) -- (0.74,-1.05) -- (-0.74,-0.65) -- (-0.74,-0.55) -- (0.5,-0.55) -- (0.5,0.55) -- (-0.74,0.55) -- cycle;
    \draw (-0.8,0.65) node[left] {$\mat{K}$};
    \draw (-0.8,-0.65) node[left] {$x$};
    \end{scope}

    \begin{scope}[shift={(-0.9,-1.9)}]
    \tikzmath{\sidelen=0.6;\maxnum=0;}
    \foreach \num in {0,...,\maxnum} {
        \tikzmath{\yshift=(\num-(\maxnum)/2)*(\sidelen + 0.3);}
        \node [draw,regular polygon, regular polygon sides=3,scale=0.5,shift={(0,\yshift)},rotate=90] {};
    }
    \node[shift={(-0.4,0)}] {$\mat{K}$};
    \end{scope}

    \begin{scope}[shift={(-0.9,-3.1)}]
    \tikzmath{\sidelen=0.6;\maxnum=0;}
    \foreach \num in {0,...,\maxnum} {
        \tikzmath{\yshift=(\num-(\maxnum)/2)*(\sidelen + 0.3);}
        \node [draw,regular polygon, regular polygon sides=3,scale=0.5,shift={(0,\yshift)},rotate=90] {};
    }
    \node[shift={(-0.4,0)}] {$x$};
    \end{scope}


    \begin{scope}[shift={(0,-2.5)}]
        % \node [draw,trapezium,minimum height=1.48cm,trapezium left angle=75,trapezium right angle=75, rotate=90,threestyle] {\rotatebox{-90}{$\mat{A}$}};
        \draw[threestyle] (-0.74,0.65) -- (0.74,1.05) -- (0.74,-1.05) -- (-0.74,-0.65) -- (-0.74,-0.55) -- (0.5,-0.55) -- (0.5,0.55) -- (-0.74,0.55) -- cycle;
    \end{scope}

    \begin{scope}[shift={(0.9,-2.5)}]
    \tikzmath{\sidelen=0.6;\maxnum=4;}
    \foreach \num in {0,...,\maxnum} {
        \tikzmath{\yshift=(\num-(\maxnum)/2)*(\sidelen + 0.3);}
        \node [draw,regular polygon, regular polygon sides=3,scale=0.5,shift={(0,\yshift)},rotate=-90] {};
    }
    \end{scope}

    \begin{scope}[shift={(2.4,-1.25)}]
        \node[draw,circle, minimum size=1cm, twostyle] {};
        \draw (0,0.5) -- (0,-0.5);
        \draw (0.5,0) -- (-0.5,0);
        % \node[shift={(0.55,-0.55)}] {$2$};        

        \draw[->, -latex] (-1.6,0.4) -- (-0.37,0.4);

        \draw[->, -latex] (-1.35,-0.4) -- (-0.37,-0.4);


        \draw[->, -latex] (0.5,0) -- (1.5,0);

    \end{scope}

    \begin{scope}[shift={(1.45,-4)}]
        \draw (0,0) node[right] {\large LPN-wPRF};
    \end{scope}

    % \begin{scope}[shift={(0.95,0)}]
    % \tikzmath{\sidelen=0.6;\maxnum=4;}
    % \foreach \num in {0,...,\maxnum} {
    %     \tikzmath{\yshift=(\num-(\maxnum)/2)*(\sidelen + 0.3);}
    %     \node [draw,regular polygon, regular polygon sides=3,scale=0.5,shift={(0,\yshift)},rotate=90] {};
    % }
    % \end{scope}
    \begin{scope}[shift={(4.65,-1)}]
    \node [draw,trapezium,minimum height=1.48cm,trapezium left angle=75,trapezium right angle=75, rotate=-90,twostyle] {\rotatebox{90}{$\mat{B}$}};
    \draw (0.75,0) node[right] {$y$};
    \end{scope}
\end{tikzpicture}
}





% \scalebox{0.8}{
% \begin{tikzpicture}
%     % \begin{scope}[scale=1.6, shift={(-0.3,0)}]
%     %     % \draw[twostyle] (-0.74,0.65) -- (0.74,1.05) -- (0.74,-1.05) -- (-0.74,-0.65) -- (-0.74,-0.55) -- (0.5,-0.55) -- (0.5,0.55) -- (-0.74,0.55) -- cycle;
%     %     %  \draw (-0.8,0.65) node[left] {$\mat{K}$};
%     %     %  \draw (-0.8,-0.65) node[left] {$x$};
%     %     \draw[twostyle] (-0.74,1.05) -- (0.74,0.65) -- (0.74,-0.65) -- (-0.74,-1.05) -- (-0.74,-0.55) -- (0.5,-0.55) -- (0.5,0.55) -- (-0.74,0.55) -- cycle;
%     %     \draw (-0.8,0.8) node[left] {$\mat{K}$};
%     %     \draw (-0.8,-0.8) node[left] {$x$};

%     % \end{scope}
%      \begin{scope}
%         \draw[twostyle] (-0.74,0.65) -- (0.74,1.05) -- (0.74,-1.05) -- (-0.74,-0.65) -- (-0.74,-0.55) -- (0.5,-0.55) -- (0.5,0.55) -- (-0.74,0.55) -- cycle;
%          \draw (-0.8,0.65) node[left] {$\mat{K}$};
%          \draw (-0.8,-0.65) node[left] {$x$};
%         % \draw[twostyle] (-0.74,1.05) -- (0.74,0.65) -- (0.74,-0.65) -- (-0.74,-1.05) -- (-0.74,-0.55) -- (0.5,-0.55) -- (0.5,0.55) -- (-0.74,0.55) -- cycle;
%         % \draw (-0.8,0.8) node[left] {$\mat{K}$};
%         % \draw (-0.8,-0.8) node[left] {$x$};
%     \end{scope}
%     \begin{scope}[shift={(0.95,0)}]
%     \tikzmath{\sidelen=0.6;\maxnum=4;}
%     \foreach \num in {0,...,\maxnum} {
%         \tikzmath{\yshift=(\num-(\maxnum)/2)*(\sidelen + 0.3);}
%         \node [draw,regular polygon, regular polygon sides=3,scale=0.5,shift={(0,\yshift)},rotate=90] {};
%     }
%     \end{scope}
%     \begin{scope}[shift={(1.85,0)}]
%     \node [draw,trapezium,minimum height=1.48cm,trapezium left angle=75,trapezium right angle=75, rotate=-90,threestyle] {\rotatebox{90}{$\mat{B}$}};
%     \draw (0.75,0) node[right] {$y$};
%     \end{scope}
% \end{tikzpicture}
% }
 \end{center}

\begin{construction}[LPN-PRG Candidate]
The \textnormal{LPN-PRG} is a length-doubling PRG candidate defined as the function $\mathsf{F}_\secparam : \Z_2^n \to \Z_2^{2n}$ with input space $\inspace_\secparam = \Z_2^n$ and output space $\outspace_\secparam = \Z_2^{2n}$. For this construction, we consider the parameters $n,m,t$ with $m \geq n, t$ and $t = 2n$. We define $\mathsf{F}(x) = \mathsf{F}_\secparam$ as follows:
    \begin{enumerate}
       \item On input $x \in \Z_2^n$, first compute $u = \LMap_2(\mat{A}, x) = \mat{A}x$.
    \item Let $x^* = \Convert_{(2,3)}(x)$. Compute $v = \Convert_{(3,2)}(\LMap^{\mat{A}}_3(x^*)) =  (\mat{A} x^*) \bmod 2$. That is, compute $(\mat{A}x \bmod 3) \bmod 2$ where both $\mat{A}$ and $x$ are first reinterpreted over $\Z_3$.
    \item Compute $w = u \xor v$ and output $y = \LMap_2^{\mat{B}}(w)$.
    \end{enumerate}
\end{construction}
% \begin{center} %!TEX root = ../../main.tex

\scalebox{0.7}{
\begin{tikzpicture}
    \begin{scope}
     \node [draw,trapezium,minimum height=1.48cm,trapezium left angle=75,trapezium right angle=75, rotate=90,twostyle] {\rotatebox{-90}{$\mat{A}$}};
    \node[shift={(-1,0)}] {$x$};
    \end{scope}

    \begin{scope}[shift={(-0.95,-2.5)}]
    \tikzmath{\sidelen=0.6;\maxnum=2;}
    \foreach \num in {0,...,\maxnum} {
        \tikzmath{\yshift=(\num-(\maxnum)/2)*(\sidelen + 0.3);}
        \node [draw,regular polygon, regular polygon sides=3,scale=0.5,shift={(0,\yshift)},rotate=90] {};
    }
    \node[shift={(-0.4,0)}] {$x$};
    \end{scope}

    \begin{scope}[shift={(0,-2.5)}]
        \node [draw,trapezium,minimum height=1.48cm,trapezium left angle=75,trapezium right angle=75, rotate=90,threestyle] {\rotatebox{-90}{$\mat{A}$}};
    \end{scope}

    \begin{scope}[shift={(0.9,-2.5)}]
    \tikzmath{\sidelen=0.6;\maxnum=4;}
    \foreach \num in {0,...,\maxnum} {
        \tikzmath{\yshift=(\num-(\maxnum)/2)*(\sidelen + 0.3);}
        \node [draw,regular polygon, regular polygon sides=3,scale=0.5,shift={(0,\yshift)},rotate=-90] {};
    }
    \end{scope}

    \begin{scope}[shift={(2.4,-1.25)}]
        \node[draw,circle, minimum size=1cm, twostyle] {};
        \draw (0,0.5) -- (0,-0.5);
        \draw (0.5,0) -- (-0.5,0);

        \draw[->, -latex] (-1.6,0.4) -- (-0.37,0.4);
        \draw[->, -latex] (-1.35,-0.4) -- (-0.37,-0.4);

        \draw[->, -latex] (0.5,0) -- (1.5,0);

    \end{scope}

    \begin{scope}[shift={(4.65,-1)}]
    \node [draw,trapezium,minimum height=1.48cm,trapezium left angle=75,trapezium right angle=75, rotate=-90,twostyle] {\rotatebox{90}{$\mat{B}$}};
    \draw (0.75,0) node[right] {$y$};
    \end{scope}
    
    \begin{scope}[shift={(1.45,-4)}]
        \draw (0,0) node[right] {\large LPN-PRG};
    \end{scope}


\end{tikzpicture}
}
 \end{center}

\medskip

\begin{minipage}{0.5\textwidth}
\begin{center} %!TEX root = ../../main.tex

\scalebox{0.7}{
\begin{tikzpicture}
    \begin{scope}
    %  \node [draw,trapezium,minimum height=1.48cm,trapezium left angle=75,trapezium right angle=75, rotate=90,twostyle] {\rotatebox{-90}{$\mat{A}$}};
    % \node[shift={(-1,0)}] {$x$};
    \draw[twostyle] (-0.74,0.65) -- (0.74,1.05) -- (0.74,-1.05) -- (-0.74,-0.65) -- (-0.74,-0.55) -- (0.5,-0.55) -- (0.5,0.55) -- (-0.74,0.55) -- cycle;
    \draw (-0.8,0.65) node[left] {$\mat{K}$};
    \draw (-0.8,-0.65) node[left] {$x$};
    \end{scope}

    \begin{scope}[shift={(-0.9,-1.9)}]
    \tikzmath{\sidelen=0.6;\maxnum=0;}
    \foreach \num in {0,...,\maxnum} {
        \tikzmath{\yshift=(\num-(\maxnum)/2)*(\sidelen + 0.3);}
        \node [draw,regular polygon, regular polygon sides=3,scale=0.5,shift={(0,\yshift)},rotate=90] {};
    }
    \node[shift={(-0.4,0)}] {$\mat{K}$};
    \end{scope}

    \begin{scope}[shift={(-0.9,-3.1)}]
    \tikzmath{\sidelen=0.6;\maxnum=0;}
    \foreach \num in {0,...,\maxnum} {
        \tikzmath{\yshift=(\num-(\maxnum)/2)*(\sidelen + 0.3);}
        \node [draw,regular polygon, regular polygon sides=3,scale=0.5,shift={(0,\yshift)},rotate=90] {};
    }
    \node[shift={(-0.4,0)}] {$x$};
    \end{scope}


    \begin{scope}[shift={(0,-2.5)}]
        % \node [draw,trapezium,minimum height=1.48cm,trapezium left angle=75,trapezium right angle=75, rotate=90,threestyle] {\rotatebox{-90}{$\mat{A}$}};
        \draw[threestyle] (-0.74,0.65) -- (0.74,1.05) -- (0.74,-1.05) -- (-0.74,-0.65) -- (-0.74,-0.55) -- (0.5,-0.55) -- (0.5,0.55) -- (-0.74,0.55) -- cycle;
    \end{scope}

    \begin{scope}[shift={(0.9,-2.5)}]
    \tikzmath{\sidelen=0.6;\maxnum=4;}
    \foreach \num in {0,...,\maxnum} {
        \tikzmath{\yshift=(\num-(\maxnum)/2)*(\sidelen + 0.3);}
        \node [draw,regular polygon, regular polygon sides=3,scale=0.5,shift={(0,\yshift)},rotate=-90] {};
    }
    \end{scope}

    \begin{scope}[shift={(2.4,-1.25)}]
        \node[draw,circle, minimum size=1cm, twostyle] {};
        \draw (0,0.5) -- (0,-0.5);
        \draw (0.5,0) -- (-0.5,0);
        % \node[shift={(0.55,-0.55)}] {$2$};        

        \draw[->, -latex] (-1.6,0.4) -- (-0.37,0.4);

        \draw[->, -latex] (-1.35,-0.4) -- (-0.37,-0.4);


        \draw[->, -latex] (0.5,0) -- (1.5,0);

    \end{scope}

    \begin{scope}[shift={(1.45,-4)}]
        \draw (0,0) node[right] {\large LPN-wPRF};
    \end{scope}

    % \begin{scope}[shift={(0.95,0)}]
    % \tikzmath{\sidelen=0.6;\maxnum=4;}
    % \foreach \num in {0,...,\maxnum} {
    %     \tikzmath{\yshift=(\num-(\maxnum)/2)*(\sidelen + 0.3);}
    %     \node [draw,regular polygon, regular polygon sides=3,scale=0.5,shift={(0,\yshift)},rotate=90] {};
    % }
    % \end{scope}
    \begin{scope}[shift={(4.65,-1)}]
    \node [draw,trapezium,minimum height=1.48cm,trapezium left angle=75,trapezium right angle=75, rotate=-90,twostyle] {\rotatebox{90}{$\mat{B}$}};
    \draw (0.75,0) node[right] {$y$};
    \end{scope}
\end{tikzpicture}
}





% \scalebox{0.8}{
% \begin{tikzpicture}
%     % \begin{scope}[scale=1.6, shift={(-0.3,0)}]
%     %     % \draw[twostyle] (-0.74,0.65) -- (0.74,1.05) -- (0.74,-1.05) -- (-0.74,-0.65) -- (-0.74,-0.55) -- (0.5,-0.55) -- (0.5,0.55) -- (-0.74,0.55) -- cycle;
%     %     %  \draw (-0.8,0.65) node[left] {$\mat{K}$};
%     %     %  \draw (-0.8,-0.65) node[left] {$x$};
%     %     \draw[twostyle] (-0.74,1.05) -- (0.74,0.65) -- (0.74,-0.65) -- (-0.74,-1.05) -- (-0.74,-0.55) -- (0.5,-0.55) -- (0.5,0.55) -- (-0.74,0.55) -- cycle;
%     %     \draw (-0.8,0.8) node[left] {$\mat{K}$};
%     %     \draw (-0.8,-0.8) node[left] {$x$};

%     % \end{scope}
%      \begin{scope}
%         \draw[twostyle] (-0.74,0.65) -- (0.74,1.05) -- (0.74,-1.05) -- (-0.74,-0.65) -- (-0.74,-0.55) -- (0.5,-0.55) -- (0.5,0.55) -- (-0.74,0.55) -- cycle;
%          \draw (-0.8,0.65) node[left] {$\mat{K}$};
%          \draw (-0.8,-0.65) node[left] {$x$};
%         % \draw[twostyle] (-0.74,1.05) -- (0.74,0.65) -- (0.74,-0.65) -- (-0.74,-1.05) -- (-0.74,-0.55) -- (0.5,-0.55) -- (0.5,0.55) -- (-0.74,0.55) -- cycle;
%         % \draw (-0.8,0.8) node[left] {$\mat{K}$};
%         % \draw (-0.8,-0.8) node[left] {$x$};
%     \end{scope}
%     \begin{scope}[shift={(0.95,0)}]
%     \tikzmath{\sidelen=0.6;\maxnum=4;}
%     \foreach \num in {0,...,\maxnum} {
%         \tikzmath{\yshift=(\num-(\maxnum)/2)*(\sidelen + 0.3);}
%         \node [draw,regular polygon, regular polygon sides=3,scale=0.5,shift={(0,\yshift)},rotate=90] {};
%     }
%     \end{scope}
%     \begin{scope}[shift={(1.85,0)}]
%     \node [draw,trapezium,minimum height=1.48cm,trapezium left angle=75,trapezium right angle=75, rotate=-90,threestyle] {\rotatebox{90}{$\mat{B}$}};
%     \draw (0.75,0) node[right] {$y$};
%     \end{scope}
% \end{tikzpicture}
% }
 \end{center}
\end{minipage}
\begin{minipage}{0.5\textwidth}
\begin{center} %!TEX root = ../../main.tex

\scalebox{0.7}{
\begin{tikzpicture}
    \begin{scope}
     \node [draw,trapezium,minimum height=1.48cm,trapezium left angle=75,trapezium right angle=75, rotate=90,twostyle] {\rotatebox{-90}{$\mat{A}$}};
    \node[shift={(-1,0)}] {$x$};
    \end{scope}

    \begin{scope}[shift={(-0.95,-2.5)}]
    \tikzmath{\sidelen=0.6;\maxnum=2;}
    \foreach \num in {0,...,\maxnum} {
        \tikzmath{\yshift=(\num-(\maxnum)/2)*(\sidelen + 0.3);}
        \node [draw,regular polygon, regular polygon sides=3,scale=0.5,shift={(0,\yshift)},rotate=90] {};
    }
    \node[shift={(-0.4,0)}] {$x$};
    \end{scope}

    \begin{scope}[shift={(0,-2.5)}]
        \node [draw,trapezium,minimum height=1.48cm,trapezium left angle=75,trapezium right angle=75, rotate=90,threestyle] {\rotatebox{-90}{$\mat{A}$}};
    \end{scope}

    \begin{scope}[shift={(0.9,-2.5)}]
    \tikzmath{\sidelen=0.6;\maxnum=4;}
    \foreach \num in {0,...,\maxnum} {
        \tikzmath{\yshift=(\num-(\maxnum)/2)*(\sidelen + 0.3);}
        \node [draw,regular polygon, regular polygon sides=3,scale=0.5,shift={(0,\yshift)},rotate=-90] {};
    }
    \end{scope}

    \begin{scope}[shift={(2.4,-1.25)}]
        \node[draw,circle, minimum size=1cm, twostyle] {};
        \draw (0,0.5) -- (0,-0.5);
        \draw (0.5,0) -- (-0.5,0);

        \draw[->, -latex] (-1.6,0.4) -- (-0.37,0.4);
        \draw[->, -latex] (-1.35,-0.4) -- (-0.37,-0.4);

        \draw[->, -latex] (0.5,0) -- (1.5,0);

    \end{scope}

    \begin{scope}[shift={(4.65,-1)}]
    \node [draw,trapezium,minimum height=1.48cm,trapezium left angle=75,trapezium right angle=75, rotate=-90,twostyle] {\rotatebox{90}{$\mat{B}$}};
    \draw (0.75,0) node[right] {$y$};
    \end{scope}
    
    \begin{scope}[shift={(1.45,-4)}]
        \draw (0,0) node[right] {\large LPN-PRG};
    \end{scope}


\end{tikzpicture}
}
 \end{center}
\end{minipage}


}
\caption{LPN-style-constructions}
\label{fig:lpn-constructions}
\end{figure}


\begin{itemize}
    \item \textbf{$(p,q)$-constructions.}
    For distinct primes $p, q$, the $(p,q)$-constructions have the following structure: On an input $x$ over $\Z_p$, first a linear $\bmod~p$ mapping is applied, followed by a linear $\bmod~q$ mapping. Note that after the $\bmod~p$ mapping, the input is first reinterpreted as a vector over $\Z_q$. For unkeyed primitives (e.g., OWF), both mappings are public, while for keyed primitives (e.g., wPRF), the key is used for the first linear mapping. The construction is parameterized by positive integers $n, m, t$ (functions of the security parameter $\secparam$) denoting the length of the input vector (over $\Z_p$), the length of the intermediate vector, and the length of the output vector (over $\Z_q$) respectively. The two linear mappings can be represented by matrices $\mat{A} \in \Z_p^{m \times n}$ and $\mat{B} \in \Z_q^{t \times m}$. For keyed primitives, the key $\mat{K} \in \Z_p^{m \times n}$ will be used instead of $\mat{A}$.

    \hspace*{1em} Concretely, given an input $x \in \Z_p^n$, the construction output is of the form $y = \mat{B}w \in \Z_q^t$ where $w = \mat{A}{x}$ is first viewed over $\Z_q$. In this paper, we will analyze this style of construction for $(p,q) = (2,3)$ and $(3,2)$ since these are arguably the simplest constructions that employ linear mappings over alternate moduli. We find that the $(2,3)$-constructions outperform the $(3,2)$-constructions and we will primarily use the former style for our constructions.

    % \greg{Why did we choose to focus on (2,3), (3,2)?}
    % \itai{At a high level the standard conversion from mod 3 to mod 2 loses information and is less efficient.
    % The LPN-style construction can be viewed as a variant of (3,2) that does not lose information.}
    % \greg{But why not (5,7) or (17, 101), etc. I think a sentence explaining why these two primes would help.}
    
    \item \textbf{LPN-style-constructions.}
    These constructions have the following general structure: On input $x$ over $\Z_2$, first a linear $\bmod~2$ mapping given by the matrix $\mat{A}$ is applied to obtain~$u$. Concurrently, the same linear mapping is also applied over $\Z_3$ (where both $x$ and $\mat{A}$ are now reinterpreted over $\Z_3$) and then reduced modulo $2$ to obtain $v$. The sum $w = u \oplus v$ is then multiplied by a second linear mapping (given by $\mat{B}$) over $\Z_2$. The mapping $\mat{B}$ is always public, while for keyed primitives, the key $\mat{K}$ is used instead of $\mat{A}$.

    \hspace*{1em} The construction is parameterized by positive integers $n,m,t$ (as functions of the security parameter $\secparam$) denoting the size of the input vector, the intermediate vector(s), and the output vector (all over $\Z_p$).
    Concretely, given $\mat{A} \in \Z_2^{m \times n}$ and a public $\mat{B} \in \Z_2^{t \times m}$, for an input $x \in \Z_2^n$, the construction first computes the intermediate vector:
   \[
        w = \left[(\mat{A}x \bmod 2) + (\mat{A}x \bmod 3) \bmod 2\right] \bmod 2.
    \]
    The output $y$ is then computed as $y = \mat{B}w \bmod 2$. The upshot of this style is that the input and the output are both over $\Z_2$. Intuitively, each intermediate vector bit can be thought of as a deterministic Learning-Parity-with-Noise (LPN) instance with a noise rate of $1/3$. The noise is deterministically generated and is dependent on the input $x$ and a specific column of $\mat{A}$. The noise for the $i^\thtext$ instance will be $1$ if and only if $\left<\mat{A}_i, x\right> = 1$.

    \hspace*{1em} A similar construction was considered in~\cite{boneh2018-darkmatter} but only for a single-bit output. Specifically, they considered $\mat{A} \in \Z_2^{1 \times n}$ and output the single bit $w$. In our construction, we additionally apply a compressive linear map (using $\mat{B}$) to get the final output. This is done to resist standard attacks on LPN (see Section~\ref{sec:cryptanalysis} and Appendix~\ref{appendix:cryptanalysis} for details).
\end{itemize}


\paragraph{Winning candidates.}
Through cryptanalysis and considering the cost for each candidate (See Sections~\ref{sec:cryptanalysis} and \ref{sec:distributed_protocols} for details), we find that some of our candidates are more suited (i.e., ``win'') for a particular setting.
\iffull
For instance, we found that the $(3,2)$-constructions are overshadowed by the corresponding $(2,3)$-constructions in all settings we consider. Specifically, out of the candidates we consider, we found the following:
\begin{enumerate}
    \item \ttwPRF is the best wPRF candidate with no restriction for input and output space. 
    \item LPN-wPRF is the best wPRF candidate where both the input and output need to be over $\Z_2$.
    \item \ttOWF is the best OWF candidate with no restriction for input and output space.
    \item LPN-PRG is the best PRG candidate.
\end{enumerate}
\else
Specifically, out of the candidates we consider, we find the following: \ttwPRF and \ttOWF are the best wPRF / OWF candidates with no restriction on the input/output space. LPN-wPRF is the best wPRF candidate when the input and output space are over $\Z_2$. LPN-PRG is the best PRG candidate.
\fi

We provide formal and pictorial descriptions of our winning candidates in Figures~\ref{fig:primary-constructions} and~\ref{fig:lpn-constructions}.

\iffull
\paragraph{Circuit description of constructions.}
We can represent our constructions by circuits consisting of the gates described previously. This approach is similar to that of~\cite{boyle2019-fss-preprocess}. We formally define computation circuit representations for our constructions in Definition~\ref{def:computation_circuit}.

\begin{definition}[Computation circuit]
A computation circuit $C$ with input space $\Gin = \prod \Gin_i$ and output space $\Gout = \prod \Gout_i$ is a (labeled) directed acyclic graph $(\mathcal{V},\mathcal{E})$ where $\mathcal{V}$ denotes the set of vertices and $\mathcal{E}$ denotes the set of edges according to the following:
\begin{itemize}

\item Each source vertex corresponds to exactly one $\Gin_i$ and vice versa. The label for the vertex is the identity function on the corresponding $\Gin_i$. Each sink vertex corresponds to exactly one $\Gout_i$ and vice versa. The label for the vertex is the identity function on the corresponding $\Gout_i$. Each non-source $V \in \mathcal{V}$ is labeled with a gate $\mathcal{G}_V \in \gateset$ that computes the function $\mathcal{G}_V: \mathbb{G}^{\textsf{in}}_V \to \mathbb{G}^{\textsf{out}}_V$. The depth of a vertex $V \in \mathcal{V}$, denoted by $\textsf{depth}(V)$ is the length of the largest directed path from a source vertex to $V$.


\item For an edge $(V_a, V_b)$, let $\Gout_{V_a} = \prod \Gout_{V_a,i}$ and $\Gin_{V_b} = \prod \Gin_{V_b, i}$. Then, there exists indices $j$ and $k$ such that $\Gout_{V_a,j} = \Gin_{V_b,k}$. Further, for each input $\Gin_{V_b,i}$ for $V_b$, there is some edge $(V_c, V_b)$ that satisfies the above.

\item The evaluation of the gate for vertex $V$ on input $x \in \Gin_V$ is defined as $y = \mathcal{G}_V(x)$. The evaluation of the circuit $C$, denoted by $\textsf{Eval}_C(x)$, where $x \in \Gin$ is the value $y \in \Gout$, that is obtained by recursively evaluating each gate function in the circuit.
\end{itemize}


Let $\mathsf{F} = \{\mathsf{F}_\secparam\}_{\secparam \in \N}$ denote a family of functions $\mathsf{F}_\secparam:\mathcal{X}_\secparam \to \mathcal{Y}_\secparam$. We say that $\{C_\secparam\}_{\secparam \in \N}$ is a family of computation circuits for $\mathsf{F}$ if all $C_\secparam$ have the same topological structure, and for all $\secparam \in \N$, $\mathsf{F}_\secparam(x) = \textsf{Eval}_C(x)$ for all $x \in \mathcal{X}_\secparam$.
\label{def:computation_circuit}
\end{definition}
\fi

\paragraph{Structured keys.}
The constructions we described previously use general matrices in, e.g., $\Z_p^{m \times n}$. For keyed primitives, this results in a key size of $mn$ elements of $\Z_p$ which is expensive to communicate within distributed protocols. Therefore, we will instead take advantage of structured matrices whose representation is only linear in $n$ and $m$. Since both $n$ and $m$ are $O(\secparam)$ in our constructions, this reduces the communication complexity from quadratic to linear in $\secparam$. Furthermore, some structured matrices also benefit from asymptotically faster algorithms (e.g., FFT-based) for matrix multiplications and matrix-vector products. We briefly describe the types of structured matrices we utilize below. For this, consider a matrix $\mat{M} \in \Z_p^{m \times n}$.

\begin{itemize}
    \item (Toeplitz matrices).
    A Toeplitz matrix, or a diagonal-constant matrix, is a matrix where each diagonal from left to right is constant. Specifically, $\mat{M}$ is Toeplitz if for all $i \in [m]; j \in [n]$, it holds that $M_{i,j} = M_{i+1, j+1}$ where $M_{i,j}$ denotes the element in row $i$ and column $j$ of $\mat{M}$. This means that a Toeplitz matrix can be represented by a single column and a single row, i.e., with $n + m - 1$ field elements.

    \item (Generalized circulant matrices). 
    A generalized circulant matrix is a matrix where each row after the first, is a cyclic rotation of the first row. Specifically, if the first row of generalized circulant matrix $\mat{M}$ is the vector $(a_1, \dots, a_n)$, then the $m^\thtext$ row of $\mat{M}$ will be given by the same vector cyclically rotated $m-1$ times. In general, $m \neq n$, but the special case of $m = n$ is called a (square) circulant matrix. Unless specified, for brevity, we will often use the term \textit{circulant} to denote either generalized circulant matrices or the more specific (square) circulant matrices. This will not matter for our setting, since both can be efficiently represented using just $n$ field elements (given the dimension of the matrix).
\end{itemize}

\iffull
We will usually instantiate our constructions using generalized circulant matrices to take advantage of their efficient representations. However, care must be taken while adding structure since this could potentially reduce the security of a construction. This is particularly relevant in case of multi-element outputs since elements in the intermediate vectors will be correlated. For example, in the LPN-style construction, the noise for each deterministic LPN instance will be correlated since two columns of a circulant matrix are just cyclic rotations. 

Our cryptanalysis in Section~\ref{sec:cryptanalysis} will therefore consider our constructions with structured matrices. Of course, general $m \times n$ matrices can also be used for all our constructions (with at least equal security) but this could result in substantially more communication costs. In many cases, it is also helpful to choose the matrices to be full-rank (see Section~\ref{sec:cryptanalysis} for details).
\else
We will usually instantiate our constructions using generalized circulant matrices to take advantage of their efficient representations. However, care must be taken while adding structure since this could potentially damage the security of a construction. Our cryptanalysis in Section~\ref{sec:cryptanalysis} will therefore consider our constructions with structured matrices.
\fi


% \greg{This section said earlier we were interested in (3,2) primitives as well, but there is isn't a figure for those, only (2,3).
% Did any of the applications end up finding (3,2) useful? }

% \greg{We have (2,3) wPRF and OWF, then LPN wPRF and PRG.  Should we comment on
% the relations between various primitives (e.g,. the LPN PRG is also an LPN
% OWF),  and point to generic constructions for, e.g., a (2,3)-PRG, which could
% be constructed from the (2,3)-OWF. Or maybe a forward reference to some of the
% applications to justify this particular set of primitives.  }

% \greg{Is there a generic wPRF to PRF construction that is efficient here? The
% one I know: PRF(x) := wPRF(RO(x)) is probably not helpful because RO is not
% MPC-friendly. Or maybe we should say why we chose wPRFs.  }


%!TEX root = ../main.tex

\newpage~

\section{Cryptanalysis}
\label{sec:cryptanalysis}

We give a summary of cryptanalysis of our constructions,
focusing on the main attacks that influence our choice of parameters.
The details are found in Appendix~\ref{appendix:cryptanalysis}.

%In general, for obtaining $s$-bit security given that the adversary performs $2^{\tau}$ work,
%we require that the success probability of key recover attacks is bounded by $2^{\tau - s}$,
%while the advantage of distinguishing attacks (when applicable) 
%is bounded by $2^{(\tau - s)/2}$. 
%This is consistent with the work of Micciancio and Walter~\cite{Micciancio018}.

\subsection{Summary of Security Evaluation of the \ttOWF}

We most interesting attack on the \ttOWF is based on a reduction to subset-sum. 
We give the full details below.

\subsubsection{Reduction to subset-sum.}

For a vector $w \in \Z_2^m$, there is an $(m -n) \times m$ (parity check) matrix $\mat{P}$ such that there exists $x \in \Z_2^n$ for which $\mat{A}x = w$ if and only if $\mat{P} w=0$.
Assume that $\hat{y}$ is the output of the \ttOWF on $x \in \Z_2^n$, and let $w = \mat{A}x$. Then,
$w$ satisfies the conditions $\mat{P} w = 0$ (over $\Z_2$) and $\mat{B}w = \hat{y}$ (over $\Z_3$).
We attempt to find such $w$ by a reduction to subset-sum, as detailed below.

Suppose we find a set $J \subseteq [m]$ such that
$$\left( \sum_{j \in J} \mat{P}e_j  \bmod 2, \sum_{j \in J} \mat{B}e_j  \bmod 3 \right) = (\vec{0},\hat{y}),$$
where $e_i \in \{0,1\}^m$ be the $i$'th unit vector.
Then, the preimage $x$ can be computed by solving the linear equation system
$\mat{A}x = \sum_{j \in J} e_j  \bmod 2$.

Thus, we have reduced the problem to subset-sum with $m$ binary variables
$(\epsilon_1, \ldots, \epsilon_m) \in \{0,1\}^m$, where we associate $\epsilon_i = 1$
with $$(\mat{P}e_i, \mat{B}e_i) \in \Z_2^{m-n} \times \Z_3^t,$$
and define the target
as $(\vec{0},\hat{y}) \in \Z_2^{m-n} \times \Z_3^t$.

\subsubsection{Solving the subset-sum problem.}
We can now apply the advanced subset-sum algorithm by
Howgrave{-}Graham and Joux~\cite{Howgrave-GrahamJ10} and the more recent ones
~\cite{BeckerCJ11,BonnetainBSS20},
which are based on the \emph{representation technique}.
These algorithms were mostly designed to solve subset-sum problems over the integers.
Below, we describe the main ideas of these algorithms
and explain how to apply them to the special subset-sum problem
we consider.

In the subset-sum problem over the integers,
we are given a list of $m$ positive  integers $(a_1,a_2,\ldots,a_m)$
and another positive integer $S$ such that $S = \sum_{i=1}^{m} \epsilon_i a_i$
for $\epsilon_i \in \{0,1\}$. The goal is to recover the unknown coefficients $\epsilon_i$.

A standard meet-in-the-middle approach for solving the problem has time complexity of about $2^{m/2}$.
The representation technique gives an improved algorithm as briefly summarized below.

Assume that a solution to the subset-sum problem is chosen uniformly from $\{0,1\}^m$
and the parameters are set such that the instance has about one solution on average. Effectively, this means that the density of the problem $d = \tfrac{n}{\log \max(\{a_i\}_{i=1}^{m})}$ is set to 1.

The main idea of the basic algorithm of Howgrave{-}Graham and Joux~\cite{Howgrave-GrahamJ10}
is to split the problem into two parts by writing
$S = \sigma_1 + \sigma_2$,
where $\sigma_1 = \sum_{i=1}^{m} \alpha_i a_i$, $\sigma_2 = \sum_{i=1}^{m} \beta_i a_i$
and $(\alpha_i,\beta_i) \in \{(0,0),(0,1),(1,0)\}$.
Thus, $\epsilon_i = \alpha_i + \beta_i$ for each $i$ is a solution to the problem.

Note that each coefficient $\epsilon_i$ with value 1 can be split into $(0,1)$, or $(1,0)$.
Thus, assuming that the solution has Hamming weight\footnote{In general, one
may guess the Hamming weight of the solution and
repeat the algorithm accordingly a polynomial number of times.} of $m/2$
(which occurs with probability $\Omega(1/\sqrt{m})$),
it has $2^{m/2}$ different \emph{representations}.
Consequently, we may focus of finding only one of these representations
by solving two subset-sum problems of Hamming weight $m/4$.
Focusing on a single representation of the solution
allows to beat the standard meet-in-the-middle approach which requires time $2^{m/2}$.

\subsubsection{Adaptation of previous subset-sum algorithms.}
The algorithm of~\cite{Howgrave-GrahamJ10} can be easily adapted
to our specialized subset-sum problem (although it is not defined over the integers).
Moreover the improved algorithm of~\cite{BeckerCJ11} considers additional representations
of the solution by allowing $\alpha_i$ and $\beta_i$ to also take the value -1
(implying that $\epsilon_i = 0$ can be decomposed into
$(\alpha_i,\beta_i) \in \{(0,0),(-1,1),(1,-1)\}$).
In our case, we associate $\alpha_i = -1$
with $(\mat{P}(-e_i), \mat{B}(-e_i)) = (\mat{P}e_i, 2 \cdot \mat{B}e_i)  \in \Z_2^{m-n} \times \Z_3^t$.
Finally, the recent improved algorithm of~\cite{BonnetainBSS20} considers representations over
$\{-1,0,1,2\}$ and we can adapt this algorithm to our setting in a similar way.

In terms of complexity, ignoring polynomial factors in $m$,
the attack of~\cite{Howgrave-GrahamJ10} runs in time $2^{0.337m}$ and uses $2^{0.256m}$ memory,
while the complexity of attack of~\cite{BonnetainBSS20} requires $2^{0.283m}$ time and memory.

Thus, conservatively ignoring polynomial factors, for $s$-bit security we require $0.283m \geq s$, or $m \geq 3.53 s$.

\subsection{Summary of Security Evaluation of the \ttwPRF}

For the \ttwPRF, the attacker obtains several samples and tries to 
mount a key recovery and or a distinguishing attack.
We restrict the number of samples produced with a single secret to $2^{40}$.

We will set the parameters such that $n - \log 3 \cdot t \geq s$,
and thus there are $2^s$ keys on average that are consistent with a single sample.
Therefore, any key recovery attack faster than $2^s$ has to make use of at least two samples.
Particularly, the subset-sum attack can also be applied to the \ttwPRF,
but it is not clear how to use it efficiently on more than one sample
(without strong relations between them).

The most important distinguishing attack looks for a bias in a linear combination
of the output over $\Z_3$.
Given a single sample $(x,\mat{B},y)$, assume there exist
$v \in \Z_3^m$ and $u \in \Z_3^t$ such that $u \mat{B} = v$ and the Hamming weight of $v$ is $\ell$. 
As $y = \mat{B}w \bmod 3$, the attacker computes $uy \bmod 3 = vw \bmod 3$ 
and thus obtains the value of a linear combination $\bmod$ 3 of $\ell$ entries of $w \in \{0,1\}^m$.
Since $w \in \Z_2^m$, if $\ell$ is sufficiently small, then this linear combination is biased.
The bias can be amplified using several samples.
Consequently, we require that the rows of $\mat{B}$ do not span a vector of low Hamming weight.
This analysis is probabilistic and leads to a lower bound on $m$. 

Overall, we set $n=m=2s$ and $t = s/\log 3$.
These are somewhat aggressive parameters as the security margin against the above attacks
in rather narrow. A choice of $n=m=2.5s$ is more conservative.

\subsection{Summary of Security Evaluation of the LPN-PRG}

\subsection{Summary of Security Evaluation of the LPN-PRF}

\itai{To be written} 

%!TEX root = ../main.tex
\newpage
\section{Distributed Protocols}
\label{sec:distributed_protocols}
We now describe efficient protocols to compute our candidate constructions in several interesting distributed settings. This section is structured as follows. First, in Section~\ref{subsec:protocol_overview}, we provide a technical overview for our overall protocol design. This includes computation protocols for each of our circuit gates as well as a generic way to compose them to obtain fully distributed protocols in the preprocessing model for all our constructions. Section~\ref{subsec:distributed_protocol} quantifies this approach by providing concrete communication and preprocessing costs for distributed evaluations for our constructions. In Section~\ref{subsec:3party_protocol}, for our \ttwPRF construction, we provide a 3-party protocol that does not require any preprocessing and is secure against one passive corruption. Finally, in Section~\ref{subsec:oprf_protocol}, we provide two OPRF protocols for our \ttwPRF construction in the preprocessing model. All provided protocols are for the semi-honest setting.

\subsection{Technical Overview}
\label{subsec:protocol_overview}
Recall that all our constructions can be succinctly represented using four basic gates. The main strategy now will be to evaluate each of these gates in a distributed manner. These gate evaluation subprotocols can then be easily composed to evaluate the  candidate constructions. 

We begin with distributed protocols to evaluate each of the four gates. Abstractly, the goal of a gate protocol is to convert shares of the inputs to shares of the outputs (or shares of the masked output). To make our formalism cleaner, the gate protocols, by themselves, will involve no communication. Instead, they can additionally take in masked versions of the inputs, and possibly some additional correlated randomness. When composing gate protocols, whenever a masked input is needed, the parties will exchange their local shares to publicly reveal the masked value. This choice also prevents redoing the same communication when the masked value is already available from earlier gate evaluations.


\paragraph{Protocol notation and considerations.}
For a protocol $\prot$, we use the notation $\prot(a_1, \dots, a_k \mid b_1, \dots, b_l)$ to denote that the values $a_1, \dots, a_k$ are provided publicly to all parties in the protocol, while the values $b_1, \dots, b_l$ are secret shared among the parties. When $\party_i$ knows the values $(a_1, \dots, a_k)$, and has shares $\sharei{b_1}, \dots, \share{b_l})$, we use the notation $\prot(a_1, \dots, a_k \mid \sharei{b_1}, \dots, \sharei{b_l})$ to denote that $\party_i$ runs the protocol with its local inputs. 

Given public values $a_1, \dots, a_k$, it is straightforward for the protocol parties to compute a sharing $\share{f(a_1, \dots, a_k)}$ for a function $f$ (for example, $\party_1$ computes the function as its share, and all other parties set their share to $0$).

\subsubsection{Distributed Computation of Circuit Gates}

% \yuval{In the following, I suggest to switch to a more structured format. For instance: an itemized list for each subprotocol, where the first item describes the functionality, the second describes the correlated randomness, and the third describes the actual protocol. It will also be good to have a table that summarizes the online and communication and correlated randomness (in bits) for each subprotocol. }

We provide detailed (local) protocols to compute each circuit gate in this section. The description of inputs (including shared correlated randomness) and outputs for each gate protocol is also summarized in Table~\ref{table:gate_protocol_summary}.

\begin{table}[t!]
\lncsresize{
\centering
{
\renewcommand{\arraystretch}{1.5}
\begin{tabular}{|c|c|c|c|c|}

\hline
Protocol & \makecell{Public \\ Inputs} & \makecell{Shared \\ Inputs} & \makecell{Shared \\ Correlated Randomness} & \makecell{Output Shares \\ (over base group $\mathbb{G}$)} \\
\hline
$\prot_{\LMap}^{\mat{A},p}$ & $\mat{A}$ & $x$ & - & $y = \mat{A}x$ (over $\Z_p$)\\
\hline
$\prot_{\Add}^{p}$ & & $x, x'$ & - & $y = x + x'$ (over $\Z_p$)\\
\hline

% $\prot_{\BLMap}^p$ & $\hat{\mat{K}}, \hat{x}$ & $\tilde{\mat{K}},\tilde{x}, \tilde{\mat{K}}\tilde{x} + \tilde{y}$ & $\hat{y} = \mat{K}x + \tilde{y}$ (over $\Z_p$)\\
$\prot_{\BLMap}^p$ & $\hat{\mat{K}}, \hat{x}$ & - & $\tilde{\mat{K}},\tilde{x}, \tilde{\mat{K}}\tilde{x}$ & $y = \mat{K}x$ (over $\Z_p$)\\
\hline
$\prot_{\Convert}^{(2,3)}$ & $\hat{x}$ (over $\Z_2$) & - & $r = \tilde{x}$ (over $\Z_3$) & $x^* = x$ (over $\Z_3$) \\
\hline
$\prot_{\Convert}^{(3,2)}$ & $\hat{x}$ (over $\Z_3$) & - & \makecell{$u = \tilde{x} \bmod 2$ (over $\Z_2$) \\ $v = (\tilde{x} + \textbf{1} \bmod 3) \bmod 2$ (over $\Z_2$)} & $x^* = x \bmod 2$ (over $\Z_2$) \\
\hline
\end{tabular}
}
}
\caption{Summary of input, output, and randomness for the circuit gate protocols. \greg{Change ``over base group'' to ``over group''?}}
\label{table:gate_protocol_summary}
\end{table}


\paragraph{Linear gate protocol $\prot_\LMap^{\mat{A}, p}$.}
The linear gate is the easiest to evaluate, and follows from the standard linear homomorphism of additive secret sharing.

\begin{itemize}
  \item \textbf{Functionality}: Each party is provided with the matrix $\mat{A}$ and shares of the input $x$ (over $\Z_p$). The goal is to compute shares of the output $y = \mat{A}x$.

  \item \textbf{Preprocessing}: None required.

  \item \textbf{Protocol details}:
  For the protocol $\prot_\LMap^{\mat{A}, p}(\mat{A} \mid x)$, each party $\party_i$ computes its output share as $\sharei{y} = \mat{A}\sharei{x}$. Note that this works because $\mat{A}x = \sum_{\party_i \in \parties} \mat{A}\sharei{x}$ as a direct consequence of the linear homomorphism of additive shares.
\end{itemize}

\paragraph{Addition gate protocol $\prot_\Add^{p}$.}
The addition modulo $p$ gate is also easy to evaluate. Given shares of $x, x'$ over $\Z_p$, for the protocol each party $\party_i$ can locally compute its share of $x+x'$ as $\sharei{x} + \sharei{x'} \bmod p$. This directly follows from the additive homomorphism of additive shares. 



\paragraph{Bilinear gate protocol $\prot_\BLMap^p$.}
The bilinear gate protocol is essentially a generalization of Beaver's multiplication triples~\cite{boyle2019-fss-preprocess,beaver1991-triples} that computes the multiplication of two shared inputs. For Beaver's protocol, to compute a sharing of $ab$ given shares of $a$ and $b$ (all sharings are over a ring $\mathcal{R}$), the protocol parties are provided shares of a randomly sampled triple of the form $(\tilde{a},\tilde{b},\tilde{a}\tilde{b})$ in the preprocessing stage. Beaver's protocol first reconstructs the masked inputs $\hat{a}$ and $\hat{b}$ after which local computation is enough to produce shares of the output. For our bilinear gate protocol, we assume that  all parties are already provided with the masked inputs (to move the communication outside of the gate protocol), along with correlated randomness similar to a Beaver triple. 

\begin{itemize}
  \item \textbf{Functionality}: 
  Abstractly, the goal of the bilinear gate protocol is to compute shares of the output $y = \mat{K}x$ given shares of both inputs $\mat{K}$ and $x$. For our purpose however, the masked inputs will have already been constructed beforehand, i.e., each party is provided with $\hat{\mat{K}}$ and $\hat{x}$ publicly, along with shares of correlated randomness similar to a Beaver triple (see below).

  \item \textbf{Preprocessing}: Each party is provided shares of $\tilde{\mat{K}}, \tilde{x}$, and $\tilde{\mat{K}}\tilde{x}$ as correlated randomness.

  \item \textbf{Protocol details}: For the protocol $\prot_\BLMap^p(\hat{\mat{K}},\hat{x}\mid \tilde{\mat{K}}, \tilde{x}, \tilde{\mat{K}}\tilde{x})$, each party $\party_i$ computes its share of $\hat{y}$ as:
  \[
    \sharei{\hat{y}} = \sharei{\hat{\mat{K}}\hat{x}} - \hat{\mat{K}}\sharei{\tilde{x}} - \sharei{\tilde{\mat{K}}}\hat{x} + \sharei{\tilde{\mat{K}}\tilde{x}}
  \]

  \noindent \textit{Correctness}. Note that this works since:
  \begin{align*}
  \sum_{\party_i \in \parties} \sharei{\hat{y}} &= \hat{\mat{K}}\hat{x} - \hat{\mat{K}}\tilde{x} - \tilde{\mat{K}}\hat{x} + \tilde{\mat{K}}\tilde{x} \\
  &= (\mat{K} + \tilde{\mat{K}})x - \tilde{\mat{K}}(x + \tilde{x}) + \tilde{\mat{K}}\tilde{x} \\
  &= \mat{K}x
  \end{align*}
\end{itemize}
Since the output of the bilinear gate will usually feed into a conversion gate which requires the input to be already masked, as an optimization, we can have the bilinear gate itself compute shares of the masked output, i.e., $\hat{y} = \mat{K}x + \tilde{y}$. This can be done by providing the correlated randomness $\tilde{\mat{K}}\tilde{x} + \tilde{y}$ instead of $\tilde{\mat{K}}\tilde{x}$. The upshot of this optimization is that one fewer piece of correlated randomness will be required.

\paragraph{$\Z_2 \to \Z_3$ conversion protocol $\prot_\Convert^{(2,3)}$.}

\begin{itemize}
  \item \textbf{Functionality}: Abstractly, the goal of the $\Z_2 \to \Z_3$ conversion protocol is to convert a sharing of $x$ over $\Z_2$ to a sharing of the same $x^* = x$, but now over $\Z_3$. For our purpose, the parties will be provided the masked input $\hat{x} = x \oplus \tilde{x}$ (i.e., masking is over $\Z_2$) directly along with correlated randomness that shares $\tilde{x}$ over $\Z_3$.

  \item \textbf{Preprocessing}: Each party is also provided with shares of the mask $r = \tilde{x}$ over $\Z_3$ as correlated randomness.

  \item \textbf{Protocol details}: For the protocol $\prot_\Convert^{(2,3)}(\hat{x} \mid r)$, each party proceeds as follows:
  \[
  \sharei{x^*} = \sharei{\hat{x}} + \sharei{r} + (\hat{x} \odot \sharei{r}) \quad \bmod 3
  \]
where $\odot$ denotes the Hammard (component-wise) product modulo 3. Here, addition is also done over $\Z_3$.

\noindent \textit{Correctness.} To see why this works, suppose that $\hat{x} \in \Z_2^l$. Consider any position $j \in [l]$, and denote by using a subscript $j$, the $j^\thtext$ position in a vector. Note that now, the position $j$ of the output can be written as:
\[
    \sharei{x^*}_j = \sharei{\hat{x}}_j + \sharei{r}_j + (\hat{x}\sharei{r}_j\bmod 3) \quad \bmod 3
\]
Consider two cases:
\begin{itemize}
\item If $\hat{x}_j = 0$, then $\tilde{x}_j = x_j$. Therefore, $\sum_{\party_i \in \parties} \sharei{x^*}_j = 0 + \tilde{x}_j = x_j$.

\item If $\hat{x}_j = 1$, then $x_j = 1 - \tilde{x}_j$. Therefore, $\sum_{\party_i \in \parties} \sharei{x^*}_j = 1 + 2\tilde{x}_j \bmod 3$. If $\tilde{x}_j = 0$, this evaluates to $1 = x_j$, while if $\tilde{x}_j = 1$, it evaluates to $0 = 1 - \tilde{x}_j = x_j$
\end{itemize}
In other words, in all cases, each component of the sum ($\bmod~3$) of shares $\sharei{x^*}$ is the same as the corresponding component of $x$. Therefore, 
\[
\sum_{\party_i \in \parties} \sharei{x^*} (\bmod~3) = x
\]
will hold.
\end{itemize}


\paragraph{$\Z_3 \to \Z_2$ conversion protocol $\prot_\Convert^{(3,2)}$.}

\begin{itemize}
  \item \textbf{Functionality}: Abstractly, the goal of the protocol is to convert a sharing of $x$ over $\Z_3$ to a sharing of $x^* = x \bmod~2$ over $\Z_2$. For our purpose, the parties will be provided with the masked input $\hat{x} = x + \tilde{x} \bmod~3$ directly, along with correlated randomness over $\Z_3$ (see below).

  \item \textbf{Preprocessing}: Each party is also given shares (over $\Z_2$) of two vectors: $u = \tilde{x} \bmod 2$ and $v = (\tilde{x} + \textbf{1} \bmod 3) \bmod 2$ as correlated randomness.


  \item \textbf{Protocol details}: For the protocol $\prot_\Convert^{(3,2)}(\hat{x} \mid u, v)$, each party computes its share of $x^*$ as follows: For each position $j \in [l]$,
\[
\sharei{x^*}_j = 
\begin{cases*}
       1 - \sharei{u}_j - \sharei{v}_j  & \quad if $\hat{x}_j = 0$ \\
       \sharei{v}_j & \quad if $\hat{x}_j = 1$ \\
       \sharei{u}_j & \quad if $\hat{x}_j = 2$
\end{cases*}
\]
\textit{Correctness.} To see why this works, consider three cases:
\begin{itemize}
\item If $\hat{x}_j = 0$, then $\sum_{\party_i \in \parties} \sharei{x^*}_j \bmod 2 = 1 - u_j - v_j$. This evaluates to $1$ only when $\tilde{x}_j = 2$, and is exactly the case when $x_j$ is also 1.

\item $\hat{x}_j = 1$, then $\sum_{\party_i \in \parties} \sharei{x^*}_j \bmod 2 = v_j = (\tilde{x}_j + 1 \bmod 3) \bmod 2)$. This evaluates to $1$ only when $\tilde{x}_j = 0$, and is exactly the case when $x_j$ is also 1.

\item $\hat{x}_j = 2$, then $\sum_{\party_i \in \parties} \sharei{x^*}_j \bmod 2 = u_j$. This evaluates to $1$ only when $\tilde{x}_j = 1$, and is exactly the case when $x_j$ is also 1.
\end{itemize}
Consequently, $\sum_{\party_i \in \parties} \sharei{x^*} \bmod 2 = x \bmod 2$ holds.
\end{itemize}


%-------------------------------%

\subsubsection{Composing Gate Protocols}
We now describe a general technique to evaluate circuit composed of the previously specified gates in a distributed fashion. We provide details for the semi-honest fully distributed setting (with preprocessing), where all inputs are secret shared between all parties initially. While the technique will also work for other settings (e.g., OPRF, public input), the concrete communication costs will be significantly worse than more specially designed protocols. For these settings, we will provide more efficient protocols than provided by this general technique.

\greg{Does it make sense to say that this very general composition protocol can
allow one to easily make protocols for variants of our primitives? ``MPC
toolbox for alternating moduli primitives'' or similar}

\paragraph{Composition protocol.} Consider a circuit $C$ (Definition~\ref{def:computation_circuit}) with input space $\Gin = \prod \Gin_i$. To evaluate $C$ with input $(x_1, \dots, x_l) \in \Gin$, in the fully distributed setting, all parties are given additive shares for each $x_i$. Now, the distributed evaluation of $C$ proceeds as follows:
\begin{itemize}
  
  \item All vertices at the same depth in $C$ are evaluated simultaneously, starting from the source vertices that contain the inputs of the computation. 

  \item The evaluation of a (non-source) vertex in the graph of $C$ is done by each party running the corresponding gate protocol locally on their share of the inputs. 

  \item For an edge $(V_a, V_b)$, suppose that the output of $V_a$ is used as one of the inputs of $V_b$. If the gate protocol corresponding to $\mathcal{G}_V$ requires this input to be masked (e.g., the bilinear gate protocol), then before evaluating $V_b$, each party first masks its share of the output. Now, all parties simultaneously reveal their shares to publicly reveal the masked value.   The masking values are provided to the parties in the preprocessing phase. The same value also need not be masked multiple times if it is required for multiple gates.

  \item The required output shares of the distributed evaluation are given by the evaluation of the sink vertices in the circuit.
\end{itemize}

\paragraph{Communication cost.}
Since the gate protocols themselves are locally computable, the communication cost during a distributed evaluation of a circuit comes solely from the public reconstructions of masked values required for gate protocols. For example, before feeding the output $x$ of a $\LMap^{\mat{A}}_2$ gate into a $\Convert_{(2,3)}$ gate, in the distributed evaluation, all parties will first mask their shares of $x$ to obtain shares of $\hat{x}$. Then, the parties will exchange messages to reconstruct the $\hat{x}$ value required for $\prot_\Convert^{(2,3)}$.

Consider $N$ parties taking part in the distributed evaluation. To reconstruct an $l$-bit value $\hat{x}$ that is additively shared among the parties, one of the following can be done.
\begin{itemize}
  \item Each party sends its share of $\hat{x}$ to each other party. Now, all parties can compute $\hat{x}$ locally. This requires only 1 online round but has a communication cost of $(N-1)l$ bits per party. Each party sends $N-1$ messages. The simplest case for this is when $n=2$, in which case, both parties can simultaneously exchange their shares, and add the two shares locally to reconstruct $\hat{x}$. This requires 1 online round, and has a communication cost of $1$ message and $l$ bits per party. 

  \item All parties can send their share to a designated party, say $\party_1$, who computes $\hat{x}$ and sends it back to everyone. This requires 2 rounds and has a communication cost of $(N-1)l$ bits for $\party_1$ and $l$ bits each for other parties. Here, $\party_1$ sends $N-1$ messages while all other parties send a single message.
\end{itemize}


\paragraph{Reducing round complexity.} It is also straightforward to parallelize the communication to reduce the number of rounds. For this, suppose that we call an edge $(V_a, V_b)$ \textit{communication-requiring} if the output of the protocol for $V_a$ needs to be masked before it is input into the protocol for $V_b$ (in other words, the gate protocol for $V_b$ requires a masked input). Now, define the communication-depth of a vertex $V$ as the maximum number of communication-requiring edges in any path from a source vertex to $V$. Now, instead of evaluating vertices with the same depth simultaneously, we will evaluate vertices with the same \textit{communication-depth} together before the next communication round. By doing so, we can reduce the total number of rounds to the maximum communication-depth.


\paragraph{Preprocessing cost.}
A na{\"i}ve technique to compute the preprocessing required is to add the preprocessing for each gate in the circuit as follows:

\begin{itemize}
\item The $\LMap$ and $\Add$ gates are computed locally and require no preprocessing.

\item Each $\BLMap_p$ gate requires a Beaver-style triple which provides masks for the two inputs and a multiplication of the two masks. Specifically, for $\BLMap_p(\mat{K}, x)$ where $\mat{K} \in \Z_p^{l_2 \times l_1}$ and $x \in \Z_p^{l_1}$, the preprocessed shares are of the form $(\tilde{\mat{K}}, \tilde{x}, \tilde{\mat{K}}\tilde{x})$. Consequently, when $\mat{K}$ is circulant, a total of $(2l_1 + l_2) \log_p$ bits needs to be provided as preprocessing to each party.

\item Each $\Convert_{(2,3)}$ gate requires shares of a random mask $\tilde{x}$ both over $\Z_2$ and $\Z_3$. In other words, for $x \in \Z_2^l$, it requires $l + \log_2{3} \cdot l$ bits of preprocessing per party.

\item Each $\Convert_{(3,2)}$ gate requires shares of a random mask $\tilde{x}$ (over $\Z_3$) as well as $u = \tilde{x}$ and $v = (\tilde{x} + \textbf{1} \bmod 3)\bmod 2$ (both over $\Z_2$). In other words, for $x \in \Z_3^l$, it requires $\log_2{3} \cdot l + 2l$ bits of preprocessing per party.
\end{itemize}

\paragraph{Compressing preprocessing required.}
We describe several optimizations to reduce the total cost of preprocessing. We assume here the presence of a trusted dealer to generate any correlations in the randomness. We note that some of the optimizations, while reducing the size of preprocessing, would be more complicated to generate if a dealer is not present.

\begin{itemize}

  \item (Reducing redundant preprocessing).
  A straightforward optimization is to not mask the same value twice. For example, if the same value $x$ is considered as input for both a $\BLMap$ gate and a $\Convert$ gate, the same mask $\tilde{x}$ can be used for both. 

  \item (Masking $\BLMap$ gate outputs.)
  The standard $\BLMap$ gate requires preprocessing of the form $(\tilde{\mat{K}}, \tilde{x}, \tilde{\mat{K}}\tilde{x})$. However, if the output of the $\BLMap$ gate is then later input to a gate that requires a masked input (e.g., a $\Convert$ gate or even another $\BLMap$ gate), the $\BLMap$ gate can directly mask its output by providing $\tilde{\mat{K}}\tilde{x} + \tilde{y}$ instead. If this is done, the parties will compute a sharing of $\hat{y} = \mat{K}x + \tilde{y}$ using the $\BLMap$ gate. This means that the parties can directly exchange their shares to reconstruct $\hat{y}$ without requiring more preprocessing to mask $y$.

  For both the $\Convert_{2,3}$ and $\Convert_{(3,2)}$ gates, if the masked version $\hat{x}$ of the input $x$ is already known to all parties, only $\log_2{3} \cdot l$ and $2l$ bits respectively of preprocessing are required per party. 

 
  \item (Compression using a PRG).
  Another standard technique for compressing the size of preprocessing is to use a PRG. Intuitively, each party is given a different PRG seed by the trusted dealer which they can use locally to generate their randomness. Only a single party has its shares given by the dealer to ensure that the randomness is appropriately correlated. 

  % For this, the dealer will first generate the randomness as usual. To share the randomness, instead of randomly sampling shares, the dealer (who knows the seeds of all parties) will compute the shares for all but one party (say $\party_1$) using the party's PRG seed. The share for $\party_1$ can now be set so that the shares form an sharing of the required randomness.

  % More specifically, for the $\BLMap$ gate, a dealer first samples a random triple $(\tilde{K}, \tilde{x}, \tilde{K}\tilde{x})$
\end{itemize}

\newpage



\subsection{Distributed Evaluation in the Preprocessing Model}
\label{subsec:distributed_protocol}
Equipped with our technical overview, we now move to constructing distributed protocols (with preprocessing) for our candidate constructions. By distributed evaluation, we mean that all inputs are secret shared between all parties and the protocol provides parties with a sharing of the output. As a concrete example, we provide the complete details of a 2-party distributed evaluation protocol for our \ttwPRF candidate. 
%Later, in Table~\ref{table:construction_costs}, for each our constructions, we provide the amount of preprocessed randomness required as well as the communication cost for the distributed protocol.

\subsubsection{$2$-Party Protocol for \ttwPRF.}
\label{subsec:2pc-wprf}
We detail a $2$-party semi-honest protocol for evaluating our \ttwPRF candidate (Construction~\ref{construction:23-central-wprf}). In this setting, two parties, denoted by $\party_1$ and $\party_2$ hold additive shares of a key $\mat{K} \in \Z_2^{m \times n}$, and an input $x \in \Z_2^n$. The goal is to compute an additive sharing of the wPRF output $y = \LMap^\mat{B}_3(\mat{K}x)$ where $\mat{B} \in \Z_3^{t \times m}$ is a publicly known matrix.


\paragraph{Preprocessing.}
Our protocol requires preprocessed randomness as follows. A dealer randomly samples masks $\tilde{\mat{K}}$ and $\tilde{x}$ for $\mat{K}$ and $x$ respectively. It also randomly samples $\tilde{w} \in \Z_2^m$ as mask for the intermediate output. Let $r \in \Z_3^m; r = \tilde{w}$ (viewed over $\Z_3$). The dealer creates additive sharings for $\tilde{\mat{K}}, \tilde{x}, \tilde{\mat{K}}\tilde{x}, \tilde{w}$, and $r$. Each $\party_{i\in \{1,2\}}$ is now provided $\sharei{\tilde{\mat{K}}}, \sharei{\tilde{x}}, \sharei{\tilde{\mat{K}}\tilde{x}}, \sharei{\tilde{w}}$, and $\sharei{r}$ as preprocessing.

\paragraph{Protocol details.}
The distributed protocol proceeds as follows:
\begin{itemize}
    \item Each party $\party_i$ computes masks their key and input shares as 
    \begin{align*}
    \sharei{\hat{\mat{K}}} &= \sharei{\mat{K}} + \sharei{\tilde{\mat{K}}} \\
    \sharei{\hat{x}} &= \sharei{x} + \sharei{\tilde{x}}
    \end{align*}
    using its given randomness. The shares are then exchanged simultaneously by both parties to reconstruct $\mat{\hat{K}}$ and $\hat{x}$.

    \item Each party $\party_i$ now locally runs $\prot_\BLMap^2 \left(\hat{\mat{K}}, \hat{x} \mid \sharei{\tilde{\mat{K}}}, \sharei{\tilde{x}}, \sharei{\tilde{\mat{K}}\tilde{x}}\right)$ and adds to it its share of $\tilde{w}$ to obtain its share (over $\Z_2$) of $\hat{w} = \mat{K}x + \tilde{w}$. The shares are then exchanged simultaneously by both parties to reconstruct $\hat{w}$.

    \item Each party $\party_i$ now locally runs $\prot_{\Convert}^{(2,3)}(\hat{w} \mid \sharei{r})$ to obtain its shares of $w^* = w$ (over $\Z_3$).

    \item Finally, each party $\party_i$ obtains its share of the final output $y$ (over $\Z_3$) by running $\prot_\LMap^{\mat{B}, 3}(\mat{B} \mid \sharei{w^*})$.
\end{itemize}



\paragraph{Cost analysis.}
The distributed protocol takes 2 communication rounds in total, with both parties sending a message in each round. When $\mat{K}$ is a circulant matrix (i.e., it can be represented by $n$ bits), each party communicates $2n$ bits in the first round and $m$ bits in the second round.

As preprocessing, each party also receives shares of $\tilde{\mat{K}}, \tilde{x}, \tilde{\mat{K}}\tilde{x}, \tilde{w}$ (over $\Z_2$) and $\tilde{r} = \tilde{w}$ (over $\Z_3$). This can also be optimized by providing shares of $\tilde{\mat{K}}\tilde{x} + \tilde{w}$ for the $\BLMap$ gate directly. In total, if circulant keys are used, the preprocessing cost per party is $2n + m + \log_2{3}\cdot m$ bits.

Furthermore, if each party is given a PRG seed, then they can use the PRG to compute their shares of $\tilde{\mat{K}}$ and $\tilde{x}$. In this case, the dealer only needs to provide $\party_1$ with its shares for $\tilde{\mat{K}}\tilde{x} + \tilde{w}$ and $r$. $\party_2$ can compute its shares for those values using its PRG seed and the dealer can use its knowledge of $\party_2$'s seed to appropriately set the shares for $\party_1$. Consequently, across both parties, only $m + \log_2{3} \cdot m$ bits of preprocessing is required apart from the PRG seeds. Short $\secparam$-bit seeds can be provided to each party beforehand and reused for multiple evaluations. 


\subsubsection{Public-input setting}
It is also straightforward to use the same generic technique to construct distributed protocols in the public-input setting. For keyed primitives, in public-input setting, the key is secret shared between the parties but the input is publicly known. The goal of the protocol is for the parties to compute shares of the output.

One useful optimization is that in the public-input setting, a $\BLMap$ gate where the input is known, essentially reduces to a linear gate where the key $\mat{K}$ is secret shared and the input $x$ is publicly known.

\paragraph{2PC public-input protocol for \ttwPRF.}
Concretely, for the evaluation of \ttwPRF in the public-input setting, the first round from the distributed protocol can be entirely skipped. The two parties can directly compute shares of $w = \mat{K}x = \sum_{i \in \{1,2\}} \sharei{\mat{K}}x$ locally. This also means that the preprocessing previously required for the $\BLMap$ gate that computed $\mat{K}x$ is no longer necessary. The rest of the evaluation can now proceed as before with both parties first using $\prot_\Convert^{(2,3)}$ to retrieve shares of $w$ over $\Z_3$, and then using $\prot_\LMap^{\mat{B},3}$ to compute shares of the final output $y$.

In total, the evaluation takes a single round and a communication of $m$ bits per party (to reconstruct $\hat{w}$). The only preprocessing required is for the $\prot_{\Convert}^{(2,3)}$ gate, for which each party will be given $m + \log_2{3} \cdot m$ bits. Furthermore, with PRG compression, $\party_2$ will require no extra preprocessing and $\party_1$ can be given only $\log_2{3} \cdot m$ bits.



%------------------%

\subsection{3-party Distributed Evaluation}
\label{subsec:3party_protocol}

In this section, we provide a 3-party (semi-honest) protocol for computing the \ttwPRF candidate that is secure against one passive corruption and does not require any preprocessing.

\paragraph{Functionality.}
Denote the servers by $\party_1, \party_2$, and $\party_3$. We assume that the servers hold replicated additive shares of the key $\mat{K}$ and the input $x$. The key is assumed to be circulant and can be represented by $k \in \Z_2^n$. Concretely, let $(k_1, k_2, k_3)$ and $(x_1,x_2,x_3)$ be additive shares of $k$ and $x$ respectively. Then, each party $\party_i$ is given $k_j$, $x_j$ with $j \neq i$. At the end of the protocol, $\party_2$ and $\party_3$ should hold $y_2$ and $y_3$ respectively such that $(y_2, y_3)$ is a sharing of the wPRF output $y$.

\paragraph{Protocol.}


\begin{itemize}
  \item First, the three servers compute additive shares of the linear mapping $\mat{K}{x}$ locally using their replicated shares. Note that this can be locally since for two secret shared values $a = a_1 + a_2 + a_3$ and $b = b_1 + b_2 + b_3$, their product can be computed as $ab = \sum_{1 \leq j,k \leq 3} a_jb_k$. Since each party holds two shares of $a,b$ in a replicated sharing scheme, each term $a_jb_k$ can be computed by at least 1 party. Suppose that the share of $\party_i$ is denoted by $\sharei{\mat{K}x}$.

  \item Now, $\party_1$ samples $\tilde{w} \getsr \Z_2^m$, $r_2 \gets \Z_3^m$, and sets $r_3 = \tilde{w} - r_2 \bmod 3$. In other words $(r_2, r_3)$ is a random $\Z_3$ sharing of $r = \tilde{w}$. $\party_1$ sends $\share{\mat{K}x}^{(1)} + \tilde{w}$ and $r_i$ to $\party_{i \in \{2,3\}}$. At the same time, $\party_2$ and $\party_3$ exchange their shares of $\mat{K}x$. All of this can be done in one round.

  \item At this point, $\party_2$ and $\party_3$ can both compute $\hat{w} = \tilde{w} + \sum_{1\leq i \leq 3} \sharei{\mat{K}x}$. Now, for $i \in \{2,3\}$, $\party_i$ can locally compute $w^*_i \gets \prot_\Convert^{(2,3)}(\hat{w} \mid r_i)$. Finally, $\party_i$ can locally compute its share $y_i$ of the output by running $\prot_\LMap^{\mat{B},3}(\mat{B} \mid w^*_i)$.
\end{itemize}


\paragraph{Cost analysis.}
\mahimna{todo}





%----------------%

\subsection{Oblivious PRF Evaluation in the Preprocessing Model}
\label{subsec:oprf_protocol}
Our distributed evaluation protocols from Section~\ref{subsec:distributed_protocol} can be used directly for semi-honest \textit{oblivious} PRF, or OPRF, evaluation in the preprocessing model. Recall that in the OPRF setting, one party $\party_1$ (called the ``server'') holds the key $\mat{K}$ and the other party $\party_2$ (called the ``client'') holds the input $x$. The goal of the protocol is to have the client learn the output of the PRF for key $\mat{K}$ and input $x$, while the server learns nothing. In the semi-honest setting, both parties can first use the distributed protocol to obtain shares of the PRF output. The server can then send its share to the client so that only the client learns the final output. Such an OPRF protocol would require one extra round over the corresponding distributed PRF protocol. We can however construct much better protocols whose efficiency rivals that of existing DDH-based OPRF protocols. Here, we provide two concrete efficient protocols for evaluating the \ttwPRF candidate (Construction~\ref{construction:23-central-wprf}) in the OPRF setting. 

\paragraph{General structure.} Both protocols take $3$ rounds and involve 2 messages from the server to the client and 1 message from the client to the server. The first server message however, is only required when the key needs to be changed (or re-masked). We call this the key-update phase. Now, when the masked key is already known, our protocols are optimal in the sense that they require only a single message from the client followed by a single message from the server. Since OPRF applications usually involve reusing the same key for many PRF invocations, in such a \textit{multi-input} setting, our protocols are comparable to other 2-round OPRF protocols in literature (e.g., DDH-based). 
\greg{For a real-world use of OPRFs where our construction might be the best option, see \url{https://www.usenix.org/system/files/sec19-thomas.pdf}. 
It uses the DDH-based OPRF for a private set intersection protocol (where the client has one element and the server has a masssive set). }

We detail the two protocols, $\prot^{\textsf{oprf}}_1$ and $\prot^{\textsf{oprf}}_2$, in Sections~\ref{subsec:oprf1} and~\ref{subsec:oprf2} respectively. In Section~\ref{subsec:oprf-comparison}, we compare our OPRF protocols to other common constructions in the literature. Later, in Section~\ref{sec:implementation_and_eval}, we also report on our protocol implementations and compare their performance (both computation and communication) with related work. To foreshadow, a key observation is that in comparison to common OPRF protocols, our protocols are much faster to compute but require preprocessing as well as slightly more communication.


\subsubsection{Oblivious PRF Protocol $\prot^{\textsf{oprf}}_1$}
\label{subsec:oprf1}
Our first OPRF protocol is in spirit similar to the distributed evaluation for the \ttwPRF construction. Since $\mat{K}$ is known to the server, and $x$ is known to the client, both parties do not need to exchange their shares to reconstruct the masked values $\hat{\mat{K}}$ and $\hat{x}$; the party that holds a value can mask it locally and send it to the other party. This allows us to decouple the server's message that masks its PRF key from the rest of the evaluation. To update the key, the server can simply send $\hat{\mat{K}} = \mat{K} + \tilde{\mat{K}}$ to the client. Many PRF evaluations can now be done using the same $\hat{\mat{K}}$.

\paragraph{Preprocessing.} The protocol requires the following preprocessed randomness. The mask $\tilde{\mat{K}}$ is given to the server only when the key-update phase needs to be run. For PRF evaluations, the trusted dealer samples $\tilde{w} \getsr \Z_2^m$ and provides the server and client $\Z_2$ shares of $\tilde{w}$ along with $\Z_3$ shares of $r = \tilde{w}$. Additionally, the dealer also generates an OLE correlation pair $(\tilde{\mat{K}}, \tilde{v})$ and $(\tilde{x}, \hat{v})$ such that $\tilde{\mat{K}} \in \Z_2^{m \times n}$ is a random circulant matrix that is same for all correlations, $\tilde{v} \getsr \Z_2^m$, $\tilde{x} \getsr \Z_2^n$, and $\hat{v} = \tilde{\mat{K}}\tilde{x} + \tilde{v}$. The server is given $(\tilde{\mat{K}}, \tilde{v})$ while the client is given $(\tilde{x}, \hat{v})$. Note that we simply use OLE correlations and do not make use of an actual OLE protocol. In practice, if the key-update phase is run after every $k$ evaluations (where $k$ is known), the OLE correlations for all evaluations can be preprocessed at the beginning. \mahimna{need to write about / point to another section for generating the OLE correlations.}

\paragraph{Protocol details.} Assuming that the masked key $\hat{\mat{K}}$ is known to the client, for an input $x$,  the evaluation protocol now proceeds as follows:
\begin{itemize}
  \item The client computes $\hat{x} = x + \tilde{x}$ and $\share{\hat{w}}^{(2)} = -\mat{\hat{K}}\tilde{x} + \hat{v} + \share{\tilde{w}}^{(2)}$ and sends both $\hat{x}$ and $\share{\hat{w}}^{(2)}$ to the server.

  \item The server first computes $\share{\hat{w}}^{(1)} = \mat{K}\hat{x} - \tilde{v} + \share{\tilde{w}}^{(1)}$, and adds to it the client's share to reconstruct $\hat{w}$. Identical to the distributed protocol, the server now runs $\prot_\Convert^{(2,3)}$ followed by $\prot_\LMap^{\mat{B},3}$ to obtain its share $\share{y}^{(1)}$ of the PRF output. Finally, it sends both $\hat{w}$ and $\share{y}^{(1)}$ to the client.

  \item The client also runs $\prot_\Convert^{(2,3)}$ followed by $\prot_\LMap^{\mat{B},3}$ to obtain its share $\share{y}^{(2)}$ of the PRF output. It can now use the server's share to reconstruct the PRF output $y$.

  % It now runs $\prot_\Convert^{(2,3)}(\hat{w} \mid \share{r}^{(1)})$ to get $\share{w^*}^{(1)}$ (over $\Z_3$) and then $\prot_\LMap^{\mat{B},3}(\mat{B} \mid \share{w^*}^{(1)})$ to obtain its share $\share{y}^{(1)}$ of the PRF output. Finally, it sends both $\hat{w}$ and $\share{y}^{(1)}$ to the client.

  % \item The client now runs $\prot_\Convert^{(2,3)}(\hat{w} \mid \share{r}^{(2)})$ to get $\share{w^*}^{(2)}$ (over $\Z_3$) and then $\prot_\LMap^{\mat{B},3}(\mat{B} \mid \share{w^*}^{(2)})$ to obtain $\share{y}^{(2)}$. Finally, it uses the share received from the server to reconstruct the PRF output $y$ 
\end{itemize}
For evaluating a client input, $\prot^{\textsf{oprf}}_1$ takes 2 rounds and involves a single message in each direction. The client sends $2n$ bits while the server sends $m$ bits and $t$ $\Z_3$ elements. For our parameters ($n=m=256, t=81$), and with proper $\Z_3$ packing, this amounts to roughly $897$ bits of total online communication. To update $\tilde{\mat{K}}$, the server sends a 256-bit message to the client.


\subsubsection{Oblivious PRF Protocol $\prot^{\textsf{oprf}}_2$}
\label{subsec:oprf2}
For the second protocol, the server masks the PRF in a different way; a multiplicative mask is used instead of an additive one. For simplicity, suppose that $n=m$ and that the key $\mat{K}$ is a random full-rank circulant matrix. Then to mask $\mat{K}$, the server computes $\bar{\mat{K}} = \mat{R}\mat{K}$ using a random matrix $\mat{R}$ that is also full-rank and circulant. $\mat{R}$ will be provided as preprocessing to the server, and can be reused for multiple PRF evaluations. The server will send $\bar{\mat{K}}$ to the client in the key-update phase. Note that since the product of two circulant matrices is also circulant, this message is only $n$ bits. Additionally, since the product $\mat{R}\mat{K}$ is essentially a convolution, it can be efficiently computed in $\Theta(n\log n)$ asymptotic runtime using the fast Fourier transform (FFT) algorithm.

\paragraph{Preprocessing.} The protocol requires the following preprocessed randomness. The mask $\mat{R}$ is given to the server only when the key-update phase needs to be run. For PRF evaluations, similar to first protocol, the dealer samples $w \getsr \Z_2^m$ and provides the server and client $\Z_2$ shares of $\tilde{w}$ along with $\Z_3$ shares of $r = \tilde{w}$. Additionally, the dealer gives $\tilde{u} \getsr \Z_2^m$ to the client and $\tilde{v} = \mat{R}^{-1}\tilde{u} + \tilde{w}$ to the server.

\paragraph{Protocol details.} Now, assuming that the masked key $\bar{\mat{K}}$ is known to the client, for an input $x$,  the evaluation protocol now proceeds as follows:
\begin{itemize}
  \item The client computes $\hat{u} = \bar{\mat{K}}x + \tilde{u}$ and sends it to the server.

  \item The server first computes $\mat{R}^{-1}\hat{u} + \tilde{v} = \mat{R}^{-1}(\mat{R}\mat{K}x + \tilde{u}) + (\mat{R}^{-1}\tilde{u} + \tilde{w}) = \hat{w} \mod 2$. Identical to the distributed protocol, the server now runs $\prot_\Convert^{(2,3)}$ followed by $\prot_\LMap^{\mat{B},3}$ to obtain its share $\share{y}^{(1)}$ of the PRF output. Finally, it sends both $\hat{w}$ and $\share{y}^{(1)}$ to the client.

  \item The client also runs $\prot_\Convert^{(2,3)}$ followed by $\prot_\LMap^{\mat{B},3}$ to obtain its share $\share{y}^{(2)}$ of the PRF output. It can now use the server's share to reconstruct the PRF output $y$.
\end{itemize}
For evaluating a client input, $\prot^{\textsf{oprf}}_2$ also takes 2 rounds and involves a single message in each direction. The client sends $n$ bits while the server sends $m$ bits and $t$ $\Z_3$ elements. This is $n$ fewer bits of communication as compared to the first protocol. The key-update phase is slower however, since it involves a convolution rather than a simple vector addition. For our parameters ($n=m=256, t=81$), and with proper $\Z_3$ packing, this amounts to roughly $641$ bits of total online communication. To update $\bar{\mat{K}}$, the server sends a 256-bit message to the client.


\subsubsection{Comparison}

\paragraph{DDH-based OPRF.}
A simple and widely used OPRF is based on the Decision Diffie-Hellman (DDH) assumption. Abstractly, given a cyclic group $\mathbb{G}$ of prime order $q$, consider a PRF $\mathsf{F}$ defined as follows: For key $k \in \keyspace$ and input $x \in \inspace$, define $\mathsf{F}(k,x) = H(x)^k$ where $H:\inspace \to \mathbb{G}$ is a hash function modeled as a random oracle. $\mathsf{F}$ is a secure PRF under the DDH assumption in the random oracle model~\cite{naor1999-oprf}.

The PRF $\mathsf{F}$ leads to a natural 2-party (semi-honest) protocol for oblivious evaluation. Concretely, suppose that the client holds $x$ and the server holds $k$. To evaluate the PRF obliviously, the client initiates the interaction by first sampling a mask $r \getsr \mathbb{Z}_q$ and then sending $a \gets H(x)^r$ to the server. The server responds with $b \gets a^k$. Finally, the client can retrieve the PRF computation as $y \gets b^{r^{-1}}$ where $r^{-1}$ is the multiplicative inverse of $r$ in $\Z_q$. Security of this OPRF protocol has been shown in~\cite{}, assuming the one-more discrete-log assumption~\cite{bellare2003-onemore}
% Also cite the security papers

\greg{One drawback of this is if the client becomes malicious, they can send a
previous server output back as input, and get the sequence $g, g^k, g^{k^2},
\ldots, g^{k^t}$. This is sometimes called a DH oracle, which lets one solve an
instance of the strong DH problem, which is easier than the DH problem
($O(\sqrt{q/t})$ group operations instead of $O(\sqrt{q})$).  Is there a similar
problem with our primitives? 
 }



%-------------------%
\subsection{Generating the correlated randomness.}



In this section, we show how to efficiently generate these correlations given OT correlations.







































\section{Table}

\mahimna{Placeholder section for cost table. Will some somewhere else later}


\newcommand{\formula}[1]{{\color{red} #1}}

\begin{table}[h]
{
\centering
\resizebox{\textwidth}{!}{%
  \begin{tabular}{|c|c|c|c|c|c|c|c|c|c|}
  \hline
  \multirow{3}{*}{Primitive} & \multirow{3}{*}{Construction} & \multirow{3}{*}{\makecell{Param. \\ $(n,m,t)$}} & \multicolumn{3}{c|}{\makecell{Distributed 2PC \\ (with preprocessing)}} & \makecell{Distributed \\ 3PC} &  \multicolumn{3}{c|}{\makecell{Public-Input 2PC \\ (with preprocessing)}} \\
  \cline{4-10}
  & & & \makecell{Online \\ Comm.} & \makecell{Prepr.} & \makecell{Offline \\ Comm.} & \makecell{Online \\ Comm.} & \makecell{Online \\ Comm.} & \makecell{Prepr.} & \makecell{Offline \\ Comm.} \\
  \hline \hline

  \multirow{3}{*}{wPRF} & \makecell{\ttwPRF} & $(256,256,81)$ & $(1536,4,2)$ & $(2348,662)$ & & $(1430,4,1)$ & $(512,2,1)$ & $(1324, 406)$ &\\
  & $(3,2)$-wPRF & $\formula{(n,m,t)}$ & $(\formula{4cn+2cm},4,2)$ & & & & $(\formula{2cm},2,1)$ & & \\
  % 2PC 
  % comm = 4cn + 2cm, 
  %preprocessing = (2(2cn+cm+2m), cm + 2m)

  % Public-input
  % Comm = 2cm
  % preprocessing = (2(cm+2m), 2m)

  & LPN-style & $(256,256,128)$ & $(2860,6,3)$ & $(4995,1730)$ & & &$(1324,4,2)$ & $(3160,918)$ &  \\

  \hline

  \multirow{3}{*}{OWF} & \makecell{\ttOWF} & $(128, 452, 81)$ & $(904,2,1)$ & $(2337, 717)$ & & & - & - & - \\
  & \makecell{$(3,2)$-OWF} &  &  &  & & & - & - & - \\
  & \makecell{LPN-OWF} &  &  &  & & & - & - & - \\

  \hline

  PRG & LPN-PRG & $(128,512,256)$ & $(1880, 4, 2)$ & $(4334, 1227)$ & & & - & - & - \\
  % 2PC 
  % Communication (2n+2cm) 
  % Preprocessing (5.17n + 7.17m, 1.585n + 2m)
  \hline
  \end{tabular}
}
\caption{Concrete MPC costs for our candidate constructions in three settings (Distributed 2PC (with preprocessing), 3PC, and Public-input 2PC) using our proposed parameters. For the distributed 2PC and the public-input 2PC settings, we provide the total online communication (bits, messages, rounds), preprocessing required in bits (without compression, with compression), and offline communication (bits, messages, rounds) to generate the preprocessing. For the compressed size of the preprocessing, we do not include values that can be reused (e.g., PRG seeds). For the distributed 3PC setting, we provide the total online communication cost (bits, messages, rounds) and do not include the reusable PRG seeds.}


}
\mahimna{Should we count PRG size in the compression.}
\end{table}

%!TEX root = ../main.tex

\newcommand{\secpar}{\kappa}
\newcommand{\aux}{\textsf{aux}}
\newcommand{\com}{\textsf{com}}
\newcommand{\CCC}{\mathcal{C}}
\newcommand{\PPP}{\mathcal{P}}
\newcommand{\ch}{\textsf{ch}}
\newcommand{\hash}{\textsf{H}}
\newcommand{\state}{\textsf{state}}
\newcommand{\seed}{\textsf{seed}}
\newcommand{\msgs}{\textsf{msgs}}
\newcommand{\Msg}{\textsf{Msg}}
\newcommand{\Expand}{\textsf{Expand}}
\newcommand{\algcomment}[1]{{\color{Gray}{\qquad//#1}}}
\newcommand{\bmat}[1]{\ensuremath{\mathbf{#1}}}
\newcommand{\matA}{\bmat{A}}
\newcommand{\matB}{\bmat{B}}

 \newtcolorbox{titlebox}[5]{
    enhanced,
    colframe=black,
    colback=white,
    boxrule={#3},
    arc={#2},
    auto outer arc,%
%   breakable,
    pad at break*=0pt,
    vfill before first,
    before={\par\medskip\noindent},
    after={\par\medskip},
    top=12pt, left=4pt, enlarge top by=7pt,%enlarge bottom by=7pt,%
    title={\rule[-.3\baselineskip]{0pt}{\baselineskip}\normalsize\sffamily\bfseries #1}, 
    varwidth boxed title*=-30pt, 
    attach boxed title to top left={yshift=-10pt,xshift=10pt}, 
    coltitle=black,
    boxed title style={colback=white,boxrule={#5},arc={#4},auto outer arc}
 }

 \newenvironment{systembox}[1]
 {\vspace{\baselineskip}\begin{titlebox}{Functionality \normalfont #1}{2.5pt}{1pt}{3.5pt}{1pt}}
 {\end{titlebox}}

 \newenvironment{protocolbox}[1]
 {\begin{titlebox}{ #1}{0.5pt}{0.5pt}{1pt}{0.75pt}}
 {\end{titlebox}}

\section{Application: Signatures with the \ttOWF}
\label{sec:applications}

Here we describe a signature scheme using the \ttOWF.  
Our presentation is tailored to the \ttOWF, but we note that this approach is
general.  All of the candidate primitives in this paper would be a suitable
choice of $\sfF$ (note that they are all OWFs when the input is chosen at
random) and we evaluated them all before settling on \ttOWF, which gives the
shortest signatures.

Abstractly, a signature scheme can be built from any one-way function $\sfF$
that has an MPC protocol to evaluate it, by setting the public key to $y =
\sfF(x)$ for a random secret $x$, and then proving knowledge of $x$, using a
proof system based on the MPC-in-the-head paradigm~\cite{ishai2007-zkmpc}.  In
addition to assuming the OWF is secure, the only other assumption required is a
secure hash function. As no additional number-theoretic assumptions are
required, these types of signatures are often proposed as secure post-quantum
schemes. 

Concretely, our design follows the Picnic signature scheme~\cite{chase2017-picnic},
specifically the variant instantiated with the KKW proof
system~\cite{CCS:KatKolWan18} (named Picnic2 and Picnic3).  We chose to use the
KKW, rather than ZKB++ proof system since our MPC protocol to evaluate the
\ttOWF is most efficient with a pre-processing phase, and KKW generally
produces shorter signatures.  We replace the LowMC block cipher~\cite{albrecht2015-lowmc} in Picnic 
with the \ttOWF, and make the corresponding changes to the MPC protocol. 

This is the first signature scheme based on the hardness of inverting the
\ttOWF (or similar function), a function with a simple mathematical
description, making it an accessible target for cryptanalysis, especially when
compared to block ciphers.  Arguably, the simplicity of the OWF can lead to
simpler implementations: the MPC protocol is simpler, and no large precomputed
constants are required. 

Our presentation is somewhat brief here, as many details are identical to Picnic, 
and the \ttOWF MPC has been described in \cref{sec:distributed_protocols}. 
\cref{appendix:picnic} includes additional details. 

\paragraph{Parameters.} Let $\secpar$ be a security parameter.  The
\ttOWF parameters are denoted $(n, m, t)$.  The KKW parameters $(N, M,
\tau)$ denote the number of parties $N$, the total number of MPC instances $M$,
and the number $\tau$ of MPC instances where the verifier checks the online
phase of simulation.  The scheme also requires a cryptographic hash function. 

\paragraph{Key generation.}
The signer chooses a random $x \in \Z_2^n$ as 
secret key, and a random seed $s \in \{0,1\}^{\secpar}$ such that $s$
expands to matrices $\matA \in \Z_2^{m\times n}$ and $\matB\in \Z_3^{m\times t}$ that are full rank (using a suitable cryptographic
function, such as the SHAKE extendable output function~\cite{sp800-185}).
Compute $y = \sfF(x)$ and set $(y, s)$ as the public key.
Recall that the \ttOWF is defined as $y = \sfF(x)$ where $x \in \Z_2^{n}$ and $y\in \Z_3^{t}$, and is computed as $y = \mat{B}(\mat{A}x)$ where $\mat{A}x$ is first cast to $\Z_3$. 
\iffull
We use unique $\matA$ and $\matB$ per signer in order to avoid multi-target attacks against $\sfF$.
We specify the scheme using random matrices, however, it could also be instantiated
with circulant matrices, which may improve the performance of sign and verify operations.
\fi

\paragraph{MPC protocol.}
By combining the protocols for the gates 
$\prot_{\Add}^{3}$,
$\prot_{\LMap}^{\mat{A},2}$,
$\prot_{\LMap}^{\mat{B},3}$, and
$\prot_{\Convert}^{(2,3)}$
described in \cref{sec:distributed_protocols} we have an $N$-party protocol for
the \ttOWF. The most challenging and costly step (in terms of communication) is
the conversion gate, all other operations are done locally by the parties.  In
\cref{appendix:picnic} we describe this protocol in full detail.  

\paragraph{Sign and verify.} 
The signature generation and verification algorithms for the
\ttOWF  signature scheme are given in \cref{fig:23-picnic} (of \cref{appendix:picnic}).
Here we give an overview. The
prover simulates the preprocessing and online phase for all $M$ MPC instances, and commits
to the preprocessing values, and MPC inputs and outputs.
Then she is challenged to open $\tau$ of the $M$ MPC instances.
The verifier will check the simulation of the online phase for these instances,
by re-computing all values as the prover did for $N-1$ of the parties, and for
remaining unopened party, the prover will provide the missing broadcast messages
and commitments so that the verifier may complete the simulation and recompute all
commitments. 
For the $M-\tau$ instances not chosen by the challenge, the verifier will check
the preprocessing phase only, by recomputing the preprocessing phase as the prover did. 

\newcommand{\textunderbrace}[2]{{%
  \underbrace{#1}_{\text{#2}}
}}

\paragraph{Parameter selection and signature size.}
In \cref{appendix:picnic} we give the formula for estimating signature size.
The impact of OWF choice is limited to one term, which is the sum of
the sizes of the MPC inputs, broadcast messages, and auxiliary values produced
by preprocessing.  Selecting the KKW parameters $(M, N, \tau)$ once the MPC costs
are known follows the approach in Picnic: a range of options are possible, and
we try to select parameters that balance speed (mostly dependent on the number of MPC executions and number of 
parties) and size.  Since the MPC costs of the \ttOWF are very close to those
of LowMC, the options follow a similar curve. 

\cref{table:sig-sizes} gives some options with $N=16, 64$ parties,
providing 128 and 256 bits of security. For each category, we highlight the row
of \ttOWF parameters that are a direct comparison to Picnic.  Signatures using
the \ttOWF are slightly shorter (five to fifteen percent) than Picnic using
LowMC. 


% \begin{table}[t]
% \begin{centering}
% \begin{tabular}{l|l|l}
% OWF Params      & KKW params        & Sig. size (KB)\\
% $(n,m,t)$       & $(N, M, \tau)$    &   \\\hline
% $(128,453,81)$  & $(16, 150, 51)$   & 13.30 \\
%                 & $(16, 168, 45)$   & 12.48 \\ 
%                 & $(16, 250, 36)$   & \textbf{11.54} \\
%                 & $(16, 303, 34)$   & 11.49 \\  
% Picnic3-L1      & $(16, 250, 36)$   & 12.60 \\ \hline
% $(128,453,81)$  & $(64, 151, 45)$   & 13.59 \\
%                 & $(64, 209, 34)$   & 11.70 \\ 
%                 & $(64, 343, 27)$   & \textbf{10.66} \\
%                 & $(64, 515, 24)$   & 10.35 \\ 
% Picnic2-L1      & $(64, 343, 27)$   & 12.36 \\ \hline \hline
% $(256,906,162)$ & $(16, 324, 92)$   & 50.19 \\ 
%                 & $(16, 400, 79)$   & 47.08 \\ 
%                 & $(16, 604, 68)$   & \textbf{45.82} \\ 
% Picnic3-L5      & $(16, 604, 68)$   & 48.72 \\ \hline
% $(256,906,162)$ & $(64, 322, 82)$   & 51.23 \\
%                 & $(64, 518, 60)$   & 44.04 \\ 
%                 & $(64, 604, 57)$   & \textbf{43.45} \\ 
% Picnic2-L5      & $(64, 604, 58)$   & 46.18 \\ \hline
% \end{tabular}
% \caption{ \label{table:sig-sizes}Signature size estimates for Picnic using
% \ttOWF, compared to Picnic using LowMC.  The first half of the table shows
% security level L1 (128 bits) with $N=16$ and $N=64$ parties, and the second
% half shows level L5 (256 bits).}
% \end{centering}
% \end{table}



\iffull 
\begin{table}[t]
\resizebox{\textwidth}{!}{ 
\begin{tabular}{l|l|l}
OWF Params      & KKW params        & Sig. size (KB)\\
$(n,m,t)$       & $(N, M, \tau)$    &   \\\hline
$(128,453,81)$  & $(16, 150, 51)$   & 13.30 \\
                & $(16, 168, 45)$   & 12.48 \\ 
                & $(16, 250, 36)$   & \textbf{11.54} \\
Picnic3-L1      & $(16, 250, 36)$   & 12.60 \\ \hline
$(128,453,81)$  & $(64, 151, 45)$   & 13.59 \\
                & $(64, 209, 34)$   & 11.70 \\ 
                & $(64, 343, 27)$   & \textbf{10.66} \\
Picnic2-L1      & $(64, 343, 27)$   & 12.36 \\ \hline
\end{tabular}

\quad
\begin{tabular}{l|l|l}
OWF Params      & KKW params        & Sig. size (KB)\\
$(n,m,t)$       & $(N, M, \tau)$    &   \\\hline
$(256,906,162)$ & $(16, 324, 92)$   & 50.19 \\ 
                & $(16, 400, 79)$   & 47.08 \\ 
                & $(16, 604, 68)$   & \textbf{45.82} \\ 
Picnic3-L5      & $(16, 604, 68)$   & 48.72 \\ \hline
$(256,906,162)$ & $(64, 322, 82)$   & 51.23 \\
                & $(64, 518, 60)$   & 44.04 \\ 
                & $(64, 604, 57)$   & \textbf{43.45} \\ 
Picnic2-L5      & $(64, 604, 58)$   & 46.18 \\ \hline
\end{tabular}
 }
\caption{ \label{table:sig-sizes}Signature size estimates for Picnic using
\ttOWF, compared to Picnic using LowMC.  The left table shows
security level L1 (128 bits) with $N=16$ and $N=64$ parties, and the right table shows level L5 (256 bits).}
\end{table}

\fi

%!TEX root = ../main.tex

\newpage
\section{Implementation and Evaluation}
\label{sec:implementation_and_eval}
We implemented our 2-party 23-wPRF (Construction~\ref{construction:23-central-wprf}) and 23-OPRF (Construction~\ref{construction:23-owf}) in C++. For the constructions, we used the parameters $n = m = 256$ and $t = 81$. In other words, the implemented 23-constructions use a circulant matrix in $\Z_2^{256 \times 256}$ as the key, take as input a vector in $\Z_2^{256}$ and output a vector in $\Z_3^{81}$. The correlated randomness was implemented as if provided by a trusted third party.  See Section~\ref{to-add} for concretely efficient protocols for securely generating the correlated randomness, which we did not implement but give efficiency estimates based on prior works.  \yuval{Add pointer when section is written.}

\subsection{Optimizations}
\label{subsec:implementation_opt}
We start with a centralized implementation of the 23-wPRF. We find optimizations that provide 5x better performance over a na\"ive implementation, which we detail below. Later, in Table~\ref{table:optimization_benchmarks}, we compare the performance gain from these optimizations.

\paragraph{Bit packing for $\Z_2$ vectors.} 
Instead of representing each element in a $\Z_2$ vector separately, we pack several elements into a machine word and operate on them together in an SIMD manner. For our architecture with 64-bit machine words, we can pack a vector in $\Z_2^{256}$ (e.g., the input $x$) into 4 words. Since the key $\mat{K}$ is circulant and can be represented with $n=256$ bits, it can also be represented by 4 words. This results in a theoretical $\times64$ maximum speedup in run-time for operations involving $\mat{K}$ and $x$. 

\paragraph{Bit slicing for $\Z_3$ vectors.}
We represent each element in $\Z_3$ using the two bits from its binary representation. For $z \in \Z_3$, the two bits are the least significant bit (LSB) $l_z = z \bmod 2$, and the most significant bit (MSB) $h_z$ which is $1$ if $z=2$ and $0$ otherwise. $\Z_3$ vectors are now also represented by two binary vectors, one containing the MSBs, and one containing the LSBs. Operations involving a $\Z_3$ vector are done on these binary vectors instead. In Table~\ref{table:z3_operations}, we specify how we perform common operations on $\Z_3$ elements using our bit slicing approach. We also take advantage of the bit packing optimization when operating on the binary vectors.


\begin{table}[ht]
\begin{center}
    \begin{tabular}{|c|c|c|}
        \hline
        Operation & Result MSB & Result LSB\\
        \hline\hline
        $z_1 + z_2 \bmod 3$ & $(l_1 \vee l_2) \oplus (l_1 \vee h_2) \oplus (l_2 \vee h_1)$  & $(h_1 \vee h_2) \oplus (l_1 \vee h_2) \oplus (l_2 \vee h_1)$\\
        $-z_1 \bmod 3$ & $l_1$ & $h_1$ \\
        $z_1z_2 \bmod 3$ & $(l_1 \wedge l_2) \oplus (h_1 \wedge h_2)$ & $(l_1 \wedge h_2) \oplus (h_1 \wedge l_2)$\\
        $\textrm{MUX}(s;z_1,z_2)$ & $(h_2 \wedge s) \vee (h_1 \wedge \neg s)$ & $(l_2 \wedge s) \vee (l_1 \wedge \neg s)$\\
        \hline
    \end{tabular}
\end{center}
\caption{Operations in $\Z_3$. $z_1$ and $z_2$ are elements in $\Z_3$ with (MSB, LSB) = $(h_1,l_1)$ and $(h_2, l_2)$ respectively. For a bit $s$, the operation $\textrm{MUX}(s;z_1, z_2)$ outputs $z_1$ when $s=0$ and $z_2$ when $s=1$.}
\label{table:z3_operations}
\mahimna{I'm debating whether we need to have this table at all. Once we specify that we split $\Z_3$ elements into 2 bits, for any $\Z_3$ operation, the corresponding operation on the bits can easily be computed using a truth table.}
\end{table}


\paragraph{Lookup table for matrix multiplication.} Recall that the \ttwPRF evaluation contains a mod-3 linear map using a public matrix $\mat{B} \in \Z_3^{81\times256}$. Specifically, it computes the matrix-vector product $\mat{B}w$ where $w \in \Z_3^{256}$. Since $\mat{B}$ is known prior to evaluation, we can use a lookup table to speedup the multiplication by $\mat{B}$. The same preprocessing can also by reused for multiple evaluations of the wPRF.

For this, we partition $\mat{B}$, which has $m=256$ columns, into $16$ slices of $16$ columns each. These matrices, denoted by $\mat{B}_1,\dots, \mat{B}_{16}$, are all in $\Z_3^{81 \times 16}$. Now, for each $\mat{B}_i$, we will effectively build a lookup table for its multiplication with any $\Z_3$ vector of length $16$. A point to note here is that since we  represent $\Z_3$ vectors by two binary vectors (from the bit slicing optimization), it is sufficient to preprocess multiplications (modulo 3) for binary vectors of length $16$. To multiply $\mat{B}_{i}$ by a vector in $\Z_3^{16}$, we can first multiply it separately by the corresponding MSB and LSB vectors, and then subtract the former from the latter modulo 3. This works since for $z_1, z_2 \in \Z_3$ the multiplication $z_1z_2 \bmod 3$ can be given by $z_1(2\cdot h_2 + l_2) \bmod 3 = z_1l_2 - z_1h_2 \bmod 3$ where $h_2, l_2$ are the MSB and LSB of $z_2$ respectively. Now, to multiply $\mat{B}$ by $v \in \Z_3^{256}$, we first evaluate all multiplications of the form $\mat{B}_iv_i$ where $v_i$ is the $\Z_3^{16}$ vector denoting the $i^\thtext$ slice of $v$ if it was split into 16-element chunks. Then, multiplication by $\mat{B}$ is given by $\mat{B}v = \sum_{i \in [16]} \mat{B}_i v_i \mod 3$.
\greg{Is this the ``method of four Russians''?\url{https://github.com/malb/m4ri}, \url{https://bitbucket.org/malb/m4ri/wiki/Further\%20Reading} }
\greg{$w$ is secret, so the access pattern of the $\mat{B}_i$ should be independent of $w$. Is this the case? If not there is a 
cache timing side-channel. The timing variation may be observable over the network (e.g., to the other party in the OPRF evaluation).  }

In general, a $\mat{B} \in \Z_3^{t \times m}$ can be partitioned into $m/c$ partitions with $c$ columns each (assume $c$ divides $m$ for simplicity), and would require a total multiplication lookup table size of $(m/c) \cdot 2^c \cdot t$ $\Z_3$ elements. Multiplying $\mat{B}$ by $v \in \Z_3^{m}$ requires $2(m/c)$ lookup table accesses (one each for MSB and LSB of $v$ per partition of $\mat{B}$) and an addition of $(m/c - 1)$ $\Z_3^{t}$ vectors.


For our parameters, this results in a table size of roughly $135$MB with proper $\Z_3$ packing. We chose $c = 16$ as a compromise between the size of the lookup table and increased computational efficiency.












% \subsection{Representing $Z_2$ vector - bit packing}

% The dark matter function manipulates elements of $Z_2$ and $Z_3$. However, machine words can hold up to 64 bits and we use this to represent many elements in just a few machine words, using bit slicing. Namely, we used each machine word as a vector of 64 bits and applied operations to this bit-vector in a SIMD manner.
% Since in our implementation each key $\textbf{K}$  is of size $m \times n = 256 \times 256$ and each input is a vector of size $n = 256$, this results in packing each key into $\dfrac {256}{64} = 4$ words and each input vector into 4 words. This may result in time saving of up to $\times 64$ of the run-time.

% \subsection{Representing $Z_3$ vector - bit slicing}

% The last part of the dark matter function takes an intermediate vector viewed as a vector of $Z_3$ elements, which is then multiplied 
% by a matrix $R$ in $Z_3$ to produce the output vector in $Z_3$. We tried different methods for implementing this matrix-vector multiplication over $Z_3$.:	

% Bit slicing is implemented  by representing a vector over $Z_3$  as two binary vectors - one LSB's and one MSB's. The operations over this $Z_3$ vector - addition, subtraction, multiplication and MUX  - were implemented in a bitwise manner as shown in table \ref{tab:multicol}. We took advantage of the fact that when representing each vector as MSB and LSB, negation (which is the same as multiplying by two) is the same as swapping the most and least significant bits. 

% \begin{table}[ht]
% 	\caption{Operations in $Z_3$. input:   ${l_1}, {m_1}$ - LSB and MSB  of first trinary number, 
% 		${l_2}, {m_2}$ - LSB and MSB  of second trinary number,  ${s}$ - selection bit for MUX}
% 	\begin{center}
% 		\begin{tabular}{|c|l|}
% 			\hline
% 			\textbf{Operations} & \textbf{Methods}\\
% 			\hline
% 			\multirow{3}{*}{Addition} & ${t} := ({l_1 \wedge m_2}) \oplus ({l_2 \wedge m_1})$\\
% 			& $m_{\mathrm{out}} := ( l_1 \wedge  l_2 ) \oplus  t $ \\
% 			& $l_{\mathrm{out}} :=m_1 \wedge m_2 ) \oplus t $ \\
% 			\hline
% 			\multirow{3}{*}{Subtraction} & ${t} := ({l_1} \wedge {l_2}) \oplus ({m_2} \wedge {m_1})$;\\
% 			& $m_{\mathrm{out}} := (l_1 \wedge m_2 ) \oplus t$;\\
% 			& $l_{\mathrm{out}} := (m_1 \wedge l_2 ) \oplus t$; \\
% 			\hline
% 			\multirow{3}{*}{Multiplication} & $m_{\mathrm{out}}:= (l_1 \vee m_2) \wedge   (m_1 \vee l_2)$; \\
% 			& $l_{\mathrm{out}} := (l_1 \vee l_2) \wedge     (m_1 \vee m_2)$;\\
% 			\hline
% 			\multirow{3}{*}{MUX} & $m_{\mathrm{out}} :=( m_2 \vee s) \wedge (m_1 \vee (\bar{s}) )$; \\
% 			& $l_{\mathrm{out}}[i] :=( l_2 \vee s) \wedge (l_1 \vee (\bar{s}) )$; \\
% 			\hline
			
% 		\end{tabular}
% 	\end{center}
% 	\label{tab:multicol}
% \end{table}

% \begin{table}[ht]
%     \caption{Operations in $Z_3$. input:   ${l_1}, {m_1}$ - LSB and MSB  of first trinary number, ${l_2}, {m_2}$ - LSB and MSB  of second trinary number,  ${s}$ - selection bit for MUX}
%     \begin{center}
%         \begin{tabular}{|c|l|}
%             \hline
%             \textbf{Operations} & \textbf{Methods}\\
%             \hline
%             \multirow{3}{*}{Addition} & ${t} := ({l_1 \BitOr m_2}) \oplus ({l_2 \BitOr m_1})$\\
%             & $m_{\mathrm{out}} := ( l_1 \BitOr  l_2 ) \oplus  t $ \\
%             & $l_{\mathrm{out}} :=(m_1 \BitOr m_2 ) \oplus t $ \\
%             \hline
%             \multirow{3}{*}{Subtraction} & ${t} := ({l_1} \BitOr {l_2}) \oplus ({m_2} \BitOr {m_1})$;\\
%             & $m_{\mathrm{out}} := (l_1 \BitOr m_2 ) \oplus t$;\\
%             & $l_{\mathrm{out}} := (m_1 \BitOr l_2 ) \oplus t$; \\
%             \hline
%             \multirow{3}{*}{Multiplication} & $m_{\mathrm{out}}:= (l_1 \BitAnd m_2) \oplus   (m_1 \BitAnd l_2)$; \\
%             & $l_{\mathrm{out}} := (l_1 \BitAnd l_2) \oplus   (m_1 \BitAnd m_2)$;\\
%             \hline
%             \multirow{3}{*}{MUX} & $m_{\mathrm{out}} :=( m_2 \BitAnd s) \BitOr (m_1 \BitAnd (\bar{s}) )$; \\
%             & $l_{\mathrm{out}} :=( l_2 \BitAnd s) \BitOr (l_1 \BitAnd (\bar{s}) )$; \\
%             \hline              
%         \end{tabular}
%     \end{center}
%     \label{tab:multicol}
% \end{table}


%Lookup table Implemetation
% \paragraph{Lookup table for matrix multiplication.}
% \mahimna{I added sufficient details from this into the earlier paragraph. If that looks good, then we can remove this}
% Even with bitslicing, implementing the matrix-vector multiplication column by column via the opertions from Table~1 is still rather slow.
% To get better performence, we capitalized on the fact that the random matrix $R$ is fixed, and we can therefore set up $R$-dependent tables and use table lookups in the implementation of the multiply-by-$R$ operation.

% Specifically, we partition the matrix $R$ (which has $m=256$ columns) into sixteen slices of 16 columns each, denoted $R_1,\ldots,R_{16}$.
% For each of these small matrices $R_i$, we then build a table with $2^{16}$ entries, holding the result of multiplying $R_i$ by every vector from $\{0,1\}^{16}$.
% Namely, let $R_{i,0},R_{i,1},\ldots,R_{i,15}$ be the sixteen columns of the matrix $R_i$.
% The table for $R_i$ is then defined as follows: For each index $0\le j < 2^{16}$, let $\vec{J}=(j_0,j_1,\ldots,j_{15})$ be the 0-1 vector holding the bits in the binary representation of~$j$, then we have
% \[
% T_i[j] = R_i \times \vec{J} = \sum_{k=0}^{15} R_{i,k} \cdot j_k ~~\bmod 3.
% \]
% Recalling that $R$ is $Z_3$ matrix of dimension $81\times 256$, every entry in each table~$T_i$ therefore holds an 81-vector over $Z_3$.
% Specifically, it holds a packed-$Z_3$ element with four words (two for the MSBs and two for the LSBs of this 81-element vector).

% Note, however, that $T_i$ can only be used directly to multiply $R_i$ by \emph{0-1 vectors}.
% To use $T_i$ when multiplying $R_i$ by a $\{0,1,2\}$ vector, multiply $R_i$ separately by the MBSs and the LSBs of that vector, and then subtract one from the other using the operations from Table~1.
% To multiply a dimension-256 $\{0,1,2\}$ vector by the matrix $R$, we partition it into sixteen vectors of dimension 16, use the approach above to multiply each one of them by the corresponding $R_i$, then use the operations from Table~1 to add them all together.
% Hence we have

% \begin{tabbing}
% 	\underline{Multiply-by-$R$(input: $\vec{V}\in\{0,1,2\}^{256}$:}\\
% 	~~1. $Acc := 0$\\
% 	~~2. For \=$i=0,\ldots,15$\\
% 	~~3. \> Let $\vec{M}_i,\vec{L}_i\in\{0,1\}^{16}$ be the MSB, LSB vectors (resp.) of $\vec{V}_i=\vec{V}[16i,\ldots,16i+15]$\\
% 	~~4. \> Set the indexes $m:=\sum_{k=0}^{15}2^i\cdot\vec{M}[i]$
% 	and $\ell:=\sum_{k=0}^{15}2^i\cdot\vec{L}[i]$\\
% 	~~5. \> $Acc := Acc + T_i[\ell] - T_i[\ell] \bmod 3$\\
% 	~~6. Output $Acc$
% \end{tabbing}

% We note that a slightly faster implementation could be obtained by braking the matrix into (say) 26 slices of upto 10 columns each, and directly identify each 10-vector over $\{0,1,2\}$ with an index $i<3^{10}=59049$.
% This implementation would have only 26 lookup operations instead of 32 in the algorithm above, so we expect it to be about 20\% faster. On the other hand it would have almost twice the table size of the implementation from above.



%--------------------------------------%

\subsection{Performance Benchmarks}
\paragraph{Experimental setup.}
We ran all our experiments on a t2.medium AWS EC2 instance with 4GiB RAM (architecture: x86-64 Intel(R) Xeon(R) CPU E5-2686 v4 @ 2.30GHz) running on Ubuntu 18.04. The performance benchmarks and timing results we provide are averaged over 1000 runs. For the distributed construction benchmarks, both parties were run on the same instance. We separately report the computational runtime for the parties, and analytically compute the communication costs. 


\paragraph{Optimization results.}
We start with benchmarks for the different optimization techniques detailed in Section~\ref{subsec:implementation_opt}. Table~\ref{table:optimization_benchmarks} contains timing results for our centralized implementation of the \ttwPRF construction with parameters $n=m=256$ and $t=81$ using these optimizations. The table also reports results for a baseline implementation which uses the naive $\Theta(nm)$ algorithm for multiplication between a $m \times n$ matrix and a $n$ length vector. We find that our best implementation is 25x faster than the baseline implementation.

\begin{table}[t]
	{
		\centering
		\begin{tabular}{|c|c|c|c|c|}
			
			\hline
			\multicolumn{3}{|c|}{Optimization} &  \multirow{2}{*}{Runtime ($\mu$s)} & \multirow{2}{*}{Evaluations / sec} \\
			Packing & Bit Slicing & Lookup Table &  & \\
			\hline\hline
			\multicolumn{3}{|c|}{Baseline implementation} & 156.41 & 6K\\
			\checkmark & & & 26.84 & 37K\\
			\checkmark & \checkmark & & 18.5 & 65K \\
			\checkmark & \checkmark & \checkmark & 6.08 & 165K \\
			\hline
		\end{tabular}
		\caption{Centralized 23-wPRF benchmarks for a baseline implementation and for different optimization techniques. Packing was done into 64-bit sized words (for both $\Z_2$ and $\Z_3$ vectors). For the lookup table optimization, a table with $81 \times 2^{20}$ $\Z_3$ elements, or roughly of size $135$MB, was preprocessed. Runtimes are all given in microseconds ($\mu$s).
		% \yuval{Specify units in Runtime column. Memory Usage column: not sure what this means, and in any case the first two rows seem too small to be meaningful (e.g., the amount of memory used by the program for maintaining variables may be bigger). I suggest removing this column, and instead specifying the lookup table size in the caption. Alternatively, we can keep this column and change the title to ``lookup table size'', keeping the first two rows empty. Regarding this size, what does $2^{20}$ count? Bits? Machine words? $\Z_3$ elements?}
		}
		\label{table:optimization_benchmarks}
	}
\end{table}


\paragraph{Distributed wPRF evaluation.}
We also implement the 2-party semi-honest distributed protocol detailed in Section~\ref{subsec:2pc-wprf} for evaluating the \ttwPRF construction and report timings for our implementation. Further, since this candidate was first proposed in~\cite{boneh2018-darkmatter}, we also implement their protocol as a comparison point. For both protocols, we use the parameters $n=m=256$, $t=81$ for the PRF and use the same optimizations (from Section~\ref{subsec:implementation_opt}). We use a trusted party to generate the preprocessed randomness. We provide a comparison in Table~\ref{table:distributed_wprf}.


% We consider two 2-party weak PRF protocols. The first protocol is described in \cite{boneh2018-darkmatter} and the second one is our construction as described in \ref{newProtocol}. Since the goal of the implementation was timing the local {\em online}run-time, the implementation used a trusted party 

\begin{table}[t]
{
\centering
\begin{tabular}{|c|c|c|c|c|c|}
	\hline
	Protocol & Runtime ($\mu$s) & Evaluations / sec & \makecell{Preprocessing \\ (bits)} & \makecell{Communication \\ (bits)} \\
	\hline
	\hline
	\makecell{\ttwPRF Protocol \\~\cite{boneh2018-darkmatter}} & 28.02 & 36K & 3533 & 2612 \\
	\hline
	\makecell{\ttwPRF Protocol \\ (Section~\ref{subsec:2pc-wprf})} & 12.12 & 82K & 2348 & 1536 \\
	\hline	
\end{tabular}
		
\caption{Comparison of 2-party semi-honest protocols for distributed evaluation of the \ttwPRF candidate (Construction~\ref{construction:23-central-wprf}) in the preprocessing model. The parameters $n=m=256$ and $t=81$ were used for the wPRF. The runtime reported is the maximum between the two parties.}
\mahimna{The preprocessing costs do not take into account optimizations (e.g., using a PRG seed). Should we report the na\"{i}ve cost here or the optimized cost? Using a PRG, the total preprocessing (with $\Z_3$ compression) is 918 bits.}
\mahimna{If we are comparing only the TCC18 protocol with ours does it even make sense to have a table? Should we just provide the comparison in text?}
\label{table:distributed_wprf}
}
\end{table}

\paragraph{OPRF evaluation.}
In Table~\ref{table:oprf_comparison}, we provide performance benchmarks for both our oblivious protocols (see Section~\ref{subsec:oprf_protocol}) for the \ttwPRF construction. We also compare our results to a DDH-based OPRF~\cite{ddh-oprf}. For our timing results, we report both the server and client runtimes (averages over 1000 runs). For each construction, we also include the size of the preprocessed correlated randomness, and the online communication cost. All constructions are parameterized appropriately to provide 128-bit security.

For our constructions, we report separately, the timings for refreshing the key and evaluating the input. For the comparison with the DDH-based OPRF construction in~\cite{ddh-oprf}, we use the libsodium library~\cite{LibSodium} for the elliptic curve scalar multiplication operation. We use the Curve25519 elliptic curve, which has a 256-bit key size, and provides 128 bits of security. 
% \yuval{I actually think that the Naor-Pinkas-Reingold99 paper cited in the TCC '18 paper refers to a different notion of distributing a PRF, namely where the PRF key is distributed between two or more servers and the input is public. Let's try to figure out the source of the simple DDH-based OPRF protocol. Can ask Stas/Hugo if needed.}

We point out that both of our OPRF protocols are much faster than the DDH-based OPRF although they require higher communication as well as preprocessing. To better represent the tradeoffs, we compute the minimum network speed for which the total time for one evaluation of our protocol is smaller than that for the DDH-based OPRF. We compare only the online costs for the evaluation. Analytically, we can compute that our two protocols are faster than the DDH-based OPRF protocol when the network speed is greater than $\approx 5.7$Mbps and $\approx 1.85$Mbps respectively. On a 50Mbps network, The communication of OPRF Protocol 1 outpaces its respective computation. The same happens with OPRF protocol 2 on an 40 Mbps network. While these numbers only present a simplified picture, we note that they should depict the overall tradeoffs for our OPRF protocols.


% \yuval{Communication costs in bits can be computed analytically. Maybe talk about estimated breakeven points: (1) minimal network speed in which computation dominates communication, and (2) minimal network speed where we estimate our OPRF to be faster than the DDH-based alternative.  }

\begin{table}[h]
{
\centering
\resizebox{\textwidth}{!}{%
	\begin{tabular}{|cc|c|c|c|c|c|c|}%8 columns	
		\hline
		\multirow{2}{*}{Protocol} & & \multicolumn{2}{c|}{Runtime ($\mu$s)} & \multirow{2}{*}{Preprocessing (bits)} & \multicolumn{2}{c|}{Communication (bits)}\\
		& & Client & Server & & Client & Server\\
		\hline\hline
		\multirow{2}{*}{OPRF Protocol 1} & Key Update & - & 0.65 & 256 & - & 256 \\
		& Evaluation & 8.54 & 9.45 & 2092 & 512 & 385 \\
		\hline
		\multirow{2}{*}{OPRF Protocol 2} & Key Update & - & 3.16 & 256 & - & 256\\
		& Evaluation & 7.91 & 8.21 & 1836 & 256 & 385 \\
		\hline
		\multicolumn{2}{|c|}{DDH-based OPRF~\cite{ddh-oprf}} & 57.38 & 28.69 & - & 256 & 256\\
		\hline
	\end{tabular}}
	\caption{Comparison of protocols for (semi-honest) OPRF evaluation in the preprocessing model. Runtimes are given in microseconds ($\mu$s). 
	% \yuval{What does ``parallel'' mean? Should be explained in the caption. Our output size has roughly 128 bits (81 is the number of $\Z_3$ elements), but it is actually meaningless because the output size can be easily extended or decreased. So let's drop this column. For preprocessing and communication, I suggest using concrete numbers as opposed to general expressions that refer to different values of $n$. }
	}
	\label{table:oprf_comparison}
}
\mahimna{ToDo: better naming for the protocols.}
\end{table}



\newpage

%==================== Start comment=========
\iffalse
\section{Implementation}
\label{sec:technical_overview}

The pre-shared randomness was generated centrally (as if its generated by a third trusted party). Only the online part was implemented in a distributed manner. 

%%%%%%%%%%%%%%%%%%%%%%%%%%%%%%%%%%%%%%%%%%%%%%%%%

\iffalse
\subsection{Representing $Z_2$ vector}

The dark matter function manipulates elements of $Z_2$ and $Z_3$. However, machine words can hold up to 64 bits and we use this to represent many elements in just a few machine words, using bit slicing. Namely, we used each machine word as a vector of 64 bits and applied operations to this bit-vector in a SIMD manner.
Since in our implementation each key $\textbf{K}$  is of size $m \times n = 256 \times 256$ and each input is a vector of size $n = 256$, this results in packing each key o $256 \dfrac 64 = 4$ words and each input vector into 4 words. This may result in time saving of up to $\times 64$ of the run-time.

\subsection{Representing $Z_3$ vector}

The last part of the dark matter function takes an intermediate vector viewed as a vector of $Z_3$ elements, which is then multiplied 
by a matrix $R$ in $Z_3$ to produce the output vector in $Z_3$. We tried different methods for implementing this matrix-vector multiplication over $Z_3$.:	

\subsubsection{Bit slicing} Bit slicing is implemented  by representing a vector over $Z_3$  as two binary vectors - one LSB's and one MSB's. The operations over this $Z_3$ vector - addition, subtraction, multiplication and MUX  - were implemented in a bitwise manner as shown in table \ref{tab:multicol}. We took advantage of the fact that when representing each vector as MSB and LSB, negation (which is the same as multiplying by two) is the same as swapping the most and least significant bits. 
\fi

% TODO: replace with a table - COMPLETED



\begin{table}[ht]
	\caption{Operations in $Z_3$. input:   ${l_1}, {m_1}$ - LSB and MSB  of first trinary number, 
		${l_2}, {m_2}$ - LSB and MSB  of second trinary number,  ${s}$ - selection bit for MUX}
	\begin{center}
		\begin{tabular}{|c|l|}
			\hline
			\textbf{Operations} & \textbf{Methods}\\
			\hline
			\multirow{3}{*}{Addition} & ${t} := ({l_1 \wedge m_2}) \oplus ({l_2 \wedge m_1})$\\
			& $m_{\mathrm{out}} := ( l_1 \wedge  l_2 ) \oplus  t $ \\
			& $l_{\mathrm{out}} :=m_1 \wedge m_2 ) \oplus t $ \\
			\hline
			\multirow{3}{*}{Subtraction} & ${t} := ({l_1} \wedge {l_2}) \oplus ({m_2} \wedge {m_1})$;\\
			& $m_{\mathrm{out}} := (l_1 \wedge m_2 ) \oplus t$;\\
			& $l_{\mathrm{out}} := (m_1 \wedge l_2 ) \oplus t$; \\
			\hline
			\multirow{3}{*}{Multiplication} & $m_{\mathrm{out}}:= (l_1 \vee m_2) \wedge   (m_1 \vee l_2)$; \\
			& $l_{\mathrm{out}} := (l_1 \vee l_2) \wedge     (m_1 \vee m_2)$;\\
			\hline
			\multirow{3}{*}{MUX} & $m_{\mathrm{out}} :=( m_2 \vee s) \wedge (m_1 \vee (\bar{s}) )$; \\
			& $l_{\mathrm{out}}[i] :=( l_2 \vee s) \wedge (l_1 \vee (\bar{s}) )$; \\
			\hline
			
		\end{tabular}
	\end{center}
	\label{tab:multicol}
\end{table}


%TODO: IS THIS CORRECT? CHECK AND CORRECT

\subsubsection{Matrix multiplication using a lookup table}

Even with bitslicing, implementing the matrix-vector multiplication column by column via the opertions from Table~1 is still rather slow.
To get better performence, we capitalized on the fact that the random matrix $R$ is fixed, and we can therefore set up $R$-dependent tables and use table lookups in the implementation of the multiply-by-$R$ operation.

Specifically, we partition the matrix $R$ (which has $m=256$ columns) into sixteen slices of 16 columns each, denoted $R_1,\ldots,R_{16}$.
For each of these small matrices $R_i$, we then build a table with $2^{16}$ entries, holding the result of multiplying $R_i$ by every vector from $\{0,1\}^{16}$.
Namely, let $R_{i,0},R_{i,1},\ldots,R_{i,15}$ be the sixteen columns of the matrix $R_i$.
The table for $R_i$ is then defined as follows: For each index $0\le j < 2^{16}$, let $\vec{J}=(j_0,j_1,\ldots,j_{15})$ be the 0-1 vector holding the bits in the binary representation of~$j$, then we have
\[
T_i[j] = R_i \times \vec{J} = \sum_{k=0}^{15} R_{i,k} \cdot j_k ~~\bmod 3.
\]
Recalling that $R$ is $Z_3$ matrix of dimension $81\times 256$, every entry in each table~$T_i$ therefore holds an 81-vector over $Z_3$.
Specifically, it holds a packed-$Z_3$ element with four words (two for the MSBs and two for the LSBs of this 81-element vector).

Note, however, that $T_i$ can only be used directly to multiply $R_i$ by \emph{0-1 vectors}.
To use $T_i$ when multiplying $R_i$ by a $\{0,1,2\}$ vector, multiply $R_i$ separately by the MBSs and the LSBs of that vector, and then subtract one from the other using the operations from Table~1.
To multiply a dimension-256 $\{0,1,2\}$ vector by the matrix $R$, we partition it into sixteen vectors of dimension 16, use the approach above to multiply each one of them by the corresponding $R_i$, then use the operations from Table~1 to add them all together.
Hence we have

\begin{tabbing}
	\underline{Multiply-by-$R$(input: $\vec{V}\in\{0,1,2\}^{256}$:}\\
	~~1. $Acc := 0$\\
	~~2. For \=$i=0,\ldots,15$\\
	~~3. \> Let $\vec{M}_i,\vec{L}_i\in\{0,1\}^{16}$ be the MSB, LSB vectors (resp.) of $\vec{V}_i=\vec{V}[16i,\ldots,16i+15]$\\
	~~4. \> Set the indexes $m:=\sum_{k=0}^{15}2^i\cdot\vec{M}[i]$
	and $\ell:=\sum_{k=0}^{15}2^i\cdot\vec{L}[i]$\\
	~~5. \> $Acc := Acc + T_i[\ell] - T_i[\ell] \bmod 3$\\
	~~6. Output $Acc$
\end{tabbing}

We note that a slightly faster implementation could be obtained by braking the matrix into (say) 26 slices of upto 10 columns each, and directly identify each 10-vector over $\{0,1,2\}$ with an index $i<3^{10}=59049$.
This implementation would have only 26 lookup operations instead of 32 in the implementation above, so we expect it to be about 20\% faster. On the other hand it would have almost twice the table size of the implementation from above.



\section{Analysis}

Table \ref{CommunicationCosts} includes the computation and communication results for the different protocol.

% Table: Communication Cost
\begin{table}[htbp]
	\label{CommunicationCosts}
	%[h]
	\begin{center}
		%\begin{minipage}{10cm}
		\begin{tabular}{|c|c|c|c|c|}
			\hline
			\multicolumn{1}{|p{5cm}|}{\centering Protocol}
			& \multicolumn{1}{|p{2cm}|}{\centering Round  \\Complexity}
			& \multicolumn{1}{|p{3cm}|}{\centering Online \\ Communication}
			& \multicolumn{1}{|p{3cm}|}{\centering Preprocessing \\ Size}
			& \multicolumn{1}{|p{3cm}|}{\centering Computation \\ Operation}\\
			\hline
			\hline
			\textbf{Centralized WPRF} & - & - & - & 9K\\
			\hline
			\textbf{Distributed WPRF \cite{boneh2018-darkmatter} } & 4 & 5 & 2K &  45K\\
			\hline
			\textbf{Our WPRF  } & 3 & 5 & 1M &  13K\\
			\hline

			\textbf{Our OPRF } & 3 & 5 & 1M & 11K\\
			%=======
			

			\hline
			
		\end{tabular}
		
		\vspace{-1mm}
		\caption{Communication Analysis of different protocols. The protocols were optimized using both the bit packing and the lookup table}
		\label{CommunicationCosts}
		%\end{minipage}
	\end{center}
	\vspace{-5mm}
\end{table}


\section{Benchmarking}

We compare our run-time to discrete-log based PRFs. To this end, we use the lib sodium library \cite{LibSodium}. The library uses elliptic curve 252 bits, and includes a function that performs scalar multiplication ('crypto\_scalarmult\_ed25519').


\section{Experimental Results}

The system was tested using Ubuntu Server 18.04 on a t2.medium AWS environment. To record the timings, the code was run in a loop 1000 times. Below are run-time results for running a single instance of PRF in microsecond. The results include both centralized and distributed versions of the PRF. In order to increase efficiency, packing and lookup tables were used. The packing indicates both the $Z_2$ and $Z_3$ packing.

%add table
\begin{table}[htbp]
	%[h]
	\begin{center}
		%\begin{minipage}{10cm}
		\begin{tabular}{|c|c|c|c|c|c|}
			\hline
			\textbf{Protocol} & \textbf{Packed }  &  \textbf{lookup} & \textbf{Runs/sec} & \textbf{Runtime($\mu$ sec)}\\
			\hline
			\hline
			\textbf{Centralized weak PRF}  & N  & N  &  50K&20.2 \\
			\hline
			\textbf{Centralized weak PRF} & Y  &  N & 65.4K &18.5  \\
			\hline
			\textbf{Centralized weak PRF} & Y  &  Y & 165K &6.08 \\
			\hline
		\end{tabular}
		
		\vspace{-1mm}
		\caption{Comparison of the run-time and computation of the different Optimized Centralized wPRF.}
		\label{CentralRuntimeTable}
		%\end{minipage}
	\end{center}
	\vspace{-5mm}
\end{table}

\begin{table}[htbp]
	%[h]
	\begin{center}
		%\begin{minipage}{10cm}
		\begin{tabular}{|c|c|c|c|}
			\hline
			\textbf{Protocol} & \textbf{Runs/sec} & \textbf{Runtime($\mu$ sec)} \\
			\hline
			\hline
			\textbf{Fully Distributed WPRF\cite{boneh2018-darkmatter}} & 24K & 40.56  \\
			\hline
			\textbf{Fully Distributed WPRF} &  82K &  12.12\\
			\hline
			\textbf{OPRF with lookup} &  104K &  9.52 \\
			\hline
			\textbf{Discrete log-based PRF } & 12K &\textbf{86.07}\\
			\hline
			
		\end{tabular}
		
		\vspace{-1mm}
		\caption{Optimized protocols runtime. These protocols utilize both the bit-packing and lookup table optimization}
		\label{RuntimeTable}
		%\end{minipage}
	\end{center}
	\vspace{-5mm}
\end{table}

\fi
%==================End Comment==============


\section*{Acknowledgments}
Itai Dinur is supported by ISF grants~573/16 and~1903/20, and by the European Research Council under the ERC starting grant agreement No.~757731 (LightCrypt).
Yuval Ishai is supported by ERC Project NTSC (742754), ISF grant 2774/20, NSF-BSF grant 2015782, and BSF grant 2018393.

\printbibliography

\appendix
\newpage
\iffull\else
\section*{SUPPLEMENTARY MATERIAL}
\fi
%!TEX root = ../main.tex

\section{Detailed Cryptanalysis}
\label{appendix:cryptanalysis}

In this appendix we describe in detail the cryptanalysis of our new constructions.
We start with some general remarks and then analyze each construction.

\subsubsection{Choosing public inputs.}

All the cryptosystems we define receive public inputs chosen at random.
For example, the \ttOWF receives matrices $\mat{A}$ and $\mat{B}$ as public inputs.
One option is to choose $\mat{A}$ and $\mat{B}$ independently per secret input.
While an alternative option is to fix one (or even both) of the matrices and reuse them.
Generally, this alternative option is more susceptible to multi-target
attacks and attacks that are based on self-similarity.
Thus, in general, the first option is considered more secure and this is the option we use.
For similar reasons we choose the public parameters independently per secret for the
PRFs and the PRG we define.

Of course, there are bad choices of the public inputs which could degrade security,
and we need to show that these are unlikely to be occur.

Finally, for the PRFs we define, this still leaves open the question of how to select the public inputs $x$ and $B$
per sample computed with a secret key. We chose to select $x$ independently per sample,
while fixing $\mat{B}$ per secret key, as this allows to optimize performance by preprocessing.
In terms of security, the choice of fixing $\mat{B}$ does allow for a wider range of attacks
that we need to consider, as we demonstrate in the security analysis.

\paragraph{Restricted public matrices.}
In order to optimize performance, we select the public matrices
for the schemes as random Toeplitz matrices. These matrices
define a pairwise independent hash family and are known to satisfy the
Gilbert–Varshamov bound for linear codes, which is the main tool we use in the analysis.

\paragraph{Private matrices for PRF constructions.}
The private matrix $\mat{K}$ in the PRF constructions is a circulant matrix
defined by rotations of a secret $k \in \{0,1\}^n$.
While a choice of $K$ with a small rank deficiency does not seem to
have a significant impact on the security,
some attacks on the schemes (particularly on the 2-3 PRF) may exploit
matrices of particularly low rank
(as $w = \mat{K}x \bmod 2$ resides in a subspace of small dimension).

Thus, if possible, it is preferable to select $k$ such that $\mat{K}$
is a full rank matrix. Yet, this may require additional communication
is distributed protocols.
Otherwise, we need to understand the rank distribution of $\mat{K}$
when $k$ is selected uniformly at random and prove that $\mat{K}$
has very small rank with negligible probability.
For particular choices of $n$ the analysis is simple.
\begin{proposition}
\label{prop:rank}
Let $n = 2^{n'}$ for a positive integer $n'$
and let $\mat{K} \in \Z_2^{n \times n}$
be a circulant matrix selected uniformly at random.
Then, for any $a \in \{0,\ldots n\}$,
$\Pr[rank(K) \leq a] = 2^{-n+a}$.
\end{proposition}

\begin{proof}
For every vector $u \in \Z_2^{n}$
we associate a polynomial of degree at most $n-1$,
$u(x) = \sum_{j=0}^{n-1} u_j x^{j} \in \Z_2[x]$.

Assume that $\mat{K}$ is formed by rotations
of $k \in \Z_2^{n}$.
Then for $i \in \{0,\ldots, n-1\}$,
the $i$'th column of $K$ is associated with the polynomial
$x^i \cdot k(x) \bmod x^n - 1$.
Thus, for $u \in \Z_2^{n}$, $\mat{K}u$ is given
by the coefficients of the polynomial
$u(x) \cdot k(x) \bmod x^n - 1$,
and $Ku = 0$ if and only if
$u(x) \cdot k(x)$ divides $x^n - 1$.

Therefore, there is a bijection between the kernel space of $\mat{K}$
of the subspace of polynomials $u(x) \in \Z_2[x]$
of degree at most $n-1$
that satisfy $u(x) \cdot k(x) \bmod x^n - 1 = 0$.
The dimension of this subspace of polynomials is equal to
the degree of $\gcd(k(x), x^n - 1)$ over $\Z_2[x]$.

Recalling that $n = 2^{n'}$, over $\Z_2[x]$ we have
$x^{2^{n'}} - 1 = (x - 1)^{2^{n'}} = (x - 1)^n$.
To conclude the proof,
$rank(\mat{K}) \leq a$ if and only if
the kernel dimension of $\mat{K}$ is at least $n-a$.
Equivalently, the degree of $\gcd(k(x), (x - 1)^n)$
is at least $n-a$, or
$k(x)$ divides $(x - 1)^{n-a}$.
There are exactly $2^a$ polynomials of degree at most
$n-1$ that divide $(x - 1)^{n-a}$
and the probability of selecting one of them is
$\tfrac{2^{a}}{2^n} = 2^{-n+a}$ as claimed.
\end{proof}




\subsubsection{Concrete security goals.}
For each scheme the goal is to obtain parameter sets $n,m,t$ such that it offers
$s$-bit security, as defined below, while achieving good performance in distributed protocols.

\paragraph{OWF security.}
Given a OWF scheme $F(\cdot)$ (applied to a secret input, where the public parameters are embedded into $F$),
we define an inversion attack game by choosing $\hat{x} \in \Z_2^n$
uniformly at random and giving $\hat{y}= F(\hat{x})$ to the adversary, whose
goal to output some $x \in \Z_2^n$ such that $F(x) = \hat{y}$.
We say that $F(\cdot)$ has $s$ bits of security if no adversary can win the inversion attack game on $F(\cdot)$ with average complexity below $2^{s-1}$.


\paragraph{Weak PRF security.}
We refer to both weak PRFs we define.

The key $k \in \Z_2^n$ that defines the secret matrix $\mat{K}$ is chosen uniformly at random.
Moreover, the public matrix
$\mat{B} \in \Z_3^{t \times m}$ (or $\mat{B} \in \Z_2^{t \times m}$) is chosen uniformly at random
(or as a random Toeplitz matrix).
For a parameter $r$, an adversary is given $2^{r}$ samples $(x^{(1)},\mat{B},y^{(1)}) ,\ldots, (x^{(2^r)},
\mat{B},y^{(2^r)})$,
where each $x^{(i)} \in \Z_2^n$ is chosen independently and uniformly at random.

We will place a restriction of $r \leq 40$,
corresponding to a practical limit of $2^{40}$ on the number of samples available to the adversary.

We consider two types of adversaries.
The first type is a distinguisher that attempts to distinguish $2^r$ samples where each $y^{(i)}$
is generated using the PRF with a fixed $k$ from
$2^r$ samples where each vector $y^{(i)}$ is chosen uniformly at random.
The second type attempts to find the secret key given samples generated using the PRF.

We say that the PRF has $s$ bits of security if given (at most) $2^r$ samples both conditions below hold.
\begin{enumerate}
  \item The advantage of a distinguishing adversary that runs in time $2^\tau$ is at most $2^{(\tau - s)/2}$.
  \item The probability that an adversary that runs in time $2^\tau$ find the key is at most $2^{\tau - s}$.
\end{enumerate}
This definition is aligned with the work of Micciancio and Walter~\cite{Micciancio018}.

\paragraph{PRG security.}
The secret seed $x \in \Z_2^n$ is chosen uniformly at random, along with the public parameters $\mat{A},\mat{B}$.
The adversary is given a single sample $\mat{A},\mat{B},y$.
As for PRFs, security is defined against the two types of adversaries and the definition is similar.

\subsubsection{Algebraic attacks.}
In algebraic attacks the attacker represents the outputs (on internal variables) of the cipher as multivariate polynomials in the secret key (or preimage), obtaining a system of polynomial equations. The attacker then attempts to the solve the system using techniques such as linearization or applying algorithms for finding a reduced representation of the ideal generated by the polynomials in the form of a Gr\"{o}bner basis.
Such methods are known to be efficient only in particular cases where the polynomials have a special structure, or the
polynomials equations are of low degree and the attacker obtains sufficiently many equations to solve the system by linearization.

In our case, the output of the schemes we define
mix between the sums $\bmod$ 2 and $\bmod$ 3.
For example, in the \ttOWF and PRF constructions each output entry is a sum $\bmod$ 3 of entries of $w$,
where each such entry is a sum $\bmod$ 2 of the unknown bits of the secret input.
Due to the mix between the sums $\bmod$ 2 and $\bmod$ 3 we conjecture (similarly to~\cite{boneh2018-darkmatter}) that
the output cannot be represented (or well-approximated) by a low degree polynomial over any specific polynomial ring.
In particular, it was shown in~\cite{boneh2018-darkmatter} that the sum $\bmod$ 3 of $\ell$ binary-valued variables is
a high-degree polynomial over $\Z_2$,
as long as $\ell$ is large (e.g., $\ell \approx n$).
Crucially, our choice of parameters will ensure that the sums $\bmod$ 2 and $\bmod$ 3
are dense and contain many terms.
In particular, for the
\ttOWF and PRF constructions
we will make sure that the linear code spanned by the rows of $B$ has large minimal distance,
except with very small probability.
Overall, we do not expect algebraic attacks to pose a threat to our schemes,
and our analysis is mainly based on combinatorial attacks that attempt to recover the secret,
or on statistical attacks whose goal is to distinguish the output from random.


\subsection{Security Evaluation of the \ttOWF}

In this section we analyze the security of the \ttOWF.
Beforehand, we note that we may assume without loss of generality that in the most efficient construction, the number of expected preimages is (about) 1. Specifically, in our case, we may assume that $n = \log 3 \cdot t$ (up to rounding factors).

Indeed, setting $\log 3 \cdot t > n$ does not reduce the average number of preimages substantially. Consequently, any attack on a scheme with $n = \log 3 \cdot t'$ can be applied to a scheme with $\log 3 \cdot t > n$ by truncating the output to be of length $\log 3 \cdot t'$. Hence a scheme in which $\log 3 \cdot t > n$ does not offer better security than the truncated one.
On the other hand, the truncated scheme has shorter output and is generally more efficient.
Similarly, if $n > \log 3 \cdot t$, an attacker can fix $n - \log 3 \cdot t$ bits of the secret input to an arbitrary value and try to invert the image of the induced scheme where $n' = \log 3 \cdot t$ (note that on average, such a preimage exists).


\subsubsection{Basic attacks.}
%\label{sec:basic}
We describe several basic attacks and analyze their complexity as a function of $n,m,t$.
First, by exhaustive search, we can invert the \ttOWF in time complexity $2^n$ or $3^t = 2^{\log 3 \cdot t}$.

Focusing on the value of $m$, by exhaustive search, we can find $x$ such that $\mat{A}x = \mat{A}\hat{x}$ (which implies that the outputs are identical) in time complexity $2^m$.
A tighter restriction on $m$ is imposed by the following attack: guess $m - t$ bits of $w = \mat{A}x$ and solve the linear equation system $\hat{y} = \mat{B}w$ over $\Z_3$ (which has $t$ equations and variables) to obtain a full suggestion for $w$. A suggestion for $w$ allows to compute $x$ by solving the linear equation system $\mat{A}x=w$ over $\Z_2$. This attack has complexity $2^{m-t}$. An improved attack is described next.

\paragraph{Enumerating $w$ values.}

We show how to enumerate over all $w \in \{0,1\}^m$ that satisfy $\mat{B} w = \hat{y}$ in time complexity of about $2^{m/2}$ if $m \leq 2 \log 3 \cdot t = 2 n$, and $2^{m - \log 3 \cdot t} = 2^{m - n}$, otherwise.

Given such an algorithm, we can test each $w$ by solving the equation system $\mat{A}x = w$ over $\Z_2$, and if a solution exists, we have successfully inverted~$\hat{y}$.

Observe that if $w$ and $w'$ do not have a common $1$ entry, then $w + w' \bmod 2 = w + w' \bmod 3$
(where the addition is performed entry-wise). Therefore,
\begin{align}
\label{eq:lineara}
\begin{split}
\mat{B}(w + w' \bmod 2) \bmod 3 = \\
\mat{B}(w + w' \bmod 3) \bmod 3 = \\
(\mat{B}w \bmod 3) + (\mat{B}w' \bmod 3) \bmod 3.
\end{split}
\end{align}

We use this observation in the following algorithm, whose complexity as claimed above.
\begin{enumerate}
  \item Partition the $m$ indices of $w$ into 2 subsets $I_1$ and $I_2 = [m] \backslash I_1$, each of size $m/2$ bits.
  \item For $i \in \{0,1,\ldots 2^{m/2} - 1\}$, let $w_i$ be the $m$-bit vector whose value on the $m/2$ indices of $I_1$ is $i$, and is 0 on the indices of $I_2$. For each such $i$,
      evaluate $\mat{B} w_i \bmod 3 = y_i$ and store the pairs $(w_i,y_i)$ in a table $\mathcal{T}$, sorted by $y_i$ values.
  \item For $j \in \{0,1,\ldots 2^{m/2} - 1\}$, let $w'_j$ be the $m$-bit vector whose value on the $m/2$ indices of $I_2$ is $j$, and is 0 on the indices of $I_1$. For each such $j$,
      evaluate $\mat{B} w'_j \bmod 3 = y'_j$ and search $\mathcal{T}$ for the value $\hat{y} - y'_j \bmod 3$. If there exists a match $y_i$ such that $y_i = \hat{y} - y'_j \bmod 3$ (or $y_i + y'_j \bmod 3 = \hat{y}$), recover the value $w_i$ such that $\mat{B} w_i \bmod 3 = y_i$ from $\mathcal{T}$
      and return $w = w_i + w'_j \bmod 2$.
\end{enumerate}

Note that the expected number of $w \in \{0,1\}^m$ that satisfy $\mat{B} w = \hat{y}$ is $2^{m - \log 3 \cdot t}$. Hence, we cannot hope to obtain better complexity than $2^{m - \log 3 \cdot t}$ without exploiting additional constraints on $w$, imposed by the matrix $\mat{A}$. Our
subset-sum reduction (given in Section~\ref{sec:cryptanalysis}) shows how this can be done.

\paragraph{Induced schemes.}
Given the scheme \ttOWF with a matrix $\mat{B}$ and output $y$ such that $\mat{B}w = y$, and any positive integer $r$, we can left-multiply both sides by any $r \times t$ matrix $\mat{C}$ over $\Z_3$ to obtain $\mat{C}\mat{B}w = \mat{C}y$. Note that each row of the matrix $\mat{CB}$ is a linear combination of the rows of $\mat{B}$. Using such a matrix $\mat{C}$, we can perform Gaussian elimination on the rows of $\mat{B}$.

We denote the resultant induced scheme by $OWF_\mat{C}(\cdot)$. Observe that if $OWF(x) = y$, then $OWF_\mat{C}(x) = \mat{C}y$.
We now describe a simple attack that uses an induced scheme where $\mat{C}$ is only a row vector.

\paragraph{Low Hamming weight combinations of the rows of $\mat{B}$.}
Assume that there is a vector $v \in \Z_3^m$ of Hamming weight $\ell$ in the row space of $\mat{B}$, namely, there exists a vector $u \in \Z_3^t$ for which $u \mat{B} = v$. If $\ell$ is sufficiently small, then we could use the induced scheme $OWF_u(\cdot)$ to speed up exhaustive search.

Denote the set of $\ell$ non-zero indices of $v$ by $I$. Given $\hat{y} = \mat{B}w \bmod 3$, we compute the value of $u\hat{y} \bmod 3 = vw \bmod 3$. We can now enumerate the values of the corresponding set $I$ of $\ell$ bits of $w$ for which $u\hat{y} \bmod 3 = vw \bmod 3$ holds. This set of bits has $\tfrac{2^\ell}{3}$ possible values. Each such $\ell$-bit value gives rise of a system of $\ell$ linear equations on $x$, and we exhaustively search its solution space of size $2^{n-\ell}$. Overall, if $\ell \leq n$ the complexity of the attack is
$\tfrac{2^{\ell}}{3}$, while if $\ell = n+1$, the complexity is $\tfrac{2^{\ell+1}}{3}$. When $\ell > n+1$, the complexity is higher than $2^n$.
Thus, we will require that such a vector $v$ of low Hamming weight about $n$ does not exist, except with small probability.
This probability is computed in Proposition~\ref{prop:hw} in Appendix~\ref{app:distance}.

If more such low Hamming weight vectors are available, then the complexity of the attack may be further reduced,
although it seems unlikely to obtain a significant advantage over exhaustive search with this approach.




%\subsubsection{Quantum attacks.}

%Attackers with access to a quantum computer can improve upon the complexity of some of the attacks described in the classical setting. In the classical setting, exhaustive search and the subset-sum based algorithm are the most relevant attacks that we found of the scheme. This also applies to the post-quantum setting.

%First, it is possible to invert $OWF(\cdot)$ with Grover's algorithm in time complexity $2^{n/2}$. Second, according to~\cite{BonnetainBSS20}, one can solve subset-sum (in $m$ binary variables) on a quantum computer in complexity $2^{0.2156 m}$ (ignoring polynomial factors).


\subsubsection{Parameter Selection for the \ttOWF.}
According to the analysis, we determine parameters $n,m,t$ for which
we conjecture that the \ttOWF has $s$ bits of security.

First, due to the exhaustive search, we require $n \geq s$.
Second,
the most restrictive constraint on $m$ is imposed by the subset-sum algorithm (Section~\ref{sec:cryptanalysis}).
If we conservatively ignore the hidden polynomial factors and the large memory
complexity of the subset-sum algorithm of~\cite{BonnetainBSS20}, we need to set
$$0.283m \geq s.$$

Overall, we obtain
$$n = \log 3 \cdot t = s,$$
and $$m = \tfrac{s}{0.283} \approx 3.53 s.$$

We now consider the attack exploiting low Hamming weight combinations of the rows of $\mat{B}$,
and in particular, Proposition~\ref{prop:hw}.
In our case, we apply the proposition with $\log 3 \cdot t = n$ and $\ell = n$.
For $m = 3.53 n$, we obtain that the probability of having a vector of Hamming weight at most $n$ is bounded by
$$2 \cdot 2^{m (H(0.283) + 2 \cdot 0.283 - \log 3)} \approx 2 \cdot 2^{-0.16m} \approx 2 \cdot 2^{- 0.56 s}.$$
For $s \geq 128$, the expression evaluates to (less than) $2^{-70}$, so it is unlikely to encounter such an event in practice.
Moreover, even if the event occurs, security only regrades by a factor of 3, and by the same
proposition the probability that two such vectors are spanned by the rows of $B$ is at most $2^{-140}$.
Nevertheless, one may increase $n$ (and correspondingly $t$)
by a few bits (at negligible cost) to defeat this attack vector completely.

A more aggressive setting of the parameters may take into account the polynomial factors of~\cite{BonnetainBSS20} (and perhaps its high memory complexity). Unfortunately, the polynomial factors associated with the complexity formulas of the relevant subset-sum algorithms have not been analyzed.
For example, if we assume that the polynomial factors are about $m^2$, and we aim for $s = 128$ bits of security, then
we require $m^2 \cdot 2^{0.283m} \geq 2^{s} = 2^{128}$. Setting $m = 400 = 3.125 s$ is sufficient for satisfying the constraint in this setting.


%\paragraph{Post-quantum setting.}

%In the post-quantum setting, we have the constraints
%$n/2 \geq s$
%due to Grover's algorithm, and
%$0.2156 m \geq s$
%due to the quantum subset-sum algorithm.

%Overall, we obtain
%$n = \log 3 \cdot t = 2s,$
%and $m = \tfrac{s}{0.2156} \approx 4.64 s.$


\subsection{Security Evaluation of the \ttwPRF}
\label{sec:basicprf}

As for the \ttOWF, we describe several attacks and analyze their complexity as a function of the parameters $n,m,t$.

Unlike the case of the \ttOWF (where the goal was to find a preimage of a give output),
we can choose a small value of $t$ regardless of the other parameters without sacrificing security. In fact, it is clear that a small value of $t$ can only contribute to security, as any attack on a scheme with a small value of $t$ can be applied to a scheme with a larger value of $t$, simply by ignoring part of the output. Consequently, we may fix $t$ to the smallest value acceptable by the application.

We also note that given a sufficiently large number of samples, we expect that the key $k$ would be uniquely determined by the samples (regardless of the value of $t$).

\subsubsection{Key recovery attacks exploiting a few samples.}
We describe key recovery attacks that make use of the minimal number of samples required to derive the secret key $k$.

First, as for the \ttOWF, exhaustive search requires $2^{n}$ time. Also,
similarly to the \ttOWF, given any sample $(x,\mat{B},y)$, we can guess $m - t$ bits of $w = \mat{K} x$ and solve the linear equation system $y = \mat{B} w \bmod 3$, which then allows to compute a suggestion for $k$ (that can be tested on the remaining samples). This attack has complexity $2^{m-t}$.

Furthermore, given a single sample, we can apply the same attack for enumerating $w$ values for the \ttOWF.
This attack has complexity which is the maximum between $2^{m/2}$ and $2^{m - \log 3 \cdot t}$.

\paragraph{Reduction to subset-sum.}
As for the \ttOWF, we can reduce the key recovery problem to the problem of solving subset-sum over the $m$ binary variables of $w$. However, it is clear that if the algorithm is applied to a single $(x,\mat{B},y)$ sample, then its expected complexity cannot drop below $2^{m - \log 3 \cdot t - (m -n)} = 2^{n - \log 3 \cdot t}$, which is the expected number of $w$ values possible given $(x,\mat{B},y)$
(the remaining key candidate after analyzing one sample are tested against another sample).

On the other hand, if we try to reduce the complexity by applying the algorithm to more than one sample (e.g., to $(x,\mat{B},y)$ and $(x',\mat{B},y')$), then we must take advantage of the dependency between $w = \mat{K} x$ and $w' = \mat{K} x'$, which are related via linear constraints, imposed by $k$ and by $x,x'$. However, it is not clear how to model these complex linear constraints in the subset-sum reduction and we were not able to improve the complexity of the single-sample attack.

\subsubsection{Exploiting multiple samples.}
The key recovery attacks described above make use of a minimal number of samples required to derive the secret key. On the other hand, when given more samples, it may be possible to exploit various relations among them to mount distinguishing and even key recovery attacks which we investigate below.

\paragraph{Output bias.}
We consider a single sample $(x,\mat{B},y)$ and analyze the bias of linear combinations of the entries of $y$ over $\Z_3$.
If any such linear combination has a sufficiently high bias towards some constant,
then an attacker can exploit it in a distinguishing attack.

Similarly to the case analyzed for the corresponding \ttOWF, assume there are vectors $v \in \Z_3^m$ and $u \in \Z_3^t$ such that $u \mat{B} = v$ and the Hamming weight of $v$ is $\ell$. Given $y = \mat{B}w \bmod 3$, the attacker computes $uy \bmod 3 = vw \bmod 3$ and thus obtains the value of a linear combination $\bmod$ 3 of $\ell$ entries of $w \in \{0,1\}^m$.
Specifically, denoting the set of $\ell$ non-zero indices of $v$ by $I$, the attacker computes $\sum_{i \in I} v_i w_i \bmod 3$.
We now calculate the bias of sum the $\bmod $ 3, assuming that $w$ is uniformly distributed in $\Z_2^m$.

Each non-zero coefficient of the linear combination $v$ is either 1 and 2.
It is easy to prove by induction on $\ell$ (or by analysis of sums of binomial coefficients) that
for any coefficients $v_i \in \{1,2\}$ where $i \in I$ and for any $a \in \{0,1,2\}$,
$$\Pr\left[\sum_{i \in I} v_i w_i \bmod 3 = a\right] \in \{\tfrac{1}{3} \pm \tfrac{1}{2^\ell}, \tfrac{1}{3} \pm \tfrac{2}{2^\ell}\},$$
where the actual probability depends on $v$, $a$ and $\ell$. Thus, the bias of $vw \bmod 3$ is bounded by $\tfrac{2}{2^\ell}$.

We use Proposition~\ref{prop:hw} to deduce
that the subspace spanned by the rows of $\mat{B}$ contains a vector of Hamming weight at most $\ell$ with probability at most
$2 \cdot 2^{m (H(\ell/m) - \log 3) + \ell + \log 3 \cdot t}$.

Our analysis is conservative, as we ignore the work performed
by the attacker to find a low Hamming weight vector $v$ spanned by the rows of $B$.
Thus, we will be interested in $\ell \approx s/2$,
as we would like to avoid having a linear combination of the output with bias at least $2^{-s/2}$
(except with small probability).
Plugging this into the formula above we bound the probability by
\begin{align}
\label{eq:bias}
2 \cdot 2^{m (H(s/2m) - \log 3) + s/2 + \log 3 \cdot t}.
\end{align}

\paragraph{Conditional output bias.}
As described above, the bias of expressions of the form $\sum_{i \in I} v_i w_i \bmod 3$,
(where $I \subseteq [m]$ is a set of size $\ell$, each $v_i \in \{1,2\}$ is fixed and $w_i \in \{0,1\}$
are independent and uniformly distributed random variables) is bounded by $2^{-\ell+1}$.

However, the bias may increase if information about the variables $w_i$ is known.
In particular, the case in which $\sum_{i \in I} w_i \bmod 2$ is known was analyzed in~\cite{CheonCKK20},
where the authors showed that the conditional biases such as
$$\left| \Pr \left[\sum_{i \in I} w_i \bmod 3 = 0 \mid \sum_{i \in I} w_i \bmod 2 = 0 \right] - 1/3 \right|$$
can be as large as about $2^{-0.21\ell}$.
While this is still exponentially small in $\ell$,
it is more significant than the unconditional bias.

The PRF candidate proposed in~\cite{boneh2018-darkmatter} and analyzed in~\cite{CheonCKK20}
is similar to the one we analyze for $n = m$,
yet the matrix $\mat{B}$ contains a single row that sums $\bmod$ 3 all the entries of $w$.
Moreover, since $\mat{K}$ is a circulant matrix and $x$ has even Hamming weight,
then $\sum_{i \in [m]} w_i \bmod 2 = 0$ and the distinguishing attack is applicable.
In our case, $\mat{B}$ is selected at random and the parameters are chosen
such that the attacker cannot obtain $\sum_{i \in I} w_i \bmod 3$ for any fixed set $I$ (and particularly $I = [m]$),
except with negligible probability.
The general distinguishing attack is analyzed below.

Assume that given a sample $(x,\mat{B},y)$,
there exists a set $I \subseteq [m]$ (that depends of $x$)
such that $\sum_{i \in I} w_i \bmod 2$ is known to the attacker.
Moreover, assume that there exists $v$ in the row span of $\mat{B}$
such that $v_i = 1$ for each $i \in I$ and $v_i = 0$
for each $i \notin I$.
Then, the value $\sum_{i \in I} w_i \bmod 3$ can be computed from
the output, and the distinguishing advantage is as high as (about) $2^{-0.21\ell}$.
Of course, if several samples are available, the distinguishing advantage can
increase by accumulating the bias,
assuming the above conditions are fulfilled for more than one sample.

An extended variant of the attack may consider a vector
$v' = v + u$ in the row span of $\mat{B}$.
Denoting that Hamming weights of $v$ and $u$ by $\ell_1$,$\ell_2$, respectively,
the conditional bias can be as high as $2^{-0.21\ell_1 - \ell_2 + 1}$.
Note that here we conservatively ignore the work required to find such $v'$.

Given $2^r$ samples (where $r \leq 40$),
we wish to show that the distinguishing advantage is small
(except with negligible probability).
Indeed, calculation shows that this distinguisher is not stronger that the
unconditional distinguisher.
Essentially, given a single sample, for any fixed $I$, the vector $v$ as described above
is in the row span of $\mat{B}$ with minuscule probability $3^{t - m}$.
In the extended attack, we consider a ball around $v$, but this ball
is much smaller than the one considered in the unconditional distinguisher (as $\ell_2 < s/2$).

Finally, in a more general variant of the attack,
the attacker may guess a parity of key bits and calculate the conditional bias over several samples,
attempting to amplify it.
Note that the desired vector $v$ changes for each sample according to $x$.
By similar calculation, once the attacker has fixed the guess given a sample,
almost all additional samples would not allow to amplify the conditional bias,
as a good vector is unlikely to be in the row span of $\mat{B}$.

\paragraph{Differential cryptanalysis and low Hamming weight samples.}
As we argue below, the \ttwPRF seems to be immune to classical statistical differential cryptanalysis.

Assume that the attacker obtains two samples $(x,\mat{B},y)$ and $(x+\delta \bmod 2,\mat{B},y')$, where $\delta \in \{0,1\}^n$.
Denote $w = \mat{K} x \bmod 2$ and $w' = \mat{K} (x + \delta) \bmod 2 = w + \mat{K} \delta \bmod 2$.
Thus, $y' = \mat{B} w' \bmod 3 = \mat{B} (w + \mat{K} \delta \bmod 2 ) \bmod 3$.
In general, $y$ and $y'$ do not seem to have any statistical relation that holds with sufficiently high probability.
Particularly, $w + \mat{K} \delta \bmod 2 \neq w + \mat{K} \delta \bmod 3$,
except with very small probability.

From an algebraic viewpoint, the attacker can consider the $m$ bits of $w = K x \bmod 2$ as variables over $\Z_2$,
but then the algebraic degree of the output over $\Z_3$ would be large.
The attacker can also consider the $m$ bits of $w$ as variables over $\Z_3$.
In this case, the attacker obtains $t$ linear equations and $t$ quadratic equations $\bmod$ 3 (of the form $(w_i)^2 - w_i = 0$) from the sample $(x,\mat{B},y)$.
On the other hand, the algebraic degree of $w' = \mat{K} (x +\delta) \bmod 2$ in the variables of $w$ would be large due the dense $\bmod$ 2 operations,
hence it is not clear how to obtain additional low degree equations.
In general, such algebraic attacks do not seem more efficient than the attack described above for enumerating $w$ values.

Another scenario which we consider is when the attacker obtains a sample $(x,\mat{B},y)$
such that $x$ is of low Hamming weight. In this case each entry of $w$ is a low Hamming weight sum $\bmod$ 2
of the bits of $k$ and can thus be described as a low degree polynomial over $\Z_3$.
Consequently, low degree polynomial equations over $\Z_3$ in the secret key can be deduced from the output.
Obtaining several such samples may allow the attacker to solve for the key.
In general, such low Hamming weight samples are avoided with high probability given the data limit,
but we can also place a restriction on the sample generation, forcing it to generate vectors $x$
with minimal Hamming weight (e.g., a lower bound of $n/4$).


\paragraph{Attacks that exploit the structure of $\mat{K}$.}
Recall the $\mat{K}$ is a circulant matrix and we consider attacks that exploit its
special structure. A notable property of $\mat{K}$ as a circulant matrices is that it preserves symmetry.
Specifically, if $x \in \Z_2^n$ is 2-symmetric (i.e., its two halves are equal), 
then $w = \mat{K}x$ is also 2-symmetric. Thus, given a sample $(x,\mat{B},y)$
such that $x$ is 2-symmetric, we know that $w$ is 2-symmetric and this fact can be used to 
uniquely recover $w$ (which is effectively of length $m/2$) in time $2^{n - m/2 - \log 3 \cdot t}$.
Consequently, we can derive $m/2$ linear equations $\bmod$ 2 on the secret key 
and perhaps use another sample to recover it completely.

The probability that such a 2-symmetric vector is found in the data is $2^{-n/2 + r}$.
We require that this quantity is negligible.

It is also possible to consider even more symmetry, such as 4-symmetric vectors, 
but these have lower probability of being picked. 

Finally, we leave it to future work to extend this attack beyond the simple case analyzed above.
For example, one may consider input vectors that are close to 2-symmetric, 
or pairs of samples $(x,\mat{B},y),(x',\mat{B},y')$ such that $x + x' \bmod 2$ is 2-symmetric
(implying that $\mat{K} (x + x')$ is 2-symmetric). 
 

\paragraph{Attacks based on self-similarity.}

There are several simple attacks that take advantage of the fact that $\mat{B}$ is fixed for all samples generated with the same secret $k$.
A basic attack looks for a collision on the $w$ values for a pair of samples, which can be detected at the output.
Given $2^r$ samples, the advantage of this attack is $2^{2r - rank(\mat{K})}$.
we will set the parameters such that the advantage of this attack is negligible for the data complexity bound
(assuming the rank of $K$ is not too small).

Another simple attack is a multi-target attack, where given $2^r$ samples, the attacker guesses $w$, computes $\mat{B} w \bmod 3$ and compares the result with all given outputs. A match allows to recover a candidate for the secret $k$.
The expected complexity of this attack is $2^{m-r}$, but cannot drop below $2^{m - \log 3 \cdot t}$ without exploiting relations between the different $w$ values. In general, given the data limit, the attack less efficient that the attack that enumerates all $w$ values considered above.

\paragraph{Simultaneous sums.}

We describe a more involved self-similarity attack, which exploits the fixed $\mat{B}$ value per secret $k$.

Assume that for some index set $I$ of size at least 3, $\sum_{i \in I} x^{(i)} \bmod 2 = 0$. Then, $\sum_{i \in I} w^{(i)} \bmod 2 = 0$. While it is not clear how this relation influences the output,
we extend this initial observation by simultaneously considering sums $\bmod$ 2 and 3, as follows:
assume that there are 4 samples (denoted for simplicity
by $ \{(x^{(i)},\mat{B},y^{(i)})\}_{i=1}^{4}$) such that for each $j \in [m]$,
$$\sum_{i = 1}^{4} w^{(i)}_j = 2.$$
Then, $\sum_{i = 0}^{4} x^{(i)} \bmod 2 = 0$ and $\sum_{i = 0}^{4} y^{(i)} \bmod 3 = B \cdot \vec{2} \bmod 3$ (where $\vec{2}$ is a vector with $m$ entries whose values are 2).
Note that a random 4-tuple of samples satisfies this simultaneous sum constraint with probability
$2^{-n - \log 3 \cdot t}$, but the probability that the constraint $\sum_{i = 0}^{4} w^{(i)}_j = 2$ is satisfied for every $j \in [m]$ is about
$$\left( \tfrac{\tbinom{4}{2}}{16} \right) ^{-m} \approx 2^{- 1.415 m},$$
which is higher than expected if $1.415 m < n + \log 3 \cdot t $.

Since the adversary has about $2^{4r}$ such 4-tuples, the probability that such a simultaneous 4-sum exists is about $2^{4r - 1.415 m}$. It can be detected in time complexity of about $2^{2r}$ using a standard matching algorithm. The important constraints for defending against this attack are
$$1.415 m > n + \log 3 \cdot t,$$ or
\begin{align}
\label{eq:sim}
2^{4r - 1.415 m}
\end{align}
is negligible, otherwise.

The simultaneous 4-sum distinguisher can be easily generalized to a
simultaneous $d$-sum distinguisher for arbitrary $d$.
In general, we look for $d$-tuples where (for example)
for each $j \in [m]$, $$\sum_{i = 1}^{d} w^{(i)}_j = c,$$
for some value $c$ (fixed $\bmod$ 6) such that $c \bmod 2 = 0$.
However, calculation shows that $d = 4$ gives the most efficient
distinguisher in our case (with the small data limit).



%\subsubsection{Quantum attacks.}
%The relevant attacks in the quantum setting are similar to ones for the OWF.
%First, Grover's algorithm, breaks the scheme (recovers the key) is time $2^{n/2}$.
%Second, the quantum subset-sum algorithm of~\cite{BonnetainBSS20} has complexity $2^{0.2156 m}$ (ignoring polynomial factors), but does not drop below $2^{n - \log 3 \cdot t}$.


\subsubsection{Parameter selection for the \ttwPRF.}
According to the analysis, we determine parameters $n,m,t$ for which
we conjecture that the \ttwPRF has $s$ bits of security.

In order to select the parameters, we may first set $t$ to it's minimal possible value
(depending on the application).
We also assume that $t$ is not too large, and particularly $\log 3 \cdot t \leq s$.

The constraints imposed by the above attacks are as follows.
First, due to exhaustive search, we require $$n \geq s.$$
Second, the subset-sum algorithm
imposes the constraint
$n - \log 3 \cdot t \geq s,$
(given that $n \geq m$ and $\log 3 \cdot t \leq s$).


We consider two sets of parameter.
The first is
$$n = m = s + \log 3 \cdot t.$$
In particular, if $\log 3 \cdot t  = s$, then
$n = m = 2s$.
The second parameter set
$$n = m = 1.25(s + \log 3 \cdot t),$$
and is less aggressive.

Next, we analyze the bias of the output based on~(\ref{eq:bias}),
assuming that $\log 3 \cdot t = s$.
For the first parameter set,
we obtain that the probability of having bias of $\tfrac{2}{2^{s/2}}$ is bounded by
\begin{align*}
2 \cdot 2^{m (H(s/2m) - \log 3) + s/2 + \log 3 \cdot t} =
2 \cdot 2^{2s (H(1/4) - \log 3) + 3s/2} \approx
2 \cdot 2^{-0.049s},
\end{align*}
which is non-negligible.
On the other hand, the consequences of the distinguishing attack given the data limit
seem relatively mild, and this parameter set may be considered by applications
where performance is critical.

For the second parameter set (assuming $\log 3 \cdot t = s$),
the probability of having bias $\tfrac{2}{2^{s/2}}$ is bounded by
\begin{align*}
2 \cdot 2^{m (H(s/2m) - \log 3) + s/2 + \log 3 \cdot t} =
2 \cdot 2^{2.5s (H(1/5) - \log 3) + 3s/2} \approx
2 \cdot 2^{-0.65s},
\end{align*}
which we consider negligible.

The advantage of the collision attack is $2^{2r - rank(\mat{K})} \leq 2^{80 - rank(\mat{K})}$.
Given that $m = 2s \geq 256$, if we choose $n$ as a power of 2,
based on Proposition~\ref{prop:rank},
the advantage is much smaller than $2^{s/2}$ (except with negligible probability).

Finally, we consider the attack that exploits the circulant matrix $\mat{K}$
and the simultaneous 4-sum distinguisher.
For defending against the first attack,
when $s \geq 128, r \leq 40$, we have $2^{-n/2 + r} < 2^{-128 + 40} = 2^{-88}$ ($n \geq 2s$),
which we consider negligible.

Recall that the probability of having such a 4-sum in the output is estimated
in~(\ref{eq:sim}) as $2^{4r - 1.415 m} \leq 2^{160 - 2.83s}$.
For a minimal choice of $s = 128$, this probability is smaller than $2^{-200}$
which is negligible.



\subsection{Security Analysis of the LPN-PRG}

We analyze the security of the LPN-PRG.

If the matrix $\mat{A}$ was a random matrix, then
the first step would consist of generating $m$ samples from the alternative weak PRF construction proposed
in~\cite{boneh2018-darkmatter}. Each sample
$w_i = \mat{A}[i] x \bmod 2 + ((\mat{A}[i] x \bmod 3) \bmod 2) \bmod 2 \in \Z_2^m$
can be viewed as adding noise $((\mat{A}[i] x \bmod 3) \bmod 2)$ to the inner product $\mat{A}[i] x \bmod 2$.
Given that $\mat{A}[i]$ is of sufficiently large Hamming weight, then
$\Pr_x[(\mat{A}[i] x \bmod 3) \bmod 2 = 1] \approx 1/3$, which is the magnitude of noise added.
The second step consists of a compressing linear transformation $B$ applied to $w$.
The idea is to increase the noise of each sample by mixing it with other samples.
This step should defeat standard attacks applied to LPN with a constant noise parameter
(such as decoding attacks).

In our case, the matrix $\mat{A}$ is structured (it is a random Toeplitz matrix),
but we were not able to exploit this in an efficient attack.



\subsubsection{Key recovery attacks.}
We begin by considering attacks that attempt to recover the secret key (seed).

Exhaustive search for the key recovery attacks requires time $2^n$.

In a different approach for recovering the key, given a sample $\mat{A},\mat{B},y$,
the attacker enumerates over the subspace of $w$ values that satisfy $\mat{B} w = y \bmod 2$.
This subspace contains $2^{m - t}$ vectors. For each such vector, the attacker attempts to recover $x$
given $\mat{A}$ and $w$. Thus, given $\mat{A},w$ the attacker has $m$ samples
generated from the LPN-like construction proposed in~\cite{boneh2018-darkmatter} as a weak PRF
(although we a structured matrix $\mat{A}$).
Given that $m$ is not too large (i.e., it is a small multiple of $n$),
then the best attack we have on this scheme simply tries to break the LPN instance (which has noise of $1/3$),
without exploiting the deterministic way in which it is generated.
The concrete security of LPN given a small number was analyzed in several publications such as~\cite{EsserKM17},
and the complexity of known attack is generally exponential in $n$.
Nevertheless, the attacker is required to solve $2^{m - t}$ related LPN instances,
and perhaps can amortize the complexity. Moreover, the matrix $\mat{A}$ is structured.
Thus, we (conservatively) estimate the total complexity of such attacks by $2^{m - t}$.

\subsubsection{Noisy linear equations.}

As noted above, each bit $w_i$ for $i \in [m]$ can be viewed as a noisy linear equation over $\Z_2$
with noise of about $1/3$, or bias $2/3 - 1/2 = 1/6$.
Our goal is to select the parameter $m$ such that
the all linear combinations of the output bits of $y$
have exponentially small bias towards a linear equation in the secret $x$.
We will (heuristically) model each bit $w_i$ as having independent bias of $1/6$.

If the linear subspace spanned by the rows of $\mat{B}$ contains a vector of Hamming weight $\ell$,
then by the piling-up lemma, the bias of the corresponding linear combination of the bits of $w$ is
\begin{align}\label{eq:bias_linear}
2^{\ell - 1} \cdot (\tfrac{1}{6})^{\ell} < 2^{-\log 3 \cdot \ell}.
\end{align}

Thus, we will require
that the rows of $\mat{B}$ do not span a vector whose Hamming weight is too low.
This is similar to the previously analyzed schemes,
but the lower bound on the Hamming weight we will enforce for the PRG will be lower.
Indeed, the bias above is calculated with respect some linear equation in the unknown secret key bits.
Such a bias is generally much less of a security concern compared to a bias towards a constant value
(e.g., the bias analyzed for the \ttwPRF construction) which can be used
to directly distinguish the output from random using statistical tests.
Particularly, an alternative scheme where we change the first transformation
to only compute the ``noise part''
($w =(\mat{A}x \bmod 3) \bmod 2$) would require larger parameters to be secure,
as we need a higher lower bound on the Hamming weight $\ell$ to avoid distinguishing attacks.

For a PRG with $s$ bits of security, we will conservatively require a bias of at most $2^{-0.1 s}$.
We are not aware of any attack that can exploit such a low bias.
Effectively, this means that the minimal distance of $B$ should be
at least $s \cdot 0.1/\log 3 < 0.07 s$ (except with small probability).


\begin{remark}
In~\cite{CheonCKK20} the authors analyze the constructions presented in~\cite{boneh2018-darkmatter}.
In particular, for the alternative PRF construction,
given a sample $(a \in \Z_2^n, x \cdot a)$,
they show that there exists $j \in [n]$ such that
$$|\Pr[a_j = 0 \mid x_j = 0 \text{ and } (a \cdot x \bmod 2 + ((a \cdot x \bmod 3) \bmod 2) \bmod 2) = 0]| \approx \tfrac{1}{2^{0.21 \ell}},$$
where $\ell$ is the Hamming weight of $a$.
This property was exploited in a distinguishing attack.

Our PRG construction seems to be immune to this type of analysis
because the attacker only has access to sufficiently dense linear combinations of (structured) samples
of the alternative PRF construction.
\end{remark}


\subsubsection{Parameter selection for the LPN-PRG.}

We determine parameters $n,m,t$ for which
we conjecture that the
PRG has $s$ bits of security.

Recall the we set $t = 2n$.
Exhaustive search implies that $n \geq s$ and we have lower
bounded the effort required in key recover by $2^{m - t}$.
Thus, a reasonable choice of parameters is $n = s$ and $m = 3s, t = 2s$.

For these parameters, we consider the maximal bias of linear combinations according to~(\ref{eq:bias_linear}),
with $\ell = 0.07 s$.
By Proposition~\ref{prop:hw},
the probability of having a vector of Hamming weight more than $0.07 s$
in the row span of $B$ is
$0.07 s \cdot 2^{3s (H(0.07/3) - 1) + 2s} < 0.07 s \cdot 2^{-0.52 s}$.
We consider this as a negligible probability
as the consequences of the this unlikely event are mild.


\subsection{Security analysis of LPN-PRF}

The overall structure of the LPN-PRF is similar to the PRG.
However, unlike PRG, in the first transformation, the key is part of the matrix and
$x$ is a public value.
Thus, $w$ can be viewed as $m$ outputs of (another) more
structured version of the construction of~\cite{boneh2018-darkmatter}.

In terms of parameters, the main differences are that we have $n \geq m$
and $t$ is not constraint to $2n$ (in fact, we will propose to set it smaller than $n$).
Furthermore, we assume that the attacker obtains $2^r$ samples for $r \leq 40$ instead of a single sample.

We note that variants of the basic attacks that were analyzed for the \ttwPRF (
such as the attack that exploit the circulant $\mat(K)$, the collision attack
and the multi-target attack) are also applicable to this construction.
As in the case of the \ttwPRF,
they do not seem to be a threat for our choice of parameters.

Overall, we have not found any class of attacks that are
applicable to this construction, but not to the previous ones.
Below we briefly consider the most important attacks that influence the
choice of parameters.

\paragraph{Summary of attacks.}
As for the PRG, we estimate the total complexity of key recovery attacks by $2^{m - t}$.
Although we have more samples that for the PRF,
it is not clear how to exploit them to obtain better complexity.

In addition, similarly to the PRG,
we require that each linear combination of the output
has bias of at most $2^{-0.1 s}$
(toward some linear combination of the key over $\Z_2$).

\paragraph{Parameter selection for the LPN-PRF.}
For $s$-bit security, we set $n=m= 2s$ and $t =s$.
By our analysis, exhaustive search requires $2^{m - t} = 2^s$ time.

We consider the maximal bias of linear combinations according to~(\ref{eq:bias_linear}),
with $\ell = 0.07 s$.
By Proposition~\ref{prop:hw},
the probability of having a vector of Hamming weight more than $0.07 s$
in the row span of $\mat{B}$ is
$0.07 s \cdot 2^{2s (H(0.07/2) - 1) + s} < 0.07 s \cdot 2^{-0.56 s}$,
which we consider negligible.


\section{Analysis of Random Linear Codes}
\label{app:distance}

We analyze the distance of random linear codes defined by a random matrix
or a random Toeplitz matrix.
The analysis is based on the probabilistic method and is
similar to the analysis used to obtain the Gilbert–Varshamov bound.

\begin{proposition}
\label{prop:hw}
Let $\mat{B} \in \Z^{t \times m}_{q}$ be a random matrix, or a random Toeplitz matrix whose rows define
a linear code.
Then, the minimal distance of (the code defined by) $\mat{B}$ is at most $\ell < m/2$ with probability at most
$$f_q(m,t,\ell) \leq q^{t-m} \cdot Vol_q(m,\ell),$$
where $Vol_q(m,\ell) = \sum_{i=1}^\ell \tbinom{m}{i} \cdot (q-1)^{i}$.
Moreover, this code contains two such linearly independent vectors with probability at most $(f_q(m,t,\ell))^2$.

Finally, let $H(p) = - p \log p - (1-p) \log(1-p)$ be the binary entropy function.
Then
$$f_2(m,t,\ell) \leq \ell \cdot 2^{m (H(\ell/m) - 1) + t},$$
and
$$f_3(m,t,\ell) \leq 2 \cdot 2^{m (H(\ell/m) - \log 3) + \ell + \log 3 \cdot t}.$$
\end{proposition}

\begin{proof}The number of non-zero vectors of Hamming weight at most $\ell$ over $\Z_q^m$ is
$Vol_q(m,\ell) = \sum_{i=1}^\ell \tbinom{m}{i} \cdot (q-1)^{i}$.

For $q = 2$, we have
$Vol_2(m,\ell) = \sum_{i=1}^\ell \tbinom{m}{i} <
\ell \cdot \tbinom{m}{\ell} \leq
\ell \cdot 2^{m H(\ell/m)}$, while for $q = 3$, we have
$Vol_3(m,\ell) = \sum_{i=1}^\ell \tbinom{m}{i} 2^i <
2 \cdot \tbinom{m}{\ell} 2^\ell \leq
2 \cdot 2^{m H(\ell/m) + \ell}.$

Since $\mat{B}$ is selected uniformly from a pairwise independent hash family (either a random matrix or a random Toeplitz matrix),
the probability that every non-zero vector is in the row space is $q^{t-m} - 1 < q^{t-m}$. By a union bound over all vectors of Hamming weight bounded by $\ell$, the probability that there exists such a vector in the row space of $B$ is at most
\begin{align*}
f_q(m,t,\ell) \leq q^{t-m} \cdot Vol_q(m,\ell).
\end{align*}
The bound $(f_q(m,t,\ell))^2$ on the probability of having two such linearly independent vectors follows by pairwise independence.

Specifically, for $q = 2$, we obtain
$f_2(m,t,\ell) \leq \ell \cdot 2^{m (H(\ell/m) - 1) + t}$,
while for $t = 3$, we obtain
$f_3(m,t,\ell) \leq 2 \cdot 3^{t-m} \cdot 2^{m H(\ell/m) + \ell} = 2 \cdot 2^{m (H(\ell/m) - \log 3) + \ell + \log 3 \cdot t}$.
\end{proof}


%-------------------------------------------------%


\section{Deferred Protocol Details}
\label{appendix:protocols}
\paragraph{Circuit description of constructions.}
We can represent our constructions by circuits consisting of the gates described previously. This approach is similar to that of~\cite{boyle2019-fss-preprocess}. We formally define computation circuit representations for our constructions in Definition~\ref{def:computation_circuit}.

\begin{definition}[Computation circuit]
A computation circuit $C$ with input space $\Gin = \prod \Gin_i$ and output space $\Gout = \prod \Gout_i$ is a (labeled) directed acyclic graph $(\mathcal{V},\mathcal{E})$ where $\mathcal{V}$ denotes the set of vertices and $\mathcal{E}$ denotes the set of edges according to the following:
\begin{itemize}

\item Each source vertex corresponds to exactly one $\Gin_i$ and vice versa. The label for the vertex is the identity function on the corresponding $\Gin_i$. Each sink vertex corresponds to exactly one $\Gout_i$ and vice versa. The label for the vertex is the identity function on the corresponding $\Gout_i$. Each non-source $V \in \mathcal{V}$ is labeled with a gate $\mathcal{G}_V \in \gateset$ that computes the function $\mathcal{G}_V: \mathbb{G}^{\textsf{in}}_V \to \mathbb{G}^{\textsf{out}}_V$. The depth of a vertex $V \in \mathcal{V}$, denoted by $\textsf{depth}(V)$ is the length of the largest directed path from a source vertex to $V$.


\item For an edge $(V_a, V_b)$, let $\Gout_{V_a} = \prod \Gout_{V_a,i}$ and $\Gin_{V_b} = \prod \Gin_{V_b, i}$. Then, there exists indices $j$ and $k$ such that $\Gout_{V_a,j} = \Gin_{V_b,k}$. Further, for each input $\Gin_{V_b,i}$ for $V_b$, there is some edge $(V_c, V_b)$ that satisfies the above.

\item The evaluation of the gate for vertex $V$ on input $x \in \Gin_V$ is defined as $y = \mathcal{G}_V(x)$. The evaluation of the circuit $C$, denoted by $\textsf{Eval}_C(x)$, where $x \in \Gin$ is the value $y \in \Gout$, that is obtained by recursively evaluating each gate function in the circuit.
\end{itemize}


Let $\mathsf{F} = \{\mathsf{F}_\secparam\}_{\secparam \in \N}$ denote a family of functions $\mathsf{F}_\secparam:\mathcal{X}_\secparam \to \mathcal{Y}_\secparam$. We say that $\{C_\secparam\}_{\secparam \in \N}$ is a family of computation circuits for $\mathsf{F}$ if all $C_\secparam$ have the same topological structure, and for all $\secparam \in \N$, $\mathsf{F}_\secparam(x) = \textsf{Eval}_C(x)$ for all $x \in \mathcal{X}_\secparam$.
\label{def:computation_circuit}
\end{definition}

\subsection{Local Protocols for Circuit Gates}
\label{appendix:protocol_gates}

Here, we provide the remaining details for the circuit gate protocols.

\paragraph{Protocol notation and considerations.}
For a protocol $\prot$, we use the notation $\prot(a_1, \dots, a_k \mid b_1, \dots, b_l)$ to denote that the values $a_1, \dots, a_k$ are provided publicly to all parties in the protocol, while the values $b_1, \dots, b_l$ are secret shared among the parties. When $\party_i$ knows the values $(a_1, \dots, a_k)$, and has shares $\sharei{b_1}, \dots, \share{b_l})$, we use the notation $\prot(a_1, \dots, a_k \mid \sharei{b_1}, \dots, \sharei{b_l})$ to denote that $\party_i$ runs the protocol with its local inputs. 

Given public values $a_1, \dots, a_k$, it is straightforward for the protocol parties to compute a sharing $\share{f(a_1, \dots, a_k)}$ for a function $f$ (for example, $\party_1$ computes the function as its share, and all other parties set their share to $0$).

\paragraph{Linear gate protocol $\prot_\LMap^{\mat{A}, p}$.}
The linear gate is the easiest to evaluate, and follows from the standard linear homomorphism of additive secret sharing.

\begin{itemize}
  \item \textbf{Functionality}: Each party is provided with the matrix $\mat{A}$ and shares of the input $x$ (over $\Z_p$). The goal is to compute shares of the output $y = \mat{A}x$.

  \item \textbf{Preprocessing}: None required.

  \item \textbf{Protocol details}:
  For the protocol $\prot_\LMap^{\mat{A}, p}(\mat{A} \mid x)$, each party $\party_i$ computes its output share as $\sharei{y} = \mat{A}\sharei{x}$. Note that this works because $\mat{A}x = \sum_{\party_i \in \parties} \mat{A}\sharei{x}$ as a direct consequence of the linear homomorphism of additive shares.
\end{itemize}

\paragraph{Addition gate protocol $\prot_\Add^{p}$.}
The addition modulo $p$ gate is also easy to evaluate. Given shares of $x, x'$ over $\Z_p$, for the protocol each party $\party_i$ can locally compute its share of $x+x'$ as $\sharei{x} + \sharei{x'} \bmod p$. This directly follows from the additive homomorphism of additive shares. 

\paragraph{Bilinear gate protocol $\prot_\BLMap^p$.}
The bilinear gate protocol is essentially a generalization of Beaver's multiplication triples~\cite{boyle2019-fss-preprocess,beaver1991-triples} that computes the multiplication of two shared inputs. For Beaver's protocol, to compute a sharing of $ab$ given shares of $a$ and $b$ (all sharings are over a ring $\mathcal{R}$), the protocol parties are provided shares of a randomly sampled triple of the form $(\tilde{a},\tilde{b},\tilde{a}\tilde{b})$ in the preprocessing stage. Beaver's protocol first reconstructs the masked inputs $\hat{a}$ and $\hat{b}$ after which local computation is enough to produce shares of the output. For our bilinear gate protocol, we assume that  all parties are already provided with the masked inputs (to move the communication outside of the gate protocol), along with correlated randomness similar to a Beaver triple. 

\begin{itemize}
  \item \textbf{Functionality}: 
  Abstractly, the goal of the bilinear gate protocol is to compute shares of the output $y = \mat{K}x$ given shares of both inputs $\mat{K}$ and $x$. For our purpose however, the masked inputs will have already been constructed beforehand, i.e., each party is provided with $\hat{\mat{K}}$ and $\hat{x}$ publicly, along with shares of correlated randomness similar to a Beaver triple (see below).

  \item \textbf{Preprocessing}: Each party is provided shares of $\tilde{\mat{K}}, \tilde{x}$, and $\tilde{\mat{K}}\tilde{x}$ as correlated randomness.

  \item \textbf{Protocol details}: For the protocol $\prot_\BLMap^p(\hat{\mat{K}},\hat{x}\mid \tilde{\mat{K}}, \tilde{x}, \tilde{\mat{K}}\tilde{x})$, each party $\party_i$ computes its share of $\hat{y}$ as:
  \[
    \sharei{\hat{y}} = \sharei{\hat{\mat{K}}\hat{x}} - \hat{\mat{K}}\sharei{\tilde{x}} - \sharei{\tilde{\mat{K}}}\hat{x} + \sharei{\tilde{\mat{K}}\tilde{x}}
  \]

  \noindent \textit{Correctness}. Note that this works since:
  \begin{align*}
  \sum_{\party_i \in \parties} \sharei{\hat{y}} &= \hat{\mat{K}}\hat{x} - \hat{\mat{K}}\tilde{x} - \tilde{\mat{K}}\hat{x} + \tilde{\mat{K}}\tilde{x} \\
  &= (\mat{K} + \tilde{\mat{K}})x - \tilde{\mat{K}}(x + \tilde{x}) + \tilde{\mat{K}}\tilde{x} \\
  &= \mat{K}x
  \end{align*}
\end{itemize}
Since the output of the bilinear gate will usually feed into a conversion gate which requires the input to be already masked, as an optimization, we can have the bilinear gate itself compute shares of the masked output, i.e., $\hat{y} = \mat{K}x + \tilde{y}$. This can be done by providing the correlated randomness $\tilde{\mat{K}}\tilde{x} + \tilde{y}$ instead of $\tilde{\mat{K}}\tilde{x}$. The upshot of this optimization is that one fewer piece of correlated randomness will be required.



\subsection{Composing Gate Protocols}
\label{appendix:protocol_generic}
We now describe a general technique to evaluate circuit composed of the previously specified gates in a distributed fashion. We provide details for the semi-honest fully distributed setting (with preprocessing), where all inputs are secret shared between all parties initially. This can also be thought of as a toolbox for constructing efficient distributed protocols for other constructions similar to ours. While the technique will also work for other settings (e.g., OPRF, public input), the concrete communication costs will be worse than more specially designed protocols. For these settings, we will provide more efficient protocols than provided by this general technique.


% \greg{Does it make sense to say that this very general composition protocol can
% allow one to easily make protocols for variants of our primitives? ``MPC
% toolbox for alternating moduli primitives'' or similar}

\paragraph{Composition protocol.} Consider a circuit $C$ (Definition~\ref{def:computation_circuit}) with input space $\Gin = \prod \Gin_i$. To evaluate $C$ with input $(x_1, \dots, x_l) \in \Gin$, in the fully distributed setting, all parties are given additive shares for each $x_i$. Now, the distributed evaluation of $C$ proceeds as follows:
\begin{itemize}
  
  \item All vertices at the same depth in $C$ are evaluated simultaneously, starting from the source vertices that contain the inputs of the computation. 

  \item The evaluation of a (non-source) vertex in the graph of $C$ is done by each party running the corresponding gate protocol locally on their share of the inputs. 

  \item For an edge $(V_a, V_b)$, suppose that the output of $V_a$ is used as one of the inputs of $V_b$. If the gate protocol corresponding to $\mathcal{G}_V$ requires this input to be masked (e.g., the bilinear gate protocol), then before evaluating $V_b$, each party first masks its share of the output. Now, all parties simultaneously reveal their shares to publicly reveal the masked value.   The masking values are provided to the parties in the preprocessing phase. The same value also need not be masked multiple times if it is required for multiple gates.

  \item The required output shares of the distributed evaluation are given by the evaluation of the sink vertices in the circuit.
\end{itemize}



\paragraph{Communication cost.}
Since the gate protocols themselves are locally computable, the communication cost during a distributed evaluation of a circuit comes solely from the public reconstructions of masked values required for gate protocols. For example, before feeding the output $x$ of a $\LMap^{\mat{A}}_2$ gate into a $\Convert_{(2,3)}$ gate, in the distributed evaluation, all parties will first mask their shares of $x$ to obtain shares of $\hat{x}$. Then, the parties will exchange messages to reconstruct the $\hat{x}$ value required for $\prot_\Convert^{(2,3)}$.

Consider $N$ parties taking part in the distributed evaluation. To reconstruct an $l$-bit value $\hat{x}$ that is additively shared among the parties, one of the following can be done.
\begin{itemize}
  \item Each party sends its share of $\hat{x}$ to each other party. Now, all parties can compute $\hat{x}$ locally. This requires only 1 online round but has a communication cost of $(N-1)l$ bits per party. Each party sends $N-1$ messages. The simplest case for this is when $n=2$, in which case, both parties can simultaneously exchange their shares, and add the two shares locally to reconstruct $\hat{x}$. This requires 1 online round, and has a communication cost of $1$ message and $l$ bits per party. 

  \item All parties can send their share to a designated party, say $\party_1$, who computes $\hat{x}$ and sends it back to everyone. This requires 2 rounds and has a communication cost of $(N-1)l$ bits for $\party_1$ and $l$ bits each for other parties. Here, $\party_1$ sends $N-1$ messages while all other parties send a single message.
\end{itemize}


\paragraph{Reducing round complexity.} It is also straightforward to parallelize the communication to reduce the number of rounds. For this, suppose that we call an edge $(V_a, V_b)$ \textit{communication-requiring} if the output of the protocol for $V_a$ needs to be masked before it is input into the protocol for $V_b$ (in other words, the gate protocol for $V_b$ requires a masked input). Now, define the communication-depth of a vertex $V$ as the maximum number of communication-requiring edges in any path from a source vertex to $V$. Now, instead of evaluating vertices with the same depth simultaneously, we will evaluate vertices with the same \textit{communication-depth} together before the next communication round. By doing so, we can reduce the total number of rounds to the maximum communication-depth.


\paragraph{Preprocessing cost.}
A na{\"i}ve technique to compute the preprocessing required is to add the preprocessing for each gate in the circuit as follows:

\begin{itemize}
\item The $\LMap$ and $\Add$ gates are computed locally and require no preprocessing.

\item Each $\BLMap_p$ gate requires a Beaver-style triple which provides masks for the two inputs and a multiplication of the two masks. Specifically, for $\BLMap_p(\mat{K}, x)$ where $\mat{K} \in \Z_p^{l_2 \times l_1}$ and $x \in \Z_p^{l_1}$, the preprocessed shares are of the form $(\tilde{\mat{K}}, \tilde{x}, \tilde{\mat{K}}\tilde{x})$. Consequently, when $\mat{K}$ is circulant, a total of $(2l_1 + l_2) \log_p$ bits needs to be provided as preprocessing to each party.

\item Each $\Convert_{(2,3)}$ gate requires shares of a random mask $\tilde{x}$ both over $\Z_2$ and $\Z_3$. In other words, for $x \in \Z_2^l$, it requires $l + \log_2{3} \cdot l$ bits of preprocessing per party.

\item Each $\Convert_{(3,2)}$ gate requires shares of a random mask $\tilde{x}$ (over $\Z_3$) as well as $u = \tilde{x}$ and $v = (\tilde{x} + \textbf{1} \bmod 3)\bmod 2$ (both over $\Z_2$). In other words, for $x \in \Z_3^l$, it requires $\log_2{3} \cdot l + 2l$ bits of preprocessing per party.
\end{itemize}

\paragraph{Compressing preprocessing required.}
We describe several optimizations to reduce the total cost of preprocessing. We assume here the presence of a trusted dealer to generate any correlations in the randomness. We note that some of the optimizations, while reducing the size of preprocessing, would be more complicated to generate if a dealer is not present.

\begin{itemize}

  \item (Reducing redundant preprocessing).
  A straightforward optimization is to not mask the same value twice. For example, if the same value $x$ is considered as input for both a $\BLMap$ gate and a $\Convert$ gate, the same mask $\tilde{x}$ can be used for both. 

  \item (Masking $\BLMap$ gate outputs.)
  The standard $\BLMap$ gate requires preprocessing of the form $(\tilde{\mat{K}}, \tilde{x}, \tilde{\mat{K}}\tilde{x})$. However, if the output of the $\BLMap$ gate is then later input to a gate that requires a masked input (e.g., a $\Convert$ gate or even another $\BLMap$ gate), the $\BLMap$ gate can directly mask its output by providing $\tilde{\mat{K}}\tilde{x} + \tilde{y}$ instead. If this is done, the parties will compute a sharing of $\hat{y} = \mat{K}x + \tilde{y}$ using the $\BLMap$ gate. This means that the parties can directly exchange their shares to reconstruct $\hat{y}$ without requiring more preprocessing to mask $y$.

  For both the $\Convert_{2,3}$ and $\Convert_{(3,2)}$ gates, if the masked version $\hat{x}$ of the input $x$ is already known to all parties, only $\log_2{3} \cdot l$ and $2l$ bits respectively of preprocessing are required per party. 

 
  \item (Compression using a PRG).
  Another standard technique for compressing the size of preprocessing is to use a PRG. Intuitively, each party is given a different PRG seed by the trusted dealer which they can use locally to generate their randomness. Only a single party has its shares given by the dealer to ensure that the randomness is appropriately correlated. 

  % For this, the dealer will first generate the randomness as usual. To share the randomness, instead of randomly sampling shares, the dealer (who knows the seeds of all parties) will compute the shares for all but one party (say $\party_1$) using the party's PRG seed. The share for $\party_1$ can now be set so that the shares form an sharing of the required randomness.

  % More specifically, for the $\BLMap$ gate, a dealer first samples a random triple $(\tilde{K}, \tilde{x}, \tilde{K}\tilde{x})$
\end{itemize}

\subsection{Distributed Evaluation Protocols}
\label{appendix:protocols_other_settings}
Here, we provide details for the evaluation of our candidates in other settings.

\subsubsection{Public-input setting}
It is also straightforward to use the same generic technique to construct distributed protocols in the public-input setting. For keyed primitives, in public-input setting, the key is secret shared between the parties but the input is publicly known. The goal of the protocol is for the parties to compute shares of the output.

One useful optimization is that in the public-input setting, a $\BLMap$ gate where the input is known, essentially reduces to a linear gate where the key $\mat{K}$ is secret shared and the input $x$ is publicly known.

\paragraph{2PC public-input protocol for \ttwPRF.}
Concretely, for the evaluation of \ttwPRF in the public-input setting, the first round from the distributed protocol can be entirely skipped. The two parties can directly compute shares of $w = \mat{K}x = \sum_{i \in \{1,2\}} \sharei{\mat{K}}x$ locally. This also means that the preprocessing previously required for the $\BLMap$ gate that computed $\mat{K}x$ is no longer necessary. The rest of the evaluation can now proceed as before with both parties first using $\prot_\Convert^{(2,3)}$ to retrieve shares of $w$ over $\Z_3$, and then using $\prot_\LMap^{\mat{B},3}$ to compute shares of the final output $y$.

In total, the evaluation takes a single round and a communication of $m$ bits per party (to reconstruct $\hat{w}$). The only preprocessing required is for the $\prot_{\Convert}^{(2,3)}$ gate, for which each party will be given $m + \log_2{3} \cdot m$ bits. Furthermore, with PRG compression, $\party_2$ will require no extra preprocessing and $\party_1$ can be given only $\log_2{3} \cdot m$ bits.

\subsubsection{3-party Distributed Evaluation}
\label{appendix:3party_protocol}

In this section, we provide a 3-party (semi-honest) protocol for computing the \ttwPRF candidate that is secure against one passive corruption and does not require any preprocessing.

\paragraph{Functionality.}
Denote the servers by $\party_1, \party_2$, and $\party_3$. We assume that the servers hold replicated additive shares of the key $\mat{K}$ and the input $x$. The key is assumed to be circulant and can be represented by $k \in \Z_2^n$. Concretely, let $(k_1, k_2, k_3)$ and $(x_1,x_2,x_3)$ be additive shares of $k$ and $x$ respectively. Then, each party $\party_i$ is given $k_j$, $x_j$ with $j \neq i$. At the end of the protocol, $\party_2$ and $\party_3$ should hold $y_2$ and $y_3$ respectively such that $(y_2, y_3)$ is a sharing of the wPRF output $y$.

\paragraph{Protocol details.}
\begin{itemize}
  \item First, the three servers compute additive shares of the linear mapping $\mat{K}{x}$ locally using their replicated shares. Note that this can be locally since for two secret shared values $a = a_1 + a_2 + a_3$ and $b = b_1 + b_2 + b_3$, their product can be computed as $ab = \sum_{1 \leq j,k \leq 3} a_jb_k$. Since each party holds two shares of $a,b$ in a replicated sharing scheme, each term $a_jb_k$ can be computed by at least 1 party. Suppose that the share of $\party_i$ is denoted by $\sharei{\mat{K}x}$.

  \item Now, $\party_1$ samples $\tilde{w} \getsr \Z_2^m$, $r_2 \gets \Z_3^m$, and sets $r_3 = \tilde{w} - r_2 \bmod 3$. In other words $(r_2, r_3)$ is a random $\Z_3$ sharing of $r = \tilde{w}$. $\party_1$ sends $\share{\mat{K}x}^{(1)} + \tilde{w}$ and $r_i$ to $\party_{i \in \{2,3\}}$. At the same time, $\party_2$ and $\party_3$ exchange their shares of $\mat{K}x$. All of this can be done in one round.

  \item At this point, $\party_2$ and $\party_3$ can both compute $\hat{w} = \tilde{w} + \sum_{1\leq i \leq 3} \sharei{\mat{K}x}$. Now, for $i \in \{2,3\}$, $\party_i$ can locally compute $w^*_i \gets \prot_\Convert^{(2,3)}(\hat{w} \mid r_i)$. Finally, $\party_i$ can locally compute its share $y_i$ of the output by running $\prot_\LMap^{\mat{B},3}(\mat{B} \mid w^*_i)$.
\end{itemize}


\paragraph{Cost analysis.}
The protocol requires only 1 round, with two messages sent by $\party_1$ and one message each sent by $\party_2$ and $\party_3$. $\party_1$ sends a total of $2m + 2 \log_2(3)\cdot m$ bits while the other two parties send $m$ bits each. $\party_1$ can also generate a reusable PRG seed and provide it to one of the parties which saves $\log_2(3)\cdot m$ bits of communication. 


\subsection{Oblivious Protocols}
\label{appendix:oprf_protocol}
Our distributed evaluation protocols from Section~\ref{subsec:distributed_protocol} can be used directly for semi-honest \textit{oblivious} PRF, or OPRF, evaluation in the preprocessing model. Recall that in the OPRF setting, one party $\party_1$ (called the ``server'') holds the key $\mat{K}$ and the other party $\party_2$ (called the ``client'') holds the input $x$. The goal of the protocol is to have the client learn the output of the PRF for key $\mat{K}$ and input $x$, while the server learns nothing. In the semi-honest setting, both parties can first use the distributed protocol to obtain shares of the PRF output. The server can then send its share to the client so that only the client learns the final output. Such an OPRF protocol would require one extra round over the corresponding distributed PRF protocol. We can however construct much better protocols whose efficiency rivals that of existing DDH-based OPRF protocols. Here, we provide two concrete efficient protocols for evaluating the \ttwPRF candidate (Construction~\ref{construction:23-central-wprf}) in the OPRF setting. 

\paragraph{General structure.} Both protocols take $3$ rounds and involve 2 messages from the server to the client and 1 message from the client to the server. The first server message however, is only required when the key needs to be changed (or re-masked). We call this the key-update phase. Now, when the masked key is already known, our protocols are optimal in the sense that they require only a single message from the client followed by a single message from the server. Since OPRF applications usually involve reusing the same key for many PRF invocations, in such a \textit{multi-input} setting, our protocols are comparable to other 2-round OPRF protocols in literature (e.g., DDH-based). 
\greg{For a real-world use of OPRFs where our construction might be the best option, see \url{https://www.usenix.org/system/files/sec19-thomas.pdf}. 
It uses the DDH-based OPRF for a private set intersection protocol (where the client has one element and the server has a masssive set). }

We detail the two protocols, $\prot^{\textsf{oprf}}_1$ and $\prot^{\textsf{oprf}}_2$, in Sections~\ref{subsec:oprf1} and~\ref{subsec:oprf2} respectively. In Section~\ref{subsec:oprf-comparison}, we compare our OPRF protocols to other common constructions in the literature. Later, in Section~\ref{sec:implementation_and_eval}, we also report on our protocol implementations and compare their performance (both computation and communication) with related work. To foreshadow, a key observation is that in comparison to common OPRF protocols, our protocols are much faster to compute but require preprocessing as well as slightly more communication.


\subsubsection{Oblivious PRF Protocol $\prot^{\textsf{oprf}}_1$}
\label{appendix:oprf1}
Our first OPRF protocol is in spirit similar to the distributed evaluation for the \ttwPRF construction. Since $\mat{K}$ is known to the server, and $x$ is known to the client, both parties do not need to exchange their shares to reconstruct the masked values $\hat{\mat{K}}$ and $\hat{x}$; the party that holds a value can mask it locally and send it to the other party. This allows us to decouple the server's message that masks its PRF key from the rest of the evaluation. To update the key, the server can simply send $\hat{\mat{K}} = \mat{K} + \tilde{\mat{K}}$ to the client. Many PRF evaluations can now be done using the same $\hat{\mat{K}}$.

\paragraph{Preprocessing.} The protocol requires the following preprocessed randomness. The mask $\tilde{\mat{K}}$ is given to the server only when the key-update phase needs to be run. For PRF evaluations, the trusted dealer samples $\tilde{w} \getsr \Z_2^m$ and provides the server and client $\Z_2$ shares of $\tilde{w}$ along with $\Z_3$ shares of $r = \tilde{w}$. Additionally, the dealer also generates an OLE correlation pair $(\tilde{\mat{K}}, \tilde{v})$ and $(\tilde{x}, \hat{v})$ such that $\tilde{\mat{K}} \in \Z_2^{m \times n}$ is a random circulant matrix that is same for all correlations, $\tilde{v} \getsr \Z_2^m$, $\tilde{x} \getsr \Z_2^n$, and $\hat{v} = \tilde{\mat{K}}\tilde{x} + \tilde{v}$. The server is given $(\tilde{\mat{K}}, \tilde{v})$ while the client is given $(\tilde{x}, \hat{v})$. Note that we simply use OLE correlations and do not make use of an actual OLE protocol. In practice, if the key-update phase is run after every $k$ evaluations (where $k$ is known), the OLE correlations for all evaluations can be preprocessed at the beginning. 
% \mahimna{need to write about / point to another section for generating the OLE correlations.}

\paragraph{Protocol details.} Assuming that the masked key $\hat{\mat{K}}$ is known to the client, for an input $x$,  the evaluation protocol now proceeds as follows:
\begin{itemize}
  \item The client computes $\hat{x} = x + \tilde{x}$ and $\share{\hat{w}}^{(2)} = -\mat{\hat{K}}\tilde{x} + \hat{v} + \share{\tilde{w}}^{(2)}$ and sends both $\hat{x}$ and $\share{\hat{w}}^{(2)}$ to the server.

  \item The server first computes $\share{\hat{w}}^{(1)} = \mat{K}\hat{x} - \tilde{v} + \share{\tilde{w}}^{(1)}$, and adds to it the client's share to reconstruct $\hat{w}$. Identical to the distributed protocol, the server now runs $\prot_\Convert^{(2,3)}$ followed by $\prot_\LMap^{\mat{B},3}$ to obtain its share $\share{y}^{(1)}$ of the PRF output. Finally, it sends both $\hat{w}$ and $\share{y}^{(1)}$ to the client.

  \item The client also runs $\prot_\Convert^{(2,3)}$ followed by $\prot_\LMap^{\mat{B},3}$ to obtain its share $\share{y}^{(2)}$ of the PRF output. It can now use the server's share to reconstruct the PRF output $y$.

  % It now runs $\prot_\Convert^{(2,3)}(\hat{w} \mid \share{r}^{(1)})$ to get $\share{w^*}^{(1)}$ (over $\Z_3$) and then $\prot_\LMap^{\mat{B},3}(\mat{B} \mid \share{w^*}^{(1)})$ to obtain its share $\share{y}^{(1)}$ of the PRF output. Finally, it sends both $\hat{w}$ and $\share{y}^{(1)}$ to the client.

  % \item The client now runs $\prot_\Convert^{(2,3)}(\hat{w} \mid \share{r}^{(2)})$ to get $\share{w^*}^{(2)}$ (over $\Z_3$) and then $\prot_\LMap^{\mat{B},3}(\mat{B} \mid \share{w^*}^{(2)})$ to obtain $\share{y}^{(2)}$. Finally, it uses the share received from the server to reconstruct the PRF output $y$ 
\end{itemize}
For evaluating a client input, $\prot^{\textsf{oprf}}_1$ takes 2 rounds and involves a single message in each direction. The client sends $2n$ bits while the server sends $m$ bits and $t$ $\Z_3$ elements. For our parameters ($n=m=256, t=81$), and with proper $\Z_3$ packing, this amounts to roughly $897$ bits of total online communication. To update $\tilde{\mat{K}}$, the server sends a 256-bit message to the client.


\subsubsection{Oblivious PRF Protocol $\prot^{\textsf{oprf}}_2$}
\label{appendix:oprf2}
For the second protocol, the server masks the PRF in a different way; a multiplicative mask is used instead of an additive one. For simplicity, suppose that $n=m$ and that the key $\mat{K}$ is a random full-rank circulant matrix. Then to mask $\mat{K}$, the server computes $\bar{\mat{K}} = \mat{R}\mat{K}$ using a random matrix $\mat{R}$ that is also full-rank and circulant. $\mat{R}$ will be provided as preprocessing to the server, and can be reused for multiple PRF evaluations. The server will send $\bar{\mat{K}}$ to the client in the key-update phase. Note that since the product of two circulant matrices is also circulant, this message is only $n$ bits. Additionally, since the product $\mat{R}\mat{K}$ is essentially a convolution, it can be efficiently computed in $\Theta(n\log n)$ asymptotic runtime using the fast Fourier transform (FFT) algorithm.

\paragraph{Preprocessing.} The protocol requires the following preprocessed randomness. The mask $\mat{R}$ is given to the server only when the key-update phase needs to be run. For PRF evaluations, similar to first protocol, the dealer samples $w \getsr \Z_2^m$ and provides the server and client $\Z_2$ shares of $\tilde{w}$ along with $\Z_3$ shares of $r = \tilde{w}$. Additionally, the dealer gives $\tilde{u} \getsr \Z_2^m$ to the client and $\tilde{v} = \mat{R}^{-1}\tilde{u} + \tilde{w}$ to the server.

\paragraph{Protocol details.} Now, assuming that the masked key $\bar{\mat{K}}$ is known to the client, for an input $x$,  the evaluation protocol now proceeds as follows:
\begin{itemize}
  \item The client computes $\hat{u} = \bar{\mat{K}}x + \tilde{u}$ and sends it to the server.

  \item The server first computes $\mat{R}^{-1}\hat{u} + \tilde{v} = \mat{R}^{-1}(\mat{R}\mat{K}x + \tilde{u}) + (\mat{R}^{-1}\tilde{u} + \tilde{w}) = \hat{w} \mod 2$. Identical to the distributed protocol, the server now runs $\prot_\Convert^{(2,3)}$ followed by $\prot_\LMap^{\mat{B},3}$ to obtain its share $\share{y}^{(1)}$ of the PRF output. Finally, it sends both $\hat{w}$ and $\share{y}^{(1)}$ to the client.

  \item The client also runs $\prot_\Convert^{(2,3)}$ followed by $\prot_\LMap^{\mat{B},3}$ to obtain its share $\share{y}^{(2)}$ of the PRF output. It can now use the server's share to reconstruct the PRF output $y$.
\end{itemize}
For evaluating a client input, $\prot^{\textsf{oprf}}_2$ also takes 2 rounds and involves a single message in each direction. The client sends $n$ bits while the server sends $m$ bits and $t$ $\Z_3$ elements. This is $n$ fewer bits of communication as compared to the first protocol. The key-update phase is slower however, since it involves a convolution rather than a simple vector addition. For our parameters ($n=m=256, t=81$), and with proper $\Z_3$ packing, this amounts to roughly $641$ bits of total online communication. To update $\bar{\mat{K}}$, the server sends a 256-bit message to the client. 


\subsubsection{Comparison to other OPRF protocols}
\label{appendix:oprf-comparison}
Here, we compare the concrete efficiency of our protocols with other protocols in literature.
\paragraph{DDH-based OPRFs.}
A simple and widely used OPRF is based on the Decision Diffie-Hellman (DDH) assumption. Abstractly, given a cyclic group $\mathbb{G}$ of prime order $q$, consider a PRF $\mathsf{F}$ defined as follows: For key $k \in \keyspace$ and input $x \in \inspace$, define $\mathsf{F}(k,x) = H(x)^k$ where $H:\inspace \to \mathbb{G}$ is a hash function modeled as a random oracle. $\mathsf{F}$ is a secure PRF under the DDH assumption in the random oracle model~\cite{naor1999-oprf}.

The PRF $\mathsf{F}$ leads to a natural 2-party (semi-honest) protocol for oblivious evaluation. Concretely, suppose that the client holds $x$ and the server holds $k$. To evaluate the PRF obliviously, the client initiates the interaction by first sampling a mask $r \getsr \mathbb{Z}_q$ and then sending $a \gets H(x)^r$ to the server. The server responds with $b \gets a^k$. Finally, the client can retrieve the PRF computation as $y \gets b^{r^{-1}}$ where $r^{-1}$ is the multiplicative inverse of $r$ in $\Z_q$. Security of this OPRF protocol has been shown in~\cite{jarecki2014-ddhoprf,jarecki2016-ddhoprf}, assuming the one-more discrete-log assumption~\cite{bellare2003-onemore} (and in the random oracle model).

We instantiate a DDH-based OPRF over the Curve25519 elliptic curve and compare its efficiency to our OPRF constructions. The key takeaway we found was that for both our constructions, the total computation time is smaller than the time it takes for a \textit{single} elliptic curve scalar multiplication. One caveat is that our protocols require preprocessing, as well as slightly higher communication. However, we show that for reasonable network speeds, the overall cost of our protocol is still smaller. We provide more details of this comparison as part of our evaluation in Section~\ref{subsec:performance}.


\paragraph{Other OPRFs.}
Many prior OPRF constructions~\cite{freedman2005-oprf,jarecki2009-oprf} require expensive exponentiations because they are based on algebraic PRFs. This means that we can expect similar performance tradeoffs for our protocols when compared to them (i.e., much faster computation, slightly more communication).~\cite{kolesnikov2016-oprf} provides an efficient batched-OPRF protocol based on OT extension in the preprocessing. While we did not perform with the same setup, based on their performance results, we found that our protocol is substantially more efficient for a single (or a small number of) evaluation, but becomes more comparable in performance as the batch-size increases which is unsurprising considering our protocols are not optimized for the batched evaluation. Recent work~\cite{seres2021-legendre} constructs an OPRF protocol from the Legendre PRF~\cite{damgard1988-legendre}. For 128-bit security, their protocol has a communication cost of $\approx 13$KB which is substantially higher than ours.




%-------------------------------------------------%


\begin{table}[t]
\resizebox{\textwidth}{!}{ 
\begin{tabular}{l|l|l}
OWF Params      & KKW params        & Sig. size (KB)\\
$(n,m,t)$       & $(N, M, \tau)$    &   \\\hline
$(128,453,81)$  & $(16, 150, 51)$   & 13.30 \\
                & $(16, 168, 45)$   & 12.48 \\ 
                & $(16, 250, 36)$   & \textbf{11.54} \\
Picnic3-L1      & $(16, 250, 36)$   & 12.60 \\ \hline
$(128,453,81)$  & $(64, 151, 45)$   & 13.59 \\
                & $(64, 209, 34)$   & 11.70 \\ 
                & $(64, 343, 27)$   & \textbf{10.66} \\
Picnic2-L1      & $(64, 343, 27)$   & 12.36 \\ \hline
\end{tabular}

\quad
\begin{tabular}{l|l|l}
OWF Params      & KKW params        & Sig. size (KB)\\
$(n,m,t)$       & $(N, M, \tau)$    &   \\\hline
$(256,906,162)$ & $(16, 324, 92)$   & 50.19 \\ 
                & $(16, 400, 79)$   & 47.08 \\ 
                & $(16, 604, 68)$   & \textbf{45.82} \\ 
Picnic3-L5      & $(16, 604, 68)$   & 48.72 \\ \hline
$(256,906,162)$ & $(64, 322, 82)$   & 51.23 \\
                & $(64, 518, 60)$   & 44.04 \\ 
                & $(64, 604, 57)$   & \textbf{43.45} \\ 
Picnic2-L5      & $(64, 604, 58)$   & 46.18 \\ \hline
\end{tabular}
 }
\caption{ \label{table:sig-sizes}Signature size estimates for Picnic using
\ttOWF, compared to Picnic using LowMC.  The left table shows
security level L1 (128 bits) with $N=16$ and $N=64$ parties, and the right table shows level L5 (256 bits).}
\end{table}


\section{Signature Scheme Details}
\label{appendix:picnic}
In this section we provide additional details on how to construct a signature scheme from our
new primitives, in particular the \ttOWF.


\paragraph{An $N$-party protocol.}
There will be $N$ parties for our MPC protocol, each holding a secret share of
$x$, who jointly compute $y = \sfF(x)$.  The protocol tolerates up to  $N-1$ 
corruptions: given the views of $N-1$ parties we can simulate the remaining
party's view, to prove that the $N-1$ parties have no information about the
remaining party's share. 

The preprocessing phase is similar to that in Picnic.  Each party has a random
tape that they can use to sample a secret sharing of a uniformly random value
(e.g.,  a scalar, vector, or a matrix with terms in $\Z_2$ or $\Z_3$).  Each
party samples their share $\share{r}$ and the shared value is implicitly defined as
$r = \sum_{i=1}^N \share{r}^{(i)}$.  
%We use $\share{r}^{(i)}$ to denote party $i$'s share of $r$,
%or simply $\share{r}$ when the context makes the party's index clear.

We must also be able to create a sharing mod 3, of a secret shared value mod 2.
Let $\tilde{w}\in\Z_2$ be secret shared.  Then to establish shares of $r = \tilde{w} \pmod
3$, the first $N-1$ parties sample a share $\share{r}$ from their random tapes. The
$N$-th party's share is chosen by the prover, so that the sum of the shares is
$r$.  We refer to the last party's share as an \emph{auxiliary value}, since
it's provided by the prover as part of pre-processing.  For efficiency, the random
tape for  party $i$ is
generated by a random seed, denoted $\seed_i$, using a PRG. The state of the first
$N-1$ parties after pre-processing is a seed value used to generate the random
tape, and for the $N$-th party the state is the seed value plus the list of
auxiliary values, denoted $\aux$. 

After pre-processing, the parties enter the \emph{online} phase of the protocol. 
The prover computes $\hat{x}= x + \tilde{x}$, where $\tilde{x}$ is a random value, established during
preprocessing so that each party has a share $\share{\tilde{x}}$. 
The parties can then compute the OWF using the homomorphic properties of the secret sharing, 
and the share conversion gate (to convert shares mod 2 to mod 3, used when computing~$z$)
setup during preprocessing that we describe below. 
During the online phase, parties broadcast values to all other parties and we write $\msgs_i$
to denote the broadcast messages of party $i$. 


\paragraph{Preprocessing phase.} Preprocessing establishes random seeds of all parties and shares of 
\begin{enumerate}
\item $\tilde{x}$: a random vector in $\Z_2^{n}$,\algcomment{Sampled from random tapes}
\item $\tilde{w}$: the vector $\matA \tilde{x}$ in $\Z_2^{m}$, 
\item $r$: a sharing of $\tilde{w} \mod 3$, shares in $\Z_3^m$,  \algcomment{Tapes + one $\aux$ value}
\item $\overline{r}$: a sharing of $1-\tilde{w} \mod 3$, shares in $\Z_3^m$. \algcomment{Computed from $\share{r}$}
\end{enumerate}
The shares of $\overline{r}$ are computed from shares of $r$ as follows (all arithmetic in $\Z_3^m$): the first
party computes $\share{\overline{r}} = 1 - \share{r}$, then the remaining parties compute
$\share{\overline{r}} = - \share{r}$.  Then observe that 
\[\sum_{i=1}^N \share{\overline{r}}^{(i)} = 1 - \share{r}^{(1)} - \ldots - \share{r}^{(N)} = 1 - \sum_{i=1}^{N} \share{r}^{(i)} = 1-r \]
 as required. 

\paragraph{Online phase.}
The public input to the online phase is $\hat{x} = x + \tilde{x}$. 
\begin{enumerate}

\item The parties locally compute $\hat{w} \in \Z_2^{m}$ as $\hat{w} = \matA\hat{x}$ (since both $\hat{x}$ and $\matA$ are public). 

\item Let $z$ be a vector in $\Z_3^m$ and let $z_i$ denote the $i$-th component. Each party defines 
\[
    \share{z_i}  = \begin{cases}
                \share{r_i}  & \text{if $\hat{w}_i = 0$} \algcomment{\text{Note that $\share{r_i} = \share{w_i'}$}}\\
                \share{\overline{r}_i}  & \text{if $\hat{w}_i = 1$} \algcomment{\text{Note that $\share{\overline{r}_i} = \share{1- w_i'}$}}\\
            \end{cases}
\]
then locally computes $\share{y} = \matB\share{z}$. All parties broadcast $\share{y}$ and reconstruct the output $y\in\Z_{3}^t$. 
In this step each party broadcasts $t$ values in $\Z_3$.
\end{enumerate}

\paragraph{Correctness.} The protocol correctly computes the \ttOWF.  The
first step computes $w = \matA x$, 
updating the public value $\hat{x} = x + x'$ with $\hat{w} = w + \tilde{w}$.  The
second step is where the bits of $w$ are cast from $\Z_2$ to $\Z_3$.  The 
parties have sharings of $\tilde{w}$ and $1-\tilde{w}$ mod 3 (we focus on a single bit here, for simplicity). The key observation is
that when $\hat{w} = 0$, then $w$ and $\tilde{w}$ are the same, and when $\hat{w} =
1$, $w$ and $\tilde{w}$ are different. So in the first case we set the shares of $ z =
w \mod 3$ to the shares of $[\tilde{w}] \mod 3$, and when $\hat{w} = 1$, we set the
shares of $z$ to the complement of $\tilde{w}$.

\paragraph{Communication costs.}
Here we quantify the cost of communication for the MPC inputs, the $\aux$ values and the broadcast $\msgs$ of one party,
as this will directly contribute to the signature size in the following section. 
Let $\ell_3$ be the bit length of an element in $\Z_3$; the direct encoding has
$\ell_3 = 2$, but with compression we can reduce $\ell_3$ to as little as
$\log_2(3) \approx 1.58$. \footnote{To compress a vector $v\in\Z_3^{n}$, convert it to the integer it represents: $V = \sum_{i=0}^n v_i^{i}$
and output the binary representation of $V$. }   The size of the
$\aux$ information is $m\ell_3$, the MPC input value has size $n$ bits, 
and the broadcast values have size $t\ell_3$ bits (per party). 
The total in bits is thus 
\begin{equation} \label{eqn:sizeMPC}
|\textsf{MPC}(n,m,t)| = m\ell_3 + n + t\ell_3\;.
\end{equation}
For the parameters 
$(n,m,t)=(128, 453, 81)$ the total is 972 bits (L1 security: 128-bit classical, 64-bit quantum)
and when $(n,m,t) = (256, 906, 162)$ the total is 1943 bits (L5 security: 256-bit classical, 128-bit quantum).  This compares favorably to
Picnic at the same security level, which communicates 1161--1328 bits at L1
and 2295--2536 bits at L5, depending on whether LowMC uses a full or partial S-box layer~\cite{kales2020-picnic}.

\paragraph{Signature scheme}
Given the MPC protocol above, we can compute the values $\hat{x}$, $\aux$ and
$\msgs$ for the \ttOWF and neatly drop it into the KKW proof system used in
Picnic.  The signature generation and verification algorithms for the
\ttOWF  signature scheme are given in \cref{fig:23-picnic}.


We use a cryptographic hash function $\hash: \{0,1\}^*\to \{0,1\}^{2\secpar}$
for computing commitments, and the function $\Expand$ takes as input a random $2\secpar$-bit string
and derives a challenge having the form $(\CCC, \PPP)$ where $\CCC$ is a subset of $[M]$
of size $\tau$, and $\PPP$ is a list of length $\tau$, with entries in $[N]$. 
The challenge $(\CCC, \PPP)$ defines $\tau$ pairs $(c,p_c)$ where $c$ is the
index of  an MPC instance for which the verifier will check the online phase,
and $p_c$ is the index of the party that will remain unopened. 


\paragraph{Optimizations and simplifications.}
For ease of presentation, \cref{fig:23-picnic} omits some optimizations that
are essential for efficiency, but are not unique to the $(2,3)$-signature schemes, 
they are exactly as in Picnic. All random seeds in a signature are derived from
a single random root seed, using a binary tree construction. First we derive
$M$ initial seeds, once for each MPC instance, then from from the initial seed
we derive the $N$ per-party seeds. This allows the signer to reveal the seeds
of $N-1$ parties by revealing only $\log_2(N)$ intermediate seeds, similarly, 
the initial seeds for $M-\tau$ of $M$ instances may be revealed by communicating
only $(\tau)\log_2(M/\tau)$ $\secpar$-bit seeds.

For the commitments $h'^{(k)}$ to the online execution, $\tau$ are recomputed
by the verifier, and the prover provides the missing $M-\tau$.  Here we compute
the $h'^{(k)}$ as the leaves of a Merkle tree,  so that the prover can provide
the missing commitments by sending only $\tau\log_2(M/\tau)$ $2\secpar$-bit
digests. 

Finally, \cref{fig:23-picnic} omits a random salt, included in each signature, as well as counter
inputs to the hash functions to prevent multi-target
attacks~\cite{dinur2019-picnic-attacks}. Also, hashing the public key when computing the
challenge, and prefixing the inputs to $\hash$ in each use for domain
separation should also be done, as in~\cite{picnic-spec}. 

\paragraph{Signature size}
The size of the signature in bits is:
\begin{align*}
\textunderbrace{\secpar\tau\log_2\left(\frac{M}{\tau}\right)}{initial seeds} + 
\textunderbrace{2\secpar\tau\log_2\left(\frac{M}{\tau}\right)}{Merkle tree commitments} + 
\tau\left( 
     \textunderbrace{\secpar\log_2N}{per-party seeds}  + 
     \textunderbrace{|\textsf{MPC}(n,m,t)|}{one MPC instance, \cref{eqn:sizeMPC}} 
\right)
\end{align*} 
and we note that the direct contribution of OWF choice is limited to
$|\textsf{MPC}(n,m,t)|$\footnote{The size $|\textsf{MPC}(n,m,t)|$ is a slight
overestimate since for $1/N$ instances we don't have to send $\aux$, if the
last party is unopened. In \cref{table:sig-sizes} our estimates include this,
but it's a very small difference as $\tau$ is quite small. }.  However, the
size of this term can impact the choice of $(N,M,\tau)$.  The Picnic parameters
$(N, M, \tau)$ must be chosen so that the soundness error, 
\begin{equation*} \label{eqn:soundness}
    \epsilon(N,M,\tau) = \max_{M-\tau \le k \le M} \left\lbrace  \frac{\binom{k}{M-\tau} }{\binom{M}{M-\tau} N^{k-M+\tau} } \right\rbrace\,.
\end{equation*}
is less than $2^{-\secpar}$.
By searching the parameter space for fixed $N$ and various options for $M,
\tau$, we get a curve, and choose from the combinations in the ``sweet spot'',
near the bend of the curve with moderate computation costs. This part of the curve is
similar as in Picnic, and we present some options from it in \cref{table:sig-sizes}.

\begin{figure}[p]
 \begin{minipage}[t]{1.1\textwidth}
 \begin{protocolbox}{\ttOWF Signatures}
 \begin{description}
    \item[Inputs] Both signer and verifier have $\sfF$, $y = \sfF(x)$, the
        message to be signed $\Msg$, and the signer has the secret key $x$.  The
            parameters of the protocol $(M, N, \tau)$ are described in the text.
    \item[Commit] For each MPC instance $k\in[M]$, the signer does the following.
    \begin{enumerate}
        \item Choose uniform $\seed^{(k)}$ and use  to generate values $(\seed^{(k)}_i)_{i \in [N]}$, and compute  
        $\aux^{(k)}$ as described in the text. 
        For $i=1,\ldots N-1$, let $\state^{(k)}_i = \seed^{(k)}_i$ and  let $\state^{(k)}_{N} = \seed^{(k)}_{N} || \aux^{(k)}$.
        \item Commit to the preprocessing phase:
        \begin{align*}
        \com^{(k)}_{i} = \hash(\state^{(k)}_i) \text{ for all } i\in [N], \quad 
        h^{(k)} = \hash(\com^{(k)}_{1},\ldots,\com^{(k)}_{N}).
        \end{align*}						
        \item Compute MPC input $\hat{x}^{(k)} = x + \tilde{x}^{(k)}$ based on the secret key $x$ and the random values $\tilde{x}^{(k)}$ defined by preprocessing.
        \item Simulate the online phase of the MPC protocol, producing $(\msgs^{(k)}_i)_{i \in [N]}$.			
        \item Commit to the online phase:
        $
         h'^{(k)}= \hash(\hat{x}^{(k)}, \msgs_{1}^{(k)}, \ldots, \msgs_{N}^{(k)} ).
        $
    \end{enumerate}
    
    \item[Challenge] 
    The signer computes $\ch = \hash(h_1, \ldots h_M, h_1', \ldots, h_M',
    \Msg)$, then expands $\ch$ to the challenge $(\CCC, \PPP) := \Expand(\ch)$, as described in the text. 
    
    \item[Signature output]
    The signature $\sigma$ on $\Msg$ is 
    \[
    \sigma = (\ch, 
              ((\seed^{(k)}, h^{(k)} )_{k\not\in\CCC}, 
              ( \com^{(k)}_{p_k}, (\state^{(k)}_{i})_{i\neq p_k}, \hat{x}^{(k)}, \msgs_{p_k}^{(k)} )_{k\in\CCC})_{k\in[M]})
    \]
    
    \item[Verification] The verifier parses $\sigma$ as above, and does the following.  
    \begin{enumerate}
        \item Check the preprocessing phase. For each $k\in[M]$:
        \begin{enumerate}
        \item If $k \in \CCC$: for all $i\in[N]$  such that $i \neq p_k$, the verifier uses $\state^{(k)}_{i}$ to compute $\com^{(k)}_{i}$ as 
            the signer did, then computes $h'^{(k)} = \hash(\com^{(k)}_{1},\ldots,\com^{(k)}_{N})$ using 
            the value $\com^{(k)}_{p_k}$ from $\sigma$. 
        \item If $k\not\in\CCC$: the verifier uses $\seed^{(k)}$ to compute $h'^{(k)}$ as the signer did.
        \end{enumerate} 
        
        \item Check the online phase:
        \begin{enumerate}
            \item For each $k \in \CCC$ the verifier simulates the online phase using $(\state^{(k)}_{i})_{i \neq p_k}$,  
                masked witness $\hat{x}$ and $\msgs^{(k)}_{p_k}$ to compute $(\msgs_i)_{i \neq p_k}$. 
                Then compute $h^{(k)}$ as the signer did. The verifier outputs `invalid' if the output of the MPC simulation is not equal to $y$.
        \end{enumerate}
    \item The verifier computes $\ch' = \hash(h_1, \ldots h_M, h_1', \ldots, h_M', \Msg)$ and outputs `valid' if $\ch' = \ch$ and `invalid' otherwise. 
    \end{enumerate}
 \end{description}
 \end{protocolbox}
 \end{minipage}
	\vspace*{-10pt}
	\caption{\label{fig:23-picnic}Picnic-like signature scheme using the \ttOWF and the KKW proof system.} 
\end{figure}


%--------------------------%
%!TEX root = ../main.tex
\section{Deferred Implementation Details}
\label{appendix:implementation}
Here, we provide the details for our optimization techniques.
\paragraph{Bit packing for $\Z_2$ vectors.} 
Instead of representing each element in a $\Z_2$ vector separately, we pack several elements into a machine word and operate on them together in an SIMD manner. For our architecture with 64-bit machine words, we can pack a vector in $\Z_2^{256}$ (e.g., the input $x$) into 4 words. Since the key $\mat{K}$ is circulant and can be represented with $n=256$ bits, it can also be represented by 4 words. This results in a theoretical $\times64$ maximum speedup in run-time for operations involving $\mat{K}$ and $x$. 

\paragraph{Bit slicing for $\Z_3$ vectors.}
We represent each element in $\Z_3$ using the two bits from its binary representation. For $z \in \Z_3$, the two bits are the least significant bit (LSB) $l_z = z \bmod 2$, and the most significant bit (MSB) $h_z$ which is $1$ if $z=2$ and $0$ otherwise. $\Z_3$ vectors are now also represented by two binary vectors, one containing the MSBs, and one containing the LSBs. Operations involving a $\Z_3$ vector are translated to operations on these binary vectors instead. We also take advantage of the bit packing optimization when operating on the binary vectors.
In Table~\ref{table:z3_operations}, we specify how we perform common operations on $\Z_3$ elements using our bit slicing approach. 

\begin{table}[ht]
\begin{center}
    \begin{tabular}{|c|c|c|}
        \hline
        Operation & Result MSB & Result LSB\\
        \hline\hline
        $z_1 + z_2 \bmod 3$ & $(l_1 \vee l_2) \oplus (l_1 \vee h_2) \oplus (l_2 \vee h_1)$  & $(h_1 \vee h_2) \oplus (l_1 \vee h_2) \oplus (l_2 \vee h_1)$\\
        $-z_1 \bmod 3$ & $l_1$ & $h_1$ \\
        $z_1z_2 \bmod 3$ & $(l_1 \wedge l_2) \oplus (h_1 \wedge h_2)$ & $(l_1 \wedge h_2) \oplus (h_1 \wedge l_2)$\\
        $\textrm{MUX}(s;z_1,z_2)$ & $(h_2 \wedge s) \vee (h_1 \wedge \neg s)$ & $(l_2 \wedge s) \vee (l_1 \wedge \neg s)$\\
        \hline
    \end{tabular}
\end{center}
\caption{Operations in $\Z_3$. $z_1$ and $z_2$ are elements in $\Z_3$ with (MSB, LSB) = $(h_1,l_1)$ and $(h_2, l_2)$ respectively. For a bit $s$, the operation $\textrm{MUX}(s;z_1, z_2)$ outputs $z_1$ when $s=0$ and $z_2$ when $s=1$.}
\label{table:z3_operations}
\mahimna{I'm debating whether we need to have this table at all. Once we specify that we split $\Z_3$ elements into 2 bits, for any $\Z_3$ operation, the corresponding operation on the bits can easily be computed using a truth table.}
\end{table}



\paragraph{Lookup table for matrix multiplication.} Recall that the \ttwPRF evaluation contains a mod-3 linear map using a public matrix $\mat{B} \in \Z_3^{81\times256}$. Specifically, it computes the matrix-vector product $\mat{B}w$ where $w \in \Z_3^{256}$. Since $\mat{B}$ is known prior to evaluation, we can use a lookup table to speedup the multiplication by $\mat{B}$\cite{arlazarov1970economical}. The same preprocessing can also by reused for multiple evaluations of the wPRF.

For this, we partition $\mat{B}$, which has $m=256$ columns, into $16$ slices of $16$ columns each. These matrices, denoted by $\mat{B}_1,\dots, \mat{B}_{16}$, are all in $\Z_3^{81 \times 16}$. Now, for each $\mat{B}_i$, we will effectively build a lookup table for its multiplication with any $\Z_3$ vector of length $16$. A point to note here is that since we  represent $\Z_3$ vectors by two binary vectors (from the bit slicing optimization), it is sufficient to preprocess multiplications (modulo 3) for binary vectors of length $16$. To multiply $\mat{B}_{i}$ by a vector in $\Z_3^{16}$, we can first multiply it separately by the corresponding MSB and LSB vectors, and then subtract the former from the latter modulo 3. This works since for $z_1, z_2 \in \Z_3$ the multiplication $z_1z_2 \bmod 3$ can be given by $z_1(2\cdot h_2 + l_2) \bmod 3 = z_1l_2 - z_1h_2 \bmod 3$ where $h_2, l_2$ are the MSB and LSB of $z_2$ respectively. Now, to multiply $\mat{B}$ by $v \in \Z_3^{256}$, we first evaluate all multiplications of the form $\mat{B}_iv_i$ where $v_i$ is the $\Z_3^{16}$ vector denoting the $i^\thtext$ slice of $v$ if it was split into 16-element chunks. Then, multiplication by $\mat{B}$ is given by $\mat{B}v = \sum_{i \in [16]} \mat{B}_i v_i \mod 3$.
\greg{Is this the ``method of four Russians''?\url{https://github.com/malb/m4ri}, \url{https://bitbucket.org/malb/m4ri/wiki/Further\%20Reading} }
\greg{$w$ is secret, so the access pattern of the $\mat{B}_i$ should be independent of $w$. Is this the case? If not there is a 
cache timing side-channel. The timing variation may be observable over the network (e.g., to the other party in the OPRF evaluation).  }

In general, a $\mat{B} \in \Z_3^{t \times m}$ can be partitioned into $m/c$ partitions with $c$ columns each (assume $c$ divides $m$ for simplicity), and would require a total multiplication lookup table size of $(m/c) \cdot 2^c \cdot t$ $\Z_3$ elements. Multiplying $\mat{B}$ by $v \in \Z_3^{m}$ requires $2(m/c)$ lookup table accesses (one each for MSB and LSB of $v$ per partition of $\mat{B}$) and an addition of $(m/c - 1)$ $\Z_3^{t}$ vectors.


For our parameters, this results in a table size of roughly $135$MB with proper $\Z_3$ packing. We chose $c = 16$ as a compromise between the size of the lookup table and increased computational efficiency.

