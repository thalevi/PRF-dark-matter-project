%!TEX root = ../main.tex

\newpage

\section{Candidate Constructions}
\label{sec:candidates}
In this section, we introduce our suite of candidate constructions for a number of cryptographic primitives: weak pseudo-random function families (wPRF), one-way functions (OWF), and pseudo-random generators (PRG). Our constructions are all based on similar interplays between mod-2 and mod-3 linear mappings. 

\subsection{Basic structure}
Given the wide range of candidates we propose, we find it useful to have a clean way to describe the operations that are performed in our candidate constructions. For this, we take inspiration from the basic formalism of the function secret sharing (FSS) approach to MPC with preprocessing, first introduced by Boyle, Gilboa, and Ishai~\cite{boyle2019-fss-preprocess}. Abstractly, the technique starts by representing an MPC functionality as a circuit, where each gate represents an operation to be performed in the distributed protocol. Then the inputs and outputs of each gate are secret shared and the gate operation is ``split'' using an FSS scheme~\cite{boyle2015-fss,boyle2016-fss-extension}. To evaluate the circuit in a distributed fashion, the dealer first shares a random mask for each input wire in the circuit, and possibly additional correlated randomness. Now, to compute a gate, the masked input is first revealed to all parties, who can then locally compute shares of the output wire or shares of the masked output. This technique can also be viewed as a generalization of the TinyTable protocol~\cite{damgard2017-tinytable}.

While we find it useful to use the formalism from~\cite{boyle2019-fss-preprocess} for representing the circuit to be computed, we do not explicitly require the FSS formalism for splitting the functionality of each gate. The individual operations are quite straightforward, and we instead chose to directly provide the distributed protocols that compute them. Further, by doing so, our protocols can make better use of correlated randomness to reduce the overall protocol cost as compared to the general techniques in~\cite{boyle2019-fss-preprocess}.


\paragraph{Circuit gates.}
We make use of just five types of basic operations, or ``gates,'' which we detail below. All our constructions can be succinctly represented using just these gates. In Section~\ref{sec:distributed_protocols}, we will provide distributed protocols to compute them. Local protocols for the first two gates follow directly from the homomorphic properties of (additive) secret sharing but we mention them for completeness, as well as to enable a pictorial representation of constructions that use them. To cleanly describe both our candidate constructions and their distributed protocols, the gates we describe here depart from the formalism in~\cite{boyle2019-fss-preprocess} in that the input values of the circuit itself are not secret shared. The MPC protocols will instead take in secret shared inputs as necessary and evaluate the circuit in a distributed fashion. We denote by $\gateset$, the set comprising of these gates.

\begin{itemize}
    \item \textbf{Mod-$p$ Public Linear Gate.}
    For a prime $p$, given a public matrix $\mat{A} \in \Z_p^{s \times l}$, the gate $\LMap^\mat{A}_p(\cdot)$ takes as input $x \in \Z_p^l$ and outputs $y = \mat{A}x \in \Z_p^s$.

    \item \textbf{Mod-$p$ Addition Gate.}
    For a prime $p$, the gate $\Add_p(\cdot,\cdot)$ takes input $x,x' \in \Z_p^{l}$ and outputs $y = x + x' \bmod p$.

    \item \textbf{Mod-$p$ Bilinear Gate.}
    For a prime $p$, and positive integers $s$ and $l$, the gate $\BLMap_p^{s,l}(\cdot, \cdot)$ takes as input a matrix $\mat{K} \in \Z_p^{s \times l}$ and a vector $x \in \Z_p^l$ and outputs $y = \mat{K}x \in \Z_p^s$. When clear from context, we will drop the superscript and simply write $\BLMap_p(\mat{K}, x)$.
    
    \item \textbf{$\Z_2 \to \Z_3$ conversion.} For a positive integer $l$, the gate $\Convert^l_{(2,3)}(\cdot)$ takes as input a vector $x \in \Z_2^l$ and returns its equivalent representation $x^*$ in $\Z_3^l$. When clear from context, we will drop the superscript and simply write $\Convert_{(2,3)}(x)$.

    \item \textbf{$\Z_3 \to \Z_2$ conversion.} For a positive integer $l$, the gate $\Convert^l_{(3,2)}(\cdot)$ takes as input a vector $x \in \Z_3^l$ and computes its mapping $x^*$ in $\Z_2^l$. For this, each $\Z_3$ element in $x$ is computed modulo 2 to get the corresponding $\Z_2$ element in the output $x^*$. Specifically, each $0$ and $2$ are mapped to $0$ while each $1$ is mapped to $1$. When clear from context, we will drop the superscript and simply write $\Convert_{(3,2)}(x)$.
\end{itemize}


\noindent Note that the difference between the $\LMap$ and the $\BLMap$ gates is seen in the context of their distributed protocols. For $\LMap$, the matrix $\mat{A}$ will be publicly available to all parties, while the input $x$ will be secret shared. On the other hand, for $\BLMap$, both the key $\mat{K}$ and the input $x$ will be secret shared. We call this gate \textit{bilinear} because its output is linear in both of its (secret-shared) inputs. Also note that although the $\Convert_{(2,3)}$ gate is effectively a no-op in a centralized evaluation, in the distributed setting, the gate will be used to convert an additive sharing over $\Z_2$ to an additive sharing over $\Z_3$.

\greg{At this point I feel that Lin and BL not that descriptive.  It's
vector-matrix multiply, and for Lin the matrix is public.  Readers know what
vector-matrix multiplication is, so why call it a Bilinear gate? Maybe as I
read further it'll be clearer why they have these names.  If it's to contrast
these against the non-linear operations (as is typical in symmtric-key
literature), then maybe point out what the nonlinear operations are.  }

As described previously, for the (bi)linear mappings, we focus only on constructions that use mod-2 and mod-3 mappings. In \fig~\ref{fig:graphical_gates}, we provide a pictorial representation for each circuit gate. We will connect these pieces together to also provide clean visual representations for all our constructions. 

The homomorphic properties of additive secret sharing directly imply that the gates $\LMap$ and $\textsf{Add}$ can be computed locally without any preprocessing or communication. For the other gates, we provide protocols to evaluate them in a distributed setting (i.e., where all inputs and outputs are secret shared) in Section~\ref{sec:distributed_protocols}.

%!TEX root = ../../main.tex

\begin{figure}[t]
\centering
\begin{tikzpicture}[scale=0.4]
    \begin{scope}[shift={(-15,0)}]
        \node [draw,rectangle,minimum width=1.6cm,minimum height=0.55cm,fill=blue!15] {$\mat{A}$};
        \node[shift={(0,-0.6)}] {$\LMap^\mat{A}_2$ gate};
        \node[shift={(-1,0)}] {$x$};
        \node[shift={(1,0)}] {$y$};

        \node [draw,rectangle,minimum width=1.6cm,minimum height=1.065cm,pattern=north west lines, pattern color=orange!70, shift={(0,-2)}] {$\mat{A}$};
        \node[shift={(0,-2.8)}] {$\LMap^\mat{A}_3$ gate};
        \node[shift={(-1,-2)}] {$x$};
        \node[shift={(1,-2)}] {$y$};

    \end{scope}
    \begin{scope}[shift={(0,0)}]
        \draw[fill=blue!15] (-2,0.68) -- (2,0.68) -- (2,-0.68) -- (-2,-0.68) -- (-2,-0.43) -- (1.75,-0.43) -- (1.75,0.43) -- (-2,0.43) -- (-2,0.68);

        \node[shift={(0,-0.6)}] {$\BLMap_2$ gate};
        \node[shift={(-1,0.25)}] {$\mat{A}$};
        \node[shift={(-1,-0.25)}] {$x$};
        \node[shift={(1,0)}] {$y$};


        \draw[pattern=north west lines, pattern color=orange!70, shift={(0,-5)}] (-2,1.3) -- (2,1.3) -- (2,-1.3) -- (-2,-1.3) -- (-2,-0.8) -- (1.5,-0.8) -- (1.5,0.8) -- (-2,0.8) -- (-2,1.3);

        \node[shift={(0,-2.8)}] {$\BLMap_3$ gate};
        \node[shift={(-1,-1.6)}] {$\mat{A}$};
        \node[shift={(-1,-2.4)}] {$x$};
        \node[shift={(1,-2)}] {$y$};

    \end{scope}   
    \begin{scope}[shift={(15,0)}]
        \node [draw,trapezium,trapezium left angle=65,trapezium right angle=65,minimum height=0.55cm,fill=blue!15,rotate=90] {};
        \node[shift={(0,-0.7)}] {$\Convert_{23}$ gate};
        \node[shift={(-0.5,0)}] {$x$};
        \node[shift={(0.6,0.05)}] {$x^*$};

        \node [draw,trapezium,trapezium left angle=65,trapezium right angle=65,minimum height=0.55cm,pattern=north west lines, pattern color=orange!70,rotate=-90, shift={(2,0)}] {};
        \node[shift={(0,-2.8)}] {$\Convert_{32}$ gate};
        \node[shift={(-0.5,-2)}] {$x$};
        \node[shift={(0.6,-1.95)}] {$x^*$};

    \end{scope}          
\end{tikzpicture}
\caption{Pictorial representations of the circuit gates.}
\label{fig:graphical_gates}
\end{figure}


\paragraph{Construction styles.}
The candidate constructions we introduce follow one of two broad styles which we detail below. wPRFs constructions for both styles were first proposed by~\cite{boneh2018-darkmatter}. Here, we also propose a suite of symmetric primitives (e.g., OWFs, PRGs) with the same basic structure.

\begin{itemize}
    \item \textbf{$(p,q)$-constructions.}
    For distinct primes $p, q$, the $(p,q)$-constructions have the following structure: On an input $x$ over $\Z_p$, first a linear $\bmod~p$ mapping is applied, followed by a linear $\bmod~q$ mapping. Note that after the $\bmod~p$ mapping, the input is first reinterpreted as a vector over $\Z_q$. For unkeyed primitives (e.g., OWF), both mappings are public, while for keyed primitives (e.g., wPRF), the key is used for the first linear mapping. The construction is parameterized by positive integers $n, m, t$ (functions of the security parameter $\secparam$) denoting the length of the input vector (over $\Z_p$), the length of the intermediate vector, and the length of the output vector (over $\Z_q$) respectively. The two linear mappings can be represented by matrices $\mat{A} \in \Z_p^{m \times n}$ and $\mat{B} \in \Z_q^{t \times m}$. For keyed primitives, the key $\mat{K} \in \Z_p^{m \times n}$ will be used instead of $\mat{A}$.

    Concretely, given an input $x \in \Z_p^n$, the construction output is of the form $y = \mat{B}w \in \Z_q^t$ where $w = \mat{A}{x}$ is first viewed over $\Z_q$. In this paper, we will analyze this style of constructions for $(p,q) = (2,3)$ and $(3,2)$ since these are arguably the simplest constructions that employ linear mappings over alternate moduli. We find that the $(2,3)$-constructions outperform the $(3,2)$-constructions and we will use primarily use the former style for our constructions.

    \greg{Why did we choose to focus on (2,3), (3,2)?}
    \itai{At a high level the standard conversion from mod 3 to mod 2 loses information and is less efficient.
    The LPN-style construction can be viewed as a variant of (3,2) that does not lose information.}
    
    \item \textbf{LPN-style-constructions.}
    These constructions have the following general structure: On input $x$ over $\Z_2$, first a linear $\bmod~2$ mapping given by the matrix $\mat{A}$ is applied to obtain $u$. Concurrently, the same linear mapping is also applied over $\Z_3$ (where both $x$ and $\mat{A}$ are now reinterpreted over $\Z_3$) and then reduced modulo $2$ to obtain $v$. The sum $w = u \oplus v$ is then multiplied by a second linear mapping (given by $\mat{B}$) over $\Z_2$. The mapping $\mat{B}$ is always public, while for keyed primitives, the key $\mat{K}$ is used instead of the first mapping $\mat{A}$.

    The construction is parameterized by positive integers $n,m,t$ (as functions of the security parameter $\secparam$) denoting the size of the input vector, the intermediate vector(s), and the output vector (all over $\Z_p$).
    Concretely, given $\mat{A} \in \Z_2^{m \times n}$ and a public $\mat{B} \in \Z_2^{t \times m}$, for an input $x \in \Z_2^n$, the construction computes an output of the form:
    \[
        y = \left[(\mat{A}x \bmod 2) + (\mat{A}x \bmod 3) \bmod 2\right] \bmod 2.
    \]
    The upshot of this style of constructions is that both the input and the output are both over $\Z_2$. Intuitively, each output bit can be thought of as a deterministic Learning-Parity-with-Noise (LPN) instance with a noise rate of $1/3$. The noise is deterministically generated and is dependent on the input $x$ and a specific column of $\mat{A}$. The noise for the $i^\thtext$ instance will be $1$ if and only if $\left<\mat{A}_i, x\right> = 1$.
\end{itemize}


\paragraph{Winning candidates.}
Through cryptanalysis and considering the cost for each candidate, we find that some of our candidates are more suited (i.e., ``win'') for a particular setting. For instance, we found that the $(3,2)$-constructions are overshadowed by the corresponding the $(2,3)$-constructions in all settings we consider. Specifically, out of the candidates we consider, we find the following:
\begin{enumerate}
    \item \ttwPRF is the best wPRF candidate with no restriction for input and output space. 
    \item LPN-wPRF is the best wPRF candidate where both the input and output need to be over $\Z_2$.
    \item \ttOWF is the best OWF candidate with no restriction for input and output space.
    \item LPN-PRG is the best PRG candidate.
\end{enumerate}
We provide formal descriptions and pictorial descriptions of our winning candidates in \fig~\ref{fig:primary-constructions} and~\ref{fig:lpn-constructions}.

%!TEX root = ../main.tex

\begin{figure}[h]

\protbox{$(2,3)$-constructions}{

\textbf{Parameters.} Let $\secparam$ be the security parameter and define parameters $n, m, t$ as functions of $\secparam$ such that $m \geq n, m \geq t$.

\textbf{Public values.}
Let $\mat{A} \in \Z_2^{m \times n}$ and $\mat{B} \in \Z_3^{t \times m}$ be fixed public matrices chosen uniformly at random.

\begin{construction}[Mod-2/Mod-3 wPRF Candidate~\cite{boneh2018-darkmatter}]
The wPRF candidate is a family of functions $\PRFfunc : \Z_2^{m \times n} \times \Z_2^n \to \Z_3^t$ with key-space $\keyspace_\secparam = \Z_2^{m \times n}$, input space $\inspace_\secparam = \Z_2^n$ and output space $\outspace_\secparam = \Z_3^t$. For a key $\mat{K} \in \keyspace_\secparam$, we define $\sfF_{\mat{K}}(x) = \PRFfunc(\mat{K}, x)$ as follows:

\begin{enumerate}[topsep=0pt]
    \item On input $x \in \Z_2^n$, first compute $w = \BLMap_2(\mat{K}, x) = \mat{K}x$.
    \item Output $y = \LMap^{\mat{B}}_3\left(\Convert_{(2,3)}(w)\right)$. That is, view $w$ as a vector over $\Z_3$ and then output $y = \mat{B}w$. 
\end{enumerate}
\label{construction:23-central-wprf}
\end{construction}

\begin{center} %!TEX root = ../../main.tex

\scalebox{0.8}{
\begin{tikzpicture}
    \begin{scope}
        \draw[twostyle] (-0.74,0.65) -- (0.74,1.05) -- (0.74,-1.05) -- (-0.74,-0.65) -- (-0.74,-0.55) -- (0.5,-0.55) -- (0.5,0.55) -- (-0.74,0.55) -- cycle;
         \draw (-0.8,0.65) node[left] {$\mat{K}$};
         \draw (-0.8,-0.65) node[left] {$x$};
    \end{scope}
    \begin{scope}[shift={(0.95,0)}]
    \tikzmath{\sidelen=0.6;\maxnum=4;}
    \foreach \num in {0,...,\maxnum} {
        \tikzmath{\yshift=(\num-(\maxnum)/2)*(\sidelen + 0.3);}
        \node [draw,regular polygon, regular polygon sides=3,scale=0.5,shift={(0,\yshift)},rotate=90] {};
    }
    \end{scope}
    \begin{scope}[shift={(1.85,0)}]
    \node [draw,trapezium,minimum height=1.48cm,trapezium left angle=75,trapezium right angle=75, rotate=-90,threestyle] {\rotatebox{90}{$\mat{B}$}};
    \draw (0.75,0) node[right] {$y$};
    \end{scope}
\end{tikzpicture}
}
 \end{center}


\begin{construction}[Mod-2/Mod-3 OWF Candidate]
The OWF candidate is a function $\OWFfunc : \Z_2^n \to \Z_3^t$ with input space $\inspace_\secparam = \Z_2^n$ and output space $\outspace_\secparam = \Z_3^t$. We define $\sfF(x) = \OWFfunc(x)$ as follows:

\begin{enumerate}[topsep=0pt]
    \item On input $x \in \Z_2^n$, first compute $w = \LMap^{\mat{A}}_2(x) = \mat{A}x$.
    \item Output $y = \LMap^{\mat{B}}_3\left(\Convert_{(2,3)}(w)\right)$. That is, view $w$ as a vector over $\Z_3$ and then output $y = \mat{B}w$.
\end{enumerate}

\begin{center} %!TEX root = ../../main.tex

\scalebox{0.8}{
\begin{tikzpicture}
    \begin{scope}
     \node [draw,trapezium,minimum height=1.48cm,trapezium left angle=75,trapezium right angle=75, rotate=90,twostyle] {\rotatebox{-90}{$\mat{A}$}};
    \node[shift={(-1,0)}] {$x$};
    \end{scope}
    \begin{scope}[shift={(0.95,0)}]
    \tikzmath{\sidelen=0.6;\maxnum=4;}
    \foreach \num in {0,...,\maxnum} {
        \tikzmath{\yshift=(\num-(\maxnum)/2)*(\sidelen + 0.3);}
        \node [draw,regular polygon, regular polygon sides=3,scale=0.5,shift={(0,\yshift)},rotate=90] {};
    }
    \end{scope}
    \begin{scope}[shift={(1.85,0)}]
    \node [draw,trapezium,minimum height=1.48cm,trapezium left angle=75,trapezium right angle=75, rotate=-90,threestyle] {\rotatebox{90}{$\mat{B}$}};
    \draw (0.75,0) node[right] {$y$};
    \end{scope}

    \begin{scope}[shift={(0,-1.5)}]
        \draw (0,0) node[right] {\ttOWF};
    \end{scope}

\end{tikzpicture}
}
 \end{center}
\label{construction:23-owf}
\end{construction}
}
\caption{$(2,3)$-constructions}
\label{fig:primary-constructions}
\end{figure}
~
%!TEX root = ../main.tex

\begin{figure}[p]
\protbox{LPN-style-constructions}{

\textbf{Parameters.} $n, m, t$ are functions of the security parameter $\secparam$.

\textbf{Public values.}
Let $\mat{A} \in \Z_2^{m \times n}$ and $\mat{B} \in \Z_2^{t \times m}$ be fixed public matrices chosen uniformly at random. Alternatively, the matrices can also be chosen to be full-rank circulant matrices.

\begin{construction}[LPN-wPRF Candidate]
The \textnormal{LPN-wPRF} candidate is a family of functions $\PRFfunc : \Z_2^{m \times n} \times \Z_2^n \to \Z_2^t$ with key-space $\keyspace_\secparam = \Z_2^{m \times n}$, input space $\inspace_\secparam = \Z_2^n$ and output space $\outspace_\secparam = \Z_2^t$. For a key $\mat{K} \in \keyspace_\secparam$, we define $\sfF_{\mat{K}}(x) = \PRFfunc(\mat{K}, x)$ as follows:

\begin{enumerate}[topsep=0pt]
    \item On input $x \in \Z_2^n$, first compute $u = \BLMap_2(\mat{K}, x) = \mat{K}x$.
    \item Let $\mat{K}^* = \Convert_{(2,3)}(\mat{K})$ and $x^* = \Convert_{(2,3)}(x)$. Compute $v = \Convert_{(3,2)}(\BLMap_3(\mat{K}^*, x^*)) = \mat{K}^* x^* \bmod 2$. That is, compute $v = (\mat{K}x \bmod 3) \bmod 2$ where both $\mat{K}$ and $x$ are first reinterpreted over $\Z_3$.
    \item Compute $w = u \xor v$ and output $y = \LMap_2^{\mat{B}}(w)$.
\end{enumerate}
\label{construction:23-central-wprf}
\end{construction}
% \begin{center} %!TEX root = ../../main.tex

\scalebox{0.7}{
\begin{tikzpicture}
    \begin{scope}
    %  \node [draw,trapezium,minimum height=1.48cm,trapezium left angle=75,trapezium right angle=75, rotate=90,twostyle] {\rotatebox{-90}{$\mat{A}$}};
    % \node[shift={(-1,0)}] {$x$};
    \draw[twostyle] (-0.74,0.65) -- (0.74,1.05) -- (0.74,-1.05) -- (-0.74,-0.65) -- (-0.74,-0.55) -- (0.5,-0.55) -- (0.5,0.55) -- (-0.74,0.55) -- cycle;
    \draw (-0.8,0.65) node[left] {$\mat{K}$};
    \draw (-0.8,-0.65) node[left] {$x$};
    \end{scope}

    \begin{scope}[shift={(-0.9,-1.9)}]
    \tikzmath{\sidelen=0.6;\maxnum=0;}
    \foreach \num in {0,...,\maxnum} {
        \tikzmath{\yshift=(\num-(\maxnum)/2)*(\sidelen + 0.3);}
        \node [draw,regular polygon, regular polygon sides=3,scale=0.5,shift={(0,\yshift)},rotate=90] {};
    }
    \node[shift={(-0.4,0)}] {$\mat{K}$};
    \end{scope}

    \begin{scope}[shift={(-0.9,-3.1)}]
    \tikzmath{\sidelen=0.6;\maxnum=0;}
    \foreach \num in {0,...,\maxnum} {
        \tikzmath{\yshift=(\num-(\maxnum)/2)*(\sidelen + 0.3);}
        \node [draw,regular polygon, regular polygon sides=3,scale=0.5,shift={(0,\yshift)},rotate=90] {};
    }
    \node[shift={(-0.4,0)}] {$x$};
    \end{scope}


    \begin{scope}[shift={(0,-2.5)}]
        % \node [draw,trapezium,minimum height=1.48cm,trapezium left angle=75,trapezium right angle=75, rotate=90,threestyle] {\rotatebox{-90}{$\mat{A}$}};
        \draw[threestyle] (-0.74,0.65) -- (0.74,1.05) -- (0.74,-1.05) -- (-0.74,-0.65) -- (-0.74,-0.55) -- (0.5,-0.55) -- (0.5,0.55) -- (-0.74,0.55) -- cycle;
    \end{scope}

    \begin{scope}[shift={(0.9,-2.5)}]
    \tikzmath{\sidelen=0.6;\maxnum=4;}
    \foreach \num in {0,...,\maxnum} {
        \tikzmath{\yshift=(\num-(\maxnum)/2)*(\sidelen + 0.3);}
        \node [draw,regular polygon, regular polygon sides=3,scale=0.5,shift={(0,\yshift)},rotate=-90] {};
    }
    \end{scope}

    \begin{scope}[shift={(2.4,-1.25)}]
        \node[draw,circle, minimum size=1cm, twostyle] {};
        \draw (0,0.5) -- (0,-0.5);
        \draw (0.5,0) -- (-0.5,0);
        % \node[shift={(0.55,-0.55)}] {$2$};        

        \draw[->, -latex] (-1.6,0.4) -- (-0.37,0.4);

        \draw[->, -latex] (-1.35,-0.4) -- (-0.37,-0.4);


        \draw[->, -latex] (0.5,0) -- (1.5,0);

    \end{scope}

    \begin{scope}[shift={(1.45,-4)}]
        \draw (0,0) node[right] {\large LPN-wPRF};
    \end{scope}

    % \begin{scope}[shift={(0.95,0)}]
    % \tikzmath{\sidelen=0.6;\maxnum=4;}
    % \foreach \num in {0,...,\maxnum} {
    %     \tikzmath{\yshift=(\num-(\maxnum)/2)*(\sidelen + 0.3);}
    %     \node [draw,regular polygon, regular polygon sides=3,scale=0.5,shift={(0,\yshift)},rotate=90] {};
    % }
    % \end{scope}
    \begin{scope}[shift={(4.65,-1)}]
    \node [draw,trapezium,minimum height=1.48cm,trapezium left angle=75,trapezium right angle=75, rotate=-90,twostyle] {\rotatebox{90}{$\mat{B}$}};
    \draw (0.75,0) node[right] {$y$};
    \end{scope}
\end{tikzpicture}
}





% \scalebox{0.8}{
% \begin{tikzpicture}
%     % \begin{scope}[scale=1.6, shift={(-0.3,0)}]
%     %     % \draw[twostyle] (-0.74,0.65) -- (0.74,1.05) -- (0.74,-1.05) -- (-0.74,-0.65) -- (-0.74,-0.55) -- (0.5,-0.55) -- (0.5,0.55) -- (-0.74,0.55) -- cycle;
%     %     %  \draw (-0.8,0.65) node[left] {$\mat{K}$};
%     %     %  \draw (-0.8,-0.65) node[left] {$x$};
%     %     \draw[twostyle] (-0.74,1.05) -- (0.74,0.65) -- (0.74,-0.65) -- (-0.74,-1.05) -- (-0.74,-0.55) -- (0.5,-0.55) -- (0.5,0.55) -- (-0.74,0.55) -- cycle;
%     %     \draw (-0.8,0.8) node[left] {$\mat{K}$};
%     %     \draw (-0.8,-0.8) node[left] {$x$};

%     % \end{scope}
%      \begin{scope}
%         \draw[twostyle] (-0.74,0.65) -- (0.74,1.05) -- (0.74,-1.05) -- (-0.74,-0.65) -- (-0.74,-0.55) -- (0.5,-0.55) -- (0.5,0.55) -- (-0.74,0.55) -- cycle;
%          \draw (-0.8,0.65) node[left] {$\mat{K}$};
%          \draw (-0.8,-0.65) node[left] {$x$};
%         % \draw[twostyle] (-0.74,1.05) -- (0.74,0.65) -- (0.74,-0.65) -- (-0.74,-1.05) -- (-0.74,-0.55) -- (0.5,-0.55) -- (0.5,0.55) -- (-0.74,0.55) -- cycle;
%         % \draw (-0.8,0.8) node[left] {$\mat{K}$};
%         % \draw (-0.8,-0.8) node[left] {$x$};
%     \end{scope}
%     \begin{scope}[shift={(0.95,0)}]
%     \tikzmath{\sidelen=0.6;\maxnum=4;}
%     \foreach \num in {0,...,\maxnum} {
%         \tikzmath{\yshift=(\num-(\maxnum)/2)*(\sidelen + 0.3);}
%         \node [draw,regular polygon, regular polygon sides=3,scale=0.5,shift={(0,\yshift)},rotate=90] {};
%     }
%     \end{scope}
%     \begin{scope}[shift={(1.85,0)}]
%     \node [draw,trapezium,minimum height=1.48cm,trapezium left angle=75,trapezium right angle=75, rotate=-90,threestyle] {\rotatebox{90}{$\mat{B}$}};
%     \draw (0.75,0) node[right] {$y$};
%     \end{scope}
% \end{tikzpicture}
% }
 \end{center}

\begin{construction}[LPN-PRG Candidate]
The \textnormal{LPN-PRG} is a length-doubling PRG candidate defined as the function $\mathsf{F}_\secparam : \Z_2^n \to \Z_2^{2n}$ with input space $\inspace_\secparam = \Z_2^n$ and output space $\outspace_\secparam = \Z_2^{2n}$. For this construction, we consider the parameters $n,m,t$ with $m \geq n, t$ and $t = 2n$. We define $\mathsf{F}(x) = \mathsf{F}_\secparam$ as follows:
    \begin{enumerate}
       \item On input $x \in \Z_2^n$, first compute $u = \LMap_2(\mat{A}, x) = \mat{A}x$.
    \item Let $x^* = \Convert_{(2,3)}(x)$. Compute $v = \Convert_{(3,2)}(\LMap^{\mat{A}}_3(x^*)) =  (\mat{A} x^*) \bmod 2$. That is, compute $(\mat{A}x \bmod 3) \bmod 2$ where both $\mat{A}$ and $x$ are first reinterpreted over $\Z_3$.
    \item Compute $w = u \xor v$ and output $y = \LMap_2^{\mat{B}}(w)$.
    \end{enumerate}
\end{construction}
% \begin{center} %!TEX root = ../../main.tex

\scalebox{0.7}{
\begin{tikzpicture}
    \begin{scope}
     \node [draw,trapezium,minimum height=1.48cm,trapezium left angle=75,trapezium right angle=75, rotate=90,twostyle] {\rotatebox{-90}{$\mat{A}$}};
    \node[shift={(-1,0)}] {$x$};
    \end{scope}

    \begin{scope}[shift={(-0.95,-2.5)}]
    \tikzmath{\sidelen=0.6;\maxnum=2;}
    \foreach \num in {0,...,\maxnum} {
        \tikzmath{\yshift=(\num-(\maxnum)/2)*(\sidelen + 0.3);}
        \node [draw,regular polygon, regular polygon sides=3,scale=0.5,shift={(0,\yshift)},rotate=90] {};
    }
    \node[shift={(-0.4,0)}] {$x$};
    \end{scope}

    \begin{scope}[shift={(0,-2.5)}]
        \node [draw,trapezium,minimum height=1.48cm,trapezium left angle=75,trapezium right angle=75, rotate=90,threestyle] {\rotatebox{-90}{$\mat{A}$}};
    \end{scope}

    \begin{scope}[shift={(0.9,-2.5)}]
    \tikzmath{\sidelen=0.6;\maxnum=4;}
    \foreach \num in {0,...,\maxnum} {
        \tikzmath{\yshift=(\num-(\maxnum)/2)*(\sidelen + 0.3);}
        \node [draw,regular polygon, regular polygon sides=3,scale=0.5,shift={(0,\yshift)},rotate=-90] {};
    }
    \end{scope}

    \begin{scope}[shift={(2.4,-1.25)}]
        \node[draw,circle, minimum size=1cm, twostyle] {};
        \draw (0,0.5) -- (0,-0.5);
        \draw (0.5,0) -- (-0.5,0);

        \draw[->, -latex] (-1.6,0.4) -- (-0.37,0.4);
        \draw[->, -latex] (-1.35,-0.4) -- (-0.37,-0.4);

        \draw[->, -latex] (0.5,0) -- (1.5,0);

    \end{scope}

    \begin{scope}[shift={(4.65,-1)}]
    \node [draw,trapezium,minimum height=1.48cm,trapezium left angle=75,trapezium right angle=75, rotate=-90,twostyle] {\rotatebox{90}{$\mat{B}$}};
    \draw (0.75,0) node[right] {$y$};
    \end{scope}
    
    \begin{scope}[shift={(1.45,-4)}]
        \draw (0,0) node[right] {\large LPN-PRG};
    \end{scope}


\end{tikzpicture}
}
 \end{center}

\medskip

\begin{minipage}{0.5\textwidth}
\begin{center} %!TEX root = ../../main.tex

\scalebox{0.7}{
\begin{tikzpicture}
    \begin{scope}
    %  \node [draw,trapezium,minimum height=1.48cm,trapezium left angle=75,trapezium right angle=75, rotate=90,twostyle] {\rotatebox{-90}{$\mat{A}$}};
    % \node[shift={(-1,0)}] {$x$};
    \draw[twostyle] (-0.74,0.65) -- (0.74,1.05) -- (0.74,-1.05) -- (-0.74,-0.65) -- (-0.74,-0.55) -- (0.5,-0.55) -- (0.5,0.55) -- (-0.74,0.55) -- cycle;
    \draw (-0.8,0.65) node[left] {$\mat{K}$};
    \draw (-0.8,-0.65) node[left] {$x$};
    \end{scope}

    \begin{scope}[shift={(-0.9,-1.9)}]
    \tikzmath{\sidelen=0.6;\maxnum=0;}
    \foreach \num in {0,...,\maxnum} {
        \tikzmath{\yshift=(\num-(\maxnum)/2)*(\sidelen + 0.3);}
        \node [draw,regular polygon, regular polygon sides=3,scale=0.5,shift={(0,\yshift)},rotate=90] {};
    }
    \node[shift={(-0.4,0)}] {$\mat{K}$};
    \end{scope}

    \begin{scope}[shift={(-0.9,-3.1)}]
    \tikzmath{\sidelen=0.6;\maxnum=0;}
    \foreach \num in {0,...,\maxnum} {
        \tikzmath{\yshift=(\num-(\maxnum)/2)*(\sidelen + 0.3);}
        \node [draw,regular polygon, regular polygon sides=3,scale=0.5,shift={(0,\yshift)},rotate=90] {};
    }
    \node[shift={(-0.4,0)}] {$x$};
    \end{scope}


    \begin{scope}[shift={(0,-2.5)}]
        % \node [draw,trapezium,minimum height=1.48cm,trapezium left angle=75,trapezium right angle=75, rotate=90,threestyle] {\rotatebox{-90}{$\mat{A}$}};
        \draw[threestyle] (-0.74,0.65) -- (0.74,1.05) -- (0.74,-1.05) -- (-0.74,-0.65) -- (-0.74,-0.55) -- (0.5,-0.55) -- (0.5,0.55) -- (-0.74,0.55) -- cycle;
    \end{scope}

    \begin{scope}[shift={(0.9,-2.5)}]
    \tikzmath{\sidelen=0.6;\maxnum=4;}
    \foreach \num in {0,...,\maxnum} {
        \tikzmath{\yshift=(\num-(\maxnum)/2)*(\sidelen + 0.3);}
        \node [draw,regular polygon, regular polygon sides=3,scale=0.5,shift={(0,\yshift)},rotate=-90] {};
    }
    \end{scope}

    \begin{scope}[shift={(2.4,-1.25)}]
        \node[draw,circle, minimum size=1cm, twostyle] {};
        \draw (0,0.5) -- (0,-0.5);
        \draw (0.5,0) -- (-0.5,0);
        % \node[shift={(0.55,-0.55)}] {$2$};        

        \draw[->, -latex] (-1.6,0.4) -- (-0.37,0.4);

        \draw[->, -latex] (-1.35,-0.4) -- (-0.37,-0.4);


        \draw[->, -latex] (0.5,0) -- (1.5,0);

    \end{scope}

    \begin{scope}[shift={(1.45,-4)}]
        \draw (0,0) node[right] {\large LPN-wPRF};
    \end{scope}

    % \begin{scope}[shift={(0.95,0)}]
    % \tikzmath{\sidelen=0.6;\maxnum=4;}
    % \foreach \num in {0,...,\maxnum} {
    %     \tikzmath{\yshift=(\num-(\maxnum)/2)*(\sidelen + 0.3);}
    %     \node [draw,regular polygon, regular polygon sides=3,scale=0.5,shift={(0,\yshift)},rotate=90] {};
    % }
    % \end{scope}
    \begin{scope}[shift={(4.65,-1)}]
    \node [draw,trapezium,minimum height=1.48cm,trapezium left angle=75,trapezium right angle=75, rotate=-90,twostyle] {\rotatebox{90}{$\mat{B}$}};
    \draw (0.75,0) node[right] {$y$};
    \end{scope}
\end{tikzpicture}
}





% \scalebox{0.8}{
% \begin{tikzpicture}
%     % \begin{scope}[scale=1.6, shift={(-0.3,0)}]
%     %     % \draw[twostyle] (-0.74,0.65) -- (0.74,1.05) -- (0.74,-1.05) -- (-0.74,-0.65) -- (-0.74,-0.55) -- (0.5,-0.55) -- (0.5,0.55) -- (-0.74,0.55) -- cycle;
%     %     %  \draw (-0.8,0.65) node[left] {$\mat{K}$};
%     %     %  \draw (-0.8,-0.65) node[left] {$x$};
%     %     \draw[twostyle] (-0.74,1.05) -- (0.74,0.65) -- (0.74,-0.65) -- (-0.74,-1.05) -- (-0.74,-0.55) -- (0.5,-0.55) -- (0.5,0.55) -- (-0.74,0.55) -- cycle;
%     %     \draw (-0.8,0.8) node[left] {$\mat{K}$};
%     %     \draw (-0.8,-0.8) node[left] {$x$};

%     % \end{scope}
%      \begin{scope}
%         \draw[twostyle] (-0.74,0.65) -- (0.74,1.05) -- (0.74,-1.05) -- (-0.74,-0.65) -- (-0.74,-0.55) -- (0.5,-0.55) -- (0.5,0.55) -- (-0.74,0.55) -- cycle;
%          \draw (-0.8,0.65) node[left] {$\mat{K}$};
%          \draw (-0.8,-0.65) node[left] {$x$};
%         % \draw[twostyle] (-0.74,1.05) -- (0.74,0.65) -- (0.74,-0.65) -- (-0.74,-1.05) -- (-0.74,-0.55) -- (0.5,-0.55) -- (0.5,0.55) -- (-0.74,0.55) -- cycle;
%         % \draw (-0.8,0.8) node[left] {$\mat{K}$};
%         % \draw (-0.8,-0.8) node[left] {$x$};
%     \end{scope}
%     \begin{scope}[shift={(0.95,0)}]
%     \tikzmath{\sidelen=0.6;\maxnum=4;}
%     \foreach \num in {0,...,\maxnum} {
%         \tikzmath{\yshift=(\num-(\maxnum)/2)*(\sidelen + 0.3);}
%         \node [draw,regular polygon, regular polygon sides=3,scale=0.5,shift={(0,\yshift)},rotate=90] {};
%     }
%     \end{scope}
%     \begin{scope}[shift={(1.85,0)}]
%     \node [draw,trapezium,minimum height=1.48cm,trapezium left angle=75,trapezium right angle=75, rotate=-90,threestyle] {\rotatebox{90}{$\mat{B}$}};
%     \draw (0.75,0) node[right] {$y$};
%     \end{scope}
% \end{tikzpicture}
% }
 \end{center}
\end{minipage}
\begin{minipage}{0.5\textwidth}
\begin{center} %!TEX root = ../../main.tex

\scalebox{0.7}{
\begin{tikzpicture}
    \begin{scope}
     \node [draw,trapezium,minimum height=1.48cm,trapezium left angle=75,trapezium right angle=75, rotate=90,twostyle] {\rotatebox{-90}{$\mat{A}$}};
    \node[shift={(-1,0)}] {$x$};
    \end{scope}

    \begin{scope}[shift={(-0.95,-2.5)}]
    \tikzmath{\sidelen=0.6;\maxnum=2;}
    \foreach \num in {0,...,\maxnum} {
        \tikzmath{\yshift=(\num-(\maxnum)/2)*(\sidelen + 0.3);}
        \node [draw,regular polygon, regular polygon sides=3,scale=0.5,shift={(0,\yshift)},rotate=90] {};
    }
    \node[shift={(-0.4,0)}] {$x$};
    \end{scope}

    \begin{scope}[shift={(0,-2.5)}]
        \node [draw,trapezium,minimum height=1.48cm,trapezium left angle=75,trapezium right angle=75, rotate=90,threestyle] {\rotatebox{-90}{$\mat{A}$}};
    \end{scope}

    \begin{scope}[shift={(0.9,-2.5)}]
    \tikzmath{\sidelen=0.6;\maxnum=4;}
    \foreach \num in {0,...,\maxnum} {
        \tikzmath{\yshift=(\num-(\maxnum)/2)*(\sidelen + 0.3);}
        \node [draw,regular polygon, regular polygon sides=3,scale=0.5,shift={(0,\yshift)},rotate=-90] {};
    }
    \end{scope}

    \begin{scope}[shift={(2.4,-1.25)}]
        \node[draw,circle, minimum size=1cm, twostyle] {};
        \draw (0,0.5) -- (0,-0.5);
        \draw (0.5,0) -- (-0.5,0);

        \draw[->, -latex] (-1.6,0.4) -- (-0.37,0.4);
        \draw[->, -latex] (-1.35,-0.4) -- (-0.37,-0.4);

        \draw[->, -latex] (0.5,0) -- (1.5,0);

    \end{scope}

    \begin{scope}[shift={(4.65,-1)}]
    \node [draw,trapezium,minimum height=1.48cm,trapezium left angle=75,trapezium right angle=75, rotate=-90,twostyle] {\rotatebox{90}{$\mat{B}$}};
    \draw (0.75,0) node[right] {$y$};
    \end{scope}
    
    \begin{scope}[shift={(1.45,-4)}]
        \draw (0,0) node[right] {\large LPN-PRG};
    \end{scope}


\end{tikzpicture}
}
 \end{center}
\end{minipage}


}
\caption{LPN-style-constructions}
\label{fig:lpn-constructions}
\end{figure}




\paragraph{Circuit description of constructions.}
We can represent our constructions by circuits consisting of the gates described previously. This approach is similar to that of~\cite{boyle2019-fss-preprocess}. We formally define computation circuit representations for our constructions in Definition~\ref{def:computation_circuit}.

\begin{definition}[Computation circuit]
A computation circuit $C$ with input space $\Gin = \prod \Gin_i$ and output space $\Gout = \prod \Gout_i$ is a (labeled) directed acyclic graph $(\mathcal{V},\mathcal{E})$ where $\mathcal{V}$ denotes the set of vertices and $\mathcal{E}$ denotes the set of edges according to the following:
\begin{itemize}

\item Each source vertex corresponds to exactly one $\Gin_i$ and vice versa. The label for the vertex is the identity function on the corresponding $\Gin_i$. Each sink vertex corresponds to exactly one $\Gout_i$ and vice versa. The label for the vertex is the identity function on the corresponding $\Gout_i$. Each non-source $V \in \mathcal{V}$ is labeled with a gate $\mathcal{G}_V \in \gateset$ that computes the function $\mathcal{G}_V: \mathbb{G}^{\textsf{in}}_V \to \mathbb{G}^{\textsf{out}}_V$. The depth of a vertex $V \in \mathcal{V}$, denoted by $\textsf{depth}(V)$ is the length of the largest directed path from a source vertex to $V$.


\item For an edge $(V_a, V_b)$, let $\Gout_{V_a} = \prod \Gout_{V_a,i}$ and $\Gin_{V_b} = \prod \Gin_{V_b, i}$. Then, there exists indices $j$ and $k$ such that $\Gout_{V_a,j} = \Gin_{V_b,k}$. Further, for each input $\Gin_{V_b,i}$ for $V_b$, there is some edge $(V_c, V_b)$ that satisfies the above.

\item The evaluation of the gate for vertex $V$ on input $x \in \Gin_V$ is defined as $y = \mathcal{G}_V(x)$. The evaluation of the circuit $C$, denoted by $\textsf{Eval}_C(x)$, where $x \in \Gin$ is the value $y \in \Gout$, that is obtained by recursively evaluating each gate function in the circuit.
\end{itemize}


Let $\mathsf{F} = \{\mathsf{F}_\secparam\}_{\secparam \in \N}$ denote a family of functions $\mathsf{F}_\secparam:\mathcal{X}_\secparam \to \mathcal{Y}_\secparam$. We say that $\{C_\secparam\}_{\secparam \in \N}$ is a family of computation circuits for $\mathsf{F}$ if all $C_\secparam$ have the same topological structure, and for all $\secparam \in \N$, $\mathsf{F}_\secparam(x) = \textsf{Eval}_C(x)$ for all $x \in \mathcal{X}_\secparam$.
\label{def:computation_circuit}
\end{definition}


\paragraph{Structured keys.}
The constructions we described previously use general matrices in, e.g., $\Z_p^{m \times n}$. For keyed primitives, this results in a key size of $mn$ elements of $\Z_p$ which is expensive to communicate within distributed protocols. Therefore, we will instead take advantage of structured matrices whose representation is only linear in $n$ and $m$. Since both $n$ and $m$ are $O(\secparam)$ in our constructions, this reduces the communication complexity from quadratic to linear in $\secparam$. Furthermore, some structured matrices also benefit from asymptotically faster algorithms (e.g., FFT-based) for matrix multiplications and matrix-vector products. We briefly describe the types of structured matrices we utilize below. For this, consider a matrix $\mat{M} \in \Z_p^{m \times n}$.

\begin{itemize}
    \item (Toeplitz matrices).
    A Toeplitz matrix, or a diagonal-constant matrix, is a matrix where each diagonal from left to right is constant. Specifically, $\mat{M}$ is Toeplitz if for all $i \in [m]; j \in [n]$, it holds that $M_{i,j} = M_{i+1, j+1}$ where $M_{i,j}$ denotes the element in row $i$ and column $j$ of $\mat{M}$. This means that a Toeplitz matrix can be represented by a single column and a single row, i.e., with $n + m - 1$ field elements.

    \item (Generalized circulant matrices). 
    An generalized circulant matrix is a matrix where each row after the first, is a cyclic rotation of the first row. Specifically, if the first row of generalized circulant matrix $\mat{M}$ is the vector $(a_1, \dots, a_n)$, then the $m^\thtext$ row of $\mat{M}$ will be given by the same vector cyclically rotated $m-1$ times. In general, $m \neq n$, but the special case of $m = n$ (i.e., when the matrix is square) is called a (square) circulant matrix. Unless specified, for brevity, we will often use the term \textit{circulant} to denote either generalized circulant matrices or the more specific (square) circulant matrices. This will not matter for our setting, since both can be efficiently represented using just $n$ field elements.
\end{itemize}

We will usually instantiate our constructions using generalized circulant matrices to take advantage of their efficient representations. However, care must be taken while adding structure since this could potentially damage the security of a construction. This is particularly relevant in case of multi-element outputs since elements in the intermediate vectors will be correlated. For example, in the LPN-style construction, the noise for each deterministic LPN instance will be correlated since two columns of a circulant matrix are just cyclic rotations. 

Our cryptanalysis in Section~\ref{sec:cryptanalysis} will therefore consider our constructions with structured matrices. Of course, general $m \times n$ matrices can also be used for all our constructions (with at least equal security) but this could result in substantially more communication costs. In many cases, it is also helpful to choose the matrices to be full-rank (see ~\ref{sec:cryptanalysis} for details).


\greg{This section said earlier we were interested in (3,2) primitives as well, but there is isn't a figure for those, only (2,3).
Did any of the applications end up finding (3,2) useful? }

\greg{We have (2,3) wPRF and OWF, then LPN wPRF and PRG.  Should we comment on
the relations between various primitives (e.g,. the LPN PRG is also an LPN
OWF),  and point to generic constructions for, e.g., a (2,3)-PRG, which could
be constructed from the (2,3)-OWF. Or maybe a forward reference to some of the
applications to justify this particular set of primitives.  }

\greg{Is there a generic wPRF to PRF construction that is efficient here? The
one I know: PRF(x) := wPRF(RO(x)) is probably not helpful because RO is not
MPC-friendly. Or maybe we should say why we chose wPRFs.  }



