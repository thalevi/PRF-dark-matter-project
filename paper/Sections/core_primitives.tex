%!TEX root = ../main.tex

\newpage

\section{Candidate Constructions}
In this section, we introduce our suite of candidate constructions for a number of cryptographic primitives: weak pseudo-random function families (wPRF), one-way functions (OWF), pseudo-random generators (PRG), and cryptographic commitment schemes. Our constructions are based on similar interplays between mod-2 and mod-3 linear mappings. 

\subsection{Basic structure}
Given the wide range of candidates we propose, we find it useful to have a clean way to describe the operations that are performed in our candidate constructions. For this, we take inspiration from the basic formalism of the function secret sharing (FSS) approach to MPC with preprocessing, first introduced by Boyle, Gilboa, and Ishai~\cite{boyle2019-fss-preprocess}. Abstractly, the key technique here is to represent an MPC functionality as a circuit, where each gate represents an operation to be performed in the distributed protocol. Inputs and outputs of each gate are secret shared and the gate operation is ``split'' using a function secret sharing (FSS) scheme. To evaluate the circuit in a distributed fashion, the dealer first shares a random mask for each input wire in the circuit, and possibly some more correlated randomness. Now, to compute a gate, the masked input is first revealed to all parties, who can then locally compute shares of the output wire or shares of the masked output.

While we find it useful to use the formalism from~\cite{boyle2019-fss-preprocess} for representing the circuit to be computed, we do not explicitly require the FSS formalism for splitting the functionality of each gate. The individual operations are quite straightforward, and we instead chose to directly provide the distributed protocols that compute them. Further, by doing so, our protocols can make better use of correlated randomness to reduce the overall protocol cost as compared to the general techniques in~\cite{boyle2019-fss-preprocess}.


\paragraph{Circuit gates.}
We make use of just four basic operations, or ``gates,'' which we detail below. All our constructions can be succinctly represented using just these gates. In Section~\ref{sec:distributed_protocols}, we will provide distributed protocols to compute them. To cleanly describe both our candidates constructions, and their distributed protocols, the gates we describe here depart from the formalism in~\cite{boyle2019-fss-preprocess} in that the input values are not secret shared. The distributed protocols will instead take in secret shared inputs as necessary.

\begin{itemize}
    \item \textbf{Mod-$p$ Public Linear Gate.}
    For a prime $p$, given a public matrix $\mat{A} \in \Z_p^{s \times l}$, the gate $\LMap^\mat{A}_p(\cdot)$ takes as input $x \in \Z_p^l$ and outputs $y = \mat{A}x \in \Z_p^s$.

    \item \textbf{Mod-$p$ Bilinear Gate.}
    For a prime $p$, and positive integers $s$ and $l$, the gate $\BLMap_p^{s,l}(\cdot, \cdot)$ takes as input a matrix $\mat{K} \in \Z_p^{s \times l}$ and a vector $x \in \Z_p^l$ and outputs $y = \mat{K}x \in \Z_p^s$. When clear from context, we will drop the superscript and simply write $\BLMap_p(\mat{K}, x)$.
    
    \item \textbf{$\Z_2 \to \Z_3$ conversion.} For a positive integer $l$, the gate $\Convert^l_{23}(\cdot)$ takes as input a vector $x \in \Z_2^l$ and returns its equivalent representation $x^*$ in $\Z_3^l$. When clear from context, we will drop the superscript and simply write $\Convert_{23}(x)$.

    \item \textbf{$\Z_3 \to \Z_2$ conversion.} For a positive integer $l$, the gate $\Convert^l_{32}(\cdot)$ takes as input a vector $x \in \Z_3^l$ and computes its mapping $x^*$ in $\Z_2^l$. For this, each $\Z_3$ element in $x$ is computed modulo 2 to get the corresponding $\Z_2$ element in the output $x^*$. Specifically, each $0$ and $2$ are mapped to $0$ while each $1$ is mapped to $1$. When clear from context, we will drop the superscript and simply write $\Convert_{32}(x)$.
\end{itemize}

\noindent Note that the difference between the $\LMap$ and the $\BLMap$ gates is seen in the context of their distributed protocols. For $\LMap$, the matrix $\mat{A}$ will be publicly available to all parties, while the input $x$ will be secret shared. On the other hand, for $\BLMap$, the both the key $\mat{K}$ and the input $x$ will be secret shared. Also note that although the $\Convert_{23}$ gate is effectively a no-op in a centralized evaluation, in the distributed setting, the gate will be used to convert an additive sharing over $\Z_2$ to an additive sharing over $\Z_3$.

As described previously, for the linear mappings, we focus only on constructions that use mod-2 and mod-3 mappings. In Figure~\ref{fig:graphical_gates}, we provide a pictorial representation for each circuit gate. We will connect these pieces together to also provide clean visual representations for all our constructions. 

%!TEX root = ../../main.tex

\begin{figure}[t]
\centering
\begin{tikzpicture}[scale=0.4]
    \begin{scope}[shift={(-15,0)}]
        \node [draw,rectangle,minimum width=1.6cm,minimum height=0.55cm,fill=blue!15] {$\mat{A}$};
        \node[shift={(0,-0.6)}] {$\LMap^\mat{A}_2$ gate};
        \node[shift={(-1,0)}] {$x$};
        \node[shift={(1,0)}] {$y$};

        \node [draw,rectangle,minimum width=1.6cm,minimum height=1.065cm,pattern=north west lines, pattern color=orange!70, shift={(0,-2)}] {$\mat{A}$};
        \node[shift={(0,-2.8)}] {$\LMap^\mat{A}_3$ gate};
        \node[shift={(-1,-2)}] {$x$};
        \node[shift={(1,-2)}] {$y$};

    \end{scope}
    \begin{scope}[shift={(0,0)}]
        \draw[fill=blue!15] (-2,0.68) -- (2,0.68) -- (2,-0.68) -- (-2,-0.68) -- (-2,-0.43) -- (1.75,-0.43) -- (1.75,0.43) -- (-2,0.43) -- (-2,0.68);

        \node[shift={(0,-0.6)}] {$\BLMap_2$ gate};
        \node[shift={(-1,0.25)}] {$\mat{A}$};
        \node[shift={(-1,-0.25)}] {$x$};
        \node[shift={(1,0)}] {$y$};


        \draw[pattern=north west lines, pattern color=orange!70, shift={(0,-5)}] (-2,1.3) -- (2,1.3) -- (2,-1.3) -- (-2,-1.3) -- (-2,-0.8) -- (1.5,-0.8) -- (1.5,0.8) -- (-2,0.8) -- (-2,1.3);

        \node[shift={(0,-2.8)}] {$\BLMap_3$ gate};
        \node[shift={(-1,-1.6)}] {$\mat{A}$};
        \node[shift={(-1,-2.4)}] {$x$};
        \node[shift={(1,-2)}] {$y$};

    \end{scope}   
    \begin{scope}[shift={(15,0)}]
        \node [draw,trapezium,trapezium left angle=65,trapezium right angle=65,minimum height=0.55cm,fill=blue!15,rotate=90] {};
        \node[shift={(0,-0.7)}] {$\Convert_{23}$ gate};
        \node[shift={(-0.5,0)}] {$x$};
        \node[shift={(0.6,0.05)}] {$x^*$};

        \node [draw,trapezium,trapezium left angle=65,trapezium right angle=65,minimum height=0.55cm,pattern=north west lines, pattern color=orange!70,rotate=-90, shift={(2,0)}] {};
        \node[shift={(0,-2.8)}] {$\Convert_{32}$ gate};
        \node[shift={(-0.5,-2)}] {$x$};
        \node[shift={(0.6,-1.95)}] {$x^*$};

    \end{scope}          
\end{tikzpicture}
\caption{Pictorial representations of the circuit gates.}
\label{fig:graphical_gates}
\end{figure}


\paragraph{Construction styles.}
The candidate constructions we introduce follow one of two broad styles, which we detail below.

\begin{itemize}
    \item \textbf{$(p,q)$-constructions.}
    For distinct primes $p, q$, the $(p,q)$-constructions have the following structure: On an input $x$ over $\Z_p$, first a linear $\bmod p$ mapping is applied, followed by a linear $\bmod q$ mapping. Note that after the $\bmod p$ mapping, the input is first reinterpreted as a vector over $q$. For unkeyed primitives (e.g., OWF), both mappings are public, while for keyed primitives (e.g., wPRF), the key is used for the first linear mapping. The construction is parameterized by positive integers $n, m, t$ (functions of the security parameter $\secparam$) denoting the length of the input vector (over $\Z_p$), the length of the intermediate vector, and the length of the output vector (over $\Z_q$) respectively. The two linear mappings can be represented by matrices $\mat{A} \in \Z_p^{m \times n}$ and $\mat{B} \in \Z_q^{t \times m}$. For keyed primitives, the key $\mat{K} \in \Z_p^{m \times n}$ will be used instead of $\mat{A}$.


    In this paper, we will analyze this style of constructions for $(p,q) = (2,3)$ and $(3,2)$. As a shorthand, we will denote these by 23-constructions and 32-constructions (e.g., 23-wPRF).


    \item \textbf{LPN-style-constructions.} \mahimna{todo}

    \item \mahimna{Also add any other constructions.}
\end{itemize}


%!TEX root = ../main.tex

\begin{figure}[h]

\protbox{$(2,3)$-constructions}{

\textbf{Parameters.} Let $\secparam$ be the security parameter and define parameters $n, m, t$ as functions of $\secparam$ such that $m \geq n, m \geq t$.

\textbf{Public values.}
Let $\mat{A} \in \Z_2^{m \times n}$ and $\mat{B} \in \Z_3^{t \times m}$ be fixed public matrices chosen uniformly at random.

\begin{construction}[Mod-2/Mod-3 wPRF Candidate~\cite{boneh2018-darkmatter}]
The wPRF candidate is a family of functions $\PRFfunc : \Z_2^{m \times n} \times \Z_2^n \to \Z_3^t$ with key-space $\keyspace_\secparam = \Z_2^{m \times n}$, input space $\inspace_\secparam = \Z_2^n$ and output space $\outspace_\secparam = \Z_3^t$. For a key $\mat{K} \in \keyspace_\secparam$, we define $\sfF_{\mat{K}}(x) = \PRFfunc(\mat{K}, x)$ as follows:

\begin{enumerate}[topsep=0pt]
    \item On input $x \in \Z_2^n$, first compute $w = \BLMap_2(\mat{K}, x) = \mat{K}x$.
    \item Output $y = \LMap^{\mat{B}}_3\left(\Convert_{(2,3)}(w)\right)$. That is, view $w$ as a vector over $\Z_3$ and then output $y = \mat{B}w$. 
\end{enumerate}
\label{construction:23-central-wprf}
\end{construction}

\begin{center} %!TEX root = ../../main.tex

\scalebox{0.8}{
\begin{tikzpicture}
    \begin{scope}
        \draw[twostyle] (-0.74,0.65) -- (0.74,1.05) -- (0.74,-1.05) -- (-0.74,-0.65) -- (-0.74,-0.55) -- (0.5,-0.55) -- (0.5,0.55) -- (-0.74,0.55) -- cycle;
         \draw (-0.8,0.65) node[left] {$\mat{K}$};
         \draw (-0.8,-0.65) node[left] {$x$};
    \end{scope}
    \begin{scope}[shift={(0.95,0)}]
    \tikzmath{\sidelen=0.6;\maxnum=4;}
    \foreach \num in {0,...,\maxnum} {
        \tikzmath{\yshift=(\num-(\maxnum)/2)*(\sidelen + 0.3);}
        \node [draw,regular polygon, regular polygon sides=3,scale=0.5,shift={(0,\yshift)},rotate=90] {};
    }
    \end{scope}
    \begin{scope}[shift={(1.85,0)}]
    \node [draw,trapezium,minimum height=1.48cm,trapezium left angle=75,trapezium right angle=75, rotate=-90,threestyle] {\rotatebox{90}{$\mat{B}$}};
    \draw (0.75,0) node[right] {$y$};
    \end{scope}
\end{tikzpicture}
}
 \end{center}


\begin{construction}[Mod-2/Mod-3 OWF Candidate]
The OWF candidate is a function $\OWFfunc : \Z_2^n \to \Z_3^t$ with input space $\inspace_\secparam = \Z_2^n$ and output space $\outspace_\secparam = \Z_3^t$. We define $\sfF(x) = \OWFfunc(x)$ as follows:

\begin{enumerate}[topsep=0pt]
    \item On input $x \in \Z_2^n$, first compute $w = \LMap^{\mat{A}}_2(x) = \mat{A}x$.
    \item Output $y = \LMap^{\mat{B}}_3\left(\Convert_{(2,3)}(w)\right)$. That is, view $w$ as a vector over $\Z_3$ and then output $y = \mat{B}w$.
\end{enumerate}

\begin{center} %!TEX root = ../../main.tex

\scalebox{0.8}{
\begin{tikzpicture}
    \begin{scope}
     \node [draw,trapezium,minimum height=1.48cm,trapezium left angle=75,trapezium right angle=75, rotate=90,twostyle] {\rotatebox{-90}{$\mat{A}$}};
    \node[shift={(-1,0)}] {$x$};
    \end{scope}
    \begin{scope}[shift={(0.95,0)}]
    \tikzmath{\sidelen=0.6;\maxnum=4;}
    \foreach \num in {0,...,\maxnum} {
        \tikzmath{\yshift=(\num-(\maxnum)/2)*(\sidelen + 0.3);}
        \node [draw,regular polygon, regular polygon sides=3,scale=0.5,shift={(0,\yshift)},rotate=90] {};
    }
    \end{scope}
    \begin{scope}[shift={(1.85,0)}]
    \node [draw,trapezium,minimum height=1.48cm,trapezium left angle=75,trapezium right angle=75, rotate=-90,threestyle] {\rotatebox{90}{$\mat{B}$}};
    \draw (0.75,0) node[right] {$y$};
    \end{scope}

    \begin{scope}[shift={(0,-1.5)}]
        \draw (0,0) node[right] {\ttOWF};
    \end{scope}

\end{tikzpicture}
}
 \end{center}
\label{construction:23-owf}
\end{construction}
}
\caption{$(2,3)$-constructions}
\label{fig:primary-constructions}
\end{figure}

\newpage




