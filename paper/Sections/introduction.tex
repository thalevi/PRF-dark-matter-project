%!TEX root = ../main.tex

\section{Introduction}
\label{sec:introduction}

\greg{Does anyone know why the PDF doesn't have the index on the left hand side with links to all of the sections?}

\greg{To me, one of the nice things about these primitives is how elegant and
simple they are.  (e.g., I showed the (2,3)-OWF to a colleague and his reaction
was, wow this is so simple, just two matrix multiplies, can it really be a
OWF?) Esp.  when we compare the (2,3)-OWF to a (MPC-friendly) block cipher. But
the simplicity doesn't really come across in the description of Section 3,
maybe the introduction is a good place to make the point (and give a
description of the (2,3)-OWF as an example).  }

\medskip

%-------------------------------------------%
Symmetric-key primitives like pseudo-random functions (PRFs)~\cite{goldreich1984-prf}, pseudo-random generators (PRGs)~\cite{?}, and one-way-functions (OWFs)~\cite{?} are deployed in innumerable settings, and are arguably some of the most fundamental building blocks of modern cryptography. While traditional usecases primarily considered settings where the function evaluation was done by a single party, recent applications (e.g., cryptocurrencies \mahimna{cite more here}) have required evaluation in a distributed fashion to avoid single points of failure. This necessitates the study of distributed evaluation (e.g., in multi-party-computation or MPC) of symmetric-key constructions. 

Towards this, a long line of work~\cite{?} has made substantial progress on distributing the computation of constructions like \mahimna{AES, SHA-256 etc}~\cite{?} which are widely used in practice. Unfortunately, the constructions themselves were not designed with distributed evaluation in mind, and are often clunky to optimize for. More recent work~\cite{grassi2016-mpcfriendly, boneh2018-darkmatter} therefore has proposed to start from scratch by designing \textit{MPC-friendly} constructions from the ground up. So far, the primary focus has been on MPC-friendly PRFs. In this work, we continue this line of research by proposing a new suite of \textit{simple} candidate constructions for a number of symmetric primitives: (weak)PRFs, OWFs, and PRGs. We focus mainly on the semi-honest setting in the preprocessing model, but also show are protocols can be adapted for malicious security or evaluation without preprocessing. Our constructions are designed with the following goals in mind:

\begin{itemize}
    \item \textit{High non-linearity but low nonlinear depth}.
    MPC protocols commonly follow one of two broad styles; the first is based on garbled circuits~\cite{} while the second is based on secret-sharing~\cite{}. Garbled circuit based protocols usually have better round complexity (independent of circuit depth) while secret-sharing based protocols have much smaller communication complexity (especially in the preprocessing model). The round complexity of the latter scales with \textit{non-linear depth}. Unfortunately, constructions like AES have a high non-linear depth, which bottlenecks their distributed evaluation.

    \hspace*{1em} In this paper, we strive to find the best of both worlds. One of the design goals for our constructions therefore, is to achieve high non-linearity for the overall computation while maintaining a low non-linear \textit{depth}. Intuitively, this will enable us to achieve low round complexity (using a secret-sharing based approach) without the communication overhead of garbled circuits.

    \item \textit{Small nonlinear size}.
    
    \item \textit{\textnormal{OT} and \textnormal{VOLE} friendliness}.


\end{itemize}
Our main technique to achieve this is to consider alternating linear functions over different (prime) moduli. The only non-linear part of the constructions will now be to switch between the different moduli, which leads to a very small round complexity. Furthermore, Boneh et al.~\cite{boneh2018-darkmatter} showed that linear functions over one moduli cannot be well approximated by low degree polynomials over a different moduli. Consequently, even a simple design that alternates between different moduli can result in a highly non-linear overall computation. In other words, this approach provides high algebraic degree with low nonlinear depth. 

As the simplest possible instance of this approach, we consider alternating linear functions over $\Z_2$ and $\Z_3$. Boneh et al. found that linear functions over $\Z_2$ are highly non-linear over $\Z_3$ while linear functions over $\Z_3$ are highly non-linear over $\Z_2$. Furthermore, using small moduli allows for efficient designs for ``switching'' between the moduli. In fact, conversions between the moduli are “small” nonlinear gates that can be implemented efficiently from OT.



\subsection{Our contributions}


\paragraph{New $(2,3)$ candidate constructions.}

\paragraph{Cryptanalysis and implications on parameter choices.}


\paragraph{Distributed protocols and optimized implementations.}


\paragraph{Applications.}


\newpage
