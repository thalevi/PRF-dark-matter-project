%!TEX root = ../main.tex
\section{Cryptanalysis}
\label{sec:cryptanalysis}

We give a summary of cryptanalysis of our constructions,
focusing on the main attacks that influence our parameters.
The details are found in Appendix~\ref{appendix:cryptanalysis}.

%In general, for obtaining $s$-bit security given that the adversary performs $2^{\tau}$ work,
%we require that the success probability of key recover attacks is bounded by $2^{\tau - s}$,
%while the advantage of distinguishing attacks (when applicable)
%is bounded by $2^{(\tau - s)/2}$.
%This is consistent with the work of Micciancio and Walter~\cite{Micciancio018}.

\subsection{Summary of Security Evaluation of the \ttOWF}

The attacker is given $\hat{y} \in \Z_3^t$ and tries to invert it.
Our most interesting attack on the \ttOWF is based on a reduction to subset-sum.
We give the full details below.

\paragraph{Reduction to subset-sum.}
For a vector $w \in \Z_2^m$, there is an $(m -n) \times m$ (parity check) matrix $\mat{P}$ such that there exists $x \in \Z_2^n$ for which $\mat{A}x = w$ if and only if $\mat{P} w=0$.
Assume that $\hat{y}$ is the output of the \ttOWF on $x \in \Z_2^n$, and let $w = \mat{A}x$. Then,
$w$ satisfies the conditions $\mat{P} w = 0$ (over $\Z_2$) and $\mat{B}w = \hat{y}$ (over $\Z_3$).
We attempt to find such $w$ by a reduction to subset-sum, as detailed below.

Suppose we find a set $J \subseteq [m]$ such that
$$\left( \sum_{j \in J} \mat{P}e_j  \bmod 2, \sum_{j \in J} \mat{B}e_j  \bmod 3 \right) = (\vec{0},\hat{y}),$$
where $e_i \in \{0,1\}^m$ is the $i$'th unit vector.
Then, the preimage $x$ can be computed by solving the linear equation system
$\mat{A}x = \sum_{j \in J} e_j  \bmod 2$.

Thus, we have reduced the problem to subset-sum with $m$ binary variables
$(\epsilon_1, \ldots, \epsilon_m) \in \{0,1\}^m$, where we associate $\epsilon_i = 1$
with $$(\mat{P}e_i, \mat{B}e_i) \in \Z_2^{m-n} \times \Z_3^t,$$
and define the target
as $(\vec{0},\hat{y}) \in \Z_2^{m-n} \times \Z_3^t$.

\paragraph{Solving the subset-sum problem.}
We can now apply the advanced subset-sum algorithm by
Howgrave{-}Graham and Joux~\cite{Howgrave-GrahamJ10} and the more recent ones
~\cite{BeckerCJ11,BonnetainBSS20},
which are based on the \emph{representation technique}.
These algorithms were mostly designed to solve subset-sum problems over the integers.
Below, we describe the main ideas of these algorithms
and explain how to apply them to the special subset-sum problem
we consider.

In the subset-sum problem over the integers,
we are given a list of $m$ positive  integers $(a_1,a_2,\ldots,a_m)$
and another positive integer $S$ such that $S = \sum_{i=1}^{m} \epsilon_i a_i$
for $\epsilon_i \in \{0,1\}$. The goal is to recover the unknown coefficients $\epsilon_i$.

A standard meet-in-the-middle approach for solving the problem has time complexity of about $2^{m/2}$.
The representation technique gives an improved algorithm as briefly summarized below.

Assume that a solution to the subset-sum problem is chosen uniformly from $\{0,1\}^m$
and the parameters are set such that the instance has about one solution on average. Effectively, this means that the density of the problem $d = \tfrac{n}{\log \max(\{a_i\}_{i=1}^{m})}$ is set to 1.

The main idea of the basic algorithm of Howgrave{-}Graham and Joux~\cite{Howgrave-GrahamJ10}
is to split the problem into two parts by writing
$S = \sigma_1 + \sigma_2$,
where $\sigma_1 = \sum_{i=1}^{m} \alpha_i a_i$, $\sigma_2 = \sum_{i=1}^{m} \beta_i a_i$
and $(\alpha_i,\beta_i) \in \{(0,0),(0,1),(1,0)\}$.
Thus, $\epsilon_i = \alpha_i + \beta_i$ for each $i$ is a solution to the problem.

Note that each coefficient $\epsilon_i$ with value 1 can be split into $(0,1)$, or $(1,0)$.
Thus, assuming that the solution has Hamming weight\footnote{In general, one
may guess the Hamming weight of the solution and
repeat the algorithm accordingly a polynomial number of times.} of $m/2$
(which occurs with probability $\Omega(1/\sqrt{m})$),
it has $2^{m/2}$ different \emph{representations}.
Consequently, we may focus on finding only one of these representations
by solving two subset-sum problems of Hamming weight $m/4$.
Focusing on a single representation of the solution
allows to beat the standard meet-in-the-middle approach which requires time $2^{m/2}$.

\paragraph{Adaptation of previous subset-sum algorithms.}
The algorithm of~\cite{Howgrave-GrahamJ10} can be easily adapted
to our specialized subset-sum problem (although it is not defined over the integers).
Moreover the improved algorithm of~\cite{BeckerCJ11} considers additional representations
of the solution by allowing $\alpha_i$ and $\beta_i$ to also take the value -1
(implying that $\epsilon_i = 0$ can be decomposed into
$(\alpha_i,\beta_i) \in \{(0,0),(-1,1),(1,-1)\}$).
In our case, we associate $\alpha_i = -1$
with $(\mat{P}(-e_i), \mat{B}(-e_i)) = (\mat{P}e_i, 2 \cdot \mat{B}e_i)  \in \Z_2^{m-n} \times \Z_3^t$.
Finally, the recent improved algorithm of~\cite{BonnetainBSS20} considers representations over
$\{-1,0,1,2\}$ and we can adapt this algorithm to our setting in a similar way.

In terms of complexity, ignoring polynomial factors in $m$,
the attack of~\cite{Howgrave-GrahamJ10} runs in time $2^{0.337m}$ and uses $2^{0.256m}$ memory,
while the complexity of attack of~\cite{BonnetainBSS20} requires $2^{0.283m}$ time and memory.

Thus, conservatively ignoring polynomial factors, for $s$-bit security we require $0.283m \geq s$, or $m \geq 3.53 s$.

\subsection{Summary of Security Evaluation of the \ttwPRF}

For the \ttwPRF, the attacker obtains several samples 
$(x^{(1)},\mat{B},y^{(1)}) ,\ldots, (x^{(2^r)},
\mat{B},y^{(2^r)})$
and tries to
mount a key recovery and/or a distinguishing attack.
We restrict the number of samples produced with a single secret to $2^{40}$.

We will set the parameters such that $n - \log 3 \cdot t \geq s$,
and thus there are $2^s$ keys on average that are consistent with a single sample.
Therefore, any key recovery attack faster than $2^s$ has to make use of at least two samples.
Particularly, the subset-sum attack can also be applied to the \ttwPRF,
but it is not clear how to use it efficiently on more than one sample
(without strong relations between them).

The most important distinguishing attack looks for a bias in a linear combination
of the output over $\Z_3$.
Given a single sample $(x,\mat{B},y)$, assume there exist
$v \in \Z_3^m$ and $u \in \Z_3^t$ such that $u \mat{B} = v$ and the Hamming weight of $v$ is $\ell$.
As $y = \mat{B}w \bmod 3$, the attacker computes $uy \bmod 3 = vw \bmod 3$
and thus obtains the value of a linear combination $\bmod$ 3 of $\ell$ entries of $w \in \{0,1\}^m$.
Since $w \in \Z_2^m$, this linear combination is biased, 
and the strength of the bias depends on how small $\ell$ is.
The bias can be amplified using several samples.
Consequently, we require that the rows of $\mat{B}$ do not span a vector of low Hamming weight.
This analysis is probabilistic and leads to a lower bound on $m$.

Another important attack we consider exploits the fact that $\mat{K}$
is circulant and preserves symmetric properties of the input $x$ (e.g., the two halves of $x$ are equal). 
This attack imposes a lower bound on $n$ so that such a symmetric vector in not found
in the data, except with negligible probability. 
We leave it as an open problem to extend this basic attack.


Overall, we set $n=m=2s$ and $t = s/\log 3$.
These are somewhat aggressive parameters as the security margin against the above attacks
in rather narrow. A choice of $n=m=2.5s$ is more conservative.

\subsection{Summary of Security Evaluation of the LPN-PRG}

The attacker is given a single sample $\mat{A},\mat{B},y$
and tries to mount a key recovery and or a distinguishing attack.

The construction differs from 
the alternative weak PRF construction proposed
in~\cite{boneh2018-darkmatter} in two ways.
The first transformation generates $t = 2n$ samples using a public matrix.
Similarly to~\cite{boneh2018-darkmatter},
each such sample can be viewed as an LPN sample, namely,
a noisy linear equation over $\Z_2$ in the bits of the seed
(although the noise is generated deterministically).
However, in~\cite{boneh2018-darkmatter} $\mat{A}$ is a random matrix,
whereas we use a (structured) Toeplitz matrix.
This may be considered as weakening of the construction. 
On the other hand, the second transformation $\mat{B}$ ``compresses''
the samples and generally strengthens the construction.

A significant consideration in selecting the parameters is that
the rows of $\mat{B}$ do not span a low Hamming weight vector,
imposing a lower bound on $m$.
Thus, only dense linear combinations of samples are available at the output,
accumulating the noise. This should defeat standard attacks against LPN. 

Overall, setting 
$n=s, m = 3s, t = 2s$ seems to provide sufficient 
resistance against the considered attacks.



\subsection{Summary of Security Evaluation of the LPN-PRF}

The attacks we consider against this primitive include a union of some
of the attacks considered for the LPN-PRG and for the \ttwPRF constructions
with some adjustments. Overall, we propose to set $n=m= 2s$ and $t =s$.
