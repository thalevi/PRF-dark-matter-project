%!TEX root = ../main.tex
\section{Implementation and Evaluation}
\label{sec:implementation_and_eval}
We implemented our 2-party protocols to compute the \ttwPRF candidate (Construction~\ref{construction:23-central-wprf}) both in the distributed setting (Section~\ref{subsec:2pc-wprf}), and the oblivious evaluation setting (Sections~\ref{subsec:oprf1} and~\ref{subsec:oprf2}). Our implementations are in C++. For the \ttwPRF construction, we used the parameters $n = m = 256$ and $t = 81$. The implemented 23-constructions uses a Toeplitz matrix in $\Z_2^{256 \times 256}$ as the key, take as input a vector in $\Z_2^{256}$ and output a vector in $\Z_3^{81}$. The correlated randomness was implemented as if provided by a trusted third party.  See Section~\ref{subsec:preprocessing} for concretely efficient protocols for securely generating the correlated randomness, which we did not implement but give efficiency estimates based on prior works. 

\iffull
\subsection{Optimizations}
\label{subsec:implementation_opt}
\else
\paragraph{Optimizations.}
We start with a centralized implementation of the 23-wPRF. We find optimizations that provide 25x better performance over a na\"ive implementation.
\iffull\else 
We use three major optimizations in our implementation. First, we use \textit{bit packing} for $\Z_2$ vectors through which we can pack several elements in a machine word and operate on them together in an SIMD manner. Second, we use \textit{bit slicing} for $\Z_3$ vectors by representing them as a pair of $\Z_2$ vectors. All operations on the $\Z_3$ vectors can now be translated to operations on the $\Z_2$ vectors. Finally, we use a lookup table optimization for the final $\Z_3$ linear mapping (i.e., multiplication by $\mat{B}$). For this, we split the 256-column matrix $\mat{B}$ into 16 pieces with 16 columns each and store multiplications with all $\Z_3^{16}$ vectors for each piece. The size for each piece was decided as a tradeoff between the lookup table size and the computational efficiency. We defer the details for our optimization techniques to Appendix~\ref{appendix:implementation}. \fi
% In Table~\ref{table:optimization_benchmarks}, we compare the performance gain.

\iffull
\paragraph{Bit packing for $\Z_2$ vectors.} 
Instead of representing each element in a $\Z_2$ vector separately, we pack several elements into a machine word and operate on them together in an SIMD manner. For our architecture with 64-bit machine words, we can pack a vector in $\Z_2^{256}$ (e.g., the input $x$) into 4 words. Since the key $\mat{K}$ is circulant and can be represented with $n=256$ bits, it can also be represented by 4 words. This results in a theoretical $\times64$ maximum speedup in run-time for operations involving $\mat{K}$ and $x$. 

\paragraph{Bit slicing for $\Z_3$ vectors.}
We represent each element in $\Z_3$ using the two bits from its binary representation. For $z \in \Z_3$, the two bits are the least significant bit (LSB) $l_z = z \bmod 2$, and the most significant bit (MSB) $h_z$ which is $1$ if $z=2$ and $0$ otherwise. $\Z_3$ vectors are now also represented by two binary vectors, one containing the MSBs, and one containing the LSBs. Operations involving a $\Z_3$ vector are translated to operations on these binary vectors instead. We also take advantage of the bit packing optimization when operating on the binary vectors.
In Table~\ref{table:z3_operations}, we specify how we perform common operations on $\Z_3$ elements using our bit slicing approach. 

\begin{table}[ht]
\begin{center}
    \begin{tabular}{|c|c|c|}
        \hline
        Operation & Result MSB & Result LSB\\
        \hline\hline
        $z_1 + z_2 \bmod 3$ & $(l_1 \vee l_2) \oplus (l_1 \vee h_2) \oplus (l_2 \vee h_1)$  & $(h_1 \vee h_2) \oplus (l_1 \vee h_2) \oplus (l_2 \vee h_1)$\\
        $-z_1 \bmod 3$ & $l_1$ & $h_1$ \\
        $z_1z_2 \bmod 3$ & $(l_1 \wedge l_2) \oplus (h_1 \wedge h_2)$ & $(l_1 \wedge h_2) \oplus (h_1 \wedge l_2)$\\
        $\textrm{MUX}(s;z_1,z_2)$ & $(h_2 \wedge s) \vee (h_1 \wedge \neg s)$ & $(l_2 \wedge s) \vee (l_1 \wedge \neg s)$\\
        \hline
    \end{tabular}
\end{center}
\caption{Operations in $\Z_3$. $z_1$ and $z_2$ are elements in $\Z_3$ with (MSB, LSB) = $(h_1,l_1)$ and $(h_2, l_2)$ respectively. For a bit $s$, the operation $\textrm{MUX}(s;z_1, z_2)$ outputs $z_1$ when $s=0$ and $z_2$ when $s=1$.}
\label{table:z3_operations}
\mahimna{I'm debating whether we need to have this table at all. Once we specify that we split $\Z_3$ elements into 2 bits, for any $\Z_3$ operation, the corresponding operation on the bits can easily be computed using a truth table.}
\end{table}



\paragraph{Lookup table for matrix multiplication.} Recall that the \ttwPRF evaluation contains a mod-3 linear map using a public matrix $\mat{B} \in \Z_3^{81\times256}$. Specifically, it computes the matrix-vector product $\mat{B}w$ where $w \in \Z_3^{256}$. Since $\mat{B}$ is known prior to evaluation, we can use a lookup table to speedup the multiplication by $\mat{B}$\cite{arlazarov1970economical}. The same preprocessing can also by reused for multiple evaluations of the wPRF.

For this, we partition $\mat{B}$, which has $m=256$ columns, into $16$ slices of $16$ columns each. These matrices, denoted by $\mat{B}_1,\dots, \mat{B}_{16}$, are all in $\Z_3^{81 \times 16}$. Now, for each $\mat{B}_i$, we will effectively build a lookup table for its multiplication with any $\Z_3$ vector of length $16$. A point to note here is that since we  represent $\Z_3$ vectors by two binary vectors (from the bit slicing optimization), it is sufficient to preprocess multiplications (modulo 3) for binary vectors of length $16$. To multiply $\mat{B}_{i}$ by a vector in $\Z_3^{16}$, we can first multiply it separately by the corresponding MSB and LSB vectors, and then subtract the former from the latter modulo 3. This works since for $z_1, z_2 \in \Z_3$ the multiplication $z_1z_2 \bmod 3$ can be given by $z_1(2\cdot h_2 + l_2) \bmod 3 = z_1l_2 - z_1h_2 \bmod 3$ where $h_2, l_2$ are the MSB and LSB of $z_2$ respectively. Now, to multiply $\mat{B}$ by $v \in \Z_3^{256}$, we first evaluate all multiplications of the form $\mat{B}_iv_i$ where $v_i$ is the $\Z_3^{16}$ vector denoting the $i^\thtext$ slice of $v$ if it was split into 16-element chunks. Then, multiplication by $\mat{B}$ is given by $\mat{B}v = \sum_{i \in [16]} \mat{B}_i v_i \mod 3$.
\greg{Is this the ``method of four Russians''?\url{https://github.com/malb/m4ri}, \url{https://bitbucket.org/malb/m4ri/wiki/Further\%20Reading} }
\greg{$w$ is secret, so the access pattern of the $\mat{B}_i$ should be independent of $w$. Is this the case? If not there is a 
cache timing side-channel. The timing variation may be observable over the network (e.g., to the other party in the OPRF evaluation).  }

In general, a $\mat{B} \in \Z_3^{t \times m}$ can be partitioned into $m/c$ partitions with $c$ columns each (assume $c$ divides $m$ for simplicity), and would require a total multiplication lookup table size of $(m/c) \cdot 2^c \cdot t$ $\Z_3$ elements. Multiplying $\mat{B}$ by $v \in \Z_3^{m}$ requires $2(m/c)$ lookup table accesses (one each for MSB and LSB of $v$ per partition of $\mat{B}$) and an addition of $(m/c - 1)$ $\Z_3^{t}$ vectors.


For our parameters, this results in a table size of roughly $135$MB with proper $\Z_3$ packing. We chose $c = 16$ as a compromise between the size of the lookup table and increased computational efficiency.


\fi









% \subsection{Representing $Z_2$ vector - bit packing}

% The dark matter function manipulates elements of $Z_2$ and $Z_3$. However, machine words can hold up to 64 bits and we use this to represent many elements in just a few machine words, using bit slicing. Namely, we used each machine word as a vector of 64 bits and applied operations to this bit-vector in a SIMD manner.
% Since in our implementation each key $\textbf{K}$  is of size $m \times n = 256 \times 256$ and each input is a vector of size $n = 256$, this results in packing each key into $\dfrac {256}{64} = 4$ words and each input vector into 4 words. This may result in time saving of up to $\times 64$ of the run-time.

% \subsection{Representing $Z_3$ vector - bit slicing}

% The last part of the dark matter function takes an intermediate vector viewed as a vector of $Z_3$ elements, which is then multiplied 
% by a matrix $R$ in $Z_3$ to produce the output vector in $Z_3$. We tried different methods for implementing this matrix-vector multiplication over $Z_3$.:	

% Bit slicing is implemented  by representing a vector over $Z_3$  as two binary vectors - one LSB's and one MSB's. The operations over this $Z_3$ vector - addition, subtraction, multiplication and MUX  - were implemented in a bitwise manner as shown in table \ref{tab:multicol}. We took advantage of the fact that when representing each vector as MSB and LSB, negation (which is the same as multiplying by two) is the same as swapping the most and least significant bits. 

% \begin{table}[ht]
% 	\caption{Operations in $Z_3$. input:   ${l_1}, {m_1}$ - LSB and MSB  of first trinary number, 
% 		${l_2}, {m_2}$ - LSB and MSB  of second trinary number,  ${s}$ - selection bit for MUX}
% 	\begin{center}
% 		\begin{tabular}{|c|l|}
% 			\hline
% 			\textbf{Operations} & \textbf{Methods}\\
% 			\hline
% 			\multirow{3}{*}{Addition} & ${t} := ({l_1 \wedge m_2}) \oplus ({l_2 \wedge m_1})$\\
% 			& $m_{\mathrm{out}} := ( l_1 \wedge  l_2 ) \oplus  t $ \\
% 			& $l_{\mathrm{out}} :=m_1 \wedge m_2 ) \oplus t $ \\
% 			\hline
% 			\multirow{3}{*}{Subtraction} & ${t} := ({l_1} \wedge {l_2}) \oplus ({m_2} \wedge {m_1})$;\\
% 			& $m_{\mathrm{out}} := (l_1 \wedge m_2 ) \oplus t$;\\
% 			& $l_{\mathrm{out}} := (m_1 \wedge l_2 ) \oplus t$; \\
% 			\hline
% 			\multirow{3}{*}{Multiplication} & $m_{\mathrm{out}}:= (l_1 \vee m_2) \wedge   (m_1 \vee l_2)$; \\
% 			& $l_{\mathrm{out}} := (l_1 \vee l_2) \wedge     (m_1 \vee m_2)$;\\
% 			\hline
% 			\multirow{3}{*}{MUX} & $m_{\mathrm{out}} :=( m_2 \vee s) \wedge (m_1 \vee (\bar{s}) )$; \\
% 			& $l_{\mathrm{out}}[i] :=( l_2 \vee s) \wedge (l_1 \vee (\bar{s}) )$; \\
% 			\hline
			
% 		\end{tabular}
% 	\end{center}
% 	\label{tab:multicol}
% \end{table}

% \begin{table}[ht]
%     \caption{Operations in $Z_3$. input:   ${l_1}, {m_1}$ - LSB and MSB  of first trinary number, ${l_2}, {m_2}$ - LSB and MSB  of second trinary number,  ${s}$ - selection bit for MUX}
%     \begin{center}
%         \begin{tabular}{|c|l|}
%             \hline
%             \textbf{Operations} & \textbf{Methods}\\
%             \hline
%             \multirow{3}{*}{Addition} & ${t} := ({l_1 \BitOr m_2}) \oplus ({l_2 \BitOr m_1})$\\
%             & $m_{\mathrm{out}} := ( l_1 \BitOr  l_2 ) \oplus  t $ \\
%             & $l_{\mathrm{out}} :=(m_1 \BitOr m_2 ) \oplus t $ \\
%             \hline
%             \multirow{3}{*}{Subtraction} & ${t} := ({l_1} \BitOr {l_2}) \oplus ({m_2} \BitOr {m_1})$;\\
%             & $m_{\mathrm{out}} := (l_1 \BitOr m_2 ) \oplus t$;\\
%             & $l_{\mathrm{out}} := (m_1 \BitOr l_2 ) \oplus t$; \\
%             \hline
%             \multirow{3}{*}{Multiplication} & $m_{\mathrm{out}}:= (l_1 \BitAnd m_2) \oplus   (m_1 \BitAnd l_2)$; \\
%             & $l_{\mathrm{out}} := (l_1 \BitAnd l_2) \oplus   (m_1 \BitAnd m_2)$;\\
%             \hline
%             \multirow{3}{*}{MUX} & $m_{\mathrm{out}} :=( m_2 \BitAnd s) \BitOr (m_1 \BitAnd (\bar{s}) )$; \\
%             & $l_{\mathrm{out}} :=( l_2 \BitAnd s) \BitOr (l_1 \BitAnd (\bar{s}) )$; \\
%             \hline              
%         \end{tabular}
%     \end{center}
%     \label{tab:multicol}
% \end{table}


%Lookup table Implemetation
% \paragraph{Lookup table for matrix multiplication.}
% \mahimna{I added sufficient details from this into the earlier paragraph. If that looks good, then we can remove this}
% Even with bitslicing, implementing the matrix-vector multiplication column by column via the opertions from Table~1 is still rather slow.
% To get better performence, we capitalized on the fact that the random matrix $R$ is fixed, and we can therefore set up $R$-dependent tables and use table lookups in the implementation of the multiply-by-$R$ operation.

% Specifically, we partition the matrix $R$ (which has $m=256$ columns) into sixteen slices of 16 columns each, denoted $R_1,\ldots,R_{16}$.
% For each of these small matrices $R_i$, we then build a table with $2^{16}$ entries, holding the result of multiplying $R_i$ by every vector from $\{0,1\}^{16}$.
% Namely, let $R_{i,0},R_{i,1},\ldots,R_{i,15}$ be the sixteen columns of the matrix $R_i$.
% The table for $R_i$ is then defined as follows: For each index $0\le j < 2^{16}$, let $\vec{J}=(j_0,j_1,\ldots,j_{15})$ be the 0-1 vector holding the bits in the binary representation of~$j$, then we have
% \[
% T_i[j] = R_i \times \vec{J} = \sum_{k=0}^{15} R_{i,k} \cdot j_k ~~\bmod 3.
% \]
% Recalling that $R$ is $Z_3$ matrix of dimension $81\times 256$, every entry in each table~$T_i$ therefore holds an 81-vector over $Z_3$.
% Specifically, it holds a packed-$Z_3$ element with four words (two for the MSBs and two for the LSBs of this 81-element vector).

% Note, however, that $T_i$ can only be used directly to multiply $R_i$ by \emph{0-1 vectors}.
% To use $T_i$ when multiplying $R_i$ by a $\{0,1,2\}$ vector, multiply $R_i$ separately by the MBSs and the LSBs of that vector, and then subtract one from the other using the operations from Table~1.
% To multiply a dimension-256 $\{0,1,2\}$ vector by the matrix $R$, we partition it into sixteen vectors of dimension 16, use the approach above to multiply each one of them by the corresponding $R_i$, then use the operations from Table~1 to add them all together.
% Hence we have

% \begin{tabbing}
% 	\underline{Multiply-by-$R$(input: $\vec{V}\in\{0,1,2\}^{256}$:}\\
% 	~~1. $Acc := 0$\\
% 	~~2. For \=$i=0,\ldots,15$\\
% 	~~3. \> Let $\vec{M}_i,\vec{L}_i\in\{0,1\}^{16}$ be the MSB, LSB vectors (resp.) of $\vec{V}_i=\vec{V}[16i,\ldots,16i+15]$\\
% 	~~4. \> Set the indexes $m:=\sum_{k=0}^{15}2^i\cdot\vec{M}[i]$
% 	and $\ell:=\sum_{k=0}^{15}2^i\cdot\vec{L}[i]$\\
% 	~~5. \> $Acc := Acc + T_i[\ell] - T_i[\ell] \bmod 3$\\
% 	~~6. Output $Acc$
% \end{tabbing}

% We note that a slightly faster implementation could be obtained by braking the matrix into (say) 26 slices of upto 10 columns each, and directly identify each 10-vector over $\{0,1,2\}$ with an index $i<3^{10}=59049$.
% This implementation would have only 26 lookup operations instead of 32 in the algorithm above, so we expect it to be about 20\% faster. On the other hand it would have almost twice the table size of the implementation from above.



%--------------------------------------%

\subsection{Performance Benchmarks}
\label{subsec:performance}
\paragraph{Experimental setup.}
We ran all our experiments on a t2.medium AWS EC2 instance with 4GiB RAM (architecture: x86-64 Intel(R) Xeon(R) CPU E5-2686 v4 @ 2.30GHz) running on Ubuntu 18.04. The performance benchmarks and timing results we provide are averaged over 1000 runs. For the distributed construction benchmarks, both parties were run on the same instance. We separately report the computational runtime for the parties, and analytically compute the communication costs. 


\paragraph{Optimization results.}
\iffull
We start with benchmarks for the different optimization techniques detailed in Section~\ref{subsec:implementation_opt}. Table~\ref{table:optimization_benchmarks} contains timing results for our centralized implementation of the \ttwPRF construction with parameters $n=m=256$ and $t=81$ using these optimizations. The table also reports results for a baseline implementation which uses the naive $\Theta(nm)$ algorithm for multiplication between a $m \times n$ matrix and a $n$ length vector. We find that our best implementation is 25x faster than the baseline implementation.
\else
Table~\ref{table:optimization_benchmarks} shows the performance gain of our optimization techniques compared to a baseline implementation.
\fi

\begin{table}[t]
	{
		\centering
        \iffull\else\resizebox{0.7\textwidth}{!}{\fi
		\begin{tabular}{|c|c|c|c|c|}
			
			\hline
			\multicolumn{3}{|c|}{Optimization} &  \multirow{2}{*}{Runtime ($\mu$s)} & \multirow{2}{*}{Evaluations / sec} \\
			Packing & Bit Slicing & Lookup Table &  & \\
			\hline\hline
			\multicolumn{3}{|c|}{Baseline implementation} & 156.41 & 6K\\
			\checkmark & & & 26.84 & 37K\\
			\checkmark & \checkmark & & 18.5 & 65K \\
			\checkmark & \checkmark & \checkmark & 6.08 & 165K \\
			\hline
		\end{tabular}
        \iffull\else}\fi
		\caption{Centralized 23-wPRF benchmarks for a baseline implementation and for different optimization techniques. Packing was done into 64-bit sized words (for both $\Z_2$ and $\Z_3$ vectors). For the lookup table optimization, a table with $81 \times 2^{20}$ $\Z_3$ elements, or roughly of size $135$MB, was preprocessed. Runtimes are all given in microseconds ($\mu$s).
		% \yuval{Specify units in Runtime column. Memory Usage column: not sure what this means, and in any case the first two rows seem too small to be meaningful (e.g., the amount of memory used by the program for maintaining variables may be bigger). I suggest removing this column, and instead specifying the lookup table size in the caption. Alternatively, we can keep this column and change the title to ``lookup table size'', keeping the first two rows empty. Regarding this size, what does $2^{20}$ count? Bits? Machine words? $\Z_3$ elements?}
		}
		\label{table:optimization_benchmarks}
	}
\end{table}


\paragraph{Distributed wPRF evaluation.}
\iffull
We also implement the 2-party semi-honest distributed protocol detailed in Section~\ref{subsec:2pc-wprf} for evaluating the \ttwPRF construction and report timings for our implementation. Further, since this candidate was first proposed in~\cite{boneh2018-darkmatter}, we also implement their protocol as a comparison point. For both protocols, we use the parameters $n=m=256$, $t=81$ for the PRF and use the same optimizations \iffull(from Section~\ref{subsec:implementation_opt})\fi. We use a trusted party to generate the preprocessed randomness. We provide a comparison in Table~\ref{table:distributed_wprf}.


% We consider two 2-party weak PRF protocols. The first protocol is described in \cite{boneh2018-darkmatter} and the second one is our construction as described in \ref{newProtocol}. Since the goal of the implementation was timing the local {\em online}run-time, the implementation used a trusted party 

\begin{table}[t]
{
\centering
\begin{tabular}{|c|c|c|c|c|c|}
	\hline
	Protocol & Runtime ($\mu$s) & Evaluations / sec & \makecell{Preprocessing \\ (bits)} & \makecell{Comm. \\ (bits)} \\
	\hline
	\hline
	\makecell{\ttwPRF Protocol \\~\cite{boneh2018-darkmatter}} & 28.02 & 36K & 3533 & 2612 \\
	\hline
	\makecell{\ttwPRF Protocol \\ (Section~\ref{subsec:2pc-wprf})} & 12.12 & 82K & 662 & 1536 \\
	\hline	
\end{tabular}
\caption{Comparison of 2-party semi-honest protocols for distributed evaluation of the \ttwPRF candidate (Construction~\ref{construction:23-central-wprf}) in the preprocessing model. The parameters $n=m=256$ and $t=81$ were used for the wPRF. The runtime reported is the maximum between the two parties.}
% \mahimna{The preprocessing costs do not take into account optimizations (e.g., using a PRG seed). Should we report the na\"{i}ve cost here or the optimized cost? Using a PRG, the total preprocessing (with $\Z_3$ compression) is 918 bits.}
% \mahimna{If we are comparing only the TCC18 protocol with ours does it even make sense to have a table? Should we just provide the comparison in text?}
\label{table:distributed_wprf}
}
\end{table}
\else
We implement our 2-party semi-honest distributed protocol for evaluating the \ttwPRF construction and report timings for our implementation. Since this candidate was first proposed in~\cite{boneh2018-darkmatter}, we also implement their protocol as a comparison point. For both protocols, we use the parameters $n=m=256$, $t=81$ for the PRF and use the same optimizations for an accurate comparison. We found that our protocol is better in all metrics. For a single evaluation, our protocol requires $12.12~\mu$s, 662 bits of preprocessing, and 1536 bits of online communication. On the other hand, the protocol from~\cite{boneh2018-darkmatter} requires $28.02\mu$s, 3533 bits of preprocessing, and 2612 bits of online communication for one evaluation.

\fi

\paragraph{OPRF evaluation.}
In Table~\ref{table:oprf_comparison}, we provide performance benchmarks for both our oblivious protocols (see Section~\ref{subsec:oprf_protocol}) for the \ttwPRF construction. We also compare our results to the standard DDH-based OPRF (see \iffull Section~\ref{subsec:oprf-comparison}\else Appendix~\ref{appendix:oprf-comparison}\fi). For our timing results, we report both the server and client runtimes (averages over 1000 runs). For each construction, we also include the size of the preprocessed correlated randomness, and the online communication cost. All constructions are parameterized appropriately to provide 128-bit security.

For our constructions, we report separately, the timings for refreshing the key and evaluating the input. For the comparison with the DDH-based OPRF construction, we use the libsodium library~\cite{LibSodium} for the elliptic curve scalar multiplication operation. We use the Curve25519 elliptic curve, which has a 256-bit key size, and provides 128 bits of security. 
% \yuval{I actually think that the Naor-Pinkas-Reingold99 paper cited in the TCC '18 paper refers to a different notion of distributing a PRF, namely where the PRF key is distributed between two or more servers and the input is public. Let's try to figure out the source of the simple DDH-based OPRF protocol. Can ask Stas/Hugo if needed.}

\iffull
We point out that both of our OPRF protocols are much faster than the DDH-based OPRF although they require higher communication as well as preprocessing. To better represent the tradeoffs, we compute the minimum network speed for which the total time for one evaluation of our protocol is smaller than that for the DDH-based OPRF. We compare only the online costs for the evaluation. Analytically, we can compute that our two protocols are faster than the DDH-based OPRF protocol when the network speed is greater than $\approx 5.7$Mbps and $\approx 1.85$Mbps respectively. On a 50Mbps network, The communication of OPRF Protocol 1 outpaces its respective computation. The same happens with OPRF protocol 2 on an 40 Mbps network. While these numbers only present a simplified picture, we note that they should depict the overall tradeoffs for our OPRF protocols.
\fi

% \yuval{Communication costs in bits can be computed analytically. Maybe talk about estimated breakeven points: (1) minimal network speed in which computation dominates communication, and (2) minimal network speed where we estimate our OPRF to be faster than the DDH-based alternative.  }

\begin{table}[t!]
{
\centering
\resizebox{0.9\textwidth}{!}{%
	\begin{tabular}{|ccc|c|c|c|c|c|c|}%8 columns	
		\hline
		\multicolumn{3}{|c|}{\multirow{2}{*}{Protocol}} & \multicolumn{2}{c|}{Runtime ($\mu$s)} & \multirow{2}{*}{Preprocessing (bits)} & \multicolumn{2}{c|}{Communication (bits)}\\
		& && Client & Server & & Client & Server\\
		\hline\hline
		\multirow{2}{*}{\makecell{$\prot_1^\textsf{oprf}$ \\ (Section~\ref{subsec:oprf1})}} &\quad\quad& Key Update & - & 0.65 & 256 & - & 256 \\
		&& Evaluation & 8.54 & 9.45 & 2092 & 512 & 385 \\
		\hline
		\multirow{2}{*}{\makecell{$\prot_2^\textsf{oprf}$ \\ (Section~\ref{subsec:oprf2})}} && Key Update & - & 3.16 & 256 & - & 256\\
		&& Evaluation & 7.91 & 8.21 & 1836 & 256 & 385 \\
		\hline
		\multicolumn{3}{|c|}{DDH-based OPRF} & 57.38 & 28.69 & - & 256 & 256\\
		\hline
	\end{tabular}}
	\caption{Comparison of protocols for (semi-honest) OPRF evaluation in the preprocessing model. Runtimes in microseconds ($\mu$s) are provided separately for refreshing the key (Key Update) and for evaluating an input (Evaluation). Communication and preprocessing are also provided separately for the two stages.}
	\label{table:oprf_comparison}
}
\end{table}














%-----------------------------------------------------------------%
% Ignore below this
%-----------------------------------------------------------------%

%==================== Start comment=========
\iffalse
\section{Implementation}
\label{sec:technical_overview}

The pre-shared randomness was generated centrally (as if its generated by a third trusted party). Only the online part was implemented in a distributed manner. 

%%%%%%%%%%%%%%%%%%%%%%%%%%%%%%%%%%%%%%%%%%%%%%%%%

\iffalse
\subsection{Representing $Z_2$ vector}

The dark matter function manipulates elements of $Z_2$ and $Z_3$. However, machine words can hold up to 64 bits and we use this to represent many elements in just a few machine words, using bit slicing. Namely, we used each machine word as a vector of 64 bits and applied operations to this bit-vector in a SIMD manner.
Since in our implementation each key $\textbf{K}$  is of size $m \times n = 256 \times 256$ and each input is a vector of size $n = 256$, this results in packing each key o $256 \dfrac 64 = 4$ words and each input vector into 4 words. This may result in time saving of up to $\times 64$ of the run-time.

\subsection{Representing $Z_3$ vector}

The last part of the dark matter function takes an intermediate vector viewed as a vector of $Z_3$ elements, which is then multiplied 
by a matrix $R$ in $Z_3$ to produce the output vector in $Z_3$. We tried different methods for implementing this matrix-vector multiplication over $Z_3$.:	

\subsubsection{Bit slicing} Bit slicing is implemented  by representing a vector over $Z_3$  as two binary vectors - one LSB's and one MSB's. The operations over this $Z_3$ vector - addition, subtraction, multiplication and MUX  - were implemented in a bitwise manner as shown in table \ref{tab:multicol}. We took advantage of the fact that when representing each vector as MSB and LSB, negation (which is the same as multiplying by two) is the same as swapping the most and least significant bits. 
\fi

% TODO: replace with a table - COMPLETED



\begin{table}[ht]
	\caption{Operations in $Z_3$. input:   ${l_1}, {m_1}$ - LSB and MSB  of first trinary number, 
		${l_2}, {m_2}$ - LSB and MSB  of second trinary number,  ${s}$ - selection bit for MUX}
	\begin{center}
		\begin{tabular}{|c|l|}
			\hline
			\textbf{Operations} & \textbf{Methods}\\
			\hline
			\multirow{3}{*}{Addition} & ${t} := ({l_1 \wedge m_2}) \oplus ({l_2 \wedge m_1})$\\
			& $m_{\mathrm{out}} := ( l_1 \wedge  l_2 ) \oplus  t $ \\
			& $l_{\mathrm{out}} :=m_1 \wedge m_2 ) \oplus t $ \\
			\hline
			\multirow{3}{*}{Subtraction} & ${t} := ({l_1} \wedge {l_2}) \oplus ({m_2} \wedge {m_1})$;\\
			& $m_{\mathrm{out}} := (l_1 \wedge m_2 ) \oplus t$;\\
			& $l_{\mathrm{out}} := (m_1 \wedge l_2 ) \oplus t$; \\
			\hline
			\multirow{3}{*}{Multiplication} & $m_{\mathrm{out}}:= (l_1 \vee m_2) \wedge   (m_1 \vee l_2)$; \\
			& $l_{\mathrm{out}} := (l_1 \vee l_2) \wedge     (m_1 \vee m_2)$;\\
			\hline
			\multirow{3}{*}{MUX} & $m_{\mathrm{out}} :=( m_2 \vee s) \wedge (m_1 \vee (\bar{s}) )$; \\
			& $l_{\mathrm{out}}[i] :=( l_2 \vee s) \wedge (l_1 \vee (\bar{s}) )$; \\
			\hline
			
		\end{tabular}
	\end{center}
	\label{tab:multicol}
\end{table}


%TODO: IS THIS CORRECT? CHECK AND CORRECT

\subsubsection{Matrix multiplication using a lookup table}

Even with bitslicing, implementing the matrix-vector multiplication column by column via the opertions from Table~1 is still rather slow.
To get better performence, we capitalized on the fact that the random matrix $R$ is fixed, and we can therefore set up $R$-dependent tables and use table lookups in the implementation of the multiply-by-$R$ operation.

Specifically, we partition the matrix $R$ (which has $m=256$ columns) into sixteen slices of 16 columns each, denoted $R_1,\ldots,R_{16}$.
For each of these small matrices $R_i$, we then build a table with $2^{16}$ entries, holding the result of multiplying $R_i$ by every vector from $\{0,1\}^{16}$.
Namely, let $R_{i,0},R_{i,1},\ldots,R_{i,15}$ be the sixteen columns of the matrix $R_i$.
The table for $R_i$ is then defined as follows: For each index $0\le j < 2^{16}$, let $\vec{J}=(j_0,j_1,\ldots,j_{15})$ be the 0-1 vector holding the bits in the binary representation of~$j$, then we have
\[
T_i[j] = R_i \times \vec{J} = \sum_{k=0}^{15} R_{i,k} \cdot j_k ~~\bmod 3.
\]
Recalling that $R$ is $Z_3$ matrix of dimension $81\times 256$, every entry in each table~$T_i$ therefore holds an 81-vector over $Z_3$.
Specifically, it holds a packed-$Z_3$ element with four words (two for the MSBs and two for the LSBs of this 81-element vector).

Note, however, that $T_i$ can only be used directly to multiply $R_i$ by \emph{0-1 vectors}.
To use $T_i$ when multiplying $R_i$ by a $\{0,1,2\}$ vector, multiply $R_i$ separately by the MBSs and the LSBs of that vector, and then subtract one from the other using the operations from Table~1.
To multiply a dimension-256 $\{0,1,2\}$ vector by the matrix $R$, we partition it into sixteen vectors of dimension 16, use the approach above to multiply each one of them by the corresponding $R_i$, then use the operations from Table~1 to add them all together.
Hence we have

\begin{tabbing}
	\underline{Multiply-by-$R$(input: $\vec{V}\in\{0,1,2\}^{256}$:}\\
	~~1. $Acc := 0$\\
	~~2. For \=$i=0,\ldots,15$\\
	~~3. \> Let $\vec{M}_i,\vec{L}_i\in\{0,1\}^{16}$ be the MSB, LSB vectors (resp.) of $\vec{V}_i=\vec{V}[16i,\ldots,16i+15]$\\
	~~4. \> Set the indexes $m:=\sum_{k=0}^{15}2^i\cdot\vec{M}[i]$
	and $\ell:=\sum_{k=0}^{15}2^i\cdot\vec{L}[i]$\\
	~~5. \> $Acc := Acc + T_i[\ell] - T_i[\ell] \bmod 3$\\
	~~6. Output $Acc$
\end{tabbing}

We note that a slightly faster implementation could be obtained by braking the matrix into (say) 26 slices of upto 10 columns each, and directly identify each 10-vector over $\{0,1,2\}$ with an index $i<3^{10}=59049$.
This implementation would have only 26 lookup operations instead of 32 in the implementation above, so we expect it to be about 20\% faster. On the other hand it would have almost twice the table size of the implementation from above.



\section{Analysis}

Table \ref{CommunicationCosts} includes the computation and communication results for the different protocol.

% Table: Communication Cost
\begin{table}[htbp]
	\label{CommunicationCosts}
	%[h]
	\begin{center}
		%\begin{minipage}{10cm}
		\begin{tabular}{|c|c|c|c|c|}
			\hline
			\multicolumn{1}{|p{5cm}|}{\centering Protocol}
			& \multicolumn{1}{|p{2cm}|}{\centering Round  \\Complexity}
			& \multicolumn{1}{|p{3cm}|}{\centering Online \\ Communication}
			& \multicolumn{1}{|p{3cm}|}{\centering Preprocessing \\ Size}
			& \multicolumn{1}{|p{3cm}|}{\centering Computation \\ Operation}\\
			\hline
			\hline
			\textbf{Centralized WPRF} & - & - & - & 9K\\
			\hline
			\textbf{Distributed WPRF \cite{boneh2018-darkmatter} } & 4 & 5 & 2K &  45K\\
			\hline
			\textbf{Our WPRF  } & 3 & 5 & 1M &  13K\\
			\hline

			\textbf{Our OPRF } & 3 & 5 & 1M & 11K\\
			%=======
			

			\hline
			
		\end{tabular}
		
		\vspace{-1mm}
		\caption{Communication Analysis of different protocols. The protocols were optimized using both the bit packing and the lookup table}
		\label{CommunicationCosts}
		%\end{minipage}
	\end{center}
	\vspace{-5mm}
\end{table}


\section{Benchmarking}

We compare our run-time to discrete-log based PRFs. To this end, we use the lib sodium library \cite{LibSodium}. The library uses elliptic curve 252 bits, and includes a function that performs scalar multiplication ('crypto\_scalarmult\_ed25519').


\section{Experimental Results}

The system was tested using Ubuntu Server 18.04 on a t2.medium AWS environment. To record the timings, the code was run in a loop 1000 times. Below are run-time results for running a single instance of PRF in microsecond. The results include both centralized and distributed versions of the PRF. In order to increase efficiency, packing and lookup tables were used. The packing indicates both the $Z_2$ and $Z_3$ packing.

%add table
\begin{table}[htbp]
	%[h]
	\begin{center}
		%\begin{minipage}{10cm}
		\begin{tabular}{|c|c|c|c|c|c|}
			\hline
			\textbf{Protocol} & \textbf{Packed }  &  \textbf{lookup} & \textbf{Runs/sec} & \textbf{Runtime($\mu$ sec)}\\
			\hline
			\hline
			\textbf{Centralized weak PRF}  & N  & N  &  50K&20.2 \\
			\hline
			\textbf{Centralized weak PRF} & Y  &  N & 65.4K &18.5  \\
			\hline
			\textbf{Centralized weak PRF} & Y  &  Y & 165K &6.08 \\
			\hline
		\end{tabular}
		
		\vspace{-1mm}
		\caption{Comparison of the run-time and computation of the different Optimized Centralized wPRF.}
		\label{CentralRuntimeTable}
		%\end{minipage}
	\end{center}
	\vspace{-5mm}
\end{table}

\begin{table}[htbp]
	%[h]
	\begin{center}
		%\begin{minipage}{10cm}
		\begin{tabular}{|c|c|c|c|}
			\hline
			\textbf{Protocol} & \textbf{Runs/sec} & \textbf{Runtime($\mu$ sec)} \\
			\hline
			\hline
			\textbf{Fully Distributed WPRF\cite{boneh2018-darkmatter}} & 24K & 40.56  \\
			\hline
			\textbf{Fully Distributed WPRF} &  82K &  12.12\\
			\hline
			\textbf{OPRF with lookup} &  104K &  9.52 \\
			\hline
			\textbf{Discrete log-based PRF } & 12K &\textbf{86.07}\\
			\hline
			
		\end{tabular}
		
		\vspace{-1mm}
		\caption{Optimized protocols runtime. These protocols utilize both the bit-packing and lookup table optimization}
		\label{RuntimeTable}
		%\end{minipage}
	\end{center}
	\vspace{-5mm}
\end{table}

\fi
%==================End Comment==============
