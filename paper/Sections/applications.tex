%!TEX root = ../main.tex

\newcommand{\secpar}{\kappa}
\newcommand{\aux}{\textsf{aux}}
\newcommand{\com}{\textsf{com}}
\newcommand{\CCC}{\mathcal{C}}
\newcommand{\PPP}{\mathcal{P}}
\newcommand{\ch}{\textsf{ch}}
\newcommand{\hash}{\textsf{H}}
\newcommand{\state}{\textsf{state}}
\newcommand{\seed}{\textsf{seed}}
\newcommand{\msgs}{\textsf{msgs}}
\newcommand{\Msg}{\textsf{Msg}}
\newcommand{\Expand}{\textsf{Expand}}
\newcommand{\algcomment}[1]{{\color{Gray}{\qquad//#1}}}
\newcommand{\bmat}[1]{\ensuremath{\mathbf{#1}}}
\newcommand{\matA}{\bmat{A}}
\newcommand{\matB}{\bmat{B}}

 \newtcolorbox{titlebox}[5]{
    enhanced,
    colframe=black,
    colback=white,
    boxrule={#3},
    arc={#2},
    auto outer arc,%
%   breakable,
    pad at break*=0pt,
    vfill before first,
    before={\par\medskip\noindent},
    after={\par\medskip},
    top=12pt, left=4pt, enlarge top by=7pt,%enlarge bottom by=7pt,%
    title={\rule[-.3\baselineskip]{0pt}{\baselineskip}\normalsize\sffamily\bfseries #1}, 
    varwidth boxed title*=-30pt, 
    attach boxed title to top left={yshift=-10pt,xshift=10pt}, 
    coltitle=black,
    boxed title style={colback=white,boxrule={#5},arc={#4},auto outer arc}
 }

 \newenvironment{systembox}[1]
 {\vspace{\baselineskip}\begin{titlebox}{Functionality \normalfont #1}{2.5pt}{1pt}{3.5pt}{1pt}}
 {\end{titlebox}}

 \newenvironment{protocolbox}[1]
 {\begin{titlebox}{ #1}{0.5pt}{0.5pt}{1pt}{0.75pt}}
 {\end{titlebox}}

\section{Applications}
\label{sec:applications}
\greg{Should the OPRF material be in the applications section? Seems like it fits here, and without it there is only one application.} 
\subsection{Signatures with the \ttOWF}
Here we describe a signature scheme using the \ttOWF.  
Our presentation is tailored to the \ttOWF, but we note that this approach is
general.  All of the candidate primitives in this paper would be a suitable
choice of $\sfF$ (note that they are all OWFs when the input is chosen at
random) and we evaluated them all before settling on \ttOWF, which gives the
shortest signatures.

Abstractly, a signature scheme can be built from any one-way function $\sfF$
that has an MPC protocol to evaluate it, by setting the public key to $y =
\sfF(x)$ for a random secret $x$, and then proving knowledge of $x$, using a
proof system based on the MPC-in-the-head paradigm~\cite{STOC:IKOS07}.  In
addition to assuming the OWF is secure, the only other assumption required is a
secure hash function. As no additional number-theoretic assumptions are
required, these type of signatures are often proposed as secure post-quantum
schemes. 

Concretely, our design follows the Picnic signature scheme~\cite{CCS:CDGORR17},
specifically the variant instantiated with the KKW proof
system~\cite{CCS:KatKolWan18} (named Picnic2 and Picnic3).  We chose to use the
KKW, rather than ZKB++ proof system since our MPC protocol to evaluate the
\ttOWF is most efficient with a pre-processing phase, and KKW generally
produces shorter signatures.  We replace the LowMC block cipher~\cite{albrecht2015-lowmc} in Picnic 
with the \ttOWF, and make the corresponding changes to the MPC protocol. 

This is the first signature scheme based on the hardness of inverting the
\ttOWF (or similar function), a function with a simple mathematical
description, making it an accesible target for cryptanalysis, especially when
compared to block ciphers.  Arguably, the simplicity of the OWF can lead to
simpler implementations: the MPC protocol is simpler, and no large precomputed
constants are required. 

Our presentation is somewhat brief here, as many details are identical to Picnic, 
and the \ttOWF MPC has been described in \cref{sec:distributed_protocols}. \cref{appendix:picnic}
includes additional details. 

\paragraph{Parameters.} Let $\secpar$ be a security parameter.  The
\ttOWF parameters are denoted $(n, m, t)$.  The KKW parameters $(N, M,
\tau)$ denote the number of parties $N$, the total number of MPC instances $M$,
and the number $\tau$ of MPC instances where the verifier checks the online
phase of simulation.  The scheme also requires a cryptographic hash function. 

\paragraph{Key generation.}
The signer chooses a random $x \in \Z_2^n$ as 
secret key, and a random seed $s \in \{0,1\}^{\secpar}$ such that $s$
expands to matrices $\matA \in \Z_2^{m\times n}$ and $\matB\in \Z_3^{m\times t}$ that is full rank (using a suitable cryptographic
function, such as the SHAKE extendable output function~\cite{sp800-185}).
Compute $y = \sfF(x)$ and set $(y, s)$ as the public key.
Recall that the \ttOWF is defined as $y = \sfF(x)$ where $x \in \Z_2^{n}$ and $y\in \Z_3^{t}$, and is computed as follows: 
\begin{enumerate}
\item Compute $w = \matA x \in \Z_2^{m}$.
\item Let $z\in\Z_3^{m}$ be $w$, where entries are intepreted as values mod 3.
\item Compute and output $y = \matB z$.
\end{enumerate}
We use unique $\matA$ and $\matB$ per signer in order to avoid multi-target attacks against $\sfF$.
We specify the scheme using random matrices, however, it could also be instantiated
with circulant matrices, which may improve the performance of sign and verify operations.

\paragraph{MPC protocol.}
By combining the protocols for the gates 
$\prot_{\Add}^{3}$,
$\prot_{\LMap}^{\mat{A},2}$,
$\prot_{\LMap}^{\mat{B},3}$, and
$\prot_{\Convert}^{(2,3)}$
described in \cref{sec:distributed_protocols} we have an $N$-party protocol for
the \ttOWF. The most challenging and costly step (in terms of communication) is
the conversion gate, all other operations are done locally by the parties.  In
\cref{appendix:picnic} we describe this protocol in full detail.  Given the MPC
protocol, we neatly drop it into the KKW proof system used in Picnic.  

\paragraph{Sign and verify.} 
The signature generation and verification algorithms for the
\ttOWF  signature scheme are given in \cref{fig:23-picnic} (of \cref{appendix:picnic}).
Here we give an overview. The structure of
the signature scheme follows a three-move $\Sigma$-protocol. In the commit phase, the
prover simulates the preprocessing phase for all $M$ MPC instances, and commits
to the output (the seeds and auxiliary values).  She then simulates the online
phase for all $M$ instances and commits to the inputs, and broadcast messages.
Then a challenge is computed by hashing all commitments, together with the
message to be signed.  The challenge selects $\tau$ of the $M$ MPC instances.
The verifier will check the simulation of the online phase for these instances,
by re-computing all values as the prover did for $N-1$ of the parties, and for
remaining unopened party, the prover will provide the missing broadcast messages
and commitments so that the verifier may complete the simulation and recompute all
commitments. 
For the $M-\tau$ instances not chosen by the challege, the verifier will check
the preprocessing phase only -- here the prover provides the random seeds of
all $N$ parties and the verifier can recompute the prover's commitment to the
preprocessing, in particular the verifier checks that the auxiliary values are
correct.
\greg{paragraph above could be shorter if we're low on space}

\newcommand{\textunderbrace}[2]{{%
  \underbrace{#1}_{\text{#2}}
}}

\paragraph{Parameter selection and signature size.}
In \cref{appendix:picnic} we give the formula for estimating signatures sizes.
The impact of OWF choice is limited to one term, which is the sum of
the sizes of the MPC inputs, broadcast messages, and auxiliary values produced
by preprocessing.  Selecting the KKW parameters $(M, N, \tau)$ once the MPC costs
are known follows the approach in Picnic: a range of options are possible, and
we try to select parameters that balance speed (number of MPC exections and number of 
parties) and size.  Since the MPC costs of the \ttOWF are very close to those
of LowMC, the options follow a similar curve. 

\cref{table:sig-sizes} gives some options with $N=16, 64$ parties,
providing 128 and 256 bits of security. For each category, we highlight the row
of \ttOWF parameters that are a direct comparison to Picnic.  Signatures using
the \ttOWF are slightly shorter (five to fifteen percent) than Picnic using
LowMC. 


\begin{table}
\begin{centering}
\begin{tabular}{l|l|l}
OWF Params      & KKW params        & Sig. size (KB)\\
$(n,m,t)$       & $(N, M, \tau)$    &   \\\hline
$(128,453,81)$  & $(16, 150, 51)$   & 13.30 \\
                & $(16, 168, 45)$   & 12.48 \\ 
                & $(16, 250, 36)$   & \textbf{11.54} \\
                & $(16, 303, 34)$   & 11.49 \\  
Picnic3-L1      & $(16, 250, 36)$   & 12.60 \\ \hline
$(128,453,81)$  & $(64, 151, 45)$   & 13.59 \\
                & $(64, 209, 34)$   & 11.70 \\ 
                & $(64, 343, 27)$   & \textbf{10.66} \\
                & $(64, 515, 24)$   & 10.35 \\ 
Picnic2-L1      & $(64, 343, 27)$   & 12.36 \\ \hline \hline
$(256,906,162)$ & $(16, 324, 92)$   & 50.19 \\ 
                & $(16, 400, 79)$   & 47.08 \\ 
                & $(16, 604, 68)$   & \textbf{45.82} \\ 
Picnic3-L5      & $(16, 604, 68)$   & 48.72 \\ \hline
$(256,906,162)$ & $(64, 322, 82)$   & 51.23 \\
                & $(64, 518, 60)$   & 44.04 \\ 
                & $(64, 604, 57)$   & \textbf{43.45} \\ 
Picnic2-L5      & $(64, 604, 58)$   & 46.18 \\ \hline
\end{tabular}
\caption{ \label{table:sig-sizes}Signature size estimates for Picnic using the
\ttOWF, compared to Picnic using LowMC.  The first half of the table shows
security level L1 (128 bits) with $N=16$ and $N=64$ parties, and the second
half shows level L5 (256 bits).}
\end{centering}
\end{table}



