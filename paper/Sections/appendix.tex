%!TEX root = ../main.tex

\section{Detailed Cryptanalysis}
\label{appendix:cryptanalysis}
\itai{Need to align notation and write summary in the body of the paper.}

In this appendix we describe in detail the cryptanalysis of our new constructions.
We start with some general remarks and then analyze each construction.

\subsubsection{Choosing public inputs.}

All the cryptosystems we define receive public inputs chosen at random.
For example, the \ttOWF receives matrices $\mat{A}$ and $\mat{B}$ as public inputs.
One option is to choose $\mat{A}$ and $\mat{B}$ independently per secret input.
While an alternative option is to fix one (or even both) of the matrices and reuse them.
Generally, this alternative option is more susceptible to multi-target
attacks and attacks that are based on self-similarity.
Thus, in general, the first option is considered more secure and this is the option we use.
For similar reasons we choose the public parameters independently per secret for the
PRFs and the PRG we define.

Of course, there are bad choices of the public inputs which could degrade security,
and we need to show that these are unlikely to be occur.

Finally, for the PRFs we define, this still leaves open the question of how to select the public inputs $x$ and $B$
per sample computed with a secret key. We chose to select $x$ independently per sample,
while fixing $\mat{B}$ per secret key, as this allows to optimize performance by preprocessing.
In terms of security, the choice of fixing $\mat{B}$ does allow for a wider range of attacks
that we need to consider, as we demonstrate in the security analysis.

\paragraph{Restricted public matrices.}
In order to optimize performance, we select the public matrices
for the schemes as random Toeplitz matrices. These matrices
define a pairwise independent hash family and are known to satisfy the
Gilbert–Varshamov bound for linear codes, which is the main tool we use in the analysis.

\paragraph{Private matrices for PRF constructions.}
The private matrix $\mat{K}$ in the PRF constructions is a circulant matrix
defined by rotations of a secret $k \in \{0,1\}^n$.
While a choice of $K$ with a small rank deficiency does not seem to
have a significant impact on the security,
some attacks on the schemes (particularly on the 2-3 PRF) may exploit
matrices of particularly low rank
(as $w = \mat{K}x \bmod 2$ resides in a subspace of small dimension).

Thus, if possible, it is preferable to select $k$ such that $\mat{K}$
is a full rank matrix. Yet, this may require additional communication
is distributed protocols.
Otherwise, we need to understand the rank distribution of $\mat{K}$
when $k$ is selected uniformly at random and prove that $\mat{K}$
has very small rank with negligible probability.
For particular choices of $n$ the analysis is simple.
\begin{proposition}
\label{prop:rank}
Let $n = 2^{n'}$ for a positive integer $n'$
and let $\mat{K} \in \Z_2^{n \times n}$
be a circulant matrix selected uniformly at random.
Then, for any $a \in \{0,\ldots n\}$,
$\Pr[rank(K) \leq a] = 2^{-n+a}$.
\end{proposition}

\begin{proof}
For every vector $u \in \Z_2^{n}$
we associate a polynomial of degree at most $n-1$,
$u(x) = \sum_{j=0}^{n-1} u_j x^{j} \in \Z_2[x]$.

Assume that $\mat{K}$ is formed by rotations
of $k \in \Z_2^{n}$.
Then for $i \in \{0,\ldots, n-1\}$,
the $i$'th column of $K$ is associated with the polynomial
$x^i \cdot k(x) \bmod x^n - 1$.
Thus, for $u \in \Z_2^{n}$, $\mat{K}u$ is given
by the coefficients of the polynomial
$u(x) \cdot k(x) \bmod x^n - 1$,
and $Ku = 0$ if and only if
$u(x) \cdot k(x)$ divides $x^n - 1$.

Therefore, there is a bijection between the kernel space of $\mat{K}$
of the subspace of polynomials $u(x) \in \Z_2[x]$
of degree at most $n-1$
that satisfy $u(x) \cdot k(x) \bmod x^n - 1 = 0$.
The dimension of this subspace of polynomials is equal to
the degree of $\gcd(k(x), x^n - 1)$ over $\Z_2[x]$.

Recalling that $n = 2^{n'}$, over $\Z_2[x]$ we have
$x^{2^{n'}} - 1 = (x - 1)^{2^{n'}} = (x - 1)^n$.
To conclude the proof,
$rank(\mat{K}) \leq a$ if and only if
the kernel dimension of $\mat{K}$ is at least $n-a$.
Equivalently, the degree of $\gcd(k(x), (x - 1)^n)$
is at least $n-a$, or
$k(x)$ divides $(x - 1)^{n-a}$.
There are exactly $2^a$ polynomials of degree at most
$n-1$ that divide $(x - 1)^{n-a}$
and the probability of selecting one of them is
$\tfrac{2^{a}}{2^n} = 2^{-n+a}$ as claimed.
\end{proof}




\subsubsection{Concrete security goals.}
For each scheme the goal is to obtain parameter sets $n,m,t$ such that it offers
$s$-bit security, as defined below, while achieving good performance in distributed protocols.

\paragraph{OWF security.}
Given a OWF scheme $F(\cdot)$ (applied to a secret input, where the public parameters are embedded into $F$),
we define an inversion attack game by choosing $\hat{x} \in \Z_2^n$
uniformly at random and giving $\hat{y}= F(\hat{x})$ to the adversary, whose
goal to output some $x \in \Z_2^n$ such that $F(x) = \hat{y}$.
We say that $F(\cdot)$ has $s$ bits of security if no adversary can win the inversion attack game on $F(\cdot)$ with average complexity below $2^{s-1}$.


\paragraph{Weak PRF security.}
We refer to both weak PRFs we define.

The key $k \in \Z_2^n$ that defines the secret matrix $\mat{K}$ is chosen uniformly at random.
Moreover, the public matrix
$\mat{B} \in \Z_3^{t \times m}$ (or $\mat{B} \in \Z_2^{t \times m}$) is chosen uniformly at random
(or as a random Toeplitz matrix).
For a parameter $r$, an adversary is given $2^{r}$ samples $(x^{(1)},\mat{B},y^{(1)}) ,\ldots, (x^{(2^r)},
\mat{B},y^{(2^r)})$,
where each $x^{(i)} \in \Z_2^n$ is chosen independently and uniformly at random.

We will place a restriction of $r \leq 40$,
corresponding to a practical limit of $2^{40}$ on the number of samples available to the adversary.

We consider two types of adversaries.
The first type is a distinguisher that attempts to distinguish $2^r$ samples where each $y^{(i)}$
is generated using the PRF with a fixed $k$ from
$2^r$ samples where each vector $y^{(i)}$ is chosen uniformly at random.
The second type attempts to find the secret key given samples generated using the PRF.

We say that the PRF has $s$ bits of security if given (at most) $2^r$ samples both conditions below hold.
\begin{enumerate}
  \item The advantage of a distinguishing adversary that runs in time $2^\tau$ is at most $2^{(\tau - s)/2}$.
  \item The probability that an adversary that runs in time $2^\tau$ find the key is at most $2^{\tau - s}$.
\end{enumerate}
This definition is aligned with the work of Micciancio and Walter~\cite{Micciancio018}.

\paragraph{PRG security.}
The secret seed $x \in \Z_2^n$ is chosen uniformly at random, along with the public parameters $\mat{A},\mat{B}$.
The adversary is given a single sample $\mat{A},\mat{B},y$.
As for PRFs, security is defined against the two types of adversaries and the definition is similar.

\subsubsection{Algebraic attacks.}
In algebraic attacks the attacker represents the outputs (on internal variables) of the cipher as multivariate polynomials in the secret key (or preimage), obtaining a system of polynomial equations. The attacker then attempts to the solve the system using techniques such as linearization or applying algorithms for finding a reduced representation of the ideal generated by the polynomials in the form of a Gr\"{o}bner basis.
Such methods are known to be efficient only in particular cases where the polynomials have a special structure, or the
polynomials equations are of low degree and the attacker obtains sufficiently many equations to solve the system by linearization.

In our case, the output of the schemes we define
mix between the sums $\bmod$ 2 and $\bmod$ 3.
For example, in the \ttOWF and PRF constructions each output entry is a sum $\bmod$ 3 of entries of $w$,
where each such entry is a sum $\bmod$ 2 of the unknown bits of the secret input.
Due to the mix between the sums $\bmod$ 2 and $\bmod$ 3 we conjecture (similarly to~\cite{boneh2018-darkmatter}) that
the output cannot be represented (or well-approximated) by a low degree polynomial over any specific polynomial ring.
In particular, it was shown in~\cite{boneh2018-darkmatter} that the sum $\bmod$ 3 of $\ell$ binary-valued variables is
a high-degree polynomial over $\Z_2$,
as long as $\ell$ is large (e.g., $\ell \approx n$).
Crucially, our choice of parameters will ensure that the sums $\bmod$ 2 and $\bmod$ 3
are dense and contain many terms.
In particular, for the
\ttOWF and PRF constructions
we will make sure that the linear code spanned by the rows of $B$ has large minimal distance,
except with very small probability.
Overall, we do not expect algebraic attacks to pose a threat to our schemes,
and our analysis is mainly based on combinatorial attacks that attempt to recover the secret,
or on statistical attacks whose goal is to distinguish the output from random.


\subsection{Security Evaluation of the \ttOWF}

In this section we analyze the security of the \ttOWF.
Beforehand, we note that we may assume without loss of generality that in the most efficient construction, the number of expected preimages is (about) 1. Specifically, in our case, we may assume that $n = \log 3 \cdot t$ (up to rounding factors).

Indeed, setting $\log 3 \cdot t > n$ does not reduce the average number of preimages substantially. Consequently, any attack on a scheme with $n = \log 3 \cdot t'$ can be applied to a scheme with $\log 3 \cdot t > n$ by truncating the output to be of length $\log 3 \cdot t'$. Hence a scheme in which $\log 3 \cdot t > n$ does not offer better security than the truncated one.
On the other hand, the truncated scheme has shorter output and is generally more efficient.
Similarly, if $n > \log 3 \cdot t$, an attacker can fix $n - \log 3 \cdot t$ bits of the secret input to an arbitrary value and try to invert the image of the induced scheme where $n' = \log 3 \cdot t$ (note that on average, such a preimage exists).


\subsubsection{Basic attacks.}
%\label{sec:basic}
We describe several basic attacks and analyze their complexity as a function of $n,m,t$.
First, by exhaustive search, we can invert the \ttOWF in time complexity $2^n$ or $3^t = 2^{\log 3 \cdot t}$.

Focusing on the value of $m$, by exhaustive search, we can find $x$ such that $\mat{A}x = \mat{A}\hat{x}$ (which implies that the outputs are identical) in time complexity $2^m$.
A tighter restriction on $m$ is imposed by the following attack: guess $m - t$ bits of $w = \mat{A}x$ and solve the linear equation system $\hat{y} = \mat{B}w$ over $\Z_3$ (which has $t$ equations and variables) to obtain a full suggestion for $w$. A suggestion for $w$ allows to compute $x$ by solving the linear equation system $\mat{A}x=w$ over $\Z_2$. This attack has complexity $2^{m-t}$. An improved attack is described next.

\paragraph{Enumerating $w$ values.}

We show how to enumerate over all $w \in \{0,1\}^m$ that satisfy $\mat{B} w = \hat{y}$ in time complexity of about $2^{m/2}$ if $m \leq 2 \log 3 \cdot t = 2 n$, and $2^{m - \log 3 \cdot t} = 2^{m - n}$, otherwise.

Given such an algorithm, we can test each $w$ by solving the equation system $\mat{A}x = w$ over $\Z_2$, and if a solution exists, we have successfully inverted~$\hat{y}$.

Observe that if $w$ and $w'$ do not have a common $1$ entry, then $w + w' \bmod 2 = w + w' \bmod 3$
(where the addition is performed entry-wise). Therefore,
\begin{align}
\label{eq:lineara}
\begin{split}
\mat{B}(w + w' \bmod 2) \bmod 3 = \\
\mat{B}(w + w' \bmod 3) \bmod 3 = \\
(\mat{B}w \bmod 3) + (\mat{B}w' \bmod 3) \bmod 3.
\end{split}
\end{align}

We use this observation in the following algorithm, whose complexity as claimed above.
\begin{enumerate}
  \item Partition the $m$ indices of $w$ into 2 subsets $I_1$ and $I_2 = [m] \backslash I_1$, each of size $m/2$ bits.
  \item For $i \in \{0,1,\ldots 2^{m/2} - 1\}$, let $w_i$ be the $m$-bit vector whose value on the $m/2$ indices of $I_1$ is $i$, and is 0 on the indices of $I_2$. For each such $i$,
      evaluate $\mat{B} w_i \bmod 3 = y_i$ and store the pairs $(w_i,y_i)$ in a table $\mathcal{T}$, sorted by $y_i$ values.
  \item For $j \in \{0,1,\ldots 2^{m/2} - 1\}$, let $w'_j$ be the $m$-bit vector whose value on the $m/2$ indices of $I_2$ is $j$, and is 0 on the indices of $I_1$. For each such $j$,
      evaluate $\mat{B} w'_j \bmod 3 = y'_j$ and search $\mathcal{T}$ for the value $\hat{y} - y'_j \bmod 3$. If there exists a match $y_i$ such that $y_i = \hat{y} - y'_j \bmod 3$ (or $y_i + y'_j \bmod 3 = \hat{y}$), recover the value $w_i$ such that $\mat{B} w_i \bmod 3 = y_i$ from $\mathcal{T}$
      and return $w = w_i + w'_j \bmod 2$.
\end{enumerate}

Note that the expected number of $w \in \{0,1\}^m$ that satisfy $\mat{B} w = \hat{y}$ is $2^{m - \log 3 \cdot t}$. Hence, we cannot hope to obtain better complexity than $2^{m - \log 3 \cdot t}$ without exploiting additional constraints on $w$, imposed by the matrix $\mat{A}$. Our
subset-sum reduction (given in Section~\ref{sec:cryptanalysis}) shows how this can be done.

\paragraph{Induced schemes.}
Given the scheme \ttOWF with a matrix $\mat{B}$ and output $y$ such that $\mat{B}w = y$, and any positive integer $r$, we can left-multiply both sides by any $r \times t$ matrix $\mat{C}$ over $\Z_3$ to obtain $\mat{C}\mat{B}w = \mat{C}y$. Note that each row of the matrix $\mat{CB}$ is a linear combination of the rows of $\mat{B}$. Using such a matrix $\mat{C}$, we can perform Gaussian elimination on the rows of $\mat{B}$.

We denote the resultant induced scheme by $OWF_\mat{C}(\cdot)$. Observe that if $OWF(x) = y$, then $OWF_\mat{C}(x) = \mat{C}y$.
We now describe a simple attack that uses an induced scheme where $\mat{C}$ is only a row vector.

\paragraph{Low Hamming weight combinations of the rows of $\mat{B}$.}
Assume that there is a vector $v \in \Z_3^m$ of Hamming weight $\ell$ in the row space of $\mat{B}$, namely, there exists a vector $u \in \Z_3^t$ for which $u \mat{B} = v$. If $\ell$ is sufficiently small, then we could use the induced scheme $OWF_u(\cdot)$ to speed up exhaustive search.

Denote the set of $\ell$ non-zero indices of $v$ by $I$. Given $\hat{y} = \mat{B}w \bmod 3$, we compute the value of $u\hat{y} \bmod 3 = vw \bmod 3$. We can now enumerate the values of the corresponding set $I$ of $\ell$ bits of $w$ for which $u\hat{y} \bmod 3 = vw \bmod 3$ holds. This set of bits has $\tfrac{2^\ell}{3}$ possible values. Each such $\ell$-bit value gives rise of a system of $\ell$ linear equations on $x$, and we exhaustively search its solution space of size $2^{n-\ell}$. Overall, if $\ell \leq n$ the complexity of the attack is
$\tfrac{2^{\ell}}{3}$, while if $\ell = n+1$, the complexity is $\tfrac{2^{\ell+1}}{3}$. When $\ell > n+1$, the complexity is higher than $2^n$.
Thus, we will require that such a vector $v$ of low Hamming weight about $n$ does not exist, except with small probability.
This probability is computed in Proposition~\ref{prop:hw} in Appendix~\ref{app:distance}.

If more such low Hamming weight vectors are available, then the complexity of the attack may be further reduced,
although it seems unlikely to obtain a significant advantage over exhaustive search with this approach.




%\subsubsection{Quantum attacks.}

%Attackers with access to a quantum computer can improve upon the complexity of some of the attacks described in the classical setting. In the classical setting, exhaustive search and the subset-sum based algorithm are the most relevant attacks that we found of the scheme. This also applies to the post-quantum setting.

%First, it is possible to invert $OWF(\cdot)$ with Grover's algorithm in time complexity $2^{n/2}$. Second, according to~\cite{BonnetainBSS20}, one can solve subset-sum (in $m$ binary variables) on a quantum computer in complexity $2^{0.2156 m}$ (ignoring polynomial factors).


\subsubsection{Parameter Selection for the \ttOWF.}
According to the analysis, we determine parameters $n,m,t$ for which
we conjecture that the \ttOWF has $s$ bits of security.

First, due to the exhaustive search, we require $n \geq s$.
Second,
the most restrictive constraint on $m$ is imposed by the subset-sum algorithm (Section~\ref{sec:cryptanalysis}).
If we conservatively ignore the hidden polynomial factors and the large memory
complexity of the subset-sum algorithm of~\cite{BonnetainBSS20}, we need to set
$$0.283m \geq s.$$

Overall, we obtain
$$n = \log 3 \cdot t = s,$$
and $$m = \tfrac{s}{0.283} \approx 3.53 s.$$

We now consider the attack exploiting low Hamming weight combinations of the rows of $\mat{B}$,
and in particular, Proposition~\ref{prop:hw}.
In our case, we apply the proposition with $\log 3 \cdot t = n$ and $\ell = n$.
For $m = 3.53 n$, we obtain that the probability of having a vector of Hamming weight at most $n$ is bounded by
$$2 \cdot 2^{m (H(0.283) + 2 \cdot 0.283 - \log 3)} \approx 2 \cdot 2^{-0.16m} \approx 2 \cdot 2^{- 0.56 s}.$$
For $s \geq 128$, the expression evaluates to (less than) $2^{-70}$, so it is unlikely to encounter such an event in practice.
Moreover, even if the event occurs, security only regrades by a factor of 3, and by the same
proposition the probability that two such vectors are spanned by the rows of $B$ is at most $2^{-140}$.
Nevertheless, one may increase $n$ (and correspondingly $t$)
by a few bits (at negligible cost) to defeat this attack vector completely.

A more aggressive setting of the parameters may take into account the polynomial factors of~\cite{BonnetainBSS20} (and perhaps its high memory complexity). Unfortunately, the polynomial factors associated with the complexity formulas of the relevant subset-sum algorithms have not been analyzed.
For example, if we assume that the polynomial factors are about $m^2$, and we aim for $s = 128$ bits of security, then
we require $m^2 \cdot 2^{0.283m} \geq 2^{s} = 2^{128}$. Setting $m = 400 = 3.125 s$ is sufficient for satisfying the constraint in this setting.


%\paragraph{Post-quantum setting.}

%In the post-quantum setting, we have the constraints
%$n/2 \geq s$
%due to Grover's algorithm, and
%$0.2156 m \geq s$
%due to the quantum subset-sum algorithm.

%Overall, we obtain
%$n = \log 3 \cdot t = 2s,$
%and $m = \tfrac{s}{0.2156} \approx 4.64 s.$


\subsection{Security Evaluation of the \ttwPRF}
\label{sec:basicprf}

As for the \ttOWF, we describe several attacks and analyze their complexity as a function of the parameters $n,m,t$.

Unlike the case of the \ttOWF (where the goal was to find a preimage of a give output),
we can choose a small value of $t$ regardless of the other parameters without sacrificing security. In fact, it is clear that a small value of $t$ can only contribute to security, as any attack on a scheme with a small value of $t$ can be applied to a scheme with a larger value of $t$, simply by ignoring part of the output. Consequently, we may fix $t$ to the smallest value acceptable by the application.

We also note that given a sufficiently large number of samples, we expect that the key $k$ would be uniquely determined by the samples (regardless of the value of $t$).

\subsubsection{Key recovery attacks exploiting a few samples.}
We describe key recovery attacks that make use of the minimal number of samples required to derive the secret key $k$.

First, as for the \ttOWF, exhaustive search requires $2^{n}$ time. Also,
similarly to the \ttOWF, given any sample $(x,\mat{B},y)$, we can guess $m - t$ bits of $w = \mat{K} x$ and solve the linear equation system $y = \mat{B} w \bmod 3$, which then allows to compute a suggestion for $k$ (that can be tested on the remaining samples). This attack has complexity $2^{m-t}$.

Furthermore, given a single sample, we can apply the same attack for enumerating $w$ values for the \ttOWF.
This attack has complexity which is the maximum between $2^{m/2}$ and $2^{m - \log 3 \cdot t}$.

\paragraph{Reduction to subset-sum.}
As for the \ttOWF, we can reduce the key recovery problem to the problem of solving subset-sum over the $m$ binary variables of $w$. However, it is clear that if the algorithm is applied to a single $(x,\mat{B},y)$ sample, then its expected complexity cannot drop below $2^{m - \log 3 \cdot t - (m -n)} = 2^{n - \log 3 \cdot t}$, which is the expected number of $w$ values possible given $(x,\mat{B},y)$
(the remaining key candidate after analyzing one sample are tested against another sample).

On the other hand, if we try to reduce the complexity by applying the algorithm to more than one sample (e.g., to $(x,\mat{B},y)$ and $(x',\mat{B},y')$), then we must take advantage of the dependency between $w = \mat{K} x$ and $w' = \mat{K} x'$, which are related via linear constraints, imposed by $k$ and by $x,x'$. However, it is not clear how to model these complex linear constraints in the subset-sum reduction and we were not able to improve the complexity of the single-sample attack.

\subsubsection{Exploiting multiple samples.}
The key recovery attacks described above make use of a minimal number of samples required to derive the secret key. On the other hand, when given more samples, it may be possible to exploit various relations among them to mount distinguishing and even key recovery attacks which we investigate below.

\paragraph{Output bias.}
We consider a single sample $(x,\mat{B},y)$ and analyze the bias of linear combinations of the entries of $y$ over $\Z_3$.
If any such linear combination has a sufficiently high bias towards some constant,
then an attacker can exploit it in a distinguishing attack.

Similarly to the case analyzed for the corresponding \ttOWF, assume there are vectors $v \in \Z_3^m$ and $u \in \Z_3^t$ such that $u \mat{B} = v$ and the Hamming weight of $v$ is $\ell$. Given $y = \mat{B}w \bmod 3$, the attacker computes $uy \bmod 3 = vw \bmod 3$ and thus obtains the value of a linear combination $\bmod$ 3 of $\ell$ entries of $w \in \{0,1\}^m$.
Specifically, denoting the set of $\ell$ non-zero indices of $v$ by $I$, the attacker computes $\sum_{i \in I} v_i w_i \bmod 3$.
We now calculate the bias of sum the $\bmod $ 3, assuming that $w$ is uniformly distributed in $\Z_2^m$.

Each non-zero coefficient of the linear combination $v$ is either 1 and 2.
It is easy to prove by induction on $\ell$ (or by analysis of sums of binomial coefficients) that
for any coefficients $v_i \in \{1,2\}$ where $i \in I$ and for any $a \in \{0,1,2\}$,
$$\Pr\left[\sum_{i \in I} v_i w_i \bmod 3 = a\right] \in \{\tfrac{1}{3} \pm \tfrac{1}{2^\ell}, \tfrac{1}{3} \pm \tfrac{2}{2^\ell}\},$$
where the actual probability depends on $v$, $a$ and $\ell$. Thus, the bias of $vw \bmod 3$ is bounded by $\tfrac{2}{2^\ell}$.

We use Proposition~\ref{prop:hw} to deduce
that the subspace spanned by the rows of $\mat{B}$ contains a vector of Hamming weight at most $\ell$ with probability at most
$2 \cdot 2^{m (H(\ell/m) - \log 3) + \ell + \log 3 \cdot t}$.

Our analysis is conservative, as we ignore the work performed
by the attacker to find a low Hamming weight vector $v$ spanned by the rows of $B$.
Thus, we will be interested in $\ell \approx s/2$,
as we would like to avoid having a linear combination of the output with bias at least $2^{-s/2}$
(except with small probability).
Plugging this into the formula above we bound the probability by
\begin{align}
\label{eq:bias}
2 \cdot 2^{m (H(s/2m) - \log 3) + s/2 + \log 3 \cdot t}.
\end{align}

\paragraph{Conditional output bias.}
As described above, the bias of expressions of the form $\sum_{i \in I} v_i w_i \bmod 3$,
(where $I \subseteq [m]$ is a set of size $\ell$, each $v_i \in \{1,2\}$ is fixed and $w_i \in \{0,1\}$
are independent and uniformly distributed random variables) is bounded by $2^{-\ell+1}$.

However, the bias may increase if information about the variables $w_i$ is known.
In particular, the case in which $\sum_{i \in I} w_i \bmod 2$ is known was analyzed in~\cite{CheonCKK20},
where the authors showed that the conditional biases such as
$$\left| \Pr \left[\sum_{i \in I} w_i \bmod 3 = 0 \mid \sum_{i \in I} w_i \bmod 2 = 0 \right] - 1/3 \right|$$
can be as large as about $2^{-0.21\ell}$.
While this is still exponentially small in $\ell$,
it is more significant than the unconditional bias.

The PRF candidate proposed in~\cite{boneh2018-darkmatter} and analyzed in~\cite{CheonCKK20}
is similar to the one we analyze for $n = m$,
yet the matrix $\mat{B}$ contains a single row that sums $\bmod$ 3 all the entries of $w$.
Moreover, since $\mat{K}$ is a circulant matrix and $x$ has even Hamming weight,
then $\sum_{i \in [m]} w_i \bmod 2 = 0$ and the distinguishing attack is applicable.
In our case, $\mat{B}$ is selected at random and the parameters are chosen
such that the attacker cannot obtain $\sum_{i \in I} w_i \bmod 3$ for any fixed set $I$ (and particularly $I = [m]$),
except with negligible probability.
The general distinguishing attack is analyzed below.

Assume that given a sample $(x,\mat{B},y)$,
there exists a set $I \subseteq [m]$ (that depends of $x$)
such that $\sum_{i \in I} w_i \bmod 2$ is known to the attacker.
Moreover, assume that there exists $v$ in the row span of $\mat{B}$
such that $v_i = 1$ for each $i \in I$ and $v_i = 0$
for each $i \notin I$.
Then, the value $\sum_{i \in I} w_i \bmod 3$ can be computed from
the output, and the distinguishing advantage is as high as (about) $2^{-0.21\ell}$.
Of course, if several samples are available, the distinguishing advantage can
increase by accumulating the bias,
assuming the above conditions are fulfilled for more than one sample.

An extended variant of the attack may consider a vector
$v' = v + u$ in the row span of $\mat{B}$.
Denoting that Hamming weights of $v$ and $u$ by $\ell_1$,$\ell_2$, respectively,
the conditional bias can be as high as $2^{-0.21\ell_1 - \ell_2 + 1}$.
Note that here we conservatively ignore the work required to find such $v'$.

Given $2^r$ samples (where $r \leq 40$),
we wish to show that the distinguishing advantage is small
(except with negligible probability).
Indeed, calculation shows that this distinguisher is not stronger that the
unconditional distinguisher.
Essentially, given a single sample, for any fixed $I$, the vector $v$ as described above
is in the row span of $\mat{B}$ with minuscule probability $3^{t - m}$.
In the extended attack, we consider a ball around $v$, but this ball
is much smaller than the one considered in the unconditional distinguisher (as $\ell_2 < s/2$).

Finally, in a more general variant of the attack,
the attacker may guess a parity of key bits and calculate the conditional bias over several samples,
attempting to amplify it.
Note that the desired vector $v$ changes for each sample according to $x$.
By similar calculation, once the attacker has fixed the guess given a sample,
almost all additional samples would not allow to amplify the conditional bias,
as a good vector is unlikely to be in the row span of $\mat{B}$.

\paragraph{Differential cryptanalysis and low Hamming weight samples.}
As we argue below, the \ttwPRF seems to be immune to classical statistical differential cryptanalysis.

Assume that the attacker obtains two samples $(x,\mat{B},y)$ and $(x+\delta \bmod 2,\mat{B},y')$, where $\delta \in \{0,1\}^n$.
Denote $w = \mat{K} x \bmod 2$ and $w' = \mat{K} (x + \delta) \bmod 2 = w + \mat{K} \delta \bmod 2$.
Thus, $y' = \mat{B} w' \bmod 3 = \mat{B} (w + \mat{K} \delta \bmod 2 ) \bmod 3$.
In general, $y$ and $y'$ do not seem to have any statistical relation that holds with sufficiently high probability.
Particularly, $w + \mat{K} \delta \bmod 2 \neq w + \mat{K} \delta \bmod 3$,
except with very small probability.

From an algebraic viewpoint, the attacker can consider the $m$ bits of $w = K x \bmod 2$ as variables over $\Z_2$,
but then the algebraic degree of the output over $\Z_3$ would be large.
The attacker can also consider the $m$ bits of $w$ as variables over $\Z_3$.
In this case, the attacker obtains $t$ linear equations and $t$ quadratic equations $\bmod$ 3 (of the form $(w_i)^2 - w_i = 0$) from the sample $(x,\mat{B},y)$.
On the other hand, the algebraic degree of $w' = \mat{K} (x +\delta) \bmod 2$ in the variables of $w$ would be large due the dense $\bmod$ 2 operations,
hence it is not clear how to obtain additional low degree equations.
In general, such algebraic attacks do not seem more efficient than the attack described above for enumerating $w$ values.

Another scenario which we consider is when the attacker obtains a sample $(x,\mat{B},y)$
such that $x$ is of low Hamming weight. In this case each entry of $w$ is a low Hamming weight sum $\bmod$ 2
of the bits of $k$ and can thus be described as a low degree polynomial over $\Z_3$.
Consequently, low degree polynomial equations over $\Z_3$ in the secret key can be deduced from the output.
Obtaining several such samples may allow the attacker to solve for the key.
In general, such low Hamming weight samples are avoided with high probability given the data limit,
but we can also place a restriction on the sample generation, forcing it to generate vectors $x$
with minimal Hamming weight (e.g., a lower bound of $n/4$).



\paragraph{Attacks based on self-similarity.}

%collision on $x$ values

There are several simple attacks that take advantage of the fact that $\mat{B}$ is fixed for all samples generated with the same secret $k$.
A basic attack looks for a collision on the $w$ values for a pair of samples, which can be detected at the output.
Given $2^r$ samples, the advantage of this attack is $2^{2r - rank(\mat{K})}$. Given the data complexity bound,
we will set the parameters such that the advantage of this attack is negligible (assuming the rank of $K$ is not too small).

\itai{Say more about the structure of K.}

Another simple attack is a multi-target attack, where given $2^r$ samples, the attacker guesses $w$, computes $\mat{B} w \bmod 3$ and compares the result with all given outputs. A match allows to recover a candidate for the secret $k$.
The expected complexity of this attack is $2^{m-r}$, but cannot drop below $2^{m - \log 3 \cdot t}$ without exploiting relations between the different $w$ values. In general, given the data limit, the attack less efficient that the attack that enumerates all $w$ values considered above.

\paragraph{Simultaneous sums.}

We describe a more involved self-similarity attack, which exploits the fixed $\mat{B}$ value per secret $k$.

Assume that for some index set $I$ of size at least 3, $\sum_{i \in I} x^{(i)} \bmod 2 = 0$. Then, $\sum_{i \in I} w^{(i)} \bmod 2 = 0$. While it is not clear how this relation influences the output,
we extend this initial observation by simultaneously considering sums $\bmod$ 2 and 3, as follows:
assume that there are 4 samples (denoted for simplicity
by $ \{(x^{(i)},\mat{B},y^{(i)})\}_{i=1}^{4}$) such that for each $j \in [m]$,
$$\sum_{i = 1}^{4} w^{(i)}_j = 2.$$
Then, $\sum_{i = 0}^{4} x^{(i)} \bmod 2 = 0$ and $\sum_{i = 0}^{4} y^{(i)} \bmod 3 = B \cdot \vec{2} \bmod 3$ (where $\vec{2}$ is a vector with $m$ entries whose values are 2).
Note that a random 4-tuple of samples satisfies this simultaneous sum constraint with probability
$2^{-n - \log 3 \cdot t}$, but the probability that the constraint $\sum_{i = 0}^{4} w^{(i)}_j = 2$ is satisfied for every $j \in [m]$ is about
$$\left( \tfrac{\tbinom{4}{2}}{16} \right) ^{-m} \approx 2^{- 1.415 m},$$
which is higher than expected if $1.415 m < n + \log 3 \cdot t $.

Since the adversary has about $2^{4r}$ such 4-tuples, the probability that such a simultaneous 4-sum exists is about $2^{4r - 1.415 m}$. It can be detected in time complexity of about $2^{2r}$ using a standard matching algorithm. The important constraints for defending against this attack are
$$1.415 m > n + \log 3 \cdot t,$$ or
\begin{align}
\label{eq:sim}
2^{4r - 1.415 m}
\end{align}
is negligible, otherwise.

The simultaneous 4-sum distinguisher can be easily generalized to a
simultaneous $d$-sum distinguisher for arbitrary $d$.
In general, we look for $d$-tuples where (for example)
for each $j \in [m]$, $$\sum_{i = 1}^{d} w^{(i)}_j = c,$$
for some value $c$ (fixed $\bmod$ 6) such that $c \bmod 2 = 0$.
However, calculation shows that $d = 4$ gives the most efficient
distinguisher in our case (with the small data limit).



%\subsubsection{Quantum attacks.}
%The relevant attacks in the quantum setting are similar to ones for the OWF.
%First, Grover's algorithm, breaks the scheme (recovers the key) is time $2^{n/2}$.
%Second, the quantum subset-sum algorithm of~\cite{BonnetainBSS20} has complexity $2^{0.2156 m}$ (ignoring polynomial factors), but does not drop below $2^{n - \log 3 \cdot t}$.


\subsubsection{Parameter selection for the \ttwPRF.}
According to the analysis, we determine parameters $n,m,t$ for which
we conjecture that the \ttwPRF has $s$ bits of security.

In order to select the parameters, we may first set $t$ to it's minimal possible value
(depending on the application).
We also assume that $t$ is not too large, and particularly $\log 3 \cdot t \leq s$.

The constraints imposed by the above attacks are as follows.
First, due to exhaustive search, we require $$n \geq s.$$
Second, the subset-sum algorithm
imposes the constraint
$n - \log 3 \cdot t \geq s,$
(given that $n \geq m$ and $\log 3 \cdot t \leq s$).


We consider two sets of parameter.
The first is
$$n = m = s + \log 3 \cdot t.$$
In particular, if $\log 3 \cdot t  = s$, then
$n = m = 2s$.
The second parameter set
$$n = m = 1.25(s + \log 3 \cdot t),$$
and is less aggressive.

Next, we analyze the bias of the output based on~(\ref{eq:bias}),
assuming that $\log 3 \cdot t = s$.
For the first parameter set,
we obtain that the probability of having bias of $\tfrac{2}{2^{s/2}}$ is bounded by
\begin{align*}
2 \cdot 2^{m (H(s/2m) - \log 3) + s/2 + \log 3 \cdot t} =
2 \cdot 2^{2s (H(1/4) - \log 3) + 3s/2} \approx
2 \cdot 2^{-0.049s},
\end{align*}
which is non-negligible.
On the other hand, the consequences of the distinguishing attack given the data limit
seem relatively mild, and this parameter set may be considered by applications
where performance is critical.

For the second parameter set (assuming $\log 3 \cdot t = s$),
the probability of having bias $\tfrac{2}{2^{s/2}}$ is bounded by
\begin{align*}
2 \cdot 2^{m (H(s/2m) - \log 3) + s/2 + \log 3 \cdot t} =
2 \cdot 2^{2.5s (H(1/5) - \log 3) + 3s/2} \approx
2 \cdot 2^{-0.65s},
\end{align*}
which we consider negligible.

The advantage of the collision attack is $2^{2r - rank(\mat{K})} \leq 2^{80 - rank(\mat{K})}$.
Given that $m = 2s \geq 256$, if we choose $n$ as a power of 2,
based on Proposition~\ref{prop:rank},
the advantage is much smaller than $2^{s/2}$ (except with negligible probability).

Finally, we consider the simultaneous 4-sum distinguisher,
and recall that the probability of having such a 4-sum in the output is estimated
in~(\ref{eq:sim}) as $2^{4r - 1.415 m} \leq 2^{160 - 2.83s}$.
For a minimal choice of $s = 128$, this probability is smaller than $2^{-200}$
which is negligible.



\subsection{Security Analysis of the LPN-PRG}

We analyze the security of the LPN-PRG.

If the matrix $\mat{A}$ was a random matrix, then
the first step would consist of generating $m$ samples from the alternative weak PRF construction proposed
in~\cite{boneh2018-darkmatter}. Each sample
$w_i = \mat{A}[i] x \bmod 2 + ((\mat{A}[i] x \bmod 3) \bmod 2) \bmod 2 \in \Z_2^m$
can be viewed as adding noise $((\mat{A}[i] x \bmod 3) \bmod 2)$ to the inner product $\mat{A}[i] x \bmod 2$.
Given that $\mat{A}[i]$ is of sufficiently large Hamming weight, then
$\Pr_x[(\mat{A}[i] x \bmod 3) \bmod 2 = 1] \approx 1/3$, which is the magnitude of noise added.
The second step consists of a compressing linear transformation $B$ applied to $w$.
The idea is to increase the noise of each sample by mixing it with other samples.
This step should defeat standard attacks applied to LPN with a constant noise parameter
(such as decoding attacks).

In our case, the matrix $\mat{A}$ is structured (it is a random Toeplitz matrix),
but we were not able to exploit this in an efficient attack.



\subsubsection{Key recovery attacks.}
We begin by considering attacks that attempt to recover the secret key (seed).

Exhaustive search for the key recovery attacks requires time $2^n$.

In a different approach for recovering the key, given a sample $\mat{A},\mat{B},y$,
the attacker enumerates over the subspace of $w$ values that satisfy $\mat{B} w = y \bmod 2$.
This subspace contains $2^{m - t}$ vectors. For each such vector, the attacker attempts to recover $x$
given $\mat{A}$ and $w$. Thus, given $\mat{A},w$ the attacker has $m$ samples
generated from the LPN-like construction proposed in~\cite{boneh2018-darkmatter} as a weak PRF
(although we a structured matrix $\mat{A}$).
Given that $m$ is not too large (i.e., it is a small multiple of $n$),
then the best attack we have on this scheme simply tries to break the LPN instance (which has noise of $1/3$),
without exploiting the deterministic way in which it is generated.
The concrete security of LPN given a small number was analyzed in several publications such as~\cite{EsserKM17},
and the complexity of known attack is generally exponential in $n$.
Nevertheless, the attacker is required to solve $2^{m - t}$ related LPN instances,
and perhaps can amortize the complexity. Moreover, the matrix $\mat{A}$ is structured.
Thus, we (conservatively) estimate the total complexity of such attacks by $2^{m - t}$.

\subsubsection{Noisy linear equations.}

As noted above, each bit $w_i$ for $i \in [m]$ can be viewed as a noisy linear equation over $\Z_2$
with noise of about $1/3$, or bias $2/3 - 1/2 = 1/6$.
Our goal is to select the parameter $m$ such that
the all linear combinations of the output bits of $y$
have exponentially small bias towards a linear equation in the secret $x$.
We will (heuristically) model each bit $w_i$ as having independent bias of $1/6$.

If the linear subspace spanned by the rows of $\mat{B}$ contains a vector of Hamming weight $\ell$,
then by the piling-up lemma, the bias of the corresponding linear combination of the bits of $w$ is
\begin{align}\label{eq:bias_linear}
2^{\ell - 1} \cdot (\tfrac{1}{6})^{\ell} < 2^{-\log 3 \cdot \ell}.
\end{align}

Thus, we will require
that the rows of $\mat{B}$ do not span a vector whose Hamming weight is too low.
This is similar to the previously analyzed schemes,
but the lower bound on the Hamming weight we will enforce for the PRG will be lower.
Indeed, the bias above is calculated with respect some linear equation in the unknown secret key bits.
Such a bias is generally much less of a security concern compared to a bias towards a constant value
(e.g., the bias analyzed for the \ttwPRF construction) which can be used
to directly distinguish the output from random using statistical tests.
Particularly, an alternative scheme where we change the first transformation
to only compute the ``noise part''
($w =(\mat{A}x \bmod 3) \bmod 2$) would require larger parameters to be secure,
as we need a higher lower bound on the Hamming weight $\ell$ to avoid distinguishing attacks.

For a PRG with $s$ bits of security, we will conservatively require a bias of at most $2^{-0.1 s}$.
We are not aware of any attack that can exploit such a low bias.
Effectively, this means that the minimal distance of $B$ should be
at least $s \cdot 0.1/\log 3 < 0.07 s$ (except with small probability).


\begin{remark}
In~\cite{CheonCKK20} the authors analyze the constructions presented in~\cite{boneh2018-darkmatter}.
In particular, for the alternative PRF construction,
given a sample $(a \in \Z_2^n, x \cdot a)$,
they show that there exists $j \in [n]$ such that
$$|\Pr[a_j = 0 \mid x_j = 0 \text{ and } (a \cdot x \bmod 2 + ((a \cdot x \bmod 3) \bmod 2) \bmod 2) = 0]| \approx \tfrac{1}{2^{0.21 \ell}},$$
where $\ell$ is the Hamming weight of $a$.
This property was exploited in a distinguishing attack.

Our PRG construction seems to be immune to this type of analysis
because the attacker only has access to sufficiently dense linear combinations of (structured) samples
of the alternative PRF construction.
\end{remark}


\subsubsection{Parameter selection for the LPN-PRG.}

We determine parameters $n,m,t$ for which
we conjecture that the
PRG has $s$ bits of security.

Recall the we set $t = 2n$.
Exhaustive search implies that $n \geq s$ and we have lower
bounded the effort required in key recover by $2^{m - t}$.
Thus, a reasonable choice of parameters is $n = s$ and $m = 3s, t = 2s$.

For these parameters, we consider the maximal bias of linear combinations according to~(\ref{eq:bias_linear}),
with $\ell = 0.07 s$.
By Proposition~\ref{prop:hw},
the probability of having a vector of Hamming weight more than $0.07 s$
in the row span of $B$ is
$0.07 s \cdot 2^{3s (H(0.07/3) - 1) + 2s} < 0.07 s \cdot 2^{-0.52 s}$.
We consider this as a negligible probability
as the consequences of the this unlikely event are mild.


\subsection{Security analysis of LPN-PRF}

The overall structure of the LPN-PRF is similar to the PRG.
However, unlike PRG, in the first transformation, the key is part of the matrix and
$x$ is a public value.
Thus, $w$ can be viewed as $m$ outputs of (another) more
structured version of the construction of~\cite{boneh2018-darkmatter}.

In terms of parameters, the main differences are that we have $n \geq m$
and $t$ is not constraint to $2n$ (in fact, we will propose to set it smaller than $n$).
Furthermore, we assume that the attacker obtains $2^r$ samples for $r \leq 40$ instead of a single sample.

We note that variants of the basic attacks that were analyzed for the \ttwPRF (such as the collision attack
and the multi-target attack) are also applicable to this construction.
As in the case of the \ttwPRF,
they do not seem to be a threat for our choice of parameters.

Overall, we have not found any class of attacks that are
applicable to this construction, but not to the previous ones.
Below we briefly consider the most important attacks that influence the
choice of parameters.

\paragraph{Summary of attacks.}
As for the PRG, we estimate the total complexity of key recovery attacks by $2^{m - t}$.
Although we have more samples that for the PRF,
it is not clear how to exploit them to obtain better complexity.

In addition, similarly to the PRG,
we require that each linear combination of the output
has bias of at most $2^{-0.1 s}$
(toward some linear combination of the key over $\Z_2$).

\paragraph{Parameter selection for the LPN-PRF.}
For $s$-bit security, we set $n=m= 2s$ and $t =s$.
By our analysis, exhaustive search requires $2^{m - t} = 2^s$ time.

We consider the maximal bias of linear combinations according to~(\ref{eq:bias_linear}),
with $\ell = 0.07 s$.
By Proposition~\ref{prop:hw},
the probability of having a vector of Hamming weight more than $0.07 s$
in the row span of $\mat{B}$ is
$0.07 s \cdot 2^{2s (H(0.07/2) - 1) + s} < 0.07 s \cdot 2^{-0.56 s}$,
which we consider negligible.


\section{Analysis of Random Linear Codes}
\label{app:distance}

We analyze the distance of random linear codes defined by a random matrix
or a random Toeplitz matrix.
The analysis is based on the probabilistic method and is
similar to the analysis used to obtain the Gilbert–Varshamov bound.

\begin{proposition}
\label{prop:hw}
Let $\mat{B} \in \Z^{t \times m}_{q}$ be a random matrix, or a random Toeplitz matrix whose rows define
a linear code.
Then, the minimal distance of (the code defined by) $\mat{B}$ is at most $\ell < m/2$ with probability at most
$$f_q(m,t,\ell) \leq q^{t-m} \cdot Vol_q(m,\ell),$$
where $Vol_q(m,\ell) = \sum_{i=1}^\ell \tbinom{m}{i} \cdot (q-1)^{i}$.
Moreover, this code contains two such linearly independent vectors with probability at most $(f_q(m,t,\ell))^2$.

Finally, let $H(p) = - p \log p - (1-p) \log(1-p)$ be the binary entropy function.
Then
$$f_2(m,t,\ell) \leq \ell \cdot 2^{m (H(\ell/m) - 1) + t},$$
and
$$f_3(m,t,\ell) \leq 2 \cdot 2^{m (H(\ell/m) - \log 3) + \ell + \log 3 \cdot t}.$$
\end{proposition}

\begin{proof}The number of non-zero vectors of Hamming weight at most $\ell$ over $\Z_q^m$ is
$Vol_q(m,\ell) = \sum_{i=1}^\ell \tbinom{m}{i} \cdot (q-1)^{i}$.

For $q = 2$, we have
$Vol_2(m,\ell) = \sum_{i=1}^\ell \tbinom{m}{i} <
\ell \cdot \tbinom{m}{\ell} \leq
\ell \cdot 2^{m H(\ell/m)}$, while for $q = 3$, we have
$Vol_3(m,\ell) = \sum_{i=1}^\ell \tbinom{m}{i} 2^i <
2 \cdot \tbinom{m}{\ell} 2^\ell \leq
2 \cdot 2^{m H(\ell/m) + \ell}.$

Since $\mat{B}$ is selected uniformly from a pairwise independent hash family (either a random matrix or a random Toeplitz matrix),
the probability that every non-zero vector is in the row space is $q^{t-m} - 1 < q^{t-m}$. By a union bound over all vectors of Hamming weight bounded by $\ell$, the probability that there exists such a vector in the row space of $B$ is at most
\begin{align*}
f_q(m,t,\ell) \leq q^{t-m} \cdot Vol_q(m,\ell).
\end{align*}
The bound $(f_q(m,t,\ell))^2$ on the probability of having two such linearly independent vectors follows by pairwise independence.

Specifically, for $q = 2$, we obtain
$f_2(m,t,\ell) \leq \ell \cdot 2^{m (H(\ell/m) - 1) + t}$,
while for $t = 3$, we obtain
$f_3(m,t,\ell) \leq 2 \cdot 3^{t-m} \cdot 2^{m H(\ell/m) + \ell} = 2 \cdot 2^{m (H(\ell/m) - \log 3) + \ell + \log 3 \cdot t}$.
\end{proof}


%-------------------------------------------------%


\section{Deferred Protocol Details}
\label{appendix:protocols}
\paragraph{Circuit description of constructions.}
We can represent our constructions by circuits consisting of the gates described previously. This approach is similar to that of~\cite{boyle2019-fss-preprocess}. We formally define computation circuit representations for our constructions in Definition~\ref{def:computation_circuit}.

\begin{definition}[Computation circuit]
A computation circuit $C$ with input space $\Gin = \prod \Gin_i$ and output space $\Gout = \prod \Gout_i$ is a (labeled) directed acyclic graph $(\mathcal{V},\mathcal{E})$ where $\mathcal{V}$ denotes the set of vertices and $\mathcal{E}$ denotes the set of edges according to the following:
\begin{itemize}

\item Each source vertex corresponds to exactly one $\Gin_i$ and vice versa. The label for the vertex is the identity function on the corresponding $\Gin_i$. Each sink vertex corresponds to exactly one $\Gout_i$ and vice versa. The label for the vertex is the identity function on the corresponding $\Gout_i$. Each non-source $V \in \mathcal{V}$ is labeled with a gate $\mathcal{G}_V \in \gateset$ that computes the function $\mathcal{G}_V: \mathbb{G}^{\textsf{in}}_V \to \mathbb{G}^{\textsf{out}}_V$. The depth of a vertex $V \in \mathcal{V}$, denoted by $\textsf{depth}(V)$ is the length of the largest directed path from a source vertex to $V$.


\item For an edge $(V_a, V_b)$, let $\Gout_{V_a} = \prod \Gout_{V_a,i}$ and $\Gin_{V_b} = \prod \Gin_{V_b, i}$. Then, there exists indices $j$ and $k$ such that $\Gout_{V_a,j} = \Gin_{V_b,k}$. Further, for each input $\Gin_{V_b,i}$ for $V_b$, there is some edge $(V_c, V_b)$ that satisfies the above.

\item The evaluation of the gate for vertex $V$ on input $x \in \Gin_V$ is defined as $y = \mathcal{G}_V(x)$. The evaluation of the circuit $C$, denoted by $\textsf{Eval}_C(x)$, where $x \in \Gin$ is the value $y \in \Gout$, that is obtained by recursively evaluating each gate function in the circuit.
\end{itemize}


Let $\mathsf{F} = \{\mathsf{F}_\secparam\}_{\secparam \in \N}$ denote a family of functions $\mathsf{F}_\secparam:\mathcal{X}_\secparam \to \mathcal{Y}_\secparam$. We say that $\{C_\secparam\}_{\secparam \in \N}$ is a family of computation circuits for $\mathsf{F}$ if all $C_\secparam$ have the same topological structure, and for all $\secparam \in \N$, $\mathsf{F}_\secparam(x) = \textsf{Eval}_C(x)$ for all $x \in \mathcal{X}_\secparam$.
\label{def:computation_circuit}
\end{definition}

\subsection{Local Protocols for Circuit Gates}
\label{appendix:protocol_gates}

Here, we provide the remaining details for the circuit gate protocols.

\paragraph{Protocol notation and considerations.}
For a protocol $\prot$, we use the notation $\prot(a_1, \dots, a_k \mid b_1, \dots, b_l)$ to denote that the values $a_1, \dots, a_k$ are provided publicly to all parties in the protocol, while the values $b_1, \dots, b_l$ are secret shared among the parties. When $\party_i$ knows the values $(a_1, \dots, a_k)$, and has shares $\sharei{b_1}, \dots, \share{b_l})$, we use the notation $\prot(a_1, \dots, a_k \mid \sharei{b_1}, \dots, \sharei{b_l})$ to denote that $\party_i$ runs the protocol with its local inputs. 

Given public values $a_1, \dots, a_k$, it is straightforward for the protocol parties to compute a sharing $\share{f(a_1, \dots, a_k)}$ for a function $f$ (for example, $\party_1$ computes the function as its share, and all other parties set their share to $0$).

\paragraph{Linear gate protocol $\prot_\LMap^{\mat{A}, p}$.}
The linear gate is the easiest to evaluate, and follows from the standard linear homomorphism of additive secret sharing.

\begin{itemize}
  \item \textbf{Functionality}: Each party is provided with the matrix $\mat{A}$ and shares of the input $x$ (over $\Z_p$). The goal is to compute shares of the output $y = \mat{A}x$.

  \item \textbf{Preprocessing}: None required.

  \item \textbf{Protocol details}:
  For the protocol $\prot_\LMap^{\mat{A}, p}(\mat{A} \mid x)$, each party $\party_i$ computes its output share as $\sharei{y} = \mat{A}\sharei{x}$. Note that this works because $\mat{A}x = \sum_{\party_i \in \parties} \mat{A}\sharei{x}$ as a direct consequence of the linear homomorphism of additive shares.
\end{itemize}

\paragraph{Addition gate protocol $\prot_\Add^{p}$.}
The addition modulo $p$ gate is also easy to evaluate. Given shares of $x, x'$ over $\Z_p$, for the protocol each party $\party_i$ can locally compute its share of $x+x'$ as $\sharei{x} + \sharei{x'} \bmod p$. This directly follows from the additive homomorphism of additive shares. 

\paragraph{Bilinear gate protocol $\prot_\BLMap^p$.}
The bilinear gate protocol is essentially a generalization of Beaver's multiplication triples~\cite{boyle2019-fss-preprocess,beaver1991-triples} that computes the multiplication of two shared inputs. For Beaver's protocol, to compute a sharing of $ab$ given shares of $a$ and $b$ (all sharings are over a ring $\mathcal{R}$), the protocol parties are provided shares of a randomly sampled triple of the form $(\tilde{a},\tilde{b},\tilde{a}\tilde{b})$ in the preprocessing stage. Beaver's protocol first reconstructs the masked inputs $\hat{a}$ and $\hat{b}$ after which local computation is enough to produce shares of the output. For our bilinear gate protocol, we assume that  all parties are already provided with the masked inputs (to move the communication outside of the gate protocol), along with correlated randomness similar to a Beaver triple. 

\begin{itemize}
  \item \textbf{Functionality}: 
  Abstractly, the goal of the bilinear gate protocol is to compute shares of the output $y = \mat{K}x$ given shares of both inputs $\mat{K}$ and $x$. For our purpose however, the masked inputs will have already been constructed beforehand, i.e., each party is provided with $\hat{\mat{K}}$ and $\hat{x}$ publicly, along with shares of correlated randomness similar to a Beaver triple (see below).

  \item \textbf{Preprocessing}: Each party is provided shares of $\tilde{\mat{K}}, \tilde{x}$, and $\tilde{\mat{K}}\tilde{x}$ as correlated randomness.

  \item \textbf{Protocol details}: For the protocol $\prot_\BLMap^p(\hat{\mat{K}},\hat{x}\mid \tilde{\mat{K}}, \tilde{x}, \tilde{\mat{K}}\tilde{x})$, each party $\party_i$ computes its share of $\hat{y}$ as:
  \[
    \sharei{\hat{y}} = \sharei{\hat{\mat{K}}\hat{x}} - \hat{\mat{K}}\sharei{\tilde{x}} - \sharei{\tilde{\mat{K}}}\hat{x} + \sharei{\tilde{\mat{K}}\tilde{x}}
  \]

  \noindent \textit{Correctness}. Note that this works since:
  \begin{align*}
  \sum_{\party_i \in \parties} \sharei{\hat{y}} &= \hat{\mat{K}}\hat{x} - \hat{\mat{K}}\tilde{x} - \tilde{\mat{K}}\hat{x} + \tilde{\mat{K}}\tilde{x} \\
  &= (\mat{K} + \tilde{\mat{K}})x - \tilde{\mat{K}}(x + \tilde{x}) + \tilde{\mat{K}}\tilde{x} \\
  &= \mat{K}x
  \end{align*}
\end{itemize}
Since the output of the bilinear gate will usually feed into a conversion gate which requires the input to be already masked, as an optimization, we can have the bilinear gate itself compute shares of the masked output, i.e., $\hat{y} = \mat{K}x + \tilde{y}$. This can be done by providing the correlated randomness $\tilde{\mat{K}}\tilde{x} + \tilde{y}$ instead of $\tilde{\mat{K}}\tilde{x}$. The upshot of this optimization is that one fewer piece of correlated randomness will be required.



\subsection{Composing Gate Protocols}
\label{appendix:protocol_generic}
We now describe a general technique to evaluate circuit composed of the previously specified gates in a distributed fashion. We provide details for the semi-honest fully distributed setting (with preprocessing), where all inputs are secret shared between all parties initially. This can also be thought of as a toolbox for constructing efficient distributed protocols for other constructions similar to ours. While the technique will also work for other settings (e.g., OPRF, public input), the concrete communication costs will be worse than more specially designed protocols. For these settings, we will provide more efficient protocols than provided by this general technique.


% \greg{Does it make sense to say that this very general composition protocol can
% allow one to easily make protocols for variants of our primitives? ``MPC
% toolbox for alternating moduli primitives'' or similar}

\paragraph{Composition protocol.} Consider a circuit $C$ (Definition~\ref{def:computation_circuit}) with input space $\Gin = \prod \Gin_i$. To evaluate $C$ with input $(x_1, \dots, x_l) \in \Gin$, in the fully distributed setting, all parties are given additive shares for each $x_i$. Now, the distributed evaluation of $C$ proceeds as follows:
\begin{itemize}
  
  \item All vertices at the same depth in $C$ are evaluated simultaneously, starting from the source vertices that contain the inputs of the computation. 

  \item The evaluation of a (non-source) vertex in the graph of $C$ is done by each party running the corresponding gate protocol locally on their share of the inputs. 

  \item For an edge $(V_a, V_b)$, suppose that the output of $V_a$ is used as one of the inputs of $V_b$. If the gate protocol corresponding to $\mathcal{G}_V$ requires this input to be masked (e.g., the bilinear gate protocol), then before evaluating $V_b$, each party first masks its share of the output. Now, all parties simultaneously reveal their shares to publicly reveal the masked value.   The masking values are provided to the parties in the preprocessing phase. The same value also need not be masked multiple times if it is required for multiple gates.

  \item The required output shares of the distributed evaluation are given by the evaluation of the sink vertices in the circuit.
\end{itemize}



\paragraph{Communication cost.}
Since the gate protocols themselves are locally computable, the communication cost during a distributed evaluation of a circuit comes solely from the public reconstructions of masked values required for gate protocols. For example, before feeding the output $x$ of a $\LMap^{\mat{A}}_2$ gate into a $\Convert_{(2,3)}$ gate, in the distributed evaluation, all parties will first mask their shares of $x$ to obtain shares of $\hat{x}$. Then, the parties will exchange messages to reconstruct the $\hat{x}$ value required for $\prot_\Convert^{(2,3)}$.

Consider $N$ parties taking part in the distributed evaluation. To reconstruct an $l$-bit value $\hat{x}$ that is additively shared among the parties, one of the following can be done.
\begin{itemize}
  \item Each party sends its share of $\hat{x}$ to each other party. Now, all parties can compute $\hat{x}$ locally. This requires only 1 online round but has a communication cost of $(N-1)l$ bits per party. Each party sends $N-1$ messages. The simplest case for this is when $n=2$, in which case, both parties can simultaneously exchange their shares, and add the two shares locally to reconstruct $\hat{x}$. This requires 1 online round, and has a communication cost of $1$ message and $l$ bits per party. 

  \item All parties can send their share to a designated party, say $\party_1$, who computes $\hat{x}$ and sends it back to everyone. This requires 2 rounds and has a communication cost of $(N-1)l$ bits for $\party_1$ and $l$ bits each for other parties. Here, $\party_1$ sends $N-1$ messages while all other parties send a single message.
\end{itemize}


\paragraph{Reducing round complexity.} It is also straightforward to parallelize the communication to reduce the number of rounds. For this, suppose that we call an edge $(V_a, V_b)$ \textit{communication-requiring} if the output of the protocol for $V_a$ needs to be masked before it is input into the protocol for $V_b$ (in other words, the gate protocol for $V_b$ requires a masked input). Now, define the communication-depth of a vertex $V$ as the maximum number of communication-requiring edges in any path from a source vertex to $V$. Now, instead of evaluating vertices with the same depth simultaneously, we will evaluate vertices with the same \textit{communication-depth} together before the next communication round. By doing so, we can reduce the total number of rounds to the maximum communication-depth.


\paragraph{Preprocessing cost.}
A na{\"i}ve technique to compute the preprocessing required is to add the preprocessing for each gate in the circuit as follows:

\begin{itemize}
\item The $\LMap$ and $\Add$ gates are computed locally and require no preprocessing.

\item Each $\BLMap_p$ gate requires a Beaver-style triple which provides masks for the two inputs and a multiplication of the two masks. Specifically, for $\BLMap_p(\mat{K}, x)$ where $\mat{K} \in \Z_p^{l_2 \times l_1}$ and $x \in \Z_p^{l_1}$, the preprocessed shares are of the form $(\tilde{\mat{K}}, \tilde{x}, \tilde{\mat{K}}\tilde{x})$. Consequently, when $\mat{K}$ is circulant, a total of $(2l_1 + l_2) \log_p$ bits needs to be provided as preprocessing to each party.

\item Each $\Convert_{(2,3)}$ gate requires shares of a random mask $\tilde{x}$ both over $\Z_2$ and $\Z_3$. In other words, for $x \in \Z_2^l$, it requires $l + \log_2{3} \cdot l$ bits of preprocessing per party.

\item Each $\Convert_{(3,2)}$ gate requires shares of a random mask $\tilde{x}$ (over $\Z_3$) as well as $u = \tilde{x}$ and $v = (\tilde{x} + \textbf{1} \bmod 3)\bmod 2$ (both over $\Z_2$). In other words, for $x \in \Z_3^l$, it requires $\log_2{3} \cdot l + 2l$ bits of preprocessing per party.
\end{itemize}

\paragraph{Compressing preprocessing required.}
We describe several optimizations to reduce the total cost of preprocessing. We assume here the presence of a trusted dealer to generate any correlations in the randomness. We note that some of the optimizations, while reducing the size of preprocessing, would be more complicated to generate if a dealer is not present.

\begin{itemize}

  \item (Reducing redundant preprocessing).
  A straightforward optimization is to not mask the same value twice. For example, if the same value $x$ is considered as input for both a $\BLMap$ gate and a $\Convert$ gate, the same mask $\tilde{x}$ can be used for both. 

  \item (Masking $\BLMap$ gate outputs.)
  The standard $\BLMap$ gate requires preprocessing of the form $(\tilde{\mat{K}}, \tilde{x}, \tilde{\mat{K}}\tilde{x})$. However, if the output of the $\BLMap$ gate is then later input to a gate that requires a masked input (e.g., a $\Convert$ gate or even another $\BLMap$ gate), the $\BLMap$ gate can directly mask its output by providing $\tilde{\mat{K}}\tilde{x} + \tilde{y}$ instead. If this is done, the parties will compute a sharing of $\hat{y} = \mat{K}x + \tilde{y}$ using the $\BLMap$ gate. This means that the parties can directly exchange their shares to reconstruct $\hat{y}$ without requiring more preprocessing to mask $y$.

  For both the $\Convert_{2,3}$ and $\Convert_{(3,2)}$ gates, if the masked version $\hat{x}$ of the input $x$ is already known to all parties, only $\log_2{3} \cdot l$ and $2l$ bits respectively of preprocessing are required per party. 

 
  \item (Compression using a PRG).
  Another standard technique for compressing the size of preprocessing is to use a PRG. Intuitively, each party is given a different PRG seed by the trusted dealer which they can use locally to generate their randomness. Only a single party has its shares given by the dealer to ensure that the randomness is appropriately correlated. 

  % For this, the dealer will first generate the randomness as usual. To share the randomness, instead of randomly sampling shares, the dealer (who knows the seeds of all parties) will compute the shares for all but one party (say $\party_1$) using the party's PRG seed. The share for $\party_1$ can now be set so that the shares form an sharing of the required randomness.

  % More specifically, for the $\BLMap$ gate, a dealer first samples a random triple $(\tilde{K}, \tilde{x}, \tilde{K}\tilde{x})$
\end{itemize}

\subsection{Distributed Evaluation Protocols}
\label{appendix:protocols_other_settings}
Here, we provide details for the evaluation of our candidates in other settings.

\subsubsection{Public-input setting}
It is also straightforward to use the same generic technique to construct distributed protocols in the public-input setting. For keyed primitives, in public-input setting, the key is secret shared between the parties but the input is publicly known. The goal of the protocol is for the parties to compute shares of the output.

One useful optimization is that in the public-input setting, a $\BLMap$ gate where the input is known, essentially reduces to a linear gate where the key $\mat{K}$ is secret shared and the input $x$ is publicly known.

\paragraph{2PC public-input protocol for \ttwPRF.}
Concretely, for the evaluation of \ttwPRF in the public-input setting, the first round from the distributed protocol can be entirely skipped. The two parties can directly compute shares of $w = \mat{K}x = \sum_{i \in \{1,2\}} \sharei{\mat{K}}x$ locally. This also means that the preprocessing previously required for the $\BLMap$ gate that computed $\mat{K}x$ is no longer necessary. The rest of the evaluation can now proceed as before with both parties first using $\prot_\Convert^{(2,3)}$ to retrieve shares of $w$ over $\Z_3$, and then using $\prot_\LMap^{\mat{B},3}$ to compute shares of the final output $y$.

In total, the evaluation takes a single round and a communication of $m$ bits per party (to reconstruct $\hat{w}$). The only preprocessing required is for the $\prot_{\Convert}^{(2,3)}$ gate, for which each party will be given $m + \log_2{3} \cdot m$ bits. Furthermore, with PRG compression, $\party_2$ will require no extra preprocessing and $\party_1$ can be given only $\log_2{3} \cdot m$ bits.

\subsubsection{3-party Distributed Evaluation}
\label{appendix:3party_protocol}

In this section, we provide a 3-party (semi-honest) protocol for computing the \ttwPRF candidate that is secure against one passive corruption and does not require any preprocessing.

\paragraph{Functionality.}
Denote the servers by $\party_1, \party_2$, and $\party_3$. We assume that the servers hold replicated additive shares of the key $\mat{K}$ and the input $x$. The key is assumed to be circulant and can be represented by $k \in \Z_2^n$. Concretely, let $(k_1, k_2, k_3)$ and $(x_1,x_2,x_3)$ be additive shares of $k$ and $x$ respectively. Then, each party $\party_i$ is given $k_j$, $x_j$ with $j \neq i$. At the end of the protocol, $\party_2$ and $\party_3$ should hold $y_2$ and $y_3$ respectively such that $(y_2, y_3)$ is a sharing of the wPRF output $y$.

\paragraph{Protocol details.}
\begin{itemize}
  \item First, the three servers compute additive shares of the linear mapping $\mat{K}{x}$ locally using their replicated shares. Note that this can be locally since for two secret shared values $a = a_1 + a_2 + a_3$ and $b = b_1 + b_2 + b_3$, their product can be computed as $ab = \sum_{1 \leq j,k \leq 3} a_jb_k$. Since each party holds two shares of $a,b$ in a replicated sharing scheme, each term $a_jb_k$ can be computed by at least 1 party. Suppose that the share of $\party_i$ is denoted by $\sharei{\mat{K}x}$.

  \item Now, $\party_1$ samples $\tilde{w} \getsr \Z_2^m$, $r_2 \gets \Z_3^m$, and sets $r_3 = \tilde{w} - r_2 \bmod 3$. In other words $(r_2, r_3)$ is a random $\Z_3$ sharing of $r = \tilde{w}$. $\party_1$ sends $\share{\mat{K}x}^{(1)} + \tilde{w}$ and $r_i$ to $\party_{i \in \{2,3\}}$. At the same time, $\party_2$ and $\party_3$ exchange their shares of $\mat{K}x$. All of this can be done in one round.

  \item At this point, $\party_2$ and $\party_3$ can both compute $\hat{w} = \tilde{w} + \sum_{1\leq i \leq 3} \sharei{\mat{K}x}$. Now, for $i \in \{2,3\}$, $\party_i$ can locally compute $w^*_i \gets \prot_\Convert^{(2,3)}(\hat{w} \mid r_i)$. Finally, $\party_i$ can locally compute its share $y_i$ of the output by running $\prot_\LMap^{\mat{B},3}(\mat{B} \mid w^*_i)$.
\end{itemize}


\paragraph{Cost analysis.}
The protocol requires only 1 round, with two messages sent by $\party_1$ and one message each sent by $\party_2$ and $\party_3$. $\party_1$ sends a total of $2m + 2 \log_2(3)\cdot m$ bits while the other two parties send $m$ bits each. $\party_1$ can also generate a reusable PRG seed and provide it to one of the parties which saves $\log_2(3)\cdot m$ bits of communication. 


\subsection{Oblivious Protocols}
\label{appendix:oprf_protocol}
Our distributed evaluation protocols from Section~\ref{subsec:distributed_protocol} can be used directly for semi-honest \textit{oblivious} PRF, or OPRF, evaluation in the preprocessing model. Recall that in the OPRF setting, one party $\party_1$ (called the ``server'') holds the key $\mat{K}$ and the other party $\party_2$ (called the ``client'') holds the input $x$. The goal of the protocol is to have the client learn the output of the PRF for key $\mat{K}$ and input $x$, while the server learns nothing. In the semi-honest setting, both parties can first use the distributed protocol to obtain shares of the PRF output. The server can then send its share to the client so that only the client learns the final output. Such an OPRF protocol would require one extra round over the corresponding distributed PRF protocol. We can however construct much better protocols whose efficiency rivals that of existing DDH-based OPRF protocols. Here, we provide two concrete efficient protocols for evaluating the \ttwPRF candidate (Construction~\ref{construction:23-central-wprf}) in the OPRF setting. 

\paragraph{General structure.} Both protocols take $3$ rounds and involve 2 messages from the server to the client and 1 message from the client to the server. The first server message however, is only required when the key needs to be changed (or re-masked). We call this the key-update phase. Now, when the masked key is already known, our protocols are optimal in the sense that they require only a single message from the client followed by a single message from the server. Since OPRF applications usually involve reusing the same key for many PRF invocations, in such a \textit{multi-input} setting, our protocols are comparable to other 2-round OPRF protocols in literature (e.g., DDH-based). 
\greg{For a real-world use of OPRFs where our construction might be the best option, see \url{https://www.usenix.org/system/files/sec19-thomas.pdf}. 
It uses the DDH-based OPRF for a private set intersection protocol (where the client has one element and the server has a masssive set). }

We detail the two protocols, $\prot^{\textsf{oprf}}_1$ and $\prot^{\textsf{oprf}}_2$, in Sections~\ref{subsec:oprf1} and~\ref{subsec:oprf2} respectively. In Section~\ref{subsec:oprf-comparison}, we compare our OPRF protocols to other common constructions in the literature. Later, in Section~\ref{sec:implementation_and_eval}, we also report on our protocol implementations and compare their performance (both computation and communication) with related work. To foreshadow, a key observation is that in comparison to common OPRF protocols, our protocols are much faster to compute but require preprocessing as well as slightly more communication.


\subsubsection{Oblivious PRF Protocol $\prot^{\textsf{oprf}}_1$}
\label{appendix:oprf1}
Our first OPRF protocol is in spirit similar to the distributed evaluation for the \ttwPRF construction. Since $\mat{K}$ is known to the server, and $x$ is known to the client, both parties do not need to exchange their shares to reconstruct the masked values $\hat{\mat{K}}$ and $\hat{x}$; the party that holds a value can mask it locally and send it to the other party. This allows us to decouple the server's message that masks its PRF key from the rest of the evaluation. To update the key, the server can simply send $\hat{\mat{K}} = \mat{K} + \tilde{\mat{K}}$ to the client. Many PRF evaluations can now be done using the same $\hat{\mat{K}}$.

\paragraph{Preprocessing.} The protocol requires the following preprocessed randomness. The mask $\tilde{\mat{K}}$ is given to the server only when the key-update phase needs to be run. For PRF evaluations, the trusted dealer samples $\tilde{w} \getsr \Z_2^m$ and provides the server and client $\Z_2$ shares of $\tilde{w}$ along with $\Z_3$ shares of $r = \tilde{w}$. Additionally, the dealer also generates an OLE correlation pair $(\tilde{\mat{K}}, \tilde{v})$ and $(\tilde{x}, \hat{v})$ such that $\tilde{\mat{K}} \in \Z_2^{m \times n}$ is a random circulant matrix that is same for all correlations, $\tilde{v} \getsr \Z_2^m$, $\tilde{x} \getsr \Z_2^n$, and $\hat{v} = \tilde{\mat{K}}\tilde{x} + \tilde{v}$. The server is given $(\tilde{\mat{K}}, \tilde{v})$ while the client is given $(\tilde{x}, \hat{v})$. Note that we simply use OLE correlations and do not make use of an actual OLE protocol. In practice, if the key-update phase is run after every $k$ evaluations (where $k$ is known), the OLE correlations for all evaluations can be preprocessed at the beginning. 
% \mahimna{need to write about / point to another section for generating the OLE correlations.}

\paragraph{Protocol details.} Assuming that the masked key $\hat{\mat{K}}$ is known to the client, for an input $x$,  the evaluation protocol now proceeds as follows:
\begin{itemize}
  \item The client computes $\hat{x} = x + \tilde{x}$ and $\share{\hat{w}}^{(2)} = -\mat{\hat{K}}\tilde{x} + \hat{v} + \share{\tilde{w}}^{(2)}$ and sends both $\hat{x}$ and $\share{\hat{w}}^{(2)}$ to the server.

  \item The server first computes $\share{\hat{w}}^{(1)} = \mat{K}\hat{x} - \tilde{v} + \share{\tilde{w}}^{(1)}$, and adds to it the client's share to reconstruct $\hat{w}$. Identical to the distributed protocol, the server now runs $\prot_\Convert^{(2,3)}$ followed by $\prot_\LMap^{\mat{B},3}$ to obtain its share $\share{y}^{(1)}$ of the PRF output. Finally, it sends both $\hat{w}$ and $\share{y}^{(1)}$ to the client.

  \item The client also runs $\prot_\Convert^{(2,3)}$ followed by $\prot_\LMap^{\mat{B},3}$ to obtain its share $\share{y}^{(2)}$ of the PRF output. It can now use the server's share to reconstruct the PRF output $y$.

  % It now runs $\prot_\Convert^{(2,3)}(\hat{w} \mid \share{r}^{(1)})$ to get $\share{w^*}^{(1)}$ (over $\Z_3$) and then $\prot_\LMap^{\mat{B},3}(\mat{B} \mid \share{w^*}^{(1)})$ to obtain its share $\share{y}^{(1)}$ of the PRF output. Finally, it sends both $\hat{w}$ and $\share{y}^{(1)}$ to the client.

  % \item The client now runs $\prot_\Convert^{(2,3)}(\hat{w} \mid \share{r}^{(2)})$ to get $\share{w^*}^{(2)}$ (over $\Z_3$) and then $\prot_\LMap^{\mat{B},3}(\mat{B} \mid \share{w^*}^{(2)})$ to obtain $\share{y}^{(2)}$. Finally, it uses the share received from the server to reconstruct the PRF output $y$ 
\end{itemize}
For evaluating a client input, $\prot^{\textsf{oprf}}_1$ takes 2 rounds and involves a single message in each direction. The client sends $2n$ bits while the server sends $m$ bits and $t$ $\Z_3$ elements. For our parameters ($n=m=256, t=81$), and with proper $\Z_3$ packing, this amounts to roughly $897$ bits of total online communication. To update $\tilde{\mat{K}}$, the server sends a 256-bit message to the client.


\subsubsection{Oblivious PRF Protocol $\prot^{\textsf{oprf}}_2$}
\label{appendix:oprf2}
For the second protocol, the server masks the PRF in a different way; a multiplicative mask is used instead of an additive one. For simplicity, suppose that $n=m$ and that the key $\mat{K}$ is a random full-rank circulant matrix. Then to mask $\mat{K}$, the server computes $\bar{\mat{K}} = \mat{R}\mat{K}$ using a random matrix $\mat{R}$ that is also full-rank and circulant. $\mat{R}$ will be provided as preprocessing to the server, and can be reused for multiple PRF evaluations. The server will send $\bar{\mat{K}}$ to the client in the key-update phase. Note that since the product of two circulant matrices is also circulant, this message is only $n$ bits. Additionally, since the product $\mat{R}\mat{K}$ is essentially a convolution, it can be efficiently computed in $\Theta(n\log n)$ asymptotic runtime using the fast Fourier transform (FFT) algorithm.

\paragraph{Preprocessing.} The protocol requires the following preprocessed randomness. The mask $\mat{R}$ is given to the server only when the key-update phase needs to be run. For PRF evaluations, similar to first protocol, the dealer samples $w \getsr \Z_2^m$ and provides the server and client $\Z_2$ shares of $\tilde{w}$ along with $\Z_3$ shares of $r = \tilde{w}$. Additionally, the dealer gives $\tilde{u} \getsr \Z_2^m$ to the client and $\tilde{v} = \mat{R}^{-1}\tilde{u} + \tilde{w}$ to the server.

\paragraph{Protocol details.} Now, assuming that the masked key $\bar{\mat{K}}$ is known to the client, for an input $x$,  the evaluation protocol now proceeds as follows:
\begin{itemize}
  \item The client computes $\hat{u} = \bar{\mat{K}}x + \tilde{u}$ and sends it to the server.

  \item The server first computes $\mat{R}^{-1}\hat{u} + \tilde{v} = \mat{R}^{-1}(\mat{R}\mat{K}x + \tilde{u}) + (\mat{R}^{-1}\tilde{u} + \tilde{w}) = \hat{w} \mod 2$. Identical to the distributed protocol, the server now runs $\prot_\Convert^{(2,3)}$ followed by $\prot_\LMap^{\mat{B},3}$ to obtain its share $\share{y}^{(1)}$ of the PRF output. Finally, it sends both $\hat{w}$ and $\share{y}^{(1)}$ to the client.

  \item The client also runs $\prot_\Convert^{(2,3)}$ followed by $\prot_\LMap^{\mat{B},3}$ to obtain its share $\share{y}^{(2)}$ of the PRF output. It can now use the server's share to reconstruct the PRF output $y$.
\end{itemize}
For evaluating a client input, $\prot^{\textsf{oprf}}_2$ also takes 2 rounds and involves a single message in each direction. The client sends $n$ bits while the server sends $m$ bits and $t$ $\Z_3$ elements. This is $n$ fewer bits of communication as compared to the first protocol. The key-update phase is slower however, since it involves a convolution rather than a simple vector addition. For our parameters ($n=m=256, t=81$), and with proper $\Z_3$ packing, this amounts to roughly $641$ bits of total online communication. To update $\bar{\mat{K}}$, the server sends a 256-bit message to the client. 


\subsubsection{Comparison to other OPRF protocols}
\label{appendix:oprf-comparison}
Here, we compare the concrete efficiency of our protocols with other protocols in literature.
\paragraph{DDH-based OPRFs.}
A simple and widely used OPRF is based on the Decision Diffie-Hellman (DDH) assumption. Abstractly, given a cyclic group $\mathbb{G}$ of prime order $q$, consider a PRF $\mathsf{F}$ defined as follows: For key $k \in \keyspace$ and input $x \in \inspace$, define $\mathsf{F}(k,x) = H(x)^k$ where $H:\inspace \to \mathbb{G}$ is a hash function modeled as a random oracle. $\mathsf{F}$ is a secure PRF under the DDH assumption in the random oracle model~\cite{naor1999-oprf}.

The PRF $\mathsf{F}$ leads to a natural 2-party (semi-honest) protocol for oblivious evaluation. Concretely, suppose that the client holds $x$ and the server holds $k$. To evaluate the PRF obliviously, the client initiates the interaction by first sampling a mask $r \getsr \mathbb{Z}_q$ and then sending $a \gets H(x)^r$ to the server. The server responds with $b \gets a^k$. Finally, the client can retrieve the PRF computation as $y \gets b^{r^{-1}}$ where $r^{-1}$ is the multiplicative inverse of $r$ in $\Z_q$. Security of this OPRF protocol has been shown in~\cite{jarecki2014-ddhoprf,jarecki2016-ddhoprf}, assuming the one-more discrete-log assumption~\cite{bellare2003-onemore} (and in the random oracle model).

We instantiate a DDH-based OPRF over the Curve25519 elliptic curve and compare its efficiency to our OPRF constructions. The key takeaway we found was that for both our constructions, the total computation time is smaller than the time it takes for a \textit{single} elliptic curve scalar multiplication. One caveat is that our protocols require preprocessing, as well as slightly higher communication. However, we show that for reasonable network speeds, the overall cost of our protocol is still smaller. We provide more details of this comparison as part of our evaluation in Section~\ref{subsec:performance}.


\paragraph{Other OPRFs.}
Many prior OPRF constructions~\cite{freedman2005-oprf,jarecki2009-oprf} require expensive exponentiations because they are based on algebraic PRFs. This means that we can expect similar performance tradeoffs for our protocols when compared to them (i.e., much faster computation, slightly more communication).~\cite{kolesnikov2016-oprf} provides an efficient batched-OPRF protocol based on OT extension in the preprocessing. While we did not perform with the same setup, based on their performance results, we found that our protocol is substantially more efficient for a single (or a small number of) evaluation, but becomes more comparable in performance as the batch-size increases which is unsurprising considering our protocols are not optimized for the batched evaluation. Recent work~\cite{seres2021-legendre} constructs an OPRF protocol from the Legendre PRF~\cite{damgard1988-legendre}. For 128-bit security, their protocol has a communication cost of $\approx 13$KB which is substantially higher than ours.




%-------------------------------------------------%




\section{Signature Scheme Details}
\label{appendix:picnic}
In this section we provide additional details on how to construct a signature scheme from our
new primitives, in particular the \ttOWF.


\paragraph{An $N$-party protocol.}
There will be $N$ parties for our MPC protocol, each holding a secret share of
$x$, who jointly compute $y = \sfF(x)$.  The protocol tolerates up to  $N-1$ 
corruptions: given the views of $N-1$ parties we can simulate the remaining
party's view, to prove that the $N-1$ parties have no information about the
remaining party's share. 

The preprocessing phase is similar to that in Picnic.  Each party has a random
tape that they can use to sample a secret sharing of a uniformly random value
(e.g.,  a scalar, vector, or a matrix with terms in $\Z_2$ or $\Z_3$).  Each
party samples their share $\share{r}$ and the shared value is implicitly defined as
$r = \sum_{i=1}^N \share{r}^{(i)}$.  
%We use $\share{r}^{(i)}$ to denote party $i$'s share of $r$,
%or simply $\share{r}$ when the context makes the party's index clear.

We must also be able to create a sharing mod 3, of a secret shared value mod 2.
Let $\tilde{w}\in\Z_2$ be secret shared.  Then to establish shares of $r = \tilde{w} \pmod
3$, the first $N-1$ parties sample a share $\share{r}$ from their random tapes. The
$N$-th party's share is chosen by the prover, so that the sum of the shares is
$r$.  We refer to the last party's share as an \emph{auxiliary value}, since
it's provided by the prover as part of pre-processing.  For efficiency, the random
tape for  party $i$ is
generated by a random seed, denoted $\seed_i$, using a PRG. The state of the first
$N-1$ parties after pre-processing is a seed value used to generate the random
tape, and for the $N$-th party the state is the seed value plus the list of
auxiliary values, denoted $\aux$. 

After pre-processing, the parties enter the \emph{online} phase of the protocol. 
The prover computes $\hat{x}= x + \tilde{x}$, where $\tilde{x}$ is a random value, established during
preprocessing so that each party has a share $\share{\tilde{x}}$. 
The parties can then compute the OWF using the homomorphic properties of the secret sharing, 
and the share conversion gate (to convert shares mod 2 to mod 3, used when computing~$z$)
setup during preprocessing that we describe below. 
During the online phase, parties broadcast values to all other parties and we write $\msgs_i$
to denote the broadcast messages of party $i$. 


\paragraph{Preprocessing phase.} Preprocessing establishes random seeds of all parties and shares of 
\begin{enumerate}
\item $\tilde{x}$: a random vector in $\Z_2^{n}$,\algcomment{Sampled from random tapes}
\item $\tilde{w}$: the vector $\matA \tilde{x}$ in $\Z_2^{m}$, 
\item $r$: a sharing of $\tilde{w} \mod 3$, shares in $\Z_3^m$,  \algcomment{Tapes + one $\aux$ value}
\item $\overline{r}$: a sharing of $1-\tilde{w} \mod 3$, shares in $\Z_3^m$. \algcomment{Computed from $\share{r}$}
\end{enumerate}
The shares of $\overline{r}$ are computed from shares of $r$ as follows (all arithmetic in $\Z_3^m$): the first
party computes $\share{\overline{r}} = 1 - \share{r}$, then the remaining parties compute
$\share{\overline{r}} = - \share{r}$.  Then observe that 
\[\sum_{i=1}^N \share{\overline{r}}^{(i)} = 1 - \share{r}^{(1)} - \ldots - \share{r}^{(N)} = 1 - \sum_{i=1}^{N} \share{r}^{(i)} = 1-r \]
 as required. 

\paragraph{Online phase.}
The public input to the online phase is $\hat{x} = x + \tilde{x}$. 
\begin{enumerate}

\item The parties locally compute $\hat{w} \in \Z_2^{m}$ as $\hat{w} = \matA\hat{x}$ (since both $\hat{x}$ and $\matA$ are public). 

\item Let $z$ be a vector in $\Z_3^m$ and let $z_i$ denote the $i$-th component. Each party defines 
\[
    \share{z_i}  = \begin{cases}
                \share{r_i}  & \text{if $\hat{w}_i = 0$} \algcomment{\text{Note that $\share{r_i} = \share{w_i'}$}}\\
                \share{\overline{r}_i}  & \text{if $\hat{w}_i = 1$} \algcomment{\text{Note that $\share{\overline{r}_i} = \share{1- w_i'}$}}\\
            \end{cases}
\]
then localy computes $\share{y} = \matB\share{z}$. All parties broadcast $\share{y}$ and reconstruct the output $y\in\Z_{3}^t$. 
In this step each party broadcasts $t$ values in $\Z_3$.
\end{enumerate}

\paragraph{Correctness.} The protocol correctly computes the \ttOWF.  The
first step computes $w = \matA x$, 
updating the public value $\hat{x} = x + x'$ with $\hat{w} = w + \tilde{w}$.  The
second step is where the bits of $w$ are cast from $\Z_2$ to $\Z_3$.  The 
parties have sharings of $\tilde{w}$ and $1-\tilde{w}$ mod 3 (we focus on a single bit here, for simplicity). The key observation is
that when $\hat{w} = 0$, then $w$ and $\tilde{w}$ are the same, and when $\hat{w} =
1$, $w$ and $\tilde{w}$ are different. So in the first case we set the shares of $ z =
w \mod 3$ to the shares of $[\tilde{w}] \mod 3$, and when $\hat{w} = 1$, we set the
shares of $z$ to the complement of $\tilde{w}$.

\paragraph{Communication costs.}
Here we quantify the cost of communication for the MPC inputs, the $\aux$ values and the broadcast $\msgs$ of one party,
as this will directly contribute to the signature size in the following section. 
Let $\ell_3$ be the bitlength of an element in $\Z_3$; the direct encoding has
$\ell_3 = 2$, but with compression we can reudce $\ell_3$ to as little as
$\log_2(3) \approx 1.58$. \footnote{To compress a vector $v\in\Z_3^{n}$, convert it to the integer it represents: $V = \sum_{i=0}^n v_i^{i}$
and output the binary representation of $V$. }   The size of the
$\aux$ information is $m\ell_3$, the MPC input value has size $n$ bits, 
and the broadcast values have size $t\ell_3$ bits (per party). 
The total in bits is thus 
\begin{equation} \label{eqn:sizeMPC}
|\textsf{MPC}(n,m,t)| = m\ell_3 + n + t\ell_3\;.
\end{equation}
For the parameters 
$(n,m,t)=(128, 453, 81)$ the total is 972 bits (L1 security: 128-bit classical, 64-bit quantum)
and when $(n,m,t) = (256, 906, 162)$ the total is 1943 bits (L5 security: 256-bit classical, 128-bit quantum).  This compares favorably to
Picnic at the same security level, which communicates 1161--1328 bits at L1
and 2295--2536 bits at L5, depending on whether LowMC uses a full or partial S-box layer~\cite{kales2020-picnic}.

\paragraph{Signature scheme}
Given the MPC protocol above, we can compute the values $\hat{x}$, $\aux$ and
$\msgs$ for the \ttOWF and neatly drop it into the KKW proof system used in
Picnic.  The signature generation and verification algorithms for the
\ttOWF  signature scheme are given in \cref{fig:23-picnic}.


We use a cryptographic hash function $\hash: \{0,1\}^*\to \{0,1\}^{2\secpar}$
for computing commitments, and the function $\Expand$ takes as input a random $2\secpar$-bit string
and derives a challenge having the form $(\CCC, \PPP)$ where $\CCC$ is a subset of $[M]$
of size $\tau$, and $\PPP$ is a list of length $\tau$, with entries in $[N]$. 
The challenge $(\CCC, \PPP)$ defines $\tau$ pairs $(c,p_c)$ where $c$ is the
index of  an MPC instance for which the verifier will check the online phase,
and $p_c$ is the index of the party that will remain unopened. 


\paragraph{Optimizations and simplifications.}
For ease of presentation, \cref{fig:23-picnic} omits some optimizations that
are essential for efficiency, but are not unique to the $(2,3)$-signature schemes, 
they are exactly as in Picnic. All random seeds in a signature are derived from
a single random root seed, using a binary tree construction. First we derive
$M$ initial seeds, once for each MPC instance, then from from the initial seed
we derive the $N$ per-party seeds. This allows the signer to reveal the seeds
of $N-1$ parties by revealing only $\log_2(N)$ intermediate seeds, similarly, 
the initial seeds for $M-\tau$ of $M$ instances may be revealed by communicating
only $(\tau)\log_2(M/\tau)$ $\secpar$-bit seeds.

For the commitments $h'^{(k)}$ to the online execution, $\tau$ are recomputed
by the verifier, and the prover provides the missing $M-\tau$.  Here we compute
the $h'^{(k)}$ as the leaves of a Merkle tree,  so that the prover can provide
the missing commitments by sending only $\tau\log_2(M/\tau)$ $2\secpar$-bit
digests. 

Finally, \cref{fig:23-picnic} omits a random salt, included in each signature, as well as counter
inputs to the hash functions to prevent multi-target
attacks~\cite{dinur2019-picnic-attacks}. Also, hashing the public key when computing the
challenge, and prefixing the inputs to $\hash$ in each use for domain
separation should also be done, as in~\cite{picnic-spec}. 

\paragraph{Signature size}
The size of the signature in bits is:
\begin{align*}
\textunderbrace{\secpar\tau\log_2\left(\frac{M}{\tau}\right)}{initial seeds} + 
\textunderbrace{2\secpar\tau\log_2\left(\frac{M}{\tau}\right)}{Merkle tree commitments} + 
\tau\left( 
     \textunderbrace{\secpar\log_2N}{per-party seeds}  + 
     \textunderbrace{|\textsf{MPC}(n,m,t)|}{one MPC instance, \cref{eqn:sizeMPC}} 
\right)
\end{align*} 
and we note that the direct contribution of OWF choice is limited to
$|\textsf{MPC}(n,m,t)|$\footnote{The size $|\textsf{MPC}(n,m,t)|$ is a slight
overestimate since for $1/N$ instances we don't have to send $\aux$, if the
last party is unopened. In \cref{table:sig-sizes} our estimates include this,
but it's a very small difference as $\tau$ is quite small. }.  However, the
size of this term can impact the choice of $(N,M,\tau)$.  The Picnic parameters
$(N, M, \tau)$ must be chosen so that the soundness error, 
\begin{equation*} \label{eqn:soundness}
    \epsilon(N,M,\tau) = \max_{M-\tau \le k \le M} \left\lbrace  \frac{\binom{k}{M-\tau} }{\binom{M}{M-\tau} N^{k-M+\tau} } \right\rbrace\,.
\end{equation*}
is less than $2^{-\secpar}$.
By searching the parameter space for fixed $N$ and various options for $M,
\tau$, we get a curve, and choose from the combinations in the ``sweet spot'',
near the bend of the curve with moderate computation costs. This part of the curve is
similar as in Picnic, and we present some options from it in \cref{table:picnic}.

\begin{figure}[p]
 \begin{minipage}[t]{1.1\textwidth}
 \begin{protocolbox}{\ttOWF Signatures}
 \begin{description}
    \item[Inputs] Both signer and verifier have $\sfF$, $y = \sfF(x)$, the
        message to be signed $\Msg$, and the signer has the secret key $x$.  The
            parameters of the protocol $(M, N, \tau)$ are described in the text.
    \item[Commit] For each MPC instance $k\in[M]$, the signer does the following.
    \begin{enumerate}
        \item Choose uniform $\seed^{(k)}$ and use  to generate values $(\seed^{(k)}_i)_{i \in [N]}$, and compute  
        $\aux^{(k)}$ as described in the text. 
        For $i=1,\ldots N-1$, let $\state^{(k)}_i = \seed^{(k)}_i$ and  let $\state^{(k)}_{N} = \seed^{(k)}_{N} || \aux^{(k)}$.
        \item Commit to the preprocessing phase:
        \begin{align*}
        \com^{(k)}_{i} = \hash(\state^{(k)}_i) \text{ for all } i\in [N], \quad 
        h^{(k)} = \hash(\com^{(k)}_{1},\ldots,\com^{(k)}_{N}).
        \end{align*}						
        \item Compute MPC input $\hat{x}^{(k)} = x + \tilde{x}^{(k)}$ based on the secret key $x$ and the random values $\tilde{x}^{(k)}$ defined by preprocessing.
        \item Simulate the online phase of the MPC protocol, producing $(\msgs^{(k)}_i)_{i \in [N]}$.			
        \item Commit to the online phase:
        $
         h'^{(k)}= \hash(\hat{x}^{(k)}, \msgs_{1}^{(k)}, \ldots, \msgs_{N}^{(k)} ).
        $
    \end{enumerate}
    
    \item[Challenge] 
    The signer computes $\ch = \hash(h_1, \ldots h_M, h_1', \ldots, h_M',
    \Msg)$, then expands $\ch$ to the challenge $(\CCC, \PPP) := \Expand(\ch)$, as described in the text. 
    
    \item[Signature output]
    The signature $\sigma$ on $\Msg$ is 
    \[
    \sigma = (\ch, 
              ((\seed^{(k)}, h^{(k)} )_{k\not\in\CCC}, 
              ( \com^{(k)}_{p_k}, (\state^{(k)}_{i})_{i\neq p_k}, \hat{x}^{(k)}, \msgs_{p_k}^{(k)} )_{k\in\CCC})_{k\in[M]})
    \]
    
    \item[Verification] The verifier parses $\sigma$ as above, and does the following.  
    \begin{enumerate}
        \item Check the preprocessing phase. For each $k\in[M]$:
        \begin{enumerate}
        \item If $k \in \CCC$: for all $i\in[N]$  such that $i \neq p_k$, the verifier uses $\state^{(k)}_{i}$ to compute $\com^{(k)}_{i}$ as 
            the signer did, then computes $h'^{(k)} = \hash(\com^{(k)}_{1},\ldots,\com^{(k)}_{N})$ using 
            the value $\com^{(k)}_{p_k}$ from $\sigma$. 
        \item If $k\not\in\CCC$: the verifier uses $\seed^{(k)}$ to compute $h'^{(k)}$ as the signer did.
        \end{enumerate} 
        
        \item Check the online phase:
        \begin{enumerate}
            \item For each $k \in \CCC$ the verifier simulates the online phase using $(\state^{(k)}_{i})_{i \neq p_k}$,  
                masked witness $\hat{x}$ and $\msgs^{(k)}_{p_k}$ to compute $(\msgs_i)_{i \neq p_k}$. 
                Then compute $h^{(k)}$ as the signer did. The verifier outputs `invalid' if the output of the MPC simulation is not equal to $y$.
        \end{enumerate}
    \item The verifier computes $\ch' = \hash(h_1, \ldots h_M, h_1', \ldots, h_M', \Msg)$ and outputs `valid' if $\ch' = \ch$ and `invalid' otherwise. 
    \end{enumerate}
 \end{description}
 \end{protocolbox}
 \end{minipage}
	\vspace*{-10pt}
	\caption{\label{fig:23-picnic}Picnic-like signature scheme using the \ttOWF and the KKW proof sytem.} 
\end{figure}


%--------------------------%
%!TEX root = ../main.tex
\section{Deferred Implementation Details}
\label{appendix:implementation}
Here, we provide the details for our optimization techniques.
\paragraph{Bit packing for $\Z_2$ vectors.} 
Instead of representing each element in a $\Z_2$ vector separately, we pack several elements into a machine word and operate on them together in an SIMD manner. For our architecture with 64-bit machine words, we can pack a vector in $\Z_2^{256}$ (e.g., the input $x$) into 4 words. Since the key $\mat{K}$ is circulant and can be represented with $n=256$ bits, it can also be represented by 4 words. This results in a theoretical $\times64$ maximum speedup in run-time for operations involving $\mat{K}$ and $x$. 

\paragraph{Bit slicing for $\Z_3$ vectors.}
We represent each element in $\Z_3$ using the two bits from its binary representation. For $z \in \Z_3$, the two bits are the least significant bit (LSB) $l_z = z \bmod 2$, and the most significant bit (MSB) $h_z$ which is $1$ if $z=2$ and $0$ otherwise. $\Z_3$ vectors are now also represented by two binary vectors, one containing the MSBs, and one containing the LSBs. Operations involving a $\Z_3$ vector are translated to operations on these binary vectors instead. We also take advantage of the bit packing optimization when operating on the binary vectors.
In Table~\ref{table:z3_operations}, we specify how we perform common operations on $\Z_3$ elements using our bit slicing approach. 

\begin{table}[ht]
\begin{center}
    \begin{tabular}{|c|c|c|}
        \hline
        Operation & Result MSB & Result LSB\\
        \hline\hline
        $z_1 + z_2 \bmod 3$ & $(l_1 \vee l_2) \oplus (l_1 \vee h_2) \oplus (l_2 \vee h_1)$  & $(h_1 \vee h_2) \oplus (l_1 \vee h_2) \oplus (l_2 \vee h_1)$\\
        $-z_1 \bmod 3$ & $l_1$ & $h_1$ \\
        $z_1z_2 \bmod 3$ & $(l_1 \wedge l_2) \oplus (h_1 \wedge h_2)$ & $(l_1 \wedge h_2) \oplus (h_1 \wedge l_2)$\\
        $\textrm{MUX}(s;z_1,z_2)$ & $(h_2 \wedge s) \vee (h_1 \wedge \neg s)$ & $(l_2 \wedge s) \vee (l_1 \wedge \neg s)$\\
        \hline
    \end{tabular}
\end{center}
\caption{Operations in $\Z_3$. $z_1$ and $z_2$ are elements in $\Z_3$ with (MSB, LSB) = $(h_1,l_1)$ and $(h_2, l_2)$ respectively. For a bit $s$, the operation $\textrm{MUX}(s;z_1, z_2)$ outputs $z_1$ when $s=0$ and $z_2$ when $s=1$.}
\label{table:z3_operations}
\mahimna{I'm debating whether we need to have this table at all. Once we specify that we split $\Z_3$ elements into 2 bits, for any $\Z_3$ operation, the corresponding operation on the bits can easily be computed using a truth table.}
\end{table}



\paragraph{Lookup table for matrix multiplication.} Recall that the \ttwPRF evaluation contains a mod-3 linear map using a public matrix $\mat{B} \in \Z_3^{81\times256}$. Specifically, it computes the matrix-vector product $\mat{B}w$ where $w \in \Z_3^{256}$. Since $\mat{B}$ is known prior to evaluation, we can use a lookup table to speedup the multiplication by $\mat{B}$\cite{arlazarov1970economical}. The same preprocessing can also by reused for multiple evaluations of the wPRF.

For this, we partition $\mat{B}$, which has $m=256$ columns, into $16$ slices of $16$ columns each. These matrices, denoted by $\mat{B}_1,\dots, \mat{B}_{16}$, are all in $\Z_3^{81 \times 16}$. Now, for each $\mat{B}_i$, we will effectively build a lookup table for its multiplication with any $\Z_3$ vector of length $16$. A point to note here is that since we  represent $\Z_3$ vectors by two binary vectors (from the bit slicing optimization), it is sufficient to preprocess multiplications (modulo 3) for binary vectors of length $16$. To multiply $\mat{B}_{i}$ by a vector in $\Z_3^{16}$, we can first multiply it separately by the corresponding MSB and LSB vectors, and then subtract the former from the latter modulo 3. This works since for $z_1, z_2 \in \Z_3$ the multiplication $z_1z_2 \bmod 3$ can be given by $z_1(2\cdot h_2 + l_2) \bmod 3 = z_1l_2 - z_1h_2 \bmod 3$ where $h_2, l_2$ are the MSB and LSB of $z_2$ respectively. Now, to multiply $\mat{B}$ by $v \in \Z_3^{256}$, we first evaluate all multiplications of the form $\mat{B}_iv_i$ where $v_i$ is the $\Z_3^{16}$ vector denoting the $i^\thtext$ slice of $v$ if it was split into 16-element chunks. Then, multiplication by $\mat{B}$ is given by $\mat{B}v = \sum_{i \in [16]} \mat{B}_i v_i \mod 3$.
\greg{Is this the ``method of four Russians''?\url{https://github.com/malb/m4ri}, \url{https://bitbucket.org/malb/m4ri/wiki/Further\%20Reading} }
\greg{$w$ is secret, so the access pattern of the $\mat{B}_i$ should be independent of $w$. Is this the case? If not there is a 
cache timing side-channel. The timing variation may be observable over the network (e.g., to the other party in the OPRF evaluation).  }

In general, a $\mat{B} \in \Z_3^{t \times m}$ can be partitioned into $m/c$ partitions with $c$ columns each (assume $c$ divides $m$ for simplicity), and would require a total multiplication lookup table size of $(m/c) \cdot 2^c \cdot t$ $\Z_3$ elements. Multiplying $\mat{B}$ by $v \in \Z_3^{m}$ requires $2(m/c)$ lookup table accesses (one each for MSB and LSB of $v$ per partition of $\mat{B}$) and an addition of $(m/c - 1)$ $\Z_3^{t}$ vectors.


For our parameters, this results in a table size of roughly $135$MB with proper $\Z_3$ packing. We chose $c = 16$ as a compromise between the size of the lookup table and increased computational efficiency.
