%!TEX root = ../main.tex

\section{Preliminaries}
\label{sec:preliminaries}

\mypara{Notation.} We start with some basic notation.
For a positive integer $k$, $[k]$ denotes the set $\{1,\dots, k\}$. $\Z_p$ denotes the ring of integers modulo $p$. We use bold uppercase letters (e.g., $\mat{A}, \mat{K}$) to denote matrices. We use $\textbf{0}^l$ and $\textbf{1}^l$ to denote the all zeros and the all ones vector respectively (of length $l$), and drop $l$ when sufficiently clear. For a vector $x$, by $x \bmod p$, we mean that each element in $x$ is taken modulo $p$. We use $x \getsr \mathcal{X}$ to denote sampling uniformly at random a set $\mathcal{X}$. $\Funcs[\inspace, \outspace]$ denotes the set of all functions from $\inspace$ to $\outspace$. $a \parallel b$ denotes concatenating the strings $a$ and $b$.

For distributed protocols with $N$ parties, we use $\parties = \{\party_1, \dots, \party_N\}$ to denote the set of parties. For a value $x$ in group $\mathbb{G}$, we use $\share{x}$ to denote an additive sharing of $x$ (in $\mathbb{G}$) among the protocol parties, and $\share{x}^{(i)}$ to denote the share of the $i^\thtext$ party. When clear from context (e.g., a local protocol for $\party_i$), we will often drop the superscript. When $\mathbb{G}' =  \mathbb{G}^l$ is a product group (e.g., $\Z_p^l$), for $x \in \mathbb{G}'$, we may also say that $\share{x}$ is a sharing \textit{over} $\mathbb{G}$, similar to the standard practice of calling $x$ a vector over $\mathbb{G}$.

For a $v \in \mathbb{G}$, we use $\randval{v}$ to denote a random mask sampled from the same group, and $\masked{v} = v + \randval{v}$ (where + is the group operation for $\mathbb{G}$) to denote $v$ masked by $\randval{v}$. We use the $+$ operator quite liberally and unless specified, it denotes the group operation (e.g., component-wise addition $\bmod~p$ for $\Z_p^l$) for the summands.


\smallskip
\noindent We now briefly recall standard symmetric primitives.

\begin{definition}[Weak Pseudorandom Function (wPRF)]
Let $\keyspace = \{\keyspace_\secparam\}_{\secparam \in \N}$, $\inspace = \{\inspace_\secparam\}_{\secparam \in \N}$, and $\outspace = \{\outspace_\secparam\}_{\secparam \in \N}$ be ensembles of finite sets indexed by a security parameter $\secparam$. Consider an efficiently computable function family $\{\mathsf{F}_\secparam\}_{\secparam \in \N}$ where each function is given by $\mathsf{F}_\secparam: \keyspace_\secparam \times \inspace_\secparam \to \outspace_\secparam$.
We say that $\{\mathsf{F}_\secparam\}_{\secparam \in \N}$ is an $(l, t, \varepsilon)$-weak pseudorandom function if for infinitely many $\secparam \in \N$ and all adversaries $\advA$ running in time at most $t(\secparam)$, the following holds: taking $f_\secparam \getsr \Funcs[\inspace_\secparam, \outspace_\secparam]$, $k \getsr \keyspace_\lambda$, and $x_1, \dots, x_l \getsr \inspace_\secparam$, we have that,
\[
    \left|\Pr\left[ \advA\left(1^\secparam, {\{ x_i, \mathsf{F}_\secparam(k,x_i)\}}_{i \in [l]} \right)\right]
    - \Pr\left[ \advA\left(1^\secparam, {\{ x_i, f_\secparam(x_i)\}}_{i \in [l]}\right)\right] \right| \leq \varepsilon(\secparam).
\]
\end{definition}


\begin{definition}[One-way Function (OWF)]
Let $\inspace = \{\inspace_\secparam\}_{\secparam \in \N}$, and $\outspace = \{\outspace_\secparam\}_{\secparam \in \N}$ be ensembles of finite sets indexed by a security parameter $\secparam$. Consider an efficiently computable function family $\{\mathsf{F}_\secparam\}_{\secparam \in \N}$ where each function is given by $\mathsf{F}_\secparam: \inspace_\secparam \to \outspace_\secparam$.
We say that $\{\mathsf{F}_\secparam\}_{\secparam \in \N}$ is a $(t, \varepsilon)$-one-way function if for infinitely many $\secparam \in \N$ and all adversaries $\advA$ running in time at most $t(\secparam)$, we have that,
\[
    \Pr\left[x \getsr \inspace; y \gets \mathsf{F}_\secparam(x) : \mathsf{F}_\secparam(\advA(1^{\abs{x}}, y)) = y \right] \leq \varepsilon(\secparam)
\]
\end{definition}

\begin{definition}[Pseudorandom Generator (PRG)]
Let $\inspace = \{\inspace_\secparam\}_{\secparam \in \N}$, and $\outspace = \{\outspace_\secparam\}_{\secparam \in \N}$ be ensembles of finite sets indexed by a security parameter $\secparam$. Consider an efficiently computable function family $\{\mathsf{F}_\secparam\}_{\secparam \in \N}$ where each function is given by $\mathsf{F}_\secparam: \inspace_\secparam \to \outspace_\secparam$.
We say that $\{\mathsf{F}_\secparam\}_{\secparam \in \N}$ is an $(l,t, \varepsilon)$-pseudorandom generator if $\mathsf{F}$ is length-expanding (i.e., $\forall \secparam, \forall x \in \inspace_\secparam$, $\len{x} < \len{\mathsf{F}_\secparam(x)}$) and for infinitely many $\secparam \in \N$ and all adversaries $\advA$ running in time at most $t(\secparam)$, the following holds: taking $x_1, \dots, x_l \getsr \inspace_\secparam$  $y_1, \dots, y_l \getsr \outspace_\secparam$, we have that,
\[
    \left|\Pr\left[ \advA\left(1^\secparam, {\{\mathsf{F}_\secparam(x_i)\}}_{i \in [l]} \right)\right]
    - \Pr\left[ \advA\left(1^\secparam, {\{y_i\}}_{i \in [l]}\right)\right] \right| \leq \varepsilon(\secparam).
\]
\end{definition}
